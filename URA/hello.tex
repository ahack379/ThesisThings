%hello.tex- First LaTeX example
%In preamble
\documentclass[10pt]{article}
\usepackage{color}
\usepackage{hyperref}
\usepackage{setspace}
%\singlespace
%\onehalfspacing
%\doublespace
%\linespread{2}

%End preamble, begin document
\begin{document}
\title{\color{blue}Pirate Typesetting or How to Structure a \LaTeX{} Document}
\author{\color{red}Burger McHack  \\That Burga Joint \\ 
			\texttt{ahack@llama.gov}}
\date{\today}
\maketitle

\renewcommand{\abstractname}{Llama Abstracts are better than normal Abstracts}
\begin{abstract}
World be where it's at.
Where-at?
Here.
Test duffle.  Duf{}fle.
H$_2$O 

\^x + \^y + \^z = \^r
\end{abstract}
%Set Table of contents depth to 3 levels
\tableofcontents
\listoffigures
\listoftables
\setcounter{tocdepth}{3} 
\phantomsection
\addcontentsline{toc}{section}{\color{red}PreIntro: A Word on Llamas}
\section*{PreIntro: A Word on Llamas}
They are gray and beaked.
\section{Introduction}
Llamas are always ignored in the introduction. I would like to make note of the llama. This article will explore the experiences of llamas on the high seas. They're pirates, at heaaaaaaaaaarrrrrrrrt.
	
	Here is a new paragraph on llamas.
	
\section{Structure}
\subsection{Structure 2.0: Llama Environment}
\subsection{Structure 3.0: Llama Hygiene}
\subsection{Structure 4.0: What Gives the Llama its Jive?}
\subsection{Structure 5.0: Llama-Friendly Meals}

\appendix
\section{Aaaarrrppendix number 1}
	
\end{document}