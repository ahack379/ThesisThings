%hello.tex- First LaTeX example
%In preamble
\documentclass[12pt]{article}

\usepackage{color}
\usepackage{hyperref}
\usepackage{setspace}
\usepackage[margin=1in]{geometry}

%End preamble, begin document
\begin{document}
\title{Proposal to Measure the Cross Section of CC Inclusive Pi0 Interaction Channel...etc }
\author{Ariana Hackenburg  \\Yale University \\ 
			\texttt{ariana.hackenburg@yale.edu}}
\date{\today}
\maketitle

\renewcommand{\abstractname}{Abstract}
\begin{abstract}
Llamas be where it's at.
\end{abstract}

\clearpage

%Set Table of contents depth to 3 levels
\tableofcontents
\listoffigures
\listoftables
\setcounter{tocdepth}{3} 
\phantomsection

\clearpage

\section{Overview and History}


Definitive proof for neutrino oscillation has been delivered over and again in the last 15 years. Alongside this proof however, have come several notoriously anomalous results. Liquid Scintillator Neutrino Detector (LSND), an oscillation experiment from the 90�s, observed an excess of low energy electron anti-neutrino events for L/E ~ 1m/MeV (Fig 1), explained at the time with a simple 2-neutrino oscillation model (Fig 2). These unexpected results led directly to Mini Booster Neutrino Experiment (MiniBooNE). After 10 years of running, MiniBooNE�s data revealed an excess of low energy neutrino events in both neutrino and anti-neutrino mode at energies below and incorporating LSND�s data(Fig 3). These results did several important things: first, MiniBooNE�s data supported LSND in anti-neutrino mode, while at the same time refuted the previous 2-neutrino oscillation explanation.  MiniBooNE also observed different amounts of excess in neutrino and anti-neutrino mode, leading to more questions about nuclear interactions and cross sections; lastly, it called into question the nature of the observed excess. 
MiniBooNE is a Cherenkov detector (Fig 4). In a Cherenkov detector, an electron and single photon are indistinguishable; how can we determine if MiniBooNE�s excess was due to neutrinos or to photon background?
\subsection{The Beginning}

Any sort of historical review of neutrino physics at some point comes to Ray Davis.  In the late 1960s, Ray Davis of Brookhaven National Lab (BNL) ventured 4850ft underground in the Homestake mine with a plan to measure various contributions to solar neutrino flux \cite{ray0}. Working closely with John Bahcall's theory team at Caltech, a tetrachloroethylene detector sensitive down to 0.814 MeV and primarily to $B^8$ solar neutrinos\cite{ray0} was built a mile underground in Homestake Mine. In 1968, the group released results which revealed that a fraction of the predicted solar neutrinos were missing (it was confirmed by a special task force that the mole people did not eat the missing $2/3$ neutrinos).  This result, known as "the solar neutrino problem", set the stage for the following half century drive towards uncovering and resolving anomolies in neutrino physics. 

A series of experiment followed throughout the 80's and 90's confirming the existence of the solar neutrino problem \cite{rayreview} \cite{kam0} \cite{sno0}, but observing slightly different fraction of missing neutrinos (about half).  Kamiokande, a 3kton water Cherenkov detector in Japan, was only sensitive to $B^8$ neutrinos because of the surroundign radioactivity.  As a result it was only able to probe down to abotu 5 MeV (vs Homestake's 0.814MeV).  
By the 1990s, the idea of neutrino oscillation as a solution to the solar neutrino problem had become popular, but remained unproven.  
\subsection{Definitive Proof of Mole People Innocence}
n experiment in Hida, Japan called Kamiokande  
\\ 2) SNO/Kamland/Super K
\\ 3) Other current neutrino experiments?


\section{Theory}
Go into some detail about the neutrino oscillations--maybe do a derivation in the appendix.  Make sure you quote all relevant resources.  

\subsection{Major Results Thus Far}
Include in this section thus far collection information on :

1) neutrino oscillation frequency $\Delta m^2$ 
\\ 2) oscillation amplitude $sin^2 2\theta$


\subsection{Experimental Anomalies}
Include mention of :


1) Gallex/Sage
\\ 2) Reactor experiments
\\ 3) Conclude with oscillation anomalies--LSND, MiniBooNE

Summarize what these things collectively can imply--transition from this into a technology we can use which will help answer some of these questions

\section{Liquid Argon Time Projection Chambers (LArTPCs)}

 There are a number of properties that make Liquid Argon (LAr) an appealing detector medium for particle physicists: it is cheap, easy to cool, and transparent to its own scintillation light.  When a charged particle travels through a TPC in LAr, it leaves a trail of ionization electrons that are drifted by an applied field to wire planes. Readout from the wires make the Y and Z coordinates of an interaction accessible, while scintillation signals seen by an array of PMTs make the drift direction accessible.  Thus a LArTPC with 3 (or even 2) planes give us the capability to reconstruct 3 dimensional configurations of events in the detector. It is important to note that LAr scintillates outside the visible spectra at 128nm. In order to �see� events in LAr, our Photo Multiplier Tubes (PMTs) must be coated in a layer of wavelength shifting material, currently Tetraphenyl Butadiene (TPB).

Liquid Argon Time Projection Chambers (LArTPCs) are ideal detectors for neutrino oscillation experiments with long baselines. 

\subsection{MicroBooNE}

MicroBooNE, the latest in a series of Booster Beam experiments located at Fermilab, is a Liquid Argon Time Projection Chamber (LArTPC) that will investigate the low energy neutrino excess seen by its predecessor, MiniBooNE. Cherenkov detectors, such as MiniBooNE, are limited by their inability to distinguish between single electrons and photons, a task LArTPCs are well suited for. With the high precision reconstruction capabilities of a LArTPC, MicroBooNE will be able to determine with high statistical certainty whether electrons or photons caused the anomalous MiniBooNE low energy excess. Of further interest to MicroBooNE are neutrino-nucleon cross-sections. Cross sections have accounted for much of the uncertainty in recent results from a variety of neutrino experiments, including **** and *****(reference) and sensitive measurements by MicroBooNE will lead to improved nuclear models and rate predictions. Beyond MicroBooNE, LArTPCs will continue to play a notable role in oscillation physics. LAr1-ND will act as a baseline for improving systematic uncertainties in MicroBooNE and investigating the nature of the MiniBooNE excess, while also acting as a small-scale phase experiment for future, bigger LArTPCs such as LAr1 and LBNE. 

\section{Current Efforts and Proposal}
MicroBooNE, a surface dweller, is subject to a constant bombardment of cosmic radiation from the upper atmosphere. Electromagnetic showers originating from cosmic $\mu^+$ and $\mu^-$ are our primary source of background in this case, carrying the ability to mimic MicroBooNE's single electron shower signal(with vertex activity). There are two subsets of events we can break these backgrounds into in order to maximize our rejection efficiency and minimize our signal rejection.  The first is the case in which the muon parent is seen in the beam window. This class of shower backgrounds can be minimized fairly effectively with a series of cuts described in DocDB \# [technote].  The second and more concerning class of showers are those who either originate from a neutral particle (ie, neutron, $\pi^0$ ), or those whose charged particle parent does not enter the detector.  While ionization electrons originating outisde the detector can be rejected as "entering showers", gammas produced outside may not Compton scatter/ pair produce until they are inside the detector; this gives it a high probablity to be tagged as a signal event if steps are not taken to minimize this background. To filter these background events from data, we can look at various parameters that may distinguish them from our desired signal. A dEdx cut can be used with 94\% efficiency to separate pair producing gmmas from cosmics frmo the Compton scatters ****DocDB\#[Andrzej technote]. Also useful is the idea that most cosmic radiation will come from above the detector; it is thus reasonable to expect many of the Compton scattered (and pair produced) showers to have a downward trajectory. In addition, the interaction length in Liquid Argon (14cm) limits the distance we can expect a Compton gamma to travel before interacting (DocDB 3693).   We explore these and other parameters in further detail in this study.
\\ \\In the first, MCC5 generated BNB files are used with (Kazu's) PDG filter to select single electron events.  Information such as energy, vertex activity, vertex coordinates and distances along electron trajectory to detector edge (both forward and backward) are stored for 121 electron events / 20,000 total BNB events.  This is intended to act as a baseline for the second study.  
\\ \\  In the second, the cosmogenic sample described above is examined.  In such a case, there are 2 subsets of interactions to sift through which might mimic our primary signal: Compton scatters and pair production.  A fiducial volume cut guarantees the vertex is inside the detector volume and dE/dx information from the first 2.4cm of a shower can help distinguish gamma from electron activity; we are then able to distinguish between these two interactions in a true sample.  Our first mode of action beyond this is thus to introduce a tag which stores properties of each interaction mode separately (I'm pretty sure I'm going into unnecessary detail here?). Using the selected sample of Compton scatters from Cosmic data, we then investigate properties of the interactions.  In particular, we (will) use geometric algorithms to calculate the distance back along the trajectory from vertex to detector edge.  Combing this information with interaction length, energy and angle, we are able to successfully ( hopefully--not done yet) reduce background Compton scatters while minimally reducing events in the true single electron sample. 
\\ \\Ultimately, these reconstruction efforst will lead to a series of interaction channel cross section measurements.  To do a cross section measurement, there are several pieces of information that we need to obtain. The cross section is given by:
\\ \\ \centerline{ $N_{obs} = \sigma * \Phi * \epsilon$, }	
\\ \\ where $N_{obs}$ is number of interactions , $\Phi$ is the flux of your interaction of interest and $\epsilon$ is the efficiency with which you are able to select events.  
The flux in BNB has been well characterized by the MiniBooNE collaboration and is given in DocBB \# [ ]. N is the number of interactions we select in this study.  This means that the real lifting comes in estimating your efficiency. There are a few aspects of estimating your efficiency of event selection.  The first is that efficiency is energy dependent.  One way to overlook this issue is to calculate a differential cross section of -channel of interest- on Ar with respect to energy; we will make this measurement for completeness.  Another issue is that estimating the efficiency of our automated reconstruction is tricky.  With a limited sample of un-blinded data, we don't have very high statistic and have to rely on our MC truth estimates to estimate our efficiency.     

\appendix
\section{Aaaarrrppendix number 1}

\begin{thebibliography}{2} %1 should be replaced with number of citations
\bibitem{ray0}
  Raymond Davis Jr and Don S. Harmer,
  \emph{Solar Neutrinos}.
  BNL,
  April 30, 1965
\bibitem{rayreview}
  MEASUREMENT OF THE SOLAR ELECTRON NEUTRINO FLUX WITH THE HOMESTAKE
  CHLORINE DETECTOR
  
\bibitem{kam0}
  Kamiokande paper

\bibitem{sno0}
  SNO paper

\end{thebibliography}



\end{document}
