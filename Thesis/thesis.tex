%In preamble
\documentclass[12pt]{article}

\usepackage{color}
\usepackage{hyperref}
\usepackage{setspace}  %\doublespacing
\usepackage{amsmath}
\usepackage{siunitx}
\usepackage{tabularx}
\usepackage{wrapfig}	 %embed figure in text
\usepackage{esvect}      %vector notation
\usepackage{indentfirst} %automatically indents the beginning of each section
\usepackage{graphicx}    %include graphics
\usepackage{fancyhdr}
\usepackage{mhchem}       %for chemistry typesetting 
\usepackage{float}        %for end page after figure [H]
\usepackage{textcomp}     %for copyright symbol
\usepackage[font=scriptsize]{caption}
\usepackage[margin=1in]{geometry}

\doublespacing
\addtolength{\evensidemargin}{-1.5in}

%End preamble, begin document
\begin{document}

\begin{center}
{\footnotesize ABSTRACT}\\
\vspace{4 mm}
{\large \textbf{MEASUREMENT OF A NEUTRINO-INDUCED CHARGED CURRENT SINGLE NEUTRAL PION CROSS SECTION AT MICROBOONE}}\\
\vspace{6 mm}
{\footnotesize ARIANA HACKENBURG\\
2017\\}
\end{center}
Micro Booster Neutrino Experiment (MicroBooNE) is a Liquid Argon Time Projection Chamber (LArTPC) operating in the Booster Neutrino Beamline (BNB) at Fermi National Accelerator Laboratory (Fermilab). MicroBooNE's primary goal is to identify the electromagnetic particle responsible for the low energy excess observed by its predecessor MiniBooNE; the results of this oscillation analysis may point to the existence of new particles and new physics. The path to this and future measurements is reliant on our understanding of interaction cross sections of neutrinos on nuclear targets.  This work details the first $\nu_{\mu}$ charged-current (CC) neutral pion ($\pi^0$) cross section on Argon using 500k Run I neutrino interactions. We also detail the challenges involved in reconstructing LArTPC event images and the strategies used to target various event topologies. We calculate the flux-averaged, total cross section for the $\nu_{\mu}$ CC-$\pi^0$ channel to be blah at $<E_{\nu}>$ = 832 MeV.   

%MicroBooNE has a short baseline (400 m) and is sensitive to lower energy neutrino interactions (0.5 - 2.0 GeV). 

%Liquid Argon Time Projection Chambers (LArTPCs) such as MicroBooNE provide excellent calorimetric information and image quality resolution. Of particular interest to MicroBooNE are electromagnetic showers, and the origin of the low energy excess. $\pi^0$’s, which decay into 2 electromagnetic showers ($\gammas$’s), form a fraction of the flagship analysis background and are thus an important topic of study. In this thesis analysis, I calculate the $\nu_{\mu}$ charged-current (CC) single $\pi^0$ cross section on Argon using 5e19 protons on target of data.  %We have chosen final state CC-$\pi^0$ interactions as our signal in order to compare with MiniBooNE results and to produce a measurement that will have potential use to other experimentalists. Finally, we will present the preliminary cross section measurement and examine the different variables that contribute to the result.
\thispagestyle{empty}
\clearpage

\singlespacing
\title{MEASUREMENT OF A NEUTRINO-INDUCED CHARGED CURRENT SINGLE NEUTRAL PION CROSS SECTION AT MICROBOONE}
%\author{Ariana Hackenburg  \\Yale University } 
\date{}
\maketitle

\vspace{4 cm}

%\date{\today}
\begin{center}
A Dissertation \\
Presented to the Faculty of the Graduate School \\
of \\
Yale University \\
in Candidacy for the Degree of \\
Doctor of Philosophy\\ 

\vspace{6 cm}
%{\small
by \\
Ariana Hackenburg \\
\vspace{3 mm}
Dissertation Director: Associate Professor Bonnie T. Fleming \\
\vspace{3 mm}
December 2017 \\ %}
\end{center}
\thispagestyle{empty}

%\begin{figure}[h!]
%\centering
%\includegraphics[scale=0.5]{Figures/ubooneLift.png}
%\end{figure}

\clearpage

\vspace*{\fill}
\begin{center}
\textcopyright 2017 by Ariana Hackenburg\\
All rights reserved.
\end{center}
\vspace*{\fill}

\clearpage

\doublespacing

%For adding the header (footer) if so desired
\pagestyle{fancy}% Change page style to fancy
%\fancyhf{}% Clear header/footer
\fancyhead[C]{}
%\fancyfoot[C]{}% \fancyfoot[R]{\thepage}
\renewcommand{\headrulewidth}{0.4pt}% Default \headrulewidth is 0.4pt
%\renewcommand{\footrulewidth}{0.4pt}% Default \footrulewidth is 0pt

\pagenumbering{roman}
%Set Table of contents depth to 3 levels
\tableofcontents
\clearpage

\listoffigures
\setcounter{tocdepth}{3} 
\phantomsection

\clearpage
\renewcommand{\thepage}{\arabic{page}}
\setcounter{page}{1}

\section{Introduction}
The idea of the neutrino was born out of an unwillingness to abandon energy conservation without a fight. At the heart of the debate was the experimentally observed energy spectrum of beta decay. According to energy conservation, two body beta decay was expected to occur at a single energy, however observations showed this distribution to be spectral; this suggested that the theory describing the process was either incomplete on incorrect.  In 1930 Wolfgang Pauli proposed the existence of an additional particle emitted during the decay which carried away some fraction of the energy to explain the observed beta spectrum \cite{bib:pauli}. These weakly interacting, spin 1/2, ``small neutral particles" were later named ``neutrinos" by Enrico Fermi \cite{bib:fermi}. 
\par The clean transformation of theoretical predictions into experimentally measurable, physical parameters can at times prove to be its own kind of science. Pauli himself thought he had done a ``frightful thing" by proposing what he believed at the time to be an undetectable particle. However, nearly 30 years after Pauli's theory was born, Clyde Cowan and Frederick Reines confirmed the existence of anti-neutrinos experimentally in a cadmium-chloride liquid scintillator detector near the Savannah River Plant \cite{bib:cowan}. In 1995, they received the Nobel prize for their work.
\par With this work, conservation of energy was safe, but new puzzles lay ahead. Nearly 90 years after it was first proposed, the neutrino remains shrouded in a number of mysteries. In this section, we explore some anomalous results and their impact on the present state of neutrino physics.  We also consider the theory behind the Standard Model, neutrino oscillation and interactions and the primary interaction of interest to this thesis work. 
%In the late 1960's, the Homestake experiment guided in a new era for neutrino experiments.
% Definitive proof for neutrino oscillation has been delivered over and again in the last 15 years. Alongside this proof, however, has come several notoriously anomalous results. 
%%As additional groups began to build experiments designed to look for neutrinos from different sources and energy scales, a number of questions arose.

\subsection{A Brief History Rife with Anomaly}
%\subsection{Solar Neutrino Problem}
%Any sort of historical review of neutrino physics at some point comes to Ray Davis, and so we might as well start our story here.  
\paragraph {Solar Neutrino Problem} In the late 1960s, Ray Davis of Brookhaven National Lab (BNL) ventured 4850ft underground in the Homestake mine with a plan to measure various contributions to solar neutrino flux \cite{ray0}. Working closely with John Bahcall's theory team at CalTech, Davis' group built a tetrachloroethylene detector sensitive down to 0.814 MeV and primarily to $B^8$ solar neutrinos a mile underground in Homestake Mine \cite{ray0}. In 1968, the group released results which revealed that a fraction of the predicted solar neutrinos were missing. The Homestake experiment ran for ~20 years, with results through the duration pointing to the same conclusion: 2/3 of the neutrinos predicted by Bahcall's Standard Solar Model (Figure \ref{fig:SSM}) were not accounted for in the data \cite{ray0}. This result became known as ``the solar neutrino problem".

\begin{wrapfigure}{r}{0.5\textwidth}
%\begin{center}
%\captionsetup{justification=centering}
\includegraphics[scale=0.5]{Figures/solarFlux.png}
%\end{center}
\caption{John Bahcall's Standard Solar Flux Model}
\label{fig:SSM}
\end{wrapfigure}

%\cite{rayreview} \cite{kam0} \cite{sno}. 
\par Throughout the 80's and 90's, a series of experimental results from various collaborations confirmed the mysterious lack of solar neutrinos. Kamiokande II, a 3 kiloton water Cherenkov detector built in Japan in the 80s, was sensitive to $B^8$ neutrinos above ~6 MeV\footnote{The original KamiokaNDE focused primarily on proton decay. After initial data taking, adjustments were made to the original detector in order to make it sensitive enough to study electron recoils from elastic neutrino scatters in water. One example addition was a surrounding layer of water which was intended to decrease background radiation.  This ``new" experiment was called Kamiokande II}. Due to surrounding radioactivity, Kamiokande II was unable to probe the keV energies of the Homestake experiment; however, Kamiokande II was a highly appealing detector due to its real time depiction of events and its ability to reconstruct both the energies and directions of those events. Due to the unique ability of this detector, Kamiokande II was able to for the first time conclude that neutrinos were coming from the sun \cite{kam0}.  In addition to this remarkable discovery, Kamiokande II also observed an apparent lack of solar neutrinos.
\par Quick to follow suit was the Sudbury Neutrino Observatory (SNO) in 2002.  SNO was a 1 kt spherical deuterium Cherenkov detector located 6800ft under ground in the Creighton Mine in Sudbury \cite{sno}. Up until now, the greatest barrier experiments had faced in detecting $\nu_\mu$ or $\nu_\tau$ interactions was that the $\mu$ and $\tau$ are heavier (105 MeV and 1777 MeV respectively) than the energy of the solar neutrino spectrum (extends to about 30MeV). Deuterium, which replaces the hydrogen in water with a proton and a neutron, has a dissociation energy of ~2 MeV. This fact uniquely equipped SNO to measure the flux of all three neutrino flavors via combination of the charged and neutral currents. SNO's final measured flux was in good agreement with Bahcall's predictions in his Standard Solar Model (SSM) \cite{sno}.  SNO had conclusively solved the solar neutrino problem.  
\paragraph{Gallium Anomaly} The Gallium Experiment, or GALLEX, was a radiochemical experiment at Gran Sasso, Italy in the 90's. GALLEX was sensitive to the pp neutrino flux energies with a threshold of 233 keV, a thus far uncharted region of the solar flux related to nuclear fusion. After 6 years of running, results showed the neutrino flux at a deficit of 60\% of that predicted by the SSM \cite{gal0}. This result was reproduced when a \ce{^{51}Cr} source was added to the detector \cite{gal2}. SAGE, another Gallium experiment set in Russia in the 90's observed a deficit comparable to GALLEX's.  A similar test of the SAGE detector's efficiency with a \ce{^{51}Cr} source confirmed it was operating near 100\% on the \ce{^{51}Cr} neutrinos \cite{sage}.  Together, these pieces are called the Gallium Anomaly.  

%\par A quick recap at this point would indicate something a bit unsettling about our story thus far. The Davis experiment began digging into Bahcall's models in the 60s; yet here we were in the 90s with still no verified explanation of what was causing the observed neutrino deficit.  Were neutrinos oscillating? Was something wrong with the sun \cite{Clarke}? 

%\paragraph{SNO : A Conclusion to the Solar Neutrino Problem}
\paragraph{Reactor Anomaly}
A third class of neutrino anomalies arises from reactor experiments. Generally built at short baselines near nuclear plants, this kind of experiment measures the flux of electron anti-neutrinos resulting from inverse beta decay. In the 80's and 90's, experiments such Bugey, ROVNO, ILL and others consistently observed an unexpected deficit in the flux \cite{bib:bugey} \cite{bib:rovno}. Though a small deficit seemed to occur consistently across experiments, it was not significant enough to draw a strong conclusion that neutrinos were missing.  Recently however, improved models of reactor antineutrino spectra \cite{bib:reactorImproved} and a reevaluation of previous reactor experiments \cite{bib:reactorGeneral} suggests more strongly that the deficit is real.  One potential explanation of the deficit is the oscillation of the electron anti-neutrinos into sterile neutrinos before they reach the detectors.  

\paragraph{Low Energy Excess} %LSND and MiniBooNE}
Finally we arrive to the predecessors of MicroBooNE. Liquid Scintillator Neutrino Detector (LSND), a scintillation detector in a stopped pion beam from the 90's, expected the majority of its events to come from $\nu_\mu$ and $\bar{\nu}_\mu$, with a small fraction of $\bar{\nu}_e$ interactions. Their results indicated an excess of low energy $\bar{\nu}_e$ events for $\frac{L}{E} \sim 1 \frac{m}{MeV}$ (Figure \ref{fig:lsnd}), explained at the time with a simple 2-neutrino oscillation model \cite{lsnd}. These unexpected results led to the construction of the Mini Booster Neutrino Experiment (MiniBooNE), a Cherenkov detector in the Booster Neutrino Beam (BNB) at Fermilab with a comparable $\frac{L}{E}$ to LSND. After 10 years of running, MiniBooNE's data revealed an excess of low energy events in both neutrino and anti-neutrino modes at energies below and incorporating LSND's data (Figure \ref{fig:lsnd}).  MiniBooNE also observed different amounts of excess in neutrino and anti-neutrino mode, leading to additional questions about nuclear interactions, cross sections and the relationship between the neutrino and antineutrino \cite{miniboone}.  These results did two important things: first, MiniBooNE's data refuted the previous 2-neutrino oscillation explanation \cite{miniboone}, and second, it called into question the nature of the observed low-energy excess. Identification of the source of the MiniBooNE low energy excess is the main question MicroBooNE was designed to answer, in addition to the valuable cross section measurements and R\&D it will produce. 
%; R\&D and cross section measurements on Argon are his flagship measurement will be background narrative to the story of this thesis. 
\begin{figure}[h!]
\centering
\includegraphics[scale=0.5]{Figures/lsnd3.png}
\hspace{1.5 mm}
\includegraphics[scale=.65]{NewFigures/minibooneNeuMode.png}
\caption{Low energy excesses seen by a) the LSND and b) MiniBooNE experiments.  MiniBooNE data is depicted here for runs taken in Neutrino Mode.}
\label{fig:lsnd}
\end{figure}

\subsection{Neutrino Oscillation}
%($\nu_e, \nu_\mu, \nu_\tau$) 
\par We are interested to know the probability of oscillation from one neutrino flavor state $\alpha$ into another (or the same) flavor state $\beta$. Each neutrino flavor can be written as a linear combination of mass eigenstates with the use of a unitary rotation matrix. We begin by looking at a simplified 2 neutrino oscillation model:
\begin{equation} \label{eq:eig}
|\nu_x> = \sum_{j=1,2} U_{ij} |\nu_j>  
\end{equation}
where $U_{ij}$ is the 2 dimensional unitary rotation matrix: 
\begin{equation}
U = \begin{bmatrix}
cos(\theta) & sin(\theta)
\\ -sin(\theta)& cos(\theta)
\end{bmatrix}
\end{equation}

\noindent Our goal is to create an equation whose measurable bases (flavors) are uniform on both sides. To start, we apply the time-dependent Hamiltonian (with propagator $\phi = Et - \vec{p}\cdot\vec{x}$) to the mass bases: 
\begin{equation} \label{eq:prop}
 \begin{bmatrix}
 \nu_1(t)
 \\ \nu_2(t)
 \end{bmatrix}
 = \begin{bmatrix}
e^{-i\phi_1} & 0
\\ 0 & e^{-i\phi_2}
\end{bmatrix} 
 \begin{bmatrix}
 \nu_1(0)
 \\ \nu_2(0)
 \end{bmatrix}
\end{equation}

\noindent Noting the inverse of equation \ref{eq:eig} at t = 0:
\begin{equation} \label{eq:osc3}
 \begin{bmatrix}
 \nu_1(0)
 \\ \nu_2(0)
 \end{bmatrix}
 = \begin{bmatrix}
cos(\theta) & -sin(\theta)
\\ sin(\theta)& cos(\theta)
\end{bmatrix} 
 \begin{bmatrix}
 \nu_\alpha(0)
 \\ \nu_\beta(0)
 \end{bmatrix}
\end{equation},
we combine equations $\ref{eq:eig}$, $\ref{eq:prop}$ and $\ref{eq:osc3}$ to write the time propogated flavor basis as:
\begin{equation} \label{eq:osc4}
 \begin{bmatrix}
 \nu_\alpha(t)
 \\ \nu_\beta(t)
 \end{bmatrix}
 = \begin{bmatrix}
cos(\theta) & sin(\theta)
\\ -sin(\theta)& cos(\theta)
\end{bmatrix} 
  \begin{bmatrix}
 e^{-i\phi_1} & 0
\\ 0 & e^{-i\phi_2}
  \end{bmatrix}
\begin{bmatrix}
cos(\theta) & -sin(\theta)
\\ sin(\theta)& cos(\theta)
\end{bmatrix}
\begin{bmatrix}
\nu_\alpha(0)
\\ \nu_\beta(0) 
\end{bmatrix}
\end{equation}

\noindent To simplify further, assume a beam is produced in a pure $\nu_\beta$ state at time = 0. To find the probability that at time = t a $\nu_\beta$ in the beam has oscillated into a $\nu_\alpha$, we calculate the square of the probability amplitude:
\begin{equation} \label{eq:osc5}
\begin{split}
 |<\nu_\beta(0)|\nu_\alpha(t)>|^2 &= |<\nu_\beta(0)|cos\theta sin\theta(e^{-i\phi_1}-e^{-i\phi_2})|\nu_\beta(0)>|^2
\\&= (cos\theta sin\theta)^2(1+1+e^{i(\phi_1-\phi_2)} + e^{-i(\phi_2-\phi_1)})
\\&= 2(cos\theta sin\theta)^2(1-cos(\phi_1-\phi_2))
\\&= sin^2 2\theta sin\big(\frac{\phi_1-\phi_2}{2}\big)
\end{split}
 \end{equation}

\noindent Finally, we recall from above the phase shift $\phi = E_i t-p_i x$ and make the following assumptions:
\par 1) At $v\sim c,\ x\sim t\sim L$
\par 2) Mass eigen states are created with equal energy\\
\noindent Now $\phi_i$ can be written (E - $p_i$)L, where $p_i$ $\approx$ E(1-$\frac{1}{2}\frac{m_i}{E}^2$). Thus, the phase difference can be written 
\begin{equation}
\\ \phi_2-\phi_1 = \frac{1}{2}\frac{L}{E}(\Delta m_2^2 - \Delta m_1^2)
\end{equation}
which leads us to the probability of oscillation from $\nu_\beta$ to $\nu_\alpha$:
%\begin{equation}
%P(\nu_\beta\rightarrow\nu_\alpha)=sin^2 (2\theta) sin\big(\frac{1}{4}\frac{L}{E}\Delta m_{12}^2\big)
%\end{equation}

%After propagating these states in time and doing some rearranging, we arrive at 
\begin{equation} \label{eq:prob}
P(\nu_\alpha \rightarrow \nu_\beta) = sin^2(2\theta)sin(1.27\Delta m^2  \frac{L}{E})
\end{equation}
where $\theta$ is the mixing angle of the oscillation, $\Delta m^2$ is the frequency of neutrino oscillation, L is the length the neutrino traveled and E is the energy of the neutrino at its source.  $\frac{L}{E}$ is the experimentally controllable parameter. 
\par It is useful to look at an example of how the free parameter affects sensitivity.  An experiment is most sensitive to $\Delta m^2$ for $\Delta m^2 \approx E/L$.  In addition, the neutrino beam diverges $\propto \frac{1}{L^2}$.  So an experiment with a short baseline (small L) has the benefit of seeing lots of events (i.e., has high sensitivity to $sin^2(2\theta)$), and is sensitive only to large values of $\Delta m^2$ \cite{warwick}. 
%\par In addition to dependance on L and E as mentioned above, the sensitivity of an experiment to $\Delta m^2$ and $sin^2(2\theta)$ depends also on event rate. We can see this by re-writing equation (\ref{eq:prob}) as
%\begin{equation} \label{eq:probN}
%N_\beta = N_\alpha sin^2(2\theta)sin(1.27\Delta m^2  \frac{L}{E})
%\end{equation}
%where $N_\beta$ is the number of signal events expected, and $N_\alpha$ is the number of original neutrinos in the absence of oscillations. Sensitivity to $\Delta m^2$ can be shown to be dependent on $L^{-\frac{1}{2}}, E,$ and $N^{-\frac{1}{4}}$ \cite{warwick}; for a given L, we can increase our sensitivity to $\Delta m^2$ by increasing event rate(generally dependent on factors like runtime, which are out of experiment's control), or energy(tricky). 




\subsection{Standard Model}
The Standard Model (SM) is a theory of fundamental forces and particles that together describe the state of matter in the world. The SM itself contains 2 different classes of particles (quarks and leptons), a group of force carriers (gauge bosons) and most recently the Higgs Boson (Figure \ref{fig:SM}a). Other recent additions to the SM include the tau neutrino observed by the DONUT experiment in 2000 \cite{bib:donut}, and the top quark in 1995 \cite{bib:cdf}. However, there are a number of physical processes that are not explained by the current SM, suggesting the SM is incomplete.  Some of these phenomena include gravitation (graviton), accelerating galaxies (dark energy), and neutrino oscillation (SM neutrinos are massless). 

\begin{figure}[h!]
%\begin{wrapfigure}{r}{0.5\textwidth}
\centering
\includegraphics[scale=0.5]{NewFigures/standardModel.png}
\hspace{6 mm}
\includegraphics[scale=0.5]{NewFigures/betaDecay.png}
\caption{a) The Standard Model of particle physics in its current state; b) Beta decay involves an exchange of $W^-$ boson, and the emission of an electron and anti electron neutrino. }
\label{fig:SM}
%\end{wrapfigure}
\end{figure}
\par The Standard Model is made up two classes of particles: fermions and bosons. Fermions are particles which have half-integer spin and obey Fermi-Dirac statistics. Leptons (particles that do not experience strong interactions, such as electrons) and baryons (3 quark particles, like protons) are both characterized as fermions. Because of their spin properties, fermions must conserve both lepton and baryon numbers, as well as obey the Pauli Exclusion Principle. 
\par Bosons, on the other hand, have integer spin, obey Bose-Einstein statistics and are unaffected by the Pauli Exclusion Principle. The 4 gauge bosons (photon, gluon, W, Z) and single scalar boson (Higgs) described by the SM are called elementary bosons. While the Higgs is thought to be the particle that gives other particles mass, the gauge bosons are force carriers and mediate interactions that occur between composite bosons (such as mesons) and fermions.  The gluon mediates the aptly named strong force, strongest of all the forces, between quarks and other gluons. Additionally, the gluon residually facilitates the nuclear force, or the interactions between nucleons and mesons, which is responsible for holding the nucleus together. A second force mediated by a gauge boson is the electromagnetic force.  This force is 137 times weaker than the strong force at distances of 1 femtometer \cite{bib:forces}, and is controlled by photon exchange between charged particles. The final force described by the SM is the weak interaction, a million times weaker than the strong force, and mediated by the W and Z gauge bosons.  The W boson is its own anti-particle and can facilitate neutrino emission and absorption, energy exchange and particle change as shown in Figure \ref{fig:SM}b. The Z boson in neutral and only allows for exchange of energy, momenta and spin between particles; this leaves the original neutrino less energy, but otherwise in tact. Gravitation, though not included in the SM, is the weakest of the forces and $10^{38}$ times weaker than the strong force.


\subsection{The Charged Current}
As mentioned earlier, the $W^\pm$ and Z bosons mediate the weak force. When an interaction occurs via the exchange of a $W^\pm$ boson, the interaction is called charged current (CC).  A neutrino can also interact via the exchange of a Z boson through the neutral current (NC) (Figures \ref{fig:CCNC}). For the remainder of this section and body of work we will focus on CC interactions. 

Neutrinos can interact with a nucleon via the charged current to produce a lepton in the final state and any number of other particles depending on the original neutrino energy. The signal definition we adopt here will cover events with a single $\mu$, single $\pi^0$ in the final state, and any number of other mesons and nucleons. This choice allows us to directly compare the resulting measurement with other experiments (MiniBooNE, SciBooNE) who have chosen the same signal definition.

% Notes for now
% - strong force decay time ~10-23, electromagnetic (eg pi0 decay) ~10-16, weak ~10-8.
% - pi lightest hadron; no electromagnetic decay. suggests conservation of lepton number
% - neutron mean lifetime ~920 s; proton limit 10+30 years
\par ****NEED TO EXPAND. Things to potentially add:
\par - Form of the CC propogator 
\par - Abridged derivation of the probability amplitude
\par - Connection of the above with theoretical cross section 


% Course: \cite{bib:weak_current}

\begin{figure}[H]
\centering
\includegraphics[scale=0.4]{NewFigures/ccpi0.png}
\hspace{3 mm}
\includegraphics[scale=0.4]{NewFigures/ncpi0.png}
\caption{a) A CC interaction has a charged lepton in the final state; b) An NC interaction will lack a final state charged lepton.} 
\label{fig:CCNC}
\end{figure}


\subsection{History of Neutrino-Induced Charged Current $\pi^0$}
 A number of previous measurements of the Charged Current (CC) $\pi^0$ cross section exist, and it is worth understanding where the field is and how we can contribute to it.  
\par In the 80's several CC $\pi^0$ cross section measurements were made by a variety of experiments.  Argonne National Laboratory (ANL) used a 12-ft bubble chamber full of hydrogen and deuterium to investigate single-pion production by the weak charged current \cite{bib:ANL1} \cite{bib:ANL2}. ANL examined an energy range of $E_\nu$ $<$ 1.5GeV in order to restrict multi-$\pi$ backgrounds entering their final sample of 273 events. They measured the final cross section as a function of energy. BNL performed similar studies in a 7ft deuterium bubble chamber in a broad band beam with average energy 1.6GeV; their signal sample was bigger at 853 events, and spanned an energy range up to 3 GeV. A few other experiments made measurements at higher energies, above the range of MicroBooNE \cite{bib:HE_unknown1} \cite{bib:HE_unknown2}.
\par More recently, experiments at Fermilab have also made this measurement. In 2011 the MiniBooNE experiment, a Cherenkov detector filled with $CH_2$ mineral oil and sitting in the Booster Neutrino Beam (BNB), made cross section and differential cross section measurements of the charged current single neutral $\pi^0$ interaction channel. They required their signal events to have an observed single $\mu^-$, single $\pi^0$, any number of additional nucleons, and no additional mesons or leptons. With 5810 signal events in their sample, they measured a flux-integrated cross section of (9.2 $\pm$ 0.3stat. $\pm$ 1.5syst.) x $10^{-39}$ $\frac{cm^2}{CH_2}$ \cite{bib:numucc_miniboone} \cite{bib:miniboone_thesis}. 
\par From 2007-2008, SciBar Booster Neutrino Experiment at Fermilab (SciBooNE) took data in the Booster Neutrino Beam.  In 2014, a SciBooNE thesis measurement showed a measured CC $\pi^0$ cross section of (5.6 $\pm$ $1.9_{fit}$ $\pm$ $0.7_{fit}$ $\pm$ $0.5_{int}$ - $0.7_{det})$ x $10^{-40}$ $\frac{cm^2}{nucleon}$ with 141 signal data events \cite{bib:sciboone_thesis}. When the MiniBooNE result is scaled per nucleon, the results agree with one another. A few things are worth noting here. First, the SciBooNE signal definition employed was different than that used by MiniBooNE in that it allowed N additional mesons in its final state. Second, the large statistical difference between final signal samples is due to both the much higher total collected Protons On Target (POT) available to the MiniBooNE measurement, and MiniBooNE's considerably higher selection efficiency. %The SciBooNE detector consists of 3 pieces: SciBar, or the Carbon/Iron Core for neutrino interactions, the Electron Catcher and the Muon Range Detec
\par In 2015, the MINERvA experiment measured $\overline{\nu}_\mu$ charged current $\pi^0$ differential cross sections against a number of variables on polystyrene \cite{bib:minerva_thesis} \cite{bib:minerva_paper}.  MINERvA lies in the Neutrinos at the Main Injector (NuMI) beamline at Fermilab, and probes an energy range of 2-10 GeV.  MicroBooNE lies in the BNB and will probe a lower energy range, however we note that MINERvA's measurement signal described here is the same as MiniBooNE with the requirement of a $\mu^+$ rather than a $\mu^-$ in the final state. This is valuable for the purpose of comparison, development and testing of theoretical models \footnote{The MiniBooNE and MINERvA neutrino induced CC $\pi^0$ results were used by a group of theorists in 2015 to test the Adler-Nussinov-Paschos model \cite{bib:min_compare}}.

\par More recently, MINERvA produced a CC $\pi^0$ $\nu_{\mu}$-induced differential cross section measurement measurement. Expand here when there is WIFI**.  

\par Lastly we mention K2K, which in 2011 produced an inclusive CC $\pi^0$ production cross section ratio with respect to the CCQE channel \cite{bib:k2k_paper}. As the MicroBooNE CCQE measurement is not mature enough yet for this comparison, we note this measurement here only for the sake of completeness.  
\par A number of different signal definitions are described above and it is prudent to carefully pick which one we will adopt before proceeding. This thought process is motivated by the hope that this measurement will not just be useful for us as experimentalists, but also for those interested in testing models across experimental results.  With this in mind, there are two things to note about the early ANL and BNL measurements which make them less desirable for direct comparison. For one, these experiments were performed on low-Z material. Because the nuclear density of the detector medium correlates strongly with the number of interactions, low-Z material for a neutrino experiment generally results in lower statistics. Additionally, the neutrino fluxes of these early experiments was not well understood, which led to high systematic errors. Looking later in time, higher-Z experiments at Fermilab (particularly in the BNB) benefited from MiniBooNE's careful and in-depth modeling of the Booster Neutrino Beamline flux \cite{bib:bnbflux}.
\par MiniBooNE's high statistics, $CH_2$ medium and utilization of the BNB make it an ideal candidate for signal adoption.  However, the MiniBooNE requirement of no meson in the final state requirement complicates the situation for MicroBooNE, whose particle ID (PID) algorithms are still under development.  Currently, requiring the final state to be meson-less artificially damages the purity of the signal measurement we obtain, while also damaging the efficiency. For this reason, we will adopt the signal definition used by SciBooNE for direct comparison and keep the MiniBooNE and MINERvA results in mind for future studies.  This will be the first CC $\pi^0$ cross section measurement on Argon.



%\newpage
%\section{Liquid Argon Time Projection Chambers (LArTPCs)}
%\begin{figure}[h!]
%\centering
%\includegraphics[scale=0.4]{Figures/tpc.png}
%\section{The MicroBooNE Experiment}
%\caption{The generalized process of readout in a TPC}
%\end{figure}
%
%\par Two of the primary benefits of using a TPC as your detector is that you get calorimetric information + image quality resolution as seen in Figure \ref{fig:argoneut}.  These event displays depict example events in which an electron, gamma were produced respectively. At first glance, a somewhat obvious distinction between the two events is the gap between the start of the shower and vertex in the display on the right.  This gap is associated with a gamma and is due to our inability to detect neutral particles directly. In other words, we do not see the gamma until it pair produces or Compton scatters in the detector, at which point we observe an electro-magnetic (EM) shower. In contrast to the birth of the gamma EM shower, the electron EM shower is seen as soon as the electron is born \footnote{That is, assuming it is above some threshold.}.  When an electron scatters off an Argon atom, it produces a Bremsstrahlung photon which then pair procuces, etc leading to an EM shower with no gap. This topological cut is a powerful tool in discriminating between the interactions that caused grief for MiniBooNE. 
%
%\begin{figure}[h!]
%\includegraphics[scale=0.4]{Figures/EVsignal.png}
%\hspace{2 mm}
%\includegraphics[scale=0.4]{Figures/EVgamma.png}
%\caption{Readout images from the Argoneut detector. The event on the left is a signal $\nu_e$ CCQE event, while the event on the right is a background event producing a gamma} 
%\label{fig:argoneut}
%\end{figure}
%\vspace{4 mm} \par There is one caveat to the story thus far, but it has a pleasant solution so fear not! Monte Carlo simulations show us that the photon can sometimes pair produce near enough to the vertex of interaction to appear gap-less. In addition, the signature of a neutral current (NC) $\nu_e$ event is a single shower with no vertex activity.  This casts doubt (or at least a decrease in event selection efficiency) on the signal sample selected by just the topology cut. But there is something we have not used yet.  An electron is a minimally ionizing particle (MIP) leaving $\sim 2 \frac{Mev}{cm}$ behind in its wake. When a gamma pair produces (accounts for $\sim 94\%$ of events above 150 MeV), it creates 2 MIPs.  Thus if we examine the first few cm ($\sim 2.4cm$) of the shower, we should see an energy deposition per cm ($\frac{dE}{dx}$) of 1 MIP for an electron shower and 2 MIPs for a gamma shower \cite{szelc}. This technique allows us to distinguish between gammas and electrons with high efficiency. A successful example of this discrimination technique is show for Argoneut data below in Figure \ref{fig:dedx}.
%\begin{figure}[h!]
%\centering
%\includegraphics[scale=.5]{Figures/dedx.png}
%\caption{ $\frac{dE}{dx}$ separation of electron vs $\gamma$ \cite{szelc}} 
%\label{fig:dedx}
%\end{figure}



\newpage

% I hope for this section to take 50 pages or so (20 hardware, 30 software)
%\section{The MicroBooNE Experiment}
\section{Hardware} \label{sec:hardware}
\subsection{Booster Neutrino Beamline}
The Booster Neutrino Beamline (BNB) is situated at Fermi National Laboratory in Batavia Illinois.  The BNB delivers a neutrino beam to the MicroBooNE detector 470 m from the target through a series of acceleration, focusing, collimating. The various stages of the beam are described here.  
\subsubsection{Proton Acceleration}
Neutrino beam creation begins with $H^-$ ion production and acceleration to 750 keV in a Cockcroft-Walton DC accelerator \cite{anderson}. The beam of H- ions are subsequently boosted to 400 MeV in a linear acceleration and stripped of their electrons before they enter the Booster \cite{bib:bnbflux}.  The Booster operates at a cycle frequency of 15 Hz, while each RF cavity operates at 84x this rate. As the protons move through the various RF cavities, they are bunched into 81 buckets 19 ns apart, where all buckets are 2 ns in width.  One collection of these 81 buckets is called a beam spill, lasts around 1.6 $\mu$s and typically consists of 1 - 5e12 Protons On Target (POT). Spills are directed into the Booster Neutrino Beamline by a switch magnet, and steered to the target by a series of dipole and focusing-defocusing (FODO) quadrupole magnets.  The configuration of this assembly is shown in Figure \ref{fig:beamline0}a. 
\subsubsection{Beam Target and Horn}
\par At this point, the bunched protons are impinged on a Beryllium Target. The target consists of 7 Be slugs, 10 cm long, 1 cm in diameter (Figure \ref{fig:beamline0}b), and is embedded at the front end of the Booster Horn. The slugs are suspended in an outer Be tube by three Be fins placed equidistant around the tube.  Air is circulated around the slugs and then passed across a heat exchanger before being recycled, to maintain the slugs at a constant temperature. This temperature is monitored at all times, and the proton beam is shut off if any abnormality in temperature or radioactivity is detected. The location of the beam is calibrated with a multi-wire chambers to ensure that it is interacting with the target itself and not surrounding material. The beam's intensity and position are monitored consistently.
\par The result of proton collision with the Be is a spray of other particles which then pass into the magnetic focusing horn. The horn is made from aluminum alloy and consists of an inner conductor along which current runs in, and an outer conductor along which it returns. This configuration of current flow induces a toroidal magnetic field between the conductors, which enables the removal of charged particles: positive in Neutrino Mode, negative in Anti-Neutrino Mode. During this process, the horn heats up and needs to be cooled.  Ports in the outer conductor allow nozzles to spray water onto the inner conductor, which then drains out the bottom of the horn.

\begin{figure}[h!]
\includegraphics[scale=0.3]{NewFigures/beamline0.png}
\hspace{3 mm}
\includegraphics[scale=0.6]{NewFigures/target0.png}
\caption{ a) Accelerator complex at Fermilab. Protons are accelerated to 8 GeV in the Booster before slamming into a Beryllium target; b) The target assembly used by MicroBooNE, with beam striking the slugs from the left. } 
\label{fig:beamline0}
\end{figure}

\begin{figure}[h!]
\centering
\includegraphics[scale=0.6]{NewFigures/beamline1.png}
\caption{ The Booster complex as seen from above ground; MicroBooNE is situated at 470 m from the target, with MiniBooNE nearby. }
\label{fig:beamline1}
\end{figure}

\subsubsection{Neutrino Beam}
\par The singularly charged beam now heads downstream to a collimator.  This 214 cm of concrete sits 214 cm from the front of the target and acts as a dump for particles from secondary interactions which will not contribute to the neutrino flux of interest. 
\par The remaining particles then enter the decay pipe.  This 2 m wide pipe is made of corrugated steel, filled with air and packed into dolomite. In the 45 m length of the decay pipe, $\pi^+$ from the initial proton-Be interaction will decay into $\mu^+$ and $\nu_\mu$.  A concrete and steel beam stop at the end of the decay pipe captures $\mu^+$'s, allowing only neutrinos to pass on.  At this point, the neutrinos travel 470 m through the ground until they reach the MicroBooNE detector (Figure \ref{fig:beamline1}).

\subsection{The MicroBooNE Detector}
The Micro Booster Neutrino Experiment (MicroBooNE) is a Liquid Argon Time Projection chamber which sits in the BNB at Fermi National Laboratory. A model and live representation of the full detector are shown in Figure \ref{fig:cryo2}. Its primary goal is to resolve the nature of MiniBooNE's observed low energy excess, in addition to making a suite of cross section measurements and performing R\&D on new technologies for the next generation of LArTPCs. The various subsystems are described in detail below.
\begin{figure}[H]
\centering
\includegraphics[scale=0.5]{Figures/cryo.png}
\hspace{4 mm}
\includegraphics[scale=0.5]{Figures/cryo2.png}
\caption{a) Model of the MicroBooNE cryostat with side transparent to expose the inner TPC; b) A picture of the MicroBooNE detector taken when the TPC was slid into the cryostat in December 2013.  MicroBooNE's cryostat is a type 304 stainless steel tube with inner diameter of 3.81 m, length of 12.2 m, and capacity of 170 tonnes. }
\label{fig:cryo2}
\end{figure}

\subsubsection{Cryogenics}
\par The stability of the detector medium has a large systematic impact on the data taken by MicroBooNE. Temperature, pressure and flow, for example, affect the drift velocity of electrons. The drift time and drift X coordinates only maintain a clear correlation when the drift velocity is well understood, and so it is crucial to control these quantities carefully. Temperature is consistently monitored at various locations in the detector by 12 Resistive Thermal Devices (RTDs), and a layer of insulating foam around the cryostat prevents Liquid Argon from boiling off due to large temperature differences in surrounding materials.
\par The LAr in the detector must also be kept very pure. The presence of too much $O_2$ attenuates ionization charge, while $N_2$ contamination quenches scintillation light. The MicroBooNE purification system consists of 2 condensers (1 used primarily for backup), 2 pumps and 2 filters. In the beginning, gaseous Argon leaves the detector and enters one of the condensers. Each condenser, designed to control the heat load on the detector, contains a liquid nitrogen coil over which the gaseous Argon passes before it's pumped through the filter in liquid form. The LAr is then pumped through 1 of 2 parallel filter skids, where each filter skid contains 2 filters. The first filter is a 4A molecular sieve tasked mainly with removing water from the Argon, while the second filter primarily removes $O_2$ and any remaining water. Finally, the Argon is passed through a stainless steel particulate filter of grate width 10 microns to prevent any materials from the piping entering the detector. 
\par Directly downstream of the filter is a 50 cm drift double-grid ion chamber which functions as a purity monitor: $Q_C$ charge is generated at the cathode, the charge drifts in an applied electric field, and $Q_A$ charge is detected at the anode. The ratio of these charges is known as the electron lifetime given by the formula below: 
%since this filter effectively removes water, it is placed after the first filter to avoid saturating the second filter with water contaminant and allowing the remaining $O_2$ to recirculate. 
\begin{equation}
  \frac{Q_A}{Q_C} = e^{-\frac{t}{\tau}} 
\end{equation}

\noindent  2 other monitors live in the detector at different depths and drift locations (one at the cathode, one at the anode). 
\par The full system is depicted in Figure \ref{fig:cryo0} and described in greater detail externally \cite{bib:purity} \cite{bib:purity2}. The cryostat design limits are expanded on and summarized in Figure \ref{fig:cryo1} \cite{bib:uboone_JINST}.

\begin{figure}[h!]
\centering
\includegraphics[scale=0.6]{NewFigures/cryo0.png}
\caption{ 3D model of the MicroBooNE's cryogenics system including pumps, filters, cooling system and monitors.  } 
\label{fig:cryo0}
\end{figure}
  

\begin{figure}[h!]
\centering
\includegraphics[scale=0.6]{NewFigures/cryoStandards0.png}
\caption{ MicroBooNE was designed with these described contaminant limits in mind. Parameter limits discussed in more detail in and summarized from detector paper \cite{bib:uboone_JINST} }
\label{fig:cryo1}
\end{figure}

%\begin{minipage}{\linewidth}
%\centering
%\captionof{table}{MicroBooNE was designed with these described contaminant limits in mind. Parameter limits discussed in more detail in and summarized from detector paper \cite{bib:uboone_JINST} } \label{tab:cryo1} 
% \begin{tabular}{| l | l | l |}
% \hline
% Parameter & Limit & Motivation \\ [0.1ex] 
% \hline \hline 
%Purity & $<$ 100 ppt $O_2$ & Particle ID across full 2.5m drift \\ 
%Purity & $<$ 2 ppm $N_2$ & Preserve Scintillation Light [ $\frac{g}{mol}$] \\
%Recirculation & 1 volume change per 2.5 days & Maintain purity \\
%Temperature gradient of LAr & $<$ 0.1 K & Maintain uniform drift velocity \\
%\hline
%\end{tabular}
%\end{minipage}


\subsubsection{Time Projection Chamber} 
\par Liquid Argon Time Projection Chambers (LArTPCs) are ideal detectors for neutrino oscillation experiments with short and long baselines.  There are a number of properties that make Liquid Argon (LAr) well suited for the TPC medium: it is cheap, easy to cool, and transparent to its own scintillation light (expanded on in Table \ref{tab:larProp}).  When a charged particle travels through a TPC in LAr, it leaves a trail of ionization electrons in its wake that are drifted in a uniform electric field to wire readout planes. Once charge is collected, signals from the wires are amplified and sent on to be stored and analyzed. The passing charge leaves induced pulses on the first 2 ``induction" planes, and collects onto the final ``collection" plane. Readout from the wires makes the Y and Z coordinates of an interaction accessible, while the drift time acts as the third X coordinate; scintillation signals seen by an array of PMTs also play an important role in making the drift direction accessible\footnote{It is important to note that the success of a LArTPC does not rely on PMTs to extract ``X"--this extra piece of information establishes $t_0$ in correspondence with the beam gate, while also tagging backgrounds events which occur in the beam window.}.  Thus, a LArTPC with 3 (or 2) planes gives us the capability to reconstruct fine-grained, 3 dimensional configurations of events in the detector. Each part of the TPC is described below, with more detailed descriptions available in the Technical Design Report \cite{bib:tdr}.

%\begin{figure}[h!]
%\centering
%\includegraphics[scale=0.8]{NewFigures/larProperties.png}
%\caption{ Properties of Liquid Argon }
%\label{fig:larProp }
%\end{figure}

\begin{minipage}{\linewidth}
\centering
\captionof{table}{General Properties of Liquid Argon \cite{bib:uboone_proposal} \label{tab:larProp}} 
 \begin{tabular}{| l | l | l |}
 \hline
 Property & Value \\ [0.1ex] \hline \hline 
 Atomic number & 18 \\ \hline
 Atomic weight & 39.95 [$\frac{g}{mol}$] \\ \hline
 Boiling point at 1 atm & 87.3 [K] \\ \hline
 Density & 1.39 [$\frac{g}{cm^3}$] \\ \hline
 Radiation length & 14.0 [cm] \\ \hline
Absorption length & 83.6 [cm] \\ \hline
 Moliere radius & 10.0 [cm] \\ \hline 
 Work function to ionizie Ar atom  & 23.6 [$\frac{eV}{pair}$] \\
\hline
\end{tabular}
\end{minipage}

%\begin{wrapfigure}{r}{0.5\textwidth}
%\begin{center}
%%\centering
%\documentclass{article}
\usepackage{color}
\usepackage{siunitx}
\usepackage{tabularx}

\begin{document}
\color{magenta}\begin{tabular}{| l | l |}
	\hline
	\multicolumn{2}{|c|}{\normalsize\color{black}\textbf{MicroBooNE Detector}} \\ \hline \hline
\color{black}Medium & \color{black} Liquid Argon \\ \hline
\color{black}Temperature & \color{black} 87.3 K 		\\ \hline
\color{black}Electric Field & \color{black} 500 V/cm 	\\ \hline
\color{black}Drift Velocity & \color{black} \SI{1.63}{\milli\meter/\micro\second}  \\ \hline
\color{black}Drift Time  & \color{black}1.63 ms \\ \hline %across 3.6m TPC    \\ \hline
\color{black}Light Collection & \color{black}32 8'' PMTs \\
  & \color{black} 4 light guide\\ 
  & \color{black}prototypes \\ \hline
\color{black}Readout & \color{black}8256 wires \\
 & \color{black} 3 planes \\
  & \color{black} 3 mm pitch \\ 
 \hline

\end{tabular}

\end{document}

%\end{center}
%\caption{Properties of MicroBooNE's Liquid Argon Time Projection Chamber}
%\end{wrapfigure}

\paragraph{Cathode} 
\par The cathode is composed of 9 type 304 stainless steel sheets which are 2.3 mm thick.  The sheets are welded together rather than screwed (Figure \ref{fig:cathodeShine}a) to mitigate any sharpness of features which could impact the electric field. The fully constructed cathode plane was tested with a laser setup for flatness and determined to be parallel to within 0.04 degrees of the anode planes \cite{bib:uboone_JINST} . The cathode is currently held at -70kV.  

\begin{figure}[h!]
\centering
\includegraphics[scale=0.4]{NewFigures/cathode.png}
\hspace{3 mm}
\includegraphics[scale=0.9]{NewFigures/tpc0.png}
\caption{a) Inside the MicroBooNE TPC, the polished cathode appears on the left; b) The fully constructed TPC from the outside }
\label{fig:cathodeShine}
\end{figure}

\paragraph{Field Cage}
\par The role of the field cage is to create a uniform electric field that steps down from the high voltage applied at the cathode to near 0 V at the anode.  This system is composed of 64 2.54 cm diameter stainless steel field cage tubes which are supported by G10 infrastructure. Most tubes have all 4 corners formed by 2 couplings and a 5.24 cm radius elbow. A number of precautions are taken to avoid rough surfaces and other potential sources of breakdown. The first tube has a wider diameter and serves an extra role as cathode frame and support; the second tube, is also unique in that it has a slightly wider radius. Additionally, the first several tubes near the cathode are welded rather than connected with screws.
\par MicroBooNE has also taken precautions in the case a breakdown does occur. The field cage tube connections are designed with several features to mitigate the damage caused in the event of a breakdown. The tubes are connected by a resistor-divider chain which enables a constant voltage step down of 1 kV (at current operating voltage) per tube across the 2.56 m width. The first 16 tubes have 499 M$\Omega$ Mettalux resistors rated at 48 kV; the remaining tubes are connected via 4 Slim-Mox 150 M$\Omega$ resistors rated to 10kV. Lower rated resistors are used further from the anode as the impact of a breakdown is attenuated when discharge occurs nearer to the anode. Finally, carefully tested surge protector varistors \cite{bib:surge} were applied between tubes 1 and 32.  Varistors are devices whose electrical resistances vary with applied voltage. These resistors exhibit high resistance under normal operation of the detector, thus influencing the circuit negligibly. In the case of breakdown, the varistors short and protect downstream electronics from damage.

\paragraph{Anode}
\par The anode plays host to 3 individual wire readout planes. The 2 induction planes (U and V) are at +60 and -60 degrees to the vertical and contain 2400 wires. The collection plane contains 3456 wires all perfectly vertical. The wires are attached to gold pins on wire carrier boards in groups of 16 for U and V and 32 for Y (Figure \ref{fig:wire_stuff}a) and wound to tension using a wire winding machine (Figure \ref{fig:wire_stuff}b). 

\begin{figure}[h!]
\centering
\includegraphics[scale=0.6]{NewFigures/wireBoard.png}
\hspace{3 mm} 
\includegraphics[scale=0.7]{NewFigures/wireWind.png}
\caption{ a) Complete wire carrier board for the Y plane; b) Wire winding machine used to apply consistent tension to all wires in each plane. }
\label{fig:wire_stuff}
\end{figure}

\par The wires themselves are 150 $\mu$m gold-plated copper.  This material shrinks when cooled from room temperature to 87.3 K, and thus must be carefully tension tested before secured into the stainless steel anode frame. 10 minute tension measurement were performed to check wire stability by applying 3 times the normal expected load value of 0.7 kg \cite{bib:uboone_JINST}.  This process was performed again when the boards were completely assembled to verify that the assembly process had no negative effects on the wires. A final tension test which used a laser and spectrum analysis software was performed once the carrier boards were installed in the anode frame. The completely installed wire readout boards are shown on the nearest face in Figure \ref{fig:cathodeShine}b.

\par A number of studies were performed to establish wire plane parameters. The planes themselves are held at different voltages with U at -110 V, V at 0 V, and Y at +230 V.  These values and ratios were established by simulations which sought to minimize the amount of charge collected on the U and V induction planes \cite{bib:uboone_proposal} \cite{bib:tdr}.  The tilt angle of the U and V plane wires were also established via simulations.  Those studies attempted to maximize the signal to noise ratio (which scales with $\frac{1}{cos\alpha}$) and minimize the number of necessary feedthroughs and channels, while maintaining the 3D tracking capability of planes at sufficiently differing angles.

\begin{wrapfigure}{r}{0.5\textwidth}
\centering
\includegraphics[scale=0.5]{NewFigures/hv.png}
\caption{ MicroBooNE HV feedthrough resting in the receptacle cup attached to the cathode. } 
\label{fig:hv_cup}
\end{wrapfigure}

\paragraph{High Voltage}
\par A negative high voltage (HV) is applied to the cathode using a Glassman LX150N12 power supply that lives outside the detector.  The output from the Glassman passes through a low pass filter to reduce ripple, and passes through a specifically designed port on the cryostat near the cathode.  The HV tube rests in a receptacle pot attached to the cathode as shown in Figure \ref{fig:hv_cup}.
\noindent Though tested up to 200kV successfully, the high voltage system is currently held at -70kV.

\subsubsection{Light Collection} \label{subsec:lightCollection}
\par The MicroBooNE light collection system enables us to infer geometric and timing interaction information from light produced in the TPC. MicroBooNE expects only 1 neutrino interaction for every 660 spills \cite{bib:first_nus} and is subject to a high cosmic rate of 5 kHz; thus it is crucial to be able to limit the amount of neutrino-less readout windows that we store. This rejection of background-only readout windows with little or no light in coincident with the beam gate is one of the primary roles of the light collection system \cite{bib:uboone_JINST}.
%Beam spill trigger signals from Fermilab Accelerator Division tell MicroBooNE when to open its 4.8 ms readout window and start recording data.

\paragraph{Photo Multiplier Tubes (PMTs)}

\par The MicroBooNE light collection system consists of 32 cryogenic Hamamatsu 5912-02MOD PMTs (Figure \ref{fig:pmt}a), each with an effective quantum efficiency of 15.3\% \cite{bib:ben_jones}. The PMTs live in the low bias +230 V field produced at the collection plane behind the anode planes, which are 86\% transparency to light.  The same PMT rack plays host to 4 R\&D light guide paddles (Figure \ref{fig:paddle}), which absorb and guide wavelength-shifted scintillation light to a PMT (or SiPM) at the end. The long flat geometry of the paddle prototypes allow larger surface area coverage with fewer design challenges than traditional bulky PMTs; these characteristics especially benefit larger detectors like DUNE and SBND currently under design.  

\begin{figure}[h!]
\centering
\includegraphics[scale=0.46]{NewFigures/single_pmt.png}
\hspace{3 mm}
\includegraphics[scale=0.9]{NewFigures/pmt1.png}
\caption{ a) A single PMT installed on the PMT rack in the MicroBooNE cryostat; b) Several installed PMT's with TPB coated plates  } 
\label{fig:pmt}
\end{figure}

\begin{wrapfigure}{r}{0.4\textwidth}
\centering
\includegraphics[scale=0.5]{NewFigures/paddle.png}
\caption{One of the 4 lightguide prototypes in the MicroBooNE PMT rack}
\label{fig:paddle}
\end{wrapfigure}




\paragraph{Scintillation}
\par Liquid Argon has a high scintillation yield of 24000 $\frac{photons}{MeV}$ at 273 $\frac{V}{cm}$.  Scintillation light has both prompt and late components and is emited at 128nm in Argon (red line in Figure \ref{fig:decay}a), outside the visible range of our PMTs (blue line in Figure \ref{fig:decay}a). Tertaphenyl butadiene (TPB) coated acrylic plates are placed in front of each PMT to wavelength shift the light out of the VUV and into the visible spectrum (green line in Figure \ref{fig:decay}a). 

\paragraph{Recombination and Exciton Luminescence}

\par Argon is ionized when charged particles pass by. Ideally, the resulting electrons will all drift to the readout planes; however, with many positively charged Argon ions floating around from ionization, it's unlikely that all electrons will make it. Recombination occurs when an ionized argon and ground state argon form a combined excited state with an electron called an excimer (Figure \ref{fig:decay}b), which then de-excites and emits a scintillation photon. Concretely, minimally ionized charge has a roughly 20\% chance of recombining with an Argon ion rather than drifting to the anode planes \cite{bib:sorel}; more highly ionizing particles are subject to greater recombination losses.This process suggests an anti-correlation between scintillation light and ionization charge.

\begin{figure}[h!]
\centering
\fbox{\includegraphics[scale=0.45]{NewFigures/tpb0.png}}
\hspace{1 mm}
\fbox{\includegraphics[scale=0.31]{NewFigures/lightExcitement}} %decayTime.png}}
\caption{ a) Spectra of VUV scintillation photons, visibility of PMT and wavelength shifted photons; b) Process of scintillation photon production for each mode  } 
\label{fig:decay}
\end{figure}


\par Recombination is not the only way to lose charge deposition information about an interaction. Roughly 17\% of the time a charged particle will excite rather than ionize the Argon \cite{bib:sorel}. From here, the excited Argon can join with a ground state Argon to form an excimer (Figure \ref{fig:decay}b, bottom). This excimer similarly de-excites via a scintillation photon at 128 nm.  A closer look at the shared excimer state of the 2 processes reveals that there are 2 ways for the electron spin state to couple to the argon (Figure \ref{fig:light}a), with 2 different decay times \cite{bib:lumin}. The result is a prompt peaks when the interaction occurs followed by a long tail of late light (Figure \ref{fig:light}b). We are generally most interested in the prompt light, and the tail can cause some problems for delayed coincidences (ex Michel electrons).

\begin{figure}[h!]
\centering
\fbox{\includegraphics[scale=0.28]{NewFigures/lightRatio.png}}
\hspace{2 mm}
\fbox{\includegraphics[scale=0.4]{NewFigures/decayTime.png}}
\caption{ a) Excimer decay rates for each state; b) PMT signal over time exhibits prompt light from singlet state of excimer de-excitation and late light from triplet. } 
\label{fig:light}
\end{figure}

\subsubsection{Electronics and Readout}
\paragraph{LArTPC}
The induction and collection plane analog response to drifted ionization charge is small, as mentioned in a previous section.  Thus, before the wire readout data can be stored and analyzed the pulses must be amplified, shaped and digitized.  To reduce noise, the boards are in the liquid argon and positioned near the ends of the wires \cite{bib:uboone_JINST}. The pre-amplification and shaping electronics live on ASIC chips on cold motherboards. There are both top and side type motherboards in the detector, where the top board connects 48 U, 48 V and 96 Y wires and the side board connects 96 U and 96 V; this design is necessary to account for the shorter U and V wires that connect along the height of the TPC. Once signal has traveled through the boards, cold cables carry the information through additional intermediate amplifiers and out feed throughs in the cryostat to 130 Analog Digital Converter (ADC) and Front End Modules (FEMs). Here, signals first pass through 12 bit ADC to be digitized at 16 MHz, where baseline for induction plane is set at half the range (2055 ADC) and baseline for collection plane is set lower (450 ADC). The sampling rate is reduced to 2 MHz on the FEM boards.
\par At this point, a decision must be made about whether or not to send information to DAQ for processing. In the case of a neutrino stream, this decision is made based on trigger information from accelerator division. If a neutrino spill has occurred, 4.8 ms readout from buffer is sent to the DAQ; else, these events are discarded. 
\paragraph{PMTs}
In the event of beam event trigger from accelerator division, PMT information must also be stored. PMT signals from the 32 PMTs are amplified and shaped with 60 ns rise time and copied into a low and high gain channel. These shaped waveforms then pass to the Front-End Modules (FEMs) where they are digitized at 64 MHz for both channels. A sampling rate of 64 MHz over a beam window of 23.4 $\mu$s means 1500 consecutive 15.625 ns time slices starting 4 $\mu$s before the BNB or NuMI beam gates (which are 1.6 and 10 $\mu$s wide, respectively). Storing information from such high sampling over the entire readout window of 4.8 ms would be prohibitive; thus, a threshold requirement is imposed on the digitized signal for events occurring outside the beam window. \\
\par As mentioned in section \ref{subsec:lightCollection}, it is prohibitive to store all 4.8 ms readout windows from every beam spill, as relatively few of these windows contain a neutrino. To handle this, an additional trigger stores only readout windows in which a required PE threshold in coincidence with 1.6 $\mu$s beam gate is met; this trigger will be discussed in more detail in Section \ref{sec:software}.

%The MicroBooNE LArTPC electronics readout system consists of both cold and warm systems. 

\subsection{Triggers and Data Streams}
\par Different triggers are employed by the MicroBooNE detector to examine various streams of data. These triggers are described briefly below and in more extensive detail in external documentation (REFERENCE).
\par A natural place to begin is with the hardware trigger.  When beam is on and producing beam spills, accelerator division sends a signal to MicroBooNE to start recording data. This signal is called the BNB trigger (or analogously the NuMI trigger, for the NuMI beamline).  This trigger causes a window to open in TPC readout that lasts for 23.4 $\mu$s, which includes the 1.6 $\mu$s spill, and a front and back readout porch.  Both beam trigger comes early, and veto all other triggers, with BNB taking priority to NuMI.  The beam trigger efficiency is 99.8\%. %DOCDB 5084 
\par As noted earlier, most beam spills do not include a neutrino interaction. To account for this, there is a requirement that the beam trigger be in coincidence with a second trigger that comes from the PMT system. This trigger is currently implemented in software rather than on the FEM boards.  A software based approach was chosen as it allows a higher level of complexity in the algorithms which decide to send a PMT trigger, in addition to guaranteeing that the algorithm used is the same for data and MC.  The software trigger works as follows: a comparison of ADC values at time tick t and tick t + S are compared.  When the difference exceeds a set value, a first window opens. A similar operation is performed a second time for a second window, and the difference between the original discriminator threshold and max ADC in the full window is recorded. The fraction of events kept for various effective PE thresholds (ADC / 20) are shown in Figure \ref{fig:swtrigger}.  The algorithms in place require 6.5 effective PE of optical activity in the 1.6 $\mu$s beam spill window for the event to be processed and stored by our DAQ; this results in a rejection of $\approx$ 97\% of events. A random sampling of beam events is stored without software trigger to study for bias and other potential effects of the algorithm; this trigger is known as BNB UNBIASED. 

\begin{figure}[h!]
\centering
\includegraphics[scale=0.75]{NewFigures/softwaretrigger.png}
\caption{ Fraction of events in BNB and NuMI which pass the software trigger for various thresholds. The threshold currently in use is 6.5 effective PE.  } 
\label{fig:swtrigger}
\end{figure}

\par There are a number of other triggers in addition to beam. The EXT trigger is the length of the BNB window (1.6 $\mu$s) and occurs orthogonally to it at a constant 0.1 Hz. This trigger has a lower efficiency of 85\%; this is expected, due to the veto power of beam triggers in coincidence with EXT.  This trigger is useful for studying pure background. EXT events are similarly passed through a software filter as there is no shortage of background events at MicroBooNE to study, and it would be prohibitive to store the all EXT readout. Similar to BNB trigger treatment, we also store EXT UNBIASED sample for later random comparison. Similar triggers exist for NuMI.
\par Readout from triggered events processed through DAQ is stored in a binary file. To open this file in ROOT file format \cite{bib:root}, we need DAQ software that knows how to open and close the original file.  The DAQ takes the output of all triggers (BNB, EXT, BNB UNBIASED, etc) and stores them in separate data streams. In the end we have N files (streams), where N is the number of triggers we began with (BNB stream, BNB UNBIASED stream, EXT stream, etc.). 

\clearpage

\newpage
\section{Software} \label{sec:software}
Beamline and detector simulations are intended to represent truth level estimations. These estimations serve as a baseline for comparison with collected data, and are colloquially referred to as Monte Carlo or MC.  For a beamline experiment of the size and complexity of MicroBooNE, there are a number of systems to model and details to account for to ensure the simulation reflects a near reality of the detector. 

%These simulations are performed in the LArSoft framework, a suite of tools available to LArTPC experiments. 
\subsection{Simulation}
Neutrino interaction simulation begins with the beamline.  The products of primary proton interactions with the Beryllium target slugs are modeled by the GENIE neutrino generator \cite{bib:genie}, and propagated along the target horn geometry and beamline using GEANT4 \cite{bib:geant4}.  The target/horn complex, beamline and MicroBooNE detector use a GDML geometrical description \cite{bib:gdml} to simulate the various materials and their locations that the neutrino beam will encounter as it travels. A detailed view of the MicroBooNE TPC, cryostat and detector hall are shown in Figures \ref{fig:gdml}a-d.
\par 

\begin{figure}[h!]
\centering
\includegraphics[scale=0.4]{NewFigures/gdml_tpc.png}
\hspace{2 mm}
\includegraphics[scale=0.4]{NewFigures/gdml_cryostat0.png}
\hspace{2 mm}
\includegraphics[scale=0.4]{NewFigures/gdml_hall0.png}
\hspace{2 mm}
\includegraphics[scale=0.55]{NewFigures/gdml_hall1.png}
\caption{GDML formatted geometry used for simulations of the MicroBooNE a) TPC; b) Cryostat and surround materials inside Liquid Argon Test Facility (LArTF); c) LArTF + surrounding world with block missing from side to allow view inside outer materials; d) LArTF + surrounding world with labels. }
\label{fig:gdml}
\end{figure}

\subsection{Reconstruction}
%Both TPC and PMT information are now available in a raw form to be used. This information on its own is probably not very useful beyond low level cutting.  In order to extract meaningful quantities for various analyses including this cross section measurement, we must now build more complex objects from our initial simulation (or data) output.

\par The current framework in place to handle simulation, reconstruction and analysis in LArTPCs is called LArSoft. LArSoft is used by several Liquid Argon experiments, including ArgoNeuT, MicroBooNE, SBND, DUNE and LAriAT. 
%LArG4
\par The LArSoft saga begins with LArG4. LArG4 (G4), a LArSoft module which interfaces with GEANT4, is the foundation for information propagation in MicroBooNE. To start, energy (in the form of charge and light) is deposited along discrete steps determined by particle interaction probabilities in GEANT4.  Electrons from this propagation are ``drifted" in groups of 600 to the wire planes\footnote{They are actually teleported via LArG4, rather than propagated. The number of electrons is calculated based on the energy deposition, and the number of electrons remaining at the wire planes are estimated using lifetime and recombination models.  The xyz positions are varied by a gaussian function and assigned to the nearest wire channel. }. 
%DetSim
\par Without agreement between data and MC, it is near impossible to draw strong conclusions about observations in data.  Thus it is crucial to model the contribution of detector effects in our MC reconstruction.  First we account for field response; electrons drifted to the anode experience different plane-dependent field responses due to different bias voltages (Figure \ref{fig:fieldResponse}). Additionally, we must account for the digitization and signal shaping that data will undergo, and model appropriately for the MC. Signal noise is acquired during both processes (as well as others), and is discussed extensively in external documentation \cite{bib:noise}.  Together, this process is called convolution, and can be represented according to Equation \ref{eq:conv}:
\begin{equation}
\label{eq:conv}
M(t_0) = \int_{t} R(t - t_0) S(t) dt
\end{equation}
, where M($t_0$) is the measured signal, R(t) is the combined field and electronics response, and S(t) is the real signal.  Convolution takes the simulation into a state which resembles data TPC waveforms.  
\par In order to now extract meaningful information from these convolved readouts, we deconvolve the signals.  Much like it sounds, deconvolution is intended to undo the detector effects we just modeled, and return the readout to a unipolar, near-Gaussian form. This is done to ease the burden on the next level reconstructions, and has the advantage of allowing planes to be treated uniformly beyond this point. The deconvolution begins with an application of a Fast Fourier Transform (FFT) to the measured signal in the time domain, M($t_0$). The result is a system representation in the frequency domain:
\begin{equation}
M(\omega) = R(\omega)S(\omega) 
\end{equation}

At this point, a frequency-based gaussian filter F($\omega$) is applied in the frequency domain to remove various noises mentioned above. Once filtered, S(t) is obtained by performing an anti-FFT on S($\omega$). 

%These models are implemented in the LArSoft framework DetSim module 

\begin{figure}[h!]
\centering
\includegraphics[scale=0.45]{NewFigures/fieldResponse.png}
\caption{Field response of each plane modeled by data-driven observation. }
\label{fig:fieldResponse}
\end{figure}


%GEANT4 takes a particle along discrete steps determined by particle interaction probabilities in Argon; energy in the form of charge and light is deposited at each step
\subsubsection{Optical Reconstruction} 
Optical reconstruction for the 32 PMTs occurs over several stages. First, output from the high and low gain streams are combined.  The high gain waveforms baseline at 2048 ADC and saturate at 4096 ADC.  With roughly 20 ADC/PE equivalent, this channel will saturate at roughly 100 PE.  In this case, low gain channel information is scaled and passed on with high gain information as waveform information. 
\par At this point, we search these raw waveforms for charge depositions.  This is done separately for cosmic and beam discriminator channels. In the case of cosmic discriminator, the first ADC sets the pedestal and is propagated through the pulse as the baseline for later area (PE) calculation. Beam discriminator calculates a local pedestal for each time tick, extrapolating through the signal region. Both discriminators use a threshold-based pulse finding algorithm; when a waveform pulse rises some number of sigma above the determined pedestal, the information is recorded and stored as an ``optical hit (ophit)".  The area of each ophit is calculated and stored with signal shaping included so as not to underestimate the PE content of the pulse.
\par To make the greatest use of the optical information, it is beneficial to combine ophits in coincidence with one another across PMTs.  The combination of these hits is called an ``optical flash (opflash)".  The current flash finding algorithm performs as follows. The 23.6 $\mu$s beam gate is divided into 0.1 $\mu$s coincidence windows. We then loop through all the optical hits we've reconstructed and search for hits with greater than 5 PE; the charge deposition and channel information of these hits are stored in the appropriate coincidence window (bin).  From here, we look to combine any late light tails with prompt light by comparing the relative charge in neighboring bins and combining bins i and i + 1 when bin i contains less charge, but more than 0.  In this final configuration, bins with greater than 10 PE are used to create a ``SimpleFlash" object.

\begin{figure}[h!]
\centering
\includegraphics[scale=0.42]{NewFigures/opticalSim_noLG.png}
\hspace{2 mm}
\includegraphics[scale=0.45]{NewFigures/opticalSim_LG.png}
\caption{a) PE distribution across all stages of optical reconstruction without low gain channel information for single simulation muon; b) PE distribution across all stages with both low and high gain for another simulated muon. The usage of both channels is essential for accurate representation of PE geometrical distributions. }
\label{fig:pedistrib}
\end{figure}

\subsubsection{TPC Reconstruction}
\paragraph{``Hit" Reconstruction}
Amplified and shaped readout from the TPC can be used to extract charge deposition information from the wire planes.  Once deconvolution is applied, a ``hit finder" looks for peaks in wire data and fits these peaks with gaussian curves.  The quality of the fit is stored in a parameter to be associated with the data product; this allows the user to later on neglect poorly reconstructed hits, such as those due to noisy wires.  Additional parameters such as the peak time and area under the gaussian fit (the hit's charge) are also stored.  

\par With hits reconstructed, we can now begin to look for patterns and form more complex objects. Particle topology can be track-like (e.g. muons, protons, charged pions) or shower-like (e.g. electrons and photons), as seen in Figure \ref{fig:trackshower}. The reconstruction of these 2 topologies has involved considerable effort and a variety of different approaches.  

\begin{figure}[h!]
\centering
\includegraphics[scale=0.4]{NewFigures/trackshower.png}
\caption{Candidate CC-$\pi^0$ event in data. This display gives a sense of the different track and shower topologies and necessary challenges to overcome to successfully reconstruct events.} 
\label{fig:trackshower}
\end{figure}

\paragraph{Track Reconstruction}
The track reconstruction used in this analysis is performed by the Pandora reconstruction package, and more extensive detail can be found externally \cite{bib:pandora}.  Pandora has reconstruction algorithms tuned separately to identify tracks likely to have resulted from cosmic rays, and those from neutrinos. Both processes begin similarly.  First, hits are grouped into clusters intended to preserve the purity of the cluster with respect to its parent particle, rather than the completeness.  As a result, these first clusters tend to be very small in the case of showers, and very long in the case of tracks.  From here, the 3 planes are scanned for 3 consistent, highly linear clusters (one cluster per plane) that overlap in the shared time coordinate.  Matching of these clusters proceeds as follows:  for every pair of 2 clusters in 2 planes that overlap in time, a predictive position for the points of the 3rd cluster in the 3rd plane is calculated. The results of all these calculations are then compared to the 3rd candidate cluster and a $\chi^2$ value is calculated to reflect the quality of the fit. From here, neutrino specific algorithms utilize a number of direction and proximity metrics to assess the likelihood that each reconstructed track originated from a neutrino interaction. 

\paragraph{Shower Reconstruction}
A full later subsection is dedicated to description of shower reconstruction due to the amount of work that has gone in to developing its various aspects.

\paragraph{Vertex Reconstruction}
3D vertex reconstruction also makes use of the Pandora reconstruction package. The vertex algorithms here use the matched 2D clusters described previously to look for kink positions and candidate vertex points. Only one point is chosen in the end per cluster pair, and is selected among the candidates based on considerations of the surrounding hit proximity and linearity. The result for a normal neutrino event in data will be many candidate neutrino vertex points; the final selection of candidate interaction vertex is described in future sections.

\paragraph{OpenCV Clustering}
OpenCV is an open source computer vision library with functions to aid in pattern recognition and image processing \cite{bib:opencv}. More on the framework and OpenCV tools developed previously are discussed in a previous technote \cite{bib:5856}. Here we describe new algorithms developed to improve the selected sample size obtained from Neutrino 2016 efforts\cite{bib:5864}.
\par A number of links in the OpenCV clustering chain have been updated since Neutrino 2016. First, we no longer use Pandora shower reconstruction to find neutrino events containing $\pi^0$. The vertex candidate associated with Selection II is used to build a simple ROI with constant 100cm bounds in every direction (Figure \ref{fig:roi}). 

\begin{figure}[h!]
\centering
\includegraphics[scale=0.35]{NewFigures/ROI.png}
\caption{New $\pi^0$ ROI around $\nu_\mu$ CC Selection II tagged pandora vertex. }
\label{fig:roi}
\end{figure}

\par Once the ROI has been built and track-like hits removed, clustering begins with a polar clustering algorithm (Figure \ref{fig:polar}) on the remaining hits. This algorithm operates by transforming image information into polar coordinates (with the origin set at the reconstructed vertex location), performing the same image manipulation described in \cite{bib:5856}, and transforming back. This strategy has the advantage over our previous simple image manipulations in that it enforces an image blur in the direction of showering. This prevents lateral over-merging in a number of events. From here we filter OpenCV defined clusters which pierce or lie outside of the previously defined ROI as they are likely not of interest to our reconstruction. 

\begin{figure}[h!]
\centering
\includegraphics[scale=0.6]{NewFigures/polar.png}
\caption{ Example from OpenCV online manual depicting polar transformation algorithm }
\label{fig:polar}
\end{figure}

\par On this reduced set of clusters, we can now run parameter finding algorithms to assign each remaining cluster a start point and direction. Start point calculation is performed in the following way: each OpenCV calculated contour has an associated minimum bounding box that surrounds it (Figure \ref{fig:flashlights}a). The start point finding module segments this bounding box into 2 segments long ways. The algorithm then locates the hit of the cluster that is closest to the ROI vertex.  The segment this hit belongs to is then chosen. Finally, within this chosen segment, we search for the hit furthest from the center of the minimum bounding box. In the adjacent segment, the hit furthest from the center is assigned to be the end point. We assign the cluster's direction to be the direction of the cluster's bounding box.
\par Now that our clusters have start points and directions we can attempt to combine charge which was not clustered together during polar clustering. We perform the merging on a reduced image, removing all clusters with less than 10 hits. This removal prevents very small clusters from being merged with larger clusters and skewing our matching results later on.  The algorithm we use to do the merging builds flash light shapes around each cluster and combines hits when the flashlights overlap (Figure \ref{fig:flashlights}). The start point of the cluster closest to the ROI vertex is used as the start point for the new cluster, while the end point is reassigned to the new cluster hit furthest from the start point.
\par Finally, we apply 4 simple filters to reduce the number of bad/uninteresting clusters passing on to matching. First, we remove clusters with fewer than 10 hits.  Next, we remove any clusters which are not aligned well with the vertex; this removes lingering cosmic rays and some mis-clustered hits. Occasionally when events are complicated (e.g. crossing muon, lot's of information, dead wires, etc.) the flashlight merging algorithm will over-merge.  To prevent extreme over-merges from continuing on to the next stage, we filter clusters whose outer contour contains the vertex. The final filter is intended to prevent matching in planes with many dead wires.  In order to avoid utilizing clusters reconstructed on a plane over a range which contains significant gaps due to ``dead'' or otherwise poorly-functioning wires, we assign a score to each plane's ROI. Dead wires closer to the vertex have a stronger weight than those wires near the ROI boundaries. This score varies event-by-event and describes the percentage of the ROI which is covered by dead or bad wires. This algorithm enforces that Y plane clusters will always pass on to matching; the clusters in the remaining U or V plane with the highest plane score are passed on to matching. 

\begin{figure}[h!]
\centering
\includegraphics[width=0.3\textwidth]{NewFigures/startPoint.png}
\hspace{3 mm}
\includegraphics[width=0.3\textwidth]{NewFigures/flashlights.png}
\caption{a) The start point finding algorithm uses the ROI vertex to determine the which segment the start point lies in   b) Depiction of flashlight merging algorithm. Flashlights (gray) are merged when they overlap. The trunk of the base flashlight is pinched at the start point to prevent over-merging near the vertex. A convex hull (red) is calculated over all final flashlights to form the new cluster boundaries. }
\label{fig:flashlights}
\end{figure}

\paragraph{Cluster Matching}
Before we can reconstruct showers and start looking for $\pi^0$'s, we first need to match cluster pairs across planes. We do this by noting that time is a shared coordinate across planes and assigning scores to cluster pairs based on their agreement in time. We quantify this score using a measure denoted as the \texttt{Intergral over Union}, or \texttt{IoU} for short. This quantity is defined as:
\begin{equation}
  {\rm IoU} = \frac{ \Delta t_1 \cap \Delta t_2  }{ \Delta t_1 \cup \Delta t_2 }
\end{equation}

With $\Delta t$ denoting the time-range associated to the hits in a given cluster.  Clusters which do not overlap are assigned a score of -1, while those that do are assigned a score between 0 and 1, with 1 being perfect overlap. At the end of the consideration of all match permutations, the highest scores are used to create matched pairs until no clusters or viable match pairs remain. We require that there be at minimum a 25\% agreement in time in order for a match to be made. We also require that one of the matched clusters come from the collection plane, as the collection plane is currently the plane used for calorimetry. 

\subsubsection{Shower Reconstruction}
Shower reconstruction uses 2D information created during the previous matching stage to create one 3D object. 
\paragraph{3D Direction}  We rely here on the reconstructed 3D interaction vertex to reconstruct the 2D projections. The 2D direction is computed as the charge-weighted average vector sum of the 2D distance from the vertex to each hit in the cluster.
\begin{equation}
  \hat{p}_{\rm 2D} = \sum_{i=0}^{N} \frac{ r_i - r_{\rm vtx} } { q_i }
\end{equation}
With N denoting the number of hits in the cluster, $r_i$ the position of the hit, $q_i$ its charge, and $r_{\rm vtx}$ the position of the projected vertex. Given two 2D weighted directions, the 3D direction is calculated using geometric relations between the planes and clusters. 

\paragraph{3D Start Point Reconstruction} We calculate the 3D start point by finding the 3D overlap position of the OpenCV reconstructed 2D start points of the matched pair of clusters. The time tick coordinates from each cluster are averaged to calculate a 3D shared time coordinate. The (Y,Z) coordinates are identified by the intersection between the wires associated with 2D start points.  Wires must intersect inside the TPC for a shower to be reconstructed. 

\begin{figure}[h!] %H]
\centering
\includegraphics[width=0.3\textwidth]{NewFigures/showers.png}
\caption{3D reconstructed showers are projected back into 2D as a visual sanity check that shower reconstruction is successful.}
\label{fig:showers}
\end{figure}

\subsection{Energy Reconstruction}
\label{sec:ereco}

\par The energy of each EM shower is reconstructed using a calorimetric energy measurement. This procedure is as follows: the integrated ADC charge measured for all the hits associated to an EM shower are converted, using a single, fixed constant value, to MeV accounting for the signal processing, electronics, and detector effects which transform deposited energy in the detector to digitized signals in our readout. For this work, only hits from the collection plane are used to reconstruct a shower's energy. The conversion from raw charge to MeV is calculated as follows:
\begin{itemize}
\item {\bf Electronics Gain}: A conversion from ADCs to number of electrons collected on a wire of 198 $e^-$ / ADC is applied. This value is obtained by accounting for the specifications of the MicroBooNE electronics. See ``Noise Characterization and Filtering in the MicroBooNE TPC''~\cite{bib:noise} for more details.
\item {\bf Lifetime Correction}: No lifetime correction is applied for data, given the exceptional Ar purity and high measured electron lifetime~\cite{bib:purity}. For MC, where an 8 ms lifetime is simulated, we correct the charge associated to each hit with an exponential correction given by $e^{t \,{\rm ms} / 8 \,{\rm ms} }$ where $t$ is the drift-time associated to a hit (which we know thanks to the fact that we are reconstructing beam-induced $\pi^0$s, generated at the trigger-time).
\item {\bf Argon Ionization}: The work function required to ionize an argon atom by a traversing charged particle is 23.6 eV, which we account for.
\item {\bf Ion Recombination}: The ionized charge which reaches the TPC wire-planes is a function of the deposited energy and the ion-recombination which quenches a fraction of the original ionization produced. Ion-recombination depends on the local density of positive and negative ions produced (and thus on the dE/dx of the particle), and on the strength of the local electric field. Electrons and photons have a smaller variation in dE/dx over the energy range of interest for MicroBooNE that for muons, pions, and protons. In addition, measuring the local dE/dx associated to an individual hit is challenging for EM showers, which consist of many branches of ionization propagating in different 3D directions. For these reasons, we apply a single, constant recombination correction of 0.38 obtained by assuming a fixed dE/dx of 2.3 MeV/cm and utilizing the Modified Box recombination model, as parametrized by the ArgoNeuT collaboration~\cite{bib:argoneut_recomb}, applied at MicroBooNE's electric field of 273 V/cm.
\end{itemize}
This gives us an energy calibration constant of:
\begin{equation}
  198 \frac{e^-}{\rm ADC} \times 23.6 \times 10^{-6} \frac{MeV}{e^-} \times \frac{1}{1-0.38} = 7.54 \times 10^{-2} \frac{\rm MeV}{\rm ADC}
\end{equation}


%\color{black}
%\subsection{MicroBooNE}

%MicroBooNE, the latest in a series of Booster Beam experiments located at Fermilab, is a Liquid Argon Time Projection Chamber (LArTPC) that will investigate the low energy neutrino excess seen by its predecessor, MiniBooNE. Cherenkov detectors, such as MiniBooNE, are limited by their inability to distinguish between single electrons and photons, a task LArTPCs are well suited for, as described in more depth above. With the high precision reconstruction capabilities of a LArTPC, MicroBooNE will be able to determine with high statistical certainty whether electrons or photons caused the anomalous MiniBooNE low energy excess. Of further interest to MicroBooNE, and the further proposal of this prospectus, are various neutrino-nucleon interaction cross-sections. Cross sections have accounted for much of the uncertainty in recent results from a variety of neutrino experiments\cite{miniboone} and sensitive measurements by MicroBooNE have the potential to lead to improved nuclear models and rate predictions. Beyond MicroBooNE, LArTPCs will continue to play a notable role in oscillation physics. LAr1-ND will act as a baseline for improving systematic uncertainties in MicroBooNE and investigating the nature of the MiniBooNE excess, while also acting as a small-scale phase experiment for future, bigger LArTPCs such as and LBNF. MicroBooNE is currently getting ready to commission the detector.


\newpage
\section {Event Selection}
With reconstructed flash, track and shower objects described in the previous section, we are now in a position to run automated reconstruction and selection chains on MC simulation of beam + background cosmic events. The selection of CC $\pi^0$ signal events takes place over several steps described in the following sections.
\subsection{CC Inclusive Filter} The first step in our automated selection is to run the $\nu_\mu$ CC inclusive selection filter (``SelectionII") developed for and since Neutrino 2016 \cite{bib:numucc}.  This selection employs a number of cuts to minimize cosmics and other background and to select neutrino induced CC interactions in the FV of the detector. Additional extensive detail on the described cuts is provided in external documentation \cite{bib:numucc}.  
\noindent The first set of cuts considers flash objects.  First, we note that neutrino interactions occurring inside the TPC should deposit light information in coincidence with the beam gate. Following this, we first require each event to have a 50 PE reconstructed flash in coincidence with the hardware trigger from accelerator division, henceforth referred to as the 1.6 $\mu$s beam gate window \cite{bib:first_nus}. This PE threshold cut allows powerful rejection of late light due to cosmic, and other few-PE noise sources. We then take the largest flash that passes this cut and flash-match it to tracks in the TPC.  The flash matching employed here uses a simple charge-weighted-in-Z calculation on both the TPC track deposition points and the PMT flash locations. For a pair to be matched, the difference in Z between the weighted positions much be < 70 cm. 
\par For the next series of cuts, we need to be able to consider units of interaction. We build interactions by considering the relationship between all reconstructed vertices and tracks in the TPC. The distance is now calculated between vertex and both track end points (recall that tracks are not reconstructed with directionality). If this distance is less than 3 cm, an association between the track and vertex is stored, and track directionality is corrected as necessary. We now have a number of candidate interactions with varying track multiplicities.  Only candidates with multiplicity $\geq$ 1 are considered for the remainder of the filter; each multiplicity is treated separately with different sets of cuts.
\paragraph{Multiplicity $>$ 1}
\par First, all interactions with greater than 1 associated tracks are considered. Our cuts require the longest track to have a maximum y directional component of $<$ 0.9.  Additionally, the longest track's y component must be $<$ 0.6 or the second longest track must be $>$ 30 cm.
\paragraph{Multiplicity = 1}
\par Interactions with multiplicity 1 must have a contained track of at least 40 cm within the FV. The absolute value of the y directional component of the track is also required to be $<$ 0.7.  This eliminates most through going muons which are mis-reconstruced on entrance into the TPC or pass in over dead wires, and pass the FV cut. Two additional parameters are currently considered.  The ratio of energy deposition near beginning and end of track must be $>$ 1.5 or if it is greater, the projected track length in the Y direction must be less than 25 cm.
\paragraph{Multiplicity = 2}
\par Cosmic tracks passing through the TPC will sometimes stop and decay into Michel electron. This decay is characterized by a kink point in the track and an increase in dE/dx of the muon right before Michel production (Figure \cite{fig:michel}).  Interactions with 2 tracks are treated separately from other multiplicities to mitigate this Michel background. The cut employed checks that 1) the dE/dx of the longer track is greater at the end than at the start, the start dedx is greater than 2.5 MeV/cm and the end dE/dx is $<$ 4.0 MeV/cm OR that the longer track's end is greater than 20 cm from the edge AND 2) the shorter track length is less than 30 cm.  Extensive detail on Michel reconstruction can be found in an external note \cite{bib:michel}. 
\par All selected interactions must have at least one track that is greater than 15 cm.
%SelectionII searches for events which have a track multiplicity of $\geq$ 1, a $\mu$ candidate track which is either contained or un-contained, a flash of 50 PE in the beam window, an agreement in Z of the reconstructed flash and candidate muon. These criteria and several calorimetric and topological cuts are described in more detail in the $\nu_\mu$ CC technote referenced above. The results of the filter are events with a $\mu$ candidate and an association between the reconstructed candidate track and pandora vertex. The most up to date filter (used here) currently has an efficiency of 42.3\% and purity of 70.4\%, and is described in DocDB 6172 \cite{bib:6172}. This SelectionII efficiency pertains specifically to CC $\nu_\mu$ inclusive events; to get a more precise idea of the efficiency of the filter on our signal of interest, we perform a short study on the MC BNB-only sample.  As mentioned above, 3496 signal events from full BNB sample have a true vertex in the fiducial volume (FV); after Selection II, we are left with 1434 signal events in the FV.  From these observations, we calculate our signal efficiency to be 41.0\%, and will use this number rather than 42.3\% in calculations for the rest of the note.

\begin{figure}[h!]
\centering
\includegraphics[scale=0.5]{NewFigures/Michel.png}
\caption{Event from cosmic data displaying candidate Michel electron. The warmer colors indicate higher ADC counts. } 
\label{fig:michel}
\end{figure}

%\par When a flash meets these conditions, the next step is to do a simple flash matching with TPC tracks to eliminate candidate tracks from our sample.  The flash object contains a charge weighted position in Y and Z, which is calculated.   SelectionII searches for events which have a track multiplicity of $\geq$ 1, a $\mu$ candidate track which is either contained or un-contained, a flash of 50 PE in the beam window, an agreement in Z of the reconstructed flash and candidate muon. These criteria and several calorimetric and topological cuts are described in more detail in the $\nu_\mu$ CC technote referenced above. The results of the filter are events with a $\mu$ candidate and an association between the reconstructed candidate track and pandora vertex. The most up to date filter (used here) currently has an efficiency of 42.3\% and purity of 70.4\%, and is described in DocDB 6172 \cite{bib:6172}. This SelectionII efficiency pertains specifically to CC $\nu_\mu$ inclusive events; to get a more precise idea of the efficiency of the filter on our signal of interest, we perform a short study on the MC BNB-only sample.  As mentioned above, 3496 signal events from full BNB sample have a true vertex in the fiducial volume (FV); after Selection II, we are left with 1434 signal events in the FV.  From these observations, we calculate our signal efficiency to be 41.0\%, and will use this number rather than 42.3\% in calculations for the rest of the note.

\subsection{Hit Removal}

At this point, we have a sample of events that are mostly CC $\nu_\mu$ induced. From here, we must narrow down the sample of CC events we've selected with SelectionII to a sample that also contains a single $\pi^0$, and reconstruct the showers associated with the particle. During past efforts, shower reconstruction-motivated clustering has been somewhat of a bottle neck at which we lose many events due to the complexity of reconstructing complicated topologies. In an attempt to mitigate this pitfall, we have chosen to add a hit removal stage. The goal of ``hit removal" is to locate induced charge which 1) originates at the vertex and 2) is associated with a shower-like object in the event. Hit removal is broken into 2 stages: cosmic-induced and neutrino-induced.  In cosmic-induced track removal, we remove clusters tagged as cosmic by the pandoraCosmic algorithms and charge which is poorly aligned with the neutrino vertex.  In neutrino-induced track removal, we remove charge induced by neutrinos that has high local linearity (eg, charge induced by track-like particles). After hit removal we are ideally left with only shower-like hits which seem to originate from the interaction vertex (Figure \ref{fig:hitremoval}). %More detail can be found in a previous technote \cite{bib:5864}. 

\begin{figure}[h!]
\centering
\fbox{\includegraphics[scale=0.27]{NewFigures/Before_HR.png}}
\hspace{1 mm}
\fbox{\includegraphics[scale=0.27]{NewFigures/After_HR.png}}
\caption{Vertex is depicted in cyan in both views. a) Before hit removal (left) and b) After hit removal (right) only shower-like hits originating from vertex remain. }
\label{fig:hitremoval}
\end{figure}

\subsection{N $\pi^0$ Filter}
We can now use the shower-like hits we've identified in our Selection II output as a handle to select $\pi^0$ events.  We approach this in the following way: first, we defined two sample sets on which to test the power of any filter we develop. The first set contains all final state $\pi^0$ events which originate from the Selection II candidate vertex (this includes both CC and NC), and the second contains everything else.  From here, we build circles of various radii around the reconstructed vertex on an event-by-event basis. Using this circle we calculate both 1) the amount of shower-like charge from our previous step that falls within the radius and 2) the amount of total charge that falls within the circle.  Ideally, the ratio of these 2 numbers will be higher for events that contain shower activity, as this information will be captured by the hit removal stage.  We do this for a variety of radii, as shown in Figure \ref{fig:all_radii}.  
\par We select a radius and ratio cut for our filter as follows:  we want to consider the hit ratio at a radius after which most showers will have converted and deposited energy.  As shown in Figure \ref{fig:all_radii}, lower radii up to roughly 35cm have notable number of $\pi^0$ entries in the 0 bin. Events in this bin have very few or no shower-like hits within the given radius, either do to longer conversion lengths or to hit track removal filtering more that it should be. To avoid losing events with longer conversion lengths and potentially creating a biased sample, we look to higher radii. We find that we maximize our efficiency * purity metric between 50 and 60 cm. We conservatively choose 60cm as our radius in order to maximize our metric and to maximize the range of radiation lengths that will make it through this filter.  At 60cm, we found that the product of efficiency and purity was maximized at a ratio cut of 0.24 which preserves 70\% of events in our sample with a primary $\pi^0$ with a purity of 36\%. The results of this study can be seen in Figure \ref{fig:separation}.

\begin{figure}[h!]
\centering
\includegraphics[scale=0.27]{NewFigures/AllRatios.png}
\caption{Ratio plots constructed by taking the ratio of shower-like charge to total charge at various radii. }
\label{fig:all_radii}
\end{figure}


\begin{figure}[H]
\centering
\fbox{\includegraphics[scale=0.27]{NewFigures/separation_withmuon_11117.png}}
%{pi0_Separation_v2.png}}
\hspace{1 mm}
\fbox{\includegraphics[scale=0.27]
{NewFigures/separation_withmuon_norm_11117.png}}
%{pi0_Separation_Norm_v2.png}}
\caption{$\pi^0$ ratio plots constructed by taking the ratio of shower-like charge to total charge in radius = 60cm a) Absolute scale with corresponding efficiency, purity and product; b) Same plot area normalized to give a sense of the distribution shapes. }
\label{fig:separation}
\end{figure}

In attempt to improve the separation power of the filter, a second smaller study was performed. This study neglected the charge information associated with the tagged muon from the total charge pool. We hoped that this choice would push the N-$\pi^0$ event ratios further to the right while leaving the 0-$\pi^0$ event ratios relatively unaffected.  The results of this study can be seen in Figure \ref{fig:separation_no_mu}.  We found that while we are able to push the N-$\pi^0$ sample further right, the 0-$\pi^0$ sample also spreads out and thus decreases our separation power. This is likely due to some mis-identification of shower hits in lower energy, shorter tracks.  If some hits are mis-identified as shower-like in 0-$\pi^0$ events, the resulting ratio without the tagged $\mu$ hits gives the false impression of shower activity. 
\par For the rest of the selection we will be using separation ratio value 0.24 obtained from the first study. A near-term future study will be done on MC in-time cosmics and triggered off-beam data events that pass SelectionII to ensure that this filter performs similarly enough between data and MC to use in a final analysis on data.

\begin{figure}[H]
\centering
\fbox{\includegraphics[scale=0.27]{NewFigures/pi0_separation_noMuon.png}}
\hspace{1 mm}
\fbox{\includegraphics[scale=0.27]{NewFigures/pi0_separation_noMuon_Norm.png}}
\caption{$\pi^0$ ratio plots constructed by taking the ratio of shower-like charge to total charge minus charge associated with the muon in radius = 60cm a) Absolute scale with corresponding efficiency, purity and product; b) Same plot area normalized to give sense distribution shapes. }
\label{fig:separation_no_mu}
\end{figure}

\subsection{$\pi^0$ Reconstruction}
\label{sec:pi0reco}
At this point we have some number of events that have made it through Selection II filter, hit removal, N-$\pi^0$ filter, OpenCV clustering and shower reconstruction. Now we can start checking events for $\pi^0$'s by examining properties of the reconstructed showers.  Currently a 2-shower pair needs to satisfy several criteria in order to be considered a $\pi^0$ candidate. First, the impact parameter of the 2 showers must be $\leq$ 4cm.  Additionally, the 3D opening angle must be $>$ 20 degrees; $\pi^0$ pairs with angles smaller than this tend to be cross-merged or overlapping. Finally, we require that the radiation length of the showers be $\leq$ 62cm with respect to the reconstructed vertex. If a pair of showers passes these criteria, they are considered to be a $\pi^0$ candidate. We do not currently handle the case where more than one viable candidate pair per event is found; these cases are simply neglected for now. We also do not include an energy or mass peak cut at this time; we hope to include these cuts in the future once we've understood our calibration scale better.

\subsection{Successfully Reconstructed Events}
The result of our reconstruction + selection chain is 585 CC $\pi^0$ candidate events. The breakdown of these events is described in the next section. Figure \ref{fig:ex2} shows a few examples of candidates in our sample. As a sanity check, we calculate the mass value of each pair in Figure \ref{fig:mass}. 

\begin{figure}[h!]
\centering
\fbox{\includegraphics[scale=0.3]{NewFigures/ex0.png}}
\hspace{1 mm}
\fbox{\includegraphics[scale=0.41]{NewFigures/ex3.png}}
\label{fig:ex0}
\end{figure}
\begin{figure}[h!]
\centering
\fbox{\includegraphics[scale=0.33]{NewFigures/ex1.png}}
\hspace{1 mm}
\fbox{\includegraphics[scale=0.5]{NewFigures/ex2.png}}
\hspace{1 mm}
\fbox{\includegraphics[scale=0.39]{NewFigures/ex4.png}}
\label{fig:ex2}
\end{figure}
\begin{figure}[h!]
\centering
\fbox{\includegraphics[scale=0.25]{NewFigures/ex5.png}}
\hspace{1 mm}
\fbox{\includegraphics[scale=0.25]{NewFigures/ex6.png}}
\caption{True CC$\pi^0$'s selected and successfully reconstructed through the full chain in MCC7 BNB + cosmics sample. }
\label{fig:ex2}
\end{figure}


\begin{figure}[h!]
\centering
\fbox{\includegraphics[scale=0.4]{NewFigures/pi0Mass_585.png}}
\caption{Calculated mass peak for the 585 selected CC $\pi^0$ events. }
\label{fig:mass}
\end{figure}



\newpage
\section {Cross Section Calculation}

\begin{figure}[H]
\centering
\includegraphics[scale=0.55]{NewFigures/xsec.png}
\caption{ Diagramatic example of an interaction; the $N_I$ blue dots are the incident particles over some beam area A, the $N_T$ green dots are the targets and the $N_S$ red dots are the number of scatters.}  
\label{fig:Xsec}
\end{figure}

%The concept of a cross section is simple.
Before an experiment is designed and built, experimentalists need to understand the rates at which they can expect different interactions to occur. This process is important; we don't want to design a detector that cannot meet the required sensitivity specifications we hope for our results to attain because we are overwhelmed by background events. In other words, given a beam of $N_I$ particles incident on $N_T$ targets, we want to be able to predict how many scatters, or interactions, we can expect to occur in our detector (Figure \ref{fig:Xsec}). This probability of interaction is called a cross section (Eq \ref{eq:xsec}). 

\begin{equation}
\sigma = A * \frac{N_S}{N_T * N_I}
\label{eq:xsec}
\end{equation}

%The cross section of a single target particle can be thought of as the 2D area it takes up; If we extend that idea of of probabilty of interaction for 1 incident particle to many, we can calculate the total probability of an interaction/or a scatter in our detector.

\par We can take this idea of predicting interaction rates further with the differential cross section.  Typically in an experiment we want to know not just the probability that an interaction will occur, but the probability that it will also result in particles with specific properties, most commonly momentum and direction. The differential probability for a particle to scatter into some volume in parameter space is given by Equation \ref{eq:diffxsec}:

\begin{equation}
\frac{d\sigma}{dX}(X',E_T,E_S) = \frac{\frac{dN_S}{dX}(X')}{N_T*N_I} * A 
\label{eq:diffxsec}
\end{equation}

, where $\frac{dN_S}{dX}$ is the number of scatters into some volume of parameter space dX, and $\frac{N_I}{A}$ is the beam flux, and more commonly written $\phi$. In its current state, Equation \ref{eq:diffxsec} suggests that the flux is independent of other variables, such as the energy of the incident beam, but this is generally not true. The equation can be rewritten to account for this as follows:

\begin{equation}
\frac{d\sigma}{dX}(X',E'_\nu) = \frac{1}{N_T*\frac{d\phi}{dE_\nu}(E'_\nu)}\frac{dN_S}{dXdE_\nu}(X',E'_\nu)
\label{eq:diffxsec_edep}
\end{equation}

There are occasions on which we won't have access to the initial neutrino energy, either because the neutrino interacted via the neutral current, or the initial energy can't be calculated based only on the information we see in our detector. In this scenario, we can still calculate the flux-averaged cross section and differential cross section over incident neutrino energy. These values reduce in this case to the following equations:

\begin{equation}
\sigma = \frac{N_S}{N_T * \phi}
\label{eq:xsec_fluxave}
\end{equation}

\begin{equation}
\Big \langle \frac{d\sigma}{dX} \Big \rangle_\phi(X') = \frac{1}{N_T\phi}\frac{dN_S}{dX}(X')
\label{eq:diffxsec_fluxave}
\end{equation}

We will use Equations \ref{eq:xsec_fluxave} and \ref{eq:diffxsec_fluxave} later to calcualte the cross section information of our channel of interest.

\subsection{True Cross Section}
Our first step is to calculate the true flux-averaged cross section on Argon. We do this using 200k events from MCC7 BNB only simulation.  The cross section can be calculated according to the following equation:

\begin{equation}
  \sigma = \frac{N_{tagged} - N_{bkgd}}{\epsilon*N_{targ}*\phi}
\end{equation}

\noindent where $N_{tagged}$, $N_{bkgd}$ are the number of tagged events and background events respectively, $\epsilon$ is the efficiency, $N_{targ}$ the number of targets and $\phi$ the flux. 
\par We use a 200k MC BNB only sample to perform this initial calculation.  Because we are using MC information to calculate a true value here, we use $\epsilon$ = 1, $N_{bkgd}$=0 and $N_{tagged}$=$N_{signal}$.  To calculate $N_{signal}$, we choose our volume of interest to be the Fiducial Volume (FV) used by SelectionII \cite{bib:numucc}, with 20cm from the wall in X and Y, and 10cm from the wall in Z. We find $N_{tagged}$ = 3496, for signal interaction vertices inside the FV. Note that while 200k events are simulated, more than half of these interactions occur outside the FV. 
\par Our next job is to calculate the number of targets in our FV:

\begin{equation} \label{eq:1}
  N_{targ} = \frac{\rho_{Ar} * V * Avogadro}{m_{mol}} 
\end{equation}
\noindent where $\rho_{Ar}$ is the density of Liquid Argon, V is the volume of interest, and $m_{mol}$ is the number of grams per mole of Argon.  Using the FV as our volume of interest, we find: 

\begin{align}
N_{targ} &= \frac{1.4 [\frac{g}{cm^3}] * 4.25e7 [cm^3] * 6.022e23 [\frac{molec}{mol}]}{39.95 [\frac{g}{mol}]} \\\\
&= 8.969e29\ molecular\ targets
\end{align}


\par Our final step is to calculate the integrated flux.  We do this by integrating over the $\nu_\mu$ flux histogram (Figure \ref{fig:flux}) provided by the Beam Working Group \cite{bib:flux}, and normalizing by the POT in our sample. The POT is calculated by integrating over the POT of all subruns under consideration; in this case, our POT is 2.42e20.  We calculate a total integrated flux of 1.20e11 $cm^{-2}$ over the range of 0.5 - 2.0 GeV (as was done by MiniBooNE) for $<E>$ = 982 MeV. 
\noindent Putting it all together we find:


\begin{align}
\sigma_{CC\pi^0} &= \frac{3496}{1.20e11 \frac{1}{cm^2} * 8.969e29 Ar } \\\\
&= (3.25 \pm 0.05) *10^{-38} \frac{cm^2}{Ar}
\end{align}

%\begin{wrapfigure}{r}{0.5\textwidth}
\begin{figure}[h!]
%\begin{center}
%\vspace{-50pt}
\centering
\includegraphics[scale=0.6]{NewFigures/BNBflux.png}
%\end{center}
\caption{$\nu_\mu$ Flux from Booster Neutrino Beam (BNB) at 470m }
\label{fig:flux}
\end{figure}
%\end{wrapfigure}

\begin{figure}[h!]
\centering
\includegraphics[scale=0.4]{NewFigures/GenieTruth.png}
\caption{Genie calculated CC $\pi^0$ cross section. The cross sections are calculated at various energies for both Carbon and Argon, and for signal definitions that include mesons in the final state and those that don't. The MiniBooNE measured cross section is displayed along with the MicroBooNE MC calculated cross section. }
\label{fig:genietruth}
\end{figure}

, where the error presented is purely statistical and dependent only on the number of signal events (for comparison with cross section measured on MC cosmics + BNB later).  This result is in comparison to the MiniBooNE result of $\sigma_{CC\pi^0}$ = (9.2 $\pm$ 0.3stat. $\pm$ 1.5syst.) * $10^{-39}$ $\frac{cm^2}{CH_2}$ at $<E_\nu>_\phi$ = 0.965 GeV \cite{bib:numucc_miniboone}.  Note that the MiniBooNE interaction medium is $CH_2$, in contrast to the Ar in MicroBooNE. The comparison to Genie model over a variety of energies is shown in Figure \ref{fig:genietruth}.

\subsection{Cross Section Calculation}

We regroup at this point by summarizing the state of our candidate sample pool at each stage described so far in Tables \ref{tab:eff1} and \ref{tab:eff2}. While our total number of reconstructed $\pi^0$ was 585 as described earlier, the final number of true CC $\pi^0$ events that we select is 254. The final efficiency is also calculated in this table to be 8.4\%.  

\begin{minipage}{\linewidth}
\centering
\captionof{table}{Efficiency losses at each stage of selection } \label{tab:eff1} 
 \begin{tabular}{| l | l | l | l |}
 \hline
 Step & Signal Events Remaining & Relative Efficiency & Total Efficiency \\ [0.5ex]
 \hline\hline

\hline
 % 120 events
  SelectionII & 1244 & 100\% & 41.0\% \\ 
\hline
  N-$\pi^0$ Filter & 872 & 70.1\% & 28.7 \% \\ \hline
  $\pi^0$ Reconstruction & 254 & 20.4\% & 8.4\% \\ \hline
   \end{tabular}
\end{minipage}

\begin{minipage}{\linewidth}
\centering
\captionof{table}{Breakdown of selected events} \label{tab:eff2} 
 \begin{tabular}{| l | l | l | l | l |}
 \hline
 Total Reco'd $\pi^0$ Events & Signal & Background & Total Efficiency & Purity \\ [0.5ex]
 \hline\hline
\hline
 % 120 events
  585 & 254 & 331 & 8.4\% & 43.4\% \\ 
\hline 
   \end{tabular}
\end{minipage}

\begin{minipage}{\linewidth}
\centering
\captionof{table}{Breakdown of background events} \label{tab:bkgd} 
 \begin{tabular}{| l | l |}
 \hline
 Background & Percent of Sample \\ [0.5ex]
 \hline\hline
\hline
 Clusters unrelated to $\pi^0$ reconstructed as $\pi^0$ & 23\% \\ \hline
  CC $\pi^0$ with mesons in final states & 22\% \\ \hline
  NC $\pi^0$ & 21\% \\ \hline
Mis-reconstructed vertex & 11\% \\ 
 \hline
 Multiple $\pi^0$ & 10\% \\ \hline
 Bad reconstruction (particles merged together) & 4\% \\ \hline
Secondary $\pi^0$ & 4\% \\ \hline 
N $\gamma$ event & 4\% \\ \hline 
Cosmic induced $\pi^0$ & $<$1\% \\ \hline 
 
   \end{tabular}
\end{minipage}
\\\\
\par A breakdown of our sample backgrounds is summarized in Table \ref{tab:bkgd}. A few comments on these categories : First, the most dominant background is made up of events in which the 2 clusters contributing to the final reco'd ``$\pi^0$" are not both gamma-looking clusters. The most common instance of this is a OpenCV clustered track originating from the vertex, and another track or shower-like particle clustered elsewhere in the ROI. A study on conversion length (near 0 for tracks) vs cluster linearity was done, however, no strong correlations were found to further limit the contribution of these events; these studies are not currently included here. It is worth noting that many of these short tracks are protons, and could potentially be removed with a dEdx cut once we fully understand calibration. Second, the signal definition excludes events with mesons in the final state, and no studies have yet been done to specifically limit this background. As a result, these events make up a large chunk of our background, however track multiplicity and track length studies will be explored in the near future. Next, the category ``bad reconstruction" refers specifically to events that contain 1 or more clusters that contain contributions from multiple particles (instances of over-merging). Many of these reconstructed events with mistakes contain at least one very low energy ``shower", and could  be removed with a modest energy cut (20-30 MeV).  There are also instances where the vertex is reconstructed in the wrong location (reco vertex $>$ 5 cm from the true vertex location). In this instance, there are sometimes secondary $\pi^0$'s reconstructed at the incorrectly reconstructed vertex, and other times mis-clustering, or clusters unrelated to the $\pi^0$ being reconstructed as $\pi^0$. Lastly, the secondary $\pi^0$'s that make it into this sample are those that are produced very near to the vertex, and so are not removed by our cluster alignment filter. 
With this information, we are now geared to calculate the CC $\pi^0$ cross section on MC cosmics + BNB using Equation \ref{eq:1}. Noting that the POT for this sample was 1.9e20, we find:

\begin{align}
\sigma_{CC\pi^0} &= \frac{585 - 331}{0.084 * 9.438e10 \frac{1}{cm^2} * 8.969e29 Ar} \\\\
&= (3.57 \pm 0.34) *10^{-38} \frac{cm^2}{Ar}
\end{align}

The error shown here is purely statistical and calculated using only $N_{tagged}$ and $N_{bkgd}$. This comparison is within 1$\sigma$ of the MC cross section, calculated to be (3.25 $\pm$ 0.05) * $10^{-38} \frac{cm^2}{Ar}$.

\clearpage
\section{Systematic Errors}
The precision and sensitivity of an experimental measurement depends exactly on how well we understand the contributing models and detector limitations. In the case of MicroBooNE, modeling of the beam flux, nominal cross section uncertainties in the GENIE neutrino generator, cuts-based variation and detector systematics all affect the final measured cross section. The total error will then ideally be the combination of independent matrices corresponding to each systematic source, as shown below:
\begin{equation}
\label{eq:sys_error}
E^{syst} = E^{flux} + E^{cross\ section} + E^{detector} + E^{cuts}
\end{equation}

\noindent In this section, we explore the degree to which each of these sources contributes to the final uncertainty. 
\subsection{Error Propagation: Background} 
There is a simple prescription to follow when approaching systematic error evaluation. We must first identify contributing parameters and their corresponding degrees of uncertainty.  Then, we vary these parameters randomly across many iterations, each time recalculating the cross section.
\par This variation must be approached in one of two ways. If a variation in parameter assignment affects only the rates of event production, a reweighing scheme can be applied to the final distributions rather than re-doing the full simulation for each generated universe.  This re-weighting strategy can be applied to error sources such as beam flux and GENIE cross sections.  For example, if the CC-neutrino interaction rate is halved by a parameter adjustment to the underlying neutrino interaction models, we can simply apply a factor of one-half to our final calculations.  On the other hand, if parameter adjustments affect event topology, re-weighting will not be sufficient, and a full generation of detector MC must be performed. One example of such a parameter which must be handled with full simulation is space charge. Space charge refers to the presence of positively charged ions that are formed when the Argon is ionized.  These ions influence the recombination of ionization electrons from new interactions, and can cause distortions in the readout.

\par What we end up with after these procedures are followed is a covariance matrix which contains all contributing error parameters and their correlations.  This matrix is given by the following equation:
\begin{equation}
\label{eq:cov_matrix}
E_{ij} = \frac{1}{n} \sum_{m=1}^{n} [N_{CV}^i - N_{m}^i] \times [N_{CV}^j - N_{m}^j] 
\end{equation}

where $E_{ij}$ is the covariance between the ith and jth variable, $N_{CV}^i$ is the nominal value of entries in the ith bin, $N_{m}^i$ is the value of entries in the $m^{th}$ generated universe.  It can also be useful to consider the correlation matrix, which gives a measure of the error in distribution shape.  This matrix can be written as 

\begin{equation}
\label{eq:cov_matrix}
\rho_{ij} = \frac{E_{ij}}{\sqrt{E_{ii}} \sqrt{E_{jj}}}
\end{equation}

\noindent and indicates the tendency of bins i and j to increase or decrease together.

% plots to include: 
% - muon momentum, muon angle, pi0 momentum, pi0 opening angle, pi0 mass?

\subsection{Flux Systematics}
When considering error due to flux, we are primarily interested in variation of particle production at the target ($\pi^+$, $\pi^-$, $K^+$, $K^-$, $K^0$) and POT.  The error due to POT (roughly 2\%, measured in the beam hall) is considerably smaller than that due to the flux itself (and the statistics in use for the  


\subsection{Genie Cross Sections}
\subsection{Detector Effects}
\paragraph{Electron Lifetime}
\paragraph{Electronics Effects}
\paragraph{Space Charge}
\paragraph{Cut-Based Variation}
The measured cross section is subject to variation based on choice of cuts values. In this section we apply a Gaussian smearing to each cut parameter used in the selection to test the final effects on the final calculation.


%\section {OTHER STUFF}
%\begin{equation} \label{eq:invMass}
% M^2_{\pi^0} = 4 E_{\gamma_1} E_{\gamma_2} sin^2\frac{\theta}{2}
% \end{equation}

 % where $N_{obs}$ is number of interactions selected from the reconstruction described above, $\Phi$ is the flux of the interaction of interest and $\epsilon$ is the efficiency with which events are selected.  The flux of the BNB has been well characterized by the MiniBooNE collaboration \cite{miniboone}. If our reconstruction were perfect, and our sample efficiency and purity were both 1, then we would divide our event rate by total flux and we would be done. But our samples will not be pure, nor will they be selected with 100\% efficiency--background events will inevitably leak in, and we lose signal to our series of cuts. Detector effects may also lead to "smearing" of kinematic variables which must be in some sense "unsmeared" before we work with them.  This means the real lifting comes in estimating our efficiency, and correcting for it. There are a few aspects of estimating efficiency of event selection.  The first is that efficiency is energy dependent.  One way to overlook this issue is to calculate a differential cross section of our channel of interest on Ar with respect to energy.  Another issue is that estimating the efficiency of our automated reconstruction is tricky.  With a limited sample of un-blinded data, we won't have high statistics and will possibly need to rely on MC truth estimates to produce this number.     

%\begin{equation} \label{eq:incoh}
%\nu_\mu N \rightarrow \mu^- \pi^0 N
%\end{equation}
%
%\begin{figure}[h!]
%\centering
%\includegraphics[scale=0.4]{Figures/EVpi0cand.png}
%\caption{NC $\pi^0$ candidate found in Argoneut data}
%\end{figure}

\newpage
\begin{thebibliography}{20} %1 should be replaced with number of citations
\singlespacing

\bibitem{bib:pauli}
W. Pauli, Phys. Today 31N9 (1978) 27

\bibitem{bib:fermi}
E. Fermi, Z. Phys. 88, 161 (1934)

\bibitem{bib:cowan}
  C. L. Cowan, Jr., F. Reines, et al., \emph{Detection of the Free Neutrino: a Confirmation}, \\
  \texttt{http://science.sciencemag.org/content/sci/124/3212/103.full.pdf}

\bibitem{ray0}
Davis, Raymond, and Don S. Harmer. ``Solar Neutrinos." Solar Neutrinos. (1964): n. page. Print.

\bibitem{rayreview}
Cleveland, Bruce T., et al. ``Measurement of the Solar Electron Neutrino Flux with the Homestake Chlorine Detector." ApJ. 496.505 (1998): n. page. Print.

\bibitem{kam0}
Hirata, K. S., et al. ``Real-time, directional measurement of 8 solar neutrinos in the Kamiokande II detector." Phys. Rev. D. 44.2241 (1991): n. page. Print.

\bibitem{gal0}
Hampel, W., et al. ``GALLEX solar neutrino observations: Results for GALLEX IV." Phys. Lett. B. 447. (1999): 127-133. Print.

\bibitem{gal2}
Hampel, W., et al. ``Final results of the 51Cr neutrino source experiments in GALLEX." Phys. Lett.. 420.1-2 (1998): 114-126. Print.
\bibitem{sage}
Abdurashitov, J. N., et al. ``The Russian-American Gallium Experiment (SAGE) Cr Neutrino Source Measurement."  Phys. Rev. Lett. (1996): n. page. Print.
\bibitem{superk}
Fukuda, Y., et al. ``Evidence for Oscillation of Atmospheric Neutrinos."  Phys. Rev. Lett.. (1998): n. page. Print.
\bibitem{Clarke}
Clarke, Arthur C. The Songs of Distant Earth. Del Rey Books, 1986. Print.

\bibitem{sno}
McGregor, Gordon A. ``FIRST RESULTS FROM THE SUDBURY NEUTRINO OBSERVATORY." arXiv. (2002): n. page. Print.

\bibitem{minos}	
Adamson, P., et al. ``Measurement of Neutrino and Antineutrino Oscillations Using Beam and Atmospheric Data in MINOS."  Phys. Rev. Lett.. 110.251801 (2013): n. page. Print.

\bibitem{warwick}
Boyd, Steve. ``Neutrino Oscillations." PX435 Neutrino Physics. N.p.. Web. 19 Feb 2015. http://www2.warwick.ac.uk/fac/sci/physics/current/teach/modulei\_home/px435/lec\_oscillations.pdf.

\bibitem{miniboone}
MiniBooNE Collaboration. ``The Neutrino Flux prediction at MiniBooNE." Phys. Rev. D. (2008): n. page. Print.

\bibitem{lsnd}
Aguilar, A., et al. \emph{Evidence for Neutrino Oscillations from the Observation of Electron Anti-neutrinos in a Muon Anti-Neutrino Beam.} Phys. Rev. D. (2001): n. page. Print.

\bibitem{bib:donut}
DONUT Collaboration, Phys. Lett. B 504 (2001) 218 [hep-ex/0012035].

\bibitem{bib:cdf}
CDF Collaboration, \emph{Observation of Top Quark Production in p $\overline{p}$ Collusions with the Collider Detector at Fermilab},\\
  \texttt{http://journals.aps.org/prl/abstract/10.1103/PhysRevLett.74.2626}

\bibitem{bib:reactorGeneral}
Mention, G., et al. \emph{The Reactor Antineutrino Anomaly},\\
  \texttt{https://arxiv.org/pdf/1101.2755.pdf}


\bibitem{bib:improvedReactor}
Mueller, Th. A., et al. \emph{Improved Predictions of Reactor Antineutrino Spectra},\\
  \texttt{https://arxiv.org/pdf/1101.2663.pdf}

\bibitem{bib:bugey}
Declais, Y., et al. \emph{Study of reactor antineutrino interaction with proton at Bugey nuclear power plant},\\
  \texttt{ https://inspirehep.net/record/39502/}

\bibitem{bib:rovno}
Kuvshinnikov, L. A., et al. \emph{Precise measurement of the cross section for the reaction $\overline{\nu_e}$ + p $\rightarrow$ n + $e^+$ at a reactor of the Rovno nuclear power plant},\\
  \texttt{http://www.jetpletters.ac.ru/ps/1248/article\_18875.pdf}




\bibitem{bib:forces}
Strassler, Matt. \emph{Of Particular Significance},\\
  \texttt{https://profmattstrassler.com/articles-and-posts/particle-physics-basics/the-known-forces-of-nature/the-strength-of-the-known-forces}

\bibitem{bib:weak_current}
Halzen, F., Martin, Alan D. \emph{Quarks and Leptons: An Introductory Course in Modern Particle Physics},\\
  \texttt{http://www.phy.olemiss.edu/~hamed/Quarks\_and\_Leptons.pdf}

\bibitem{szelc}
Szlec, A. M. ``Recent Results from ArgoNeuT and Status of MicroBooNE." Proceedings from Neutrino 2014. (2014): n. page. Print.

\bibitem{anderson}
Anderson, Colin. ``Measurement of Muon Neutrino and AntiNeutrino Induced Single Neutral Pion Production Cross Section." Thesis. (2011): n. page. Print.

\bibitem{technote}
Caratelli, D., et al. ``Electron Neutrino Cosmogenic Background Mitigation in MicroBooNE." Technote 3978. (2014): n. page. Print. http://www-microboone.fnal.gov/at\_work/technotes.html.

\bibitem{k2k}
Nakayama, S., et al. ``Measurement of single $\pi^0$ production in neutral current neutrino interactions with water by a 1.3 GeV wide band muon neutrino beam." Phys. Lett. B. 619.255-262 (2004): n. page. Print.

\bibitem{sciboone}
Kurimoto, Y., et al ``Measurement of inclusive neutral current $\pi^0$ production on carbon in a few-GeV neutrino beam." Phys. Rev. D. 81. (2010): n. page. Print.

\bibitem{bib:bnbflux}
  Aguilar, A., et al., \emph{The Neutrino Flux at MiniBooNE},\\
    \texttt{https://arxiv.org/pdf/0806.1449v2.pdf}

\bibitem{bib:ANL1}
  S. B. Barish et al., Phys. Rev. D., 19, 2521 (1979).

\bibitem{bib:ANL2}
 G. M. Radecky et al., Phys. Rev. D., 25, 1161 (1982)
 
\bibitem{bib:BNL}
 T. Kitagaki et al., Phys. Rev. D., 34, 2554 (1986)
 
\bibitem{bib:HE_unknown1}
 D. Allasia et al., Nucl. Phys. B., 343, 285 (1990).
 \bibitem{bib:HE_unknown2}
 H. J. Grabosch et al., Zeit. Phys. C., 41, 527 (1989).

\bibitem{bib:HE_unknown2}
J. Catala-Perez, \emph{Measurement of neutrino induced charged current neutral pion production cross section at SciBooNE.}
https://inspirehep.net/record/1280829?ln=en

\bibitem{bib:numucc_miniboone}
  A. A. Aguilar-Arevalo, et al., \emph{Measurement of $\nu_\mu$-induced charged-current neutral pion production cross sections on mineral oil at $E_\nu$ = 0.5-2.0 GeV},\\
  \texttt{https://arxiv.org/pdf/1010.3264v4.pdf}

\bibitem{bib:miniboone_thesis}
  Robert H. Nelson, Thesis, \emph{A Measurement of Neutrino-Induced Charged-Current Neutral Pion Production},\\
  \texttt{https://www-boone.fnal.gov/publications/Papers/rhn\_thesis.pdf}
  
\bibitem{bib:sciboone_thesis}
  Joan Catala Perez, Thesis, \emph{Measurement of neutrino induced charged current neutral pion production cross section at SciBooNE.},\\
  \texttt{http://lss.fnal.gov/archive/thesis/2000/fermilab-thesis-2014-03.pdf}  

\bibitem{bib:minerva_thesis}
  Jose Luis Palomino Gallo, Thesis, \emph{First Measurement of $\overline{\nu_\mu}$ of Induced Charged-Current $\pi^0$ Production Cross Sections on Polystyrene at $E_{\overline{\nu_\mu}}$ 2-10 GeV},\\
  \texttt{http://inspirehep.net/record/1247736/files/fermilab-thesis-2012-56.pdf}  
  
\bibitem{bib:minerva_paper}
   \emph{Single neutral pion production by charged current $\overline{\nu_\mu}$ interactions on hydrocarbon at $< E_\nu >$ =3.6 GeV},\\
  \texttt{http://www.sciencedirect.com/science/article/pii/S0370269315005493}  

\bibitem{bib:min_compare}
  J. Y. Yu, E. A. Paschos, and I. Schienbein \emph{Comparison of the Adler-Nussinov-Paschos model with the data for neutrino induced single pion production from the MiniBooNE and MINERvA experiments},\\
  \texttt{http://www.sciencedirect.com/science/article/pii/S0370269315005493}  
  
\bibitem{bib:k2k_paper}
  C. Mariani, et al., \emph{Measurement of inclusive $\pi^0$ production in the Charged-Current Interactions of Neutrinos in a 1.3-GeV wide band beam},\\
  \texttt{arXiv:1012.1794}

\bibitem{bib:uboone_JINST}
  The MicroBooNE Collaboration, \emph{Design and Construction of the MicroBooNE Detector},\\
  \texttt{https://arxiv.org/pdf/1612.05824v2.pdf}

\bibitem{bib:uboone_proposal}
  The MicroBooNE Collaboration, \emph{A Proposal for a New Experiment Using the Booster and NuMI Neutrino Beamlines: MicroBooNE},\\
  \texttt{http://www-microboone.fnal.gov/public/MicroBooNE\_10152007.pdf}

\bibitem{bib:first_nus}
  The MicroBooNE Collaboration, \emph{First Neutrino Interactions Observed with the MicroBooNE Liquid-Argon TPC Detector},\\
  \texttt{http://www-microboone.fnal.gov/publications/publicnotes/MICROBOONE-NOTE-1002-PUB.pdf}

\bibitem{bib:ben_jones}
  Benjamin J.P. Jones, Thesis, \emph{Steril Neutrinos in Cold Climates},\\
  \texttt{https://inspirehep.net/record/1389805/files/fermilab-thesis-2015-17.pdf}

\bibitem{bib:purity}
  The MicroBooNE Collaboration, \emph{Measurement of the Electronegative Contaminants and Drift Electron Lifetime in the MicroBooNE Experiment},\\
  \texttt{http://www-microboone.fnal.gov/publications/publicnotes/MICROBOONE-NOTE-1003-PUB.pdf}

\bibitem{bib:purity2}
  R. Acciarri, et al., \emph{Liquid Argon Dielectric Breakdown Studies with the MicroBooNE Purification System}, \\
  \texttt{https://arxiv.org/pdf/1408.0264v1.pdf}

\bibitem{bib:tdr}
  The MicroBooNE Collaboration, \emph{The MicroBooNE Technical Design Report}, \\
  \texttt{http://www-microboone.fnal.gov/publications/TDRCD3.pdf}

\bibitem{bib:surge}
  J. Asaadi, et al., \emph{Testing of High Voltage Surge Protection Devices for Use in Liquid Argon TPC Detectors}, \\
  \texttt{https://arxiv.org/pdf/1406.5216v2.pdf}

\bibitem{bib:sorel}
  M. Sorel, \emph{Expected performance of an ideal liquid argon neutrino detector with enhanced sensitivity to scintillation light}, \\
  \texttt{https://arxiv.org/pdf/1405.0848.pdf}

\bibitem{bib:lumin}
  M. Sorel, \emph{Argon, kryton, and xenon excimer luminescence: From the dilute gas to the condensed phase}, \\
  \texttt{http://aip.scitation.org/doi/pdf/10.1063/1.457108}

\bibitem{bib:root}
  Rene Brun, Fons Rademakers., \emph{ROOT Documentation}, \\
  \texttt{https://root.cern/documentation}

\bibitem{bib:genie}
  C. Andreopoulos et al., \emph{The GENIE Neutrino Monte Carlo Generator}, \\
  \texttt{http://aip.scitation.org/doi/pdf/10.1063/1.457108}

\bibitem{bib:gdml}
  R. Chytracek, J. McCormick, W. Pokorski, and G. Santin, \emph{Geometry Description Markup Language for Physics Simulation and Analysis Applications}, \\
  \texttt{http://ieeexplore.ieee.org/document/1710291}

\bibitem{bib:geant4}
  S. Agostinelli et al., \emph{Geant4--a simulation toolkit},\\
  \texttt{http://www.sciencedirect.com/science/article/pii/S0168900203013688}

\bibitem{bib:pandora}
  The MicroBooNE Collaboration, \emph{The Pandora multi-algorithm approach to automated pattern recognition in LArTPC detectors},\\
  \texttt{http://www-microboone.fnal.gov/publications/publicnotes/MICROBOONE-NOTE-1015-PUB.pdf}

\bibitem{bib:spaceCharge}
  The MicroBooNE Collaboration, \emph{Study of Space Charge Effects in MicroBooNE},\\
  \texttt{http://www-microboone.fnal.gov/publications/publicnotes/MICROBOONE-NOTE-1018-PUB.pdf}

\bibitem{bib:noise}
  The MicroBooNE Collaboration, \emph{Noise Characterization and Filtering in the MicroBooNE TPC},\\
  \texttt{http://www-microboone.fnal.gov/publications/publicnotes/MICROBOONE-NOTE-1016-PUB.pdf}

\bibitem{bib:michel}
  The MicroBooNE Collaboration, \emph{Michel Electron Reconstruction Using the MicroBooNE LArTPC Cosmic Data},\\
  \texttt{https://www-microboone.fnal.gov/publications/publicnotes/MICROBOONE-NOTE-1008-PUB.pdf}

\bibitem{bib:opencv}
  \emph{OpenCV Documentation},\\
  \texttt{http://opencv.org/documentation.html}


\end{thebibliography}



\end{document}
