\clearpage
\section{$\pi^0$ Selection} 

In this chapter, reconstructed shower information is used to identify events that contain a $\pi^0$ candidate. This is broken into two separate but complementary paths: one that considers all events with at least two reconstructed showers and another that considers all events with at least one reconstructed shower. There is a large overlap between these two final samples, but one is not a subset of the other. The two shower selection is discussed first.  

\subsection{Candidates With At Least Two Showers}
A breakdown of events with at least two reconstructed showers is considered in Figure \ref{fig:physics_pi0_nshrs}.  By requiring at least two showers, the remaining set after CC Inclusive selection is reduced by 95\%, and a highly signal-enriched sample is created. Pass rate and composition breakdowns after this at-least-two-showers requirement are shown in Tables \ref{tab:pi0_2showers_eventrates} and \ref{tab:pi0_2showers_composition}. 

\begin{figure}[H]
  \begin{subfigure}[t]{0.3\textwidth}
\includegraphics[scale=0.3]{Pi0_Cut_Section/Physics_sel2gt1shower_onoffseparate_nshrs.png}
  \caption{ }
  \end{subfigure} 
  \hspace{35mm}
  \begin{subfigure}[t]{0.3\textwidth}
    \includegraphics[scale=0.3]{Pi0_Cut_Section/Physics_sel2gt1shower_onoffseparate_nshrs_log.png}
  \caption{ }
  \end{subfigure} 
\caption{ Data to simulation comparison of number of reconstructed showers in a) linear and b) log scale. }
\label{fig:physics_pi0_nshrs}
\end{figure}

\begin{table}[H]
\centering
\captionof{table}{ Pass rate breakdown after two shower requirement scaled to OnBeam POT 
\label{tab:pi0_2showers_eventrates}}
 \begin{tabular}{| l | l | l |l|l|l|l|l|}
 \hline
 & CC1$\pi^0$ & CC0$\pi^0$ & NC$\pi^0$ & NC0$\pi^0$ & Other & Total \\ [0.1ex] \hline
CC Inclusive & 0.331 & 0.106 & 0.008 & 0.010 & 0.030 & 0.087 \\
$\geq$2 Showers & 0.106 & 0.001 & 0.002 & 0.0002 & 0.008 & 0.003 \\ \hline
\end{tabular}
\end{table}

\begin{table}[H]
\centering
\captionof{table}{ Sample composition after two shower requirement scaled to OnBeam POT 
\label{tab:pi0_2showers_composition}}
 \begin{tabular}{| l | l | l |l|l|l|l|l|}
 \hline
 & CC1$\pi^0$ & CC0$\pi^0$ & NC$\pi^0$ & NC0$\pi^0$ & Other &Cosmic+$\nu$& Cosmic\\ [0.1ex] \hline
CC Inclusive & 0.060 & 0.743 & 0.004 & 0.020 & 0.014 & 0.042 & 0.117 \\
$\geq$2 Showers & 0.624 & 0.179 & 0.037 & 0.010 & 0.120 & 0.016 & 0.014\\ \hline
\end{tabular}
\end{table}

Before using this sample to extract a cross section, two additional items are considered. First, because the remaining sample now contains both a candidate $\mu$ AND candidate showers, background breakdowns are reclassified to include shower information. Second, a series of quality cuts are applied to ensure that the selected sample maintains a low cosmic background and is well-reconstructed.

\subsubsection{Signal and Backgrounds for $\pi^0\rightarrow\geq 2 \gamma$}
The background breakdowns discussed throughout the note thus far were defined based on the truth-origin of the reconstructed candidate $\mu$ in conjunction with GENIE truth information. At this point, the candidate showers provide a richer profile of the reconstruction quality of each interaction. Because of this new information, all interactions in which a final candidate shower (after quality cuts) is of cosmic origin is sorted into a `cosmic' category. These new backgrounds are described below and used in all following data-MC comparison plots. However, for continuity with the previous discussion at the CC Inclusive stage, pass rate and composition tables will be presented using the backgrounds described at the beginning of the note. 

\paragraph{ Cosmics}
This category remains un-changed from the previously described background breakdowns.  It represents the OffBeam data contribution to the stack. 

\paragraph{ Cosmic + Neutrino}
Events previously tagged as ``Cosmic + Neutrino" enter this category automatically. Additionally, if either candidate shower is of cosmic origin, the event is added to this category. Two examples of this type of event are shown in Figure \ref{fig:backgrounds_cos}.

\begin{figure}[H]
\centering
\includegraphics[scale=0.5]{Pi0_Cut_Section/backgrounds_cosmic.png}
\caption{ Examples of Cosmic+Neutrino backgrounds at final stage; tagged track is shown in solid red, tagged vertex is shown in cyan, 3d-projected candidate $\pi^0$ showers are shown in blue and green triangles, track-like hits are shown in black, shower-like hits are shown in red. In both examples, the candidate track is neutrino-induced. However, in a) both showers are brem radiation from a cosmic, and in b) one shower is from a $\pi^0$, while the other is a delta ray brem photon.}
\label{fig:backgrounds_cos}
\end{figure}

\paragraph{ $\nu_\mu$ CC 1 $\pi^0$ } Signal classification is the same as before with one additional requirement: both reconstructed showers must be neutrino-induced.  All signals with cosmic-origin showers are reclassified as ``Cosmic + Neutrino" under this scheme.  Two examples of signal topology at this stage are shown in Figure \ref{fig:backgrounds_signal}.

\begin{figure}[H]
\centering
\includegraphics[scale=0.45]{Pi0_Cut_Section/signal.png}
\caption{ Examples of signal events at final stage; tagged track is shown in solid red, tagged vertex is shown in cyan, 3d-projected candidate $\pi^0$ showers are shown in blue and green triangles, track-like hits are shown in black, shower-like hits are shown in red.}
\label{fig:backgrounds_signal}
\end{figure}

\paragraph{$\nu_\mu$ CC Charge Exchange}
This background contains all instances of $\nu_\mu$-induced CC charge exchange outside of the initial neutrino interaction point in which both showers are of $\nu$-origin. An example of this interaction type in the final sample is shown in Figure \ref{fig:backgrounds_cc}a. 

\paragraph{$\nu_\mu$ CC Multiple $\pi^0$}
This background contains all instances of $\nu_\mu$-induced CC multiple $\pi^0$ production in which both candidate showers are of $\nu$-origin.  An example of this interaction type in the final sample is shown in Figure \ref{fig:backgrounds_cc}b.

\paragraph{$\nu_\mu$ NC $\pi^0$}
This background contains all instances of $\nu_\mu$-induced NC $\pi^0$ production in which both candidate showers are of $\nu$-origin. Note that NC multiple $\pi^0$ events are included in this category.  An example of this interaction type in the final sample is shown in Figure \ref{fig:backgrounds_cc}c. 

\begin{figure}[H]
\centering
\includegraphics[scale=0.93]{Pi0_Cut_Section/backgrounds_cc_v1.png}
\caption{ Examples of backgrounds in the final selected sample; tagged track is shown in solid red, tagged vertex is shown in cyan, 3d-projected candidate $\pi^0$ showers are shown in blue and green triangles, track-like hits are shown in black, shower-like hits are shown in red. a) CC charge exchange; b) CC Multiple $\pi^0$; c) NC $\pi^0$ }
\label{fig:backgrounds_cc}
\end{figure}

\paragraph{ $\nu_\mu$ Final State Electro-Magnetic Activity ($\nu_\mu$ FSEM)}
Remaining events are sorted into classifications of interactions with and without electromagnetic activity in the final state.  Final State EM events are $\nu_\mu$ events in which both reconstructed showers are of $\nu$-origin, but don't fit in to any of the previous categories. This set contains proton and neutron inelastic scatters that produce EM showers (Figures \ref{fig:backgrounds_nc}a,b), kaon decays (Figure \ref{fig:backgrounds_nc}d),  interactions with 1 or more photons in the final state such as $\eta$ decay (Figure \ref{fig:backgrounds_nc}e), $\nu_\mu$ CC 1$\pi^0$ out of the FV (Figure \ref{fig:backgrounds_nc}f).

\begin{figure}[H]
\centering
\includegraphics[scale=0.54]{Pi0_Cut_Section/backgrounds_fem.png}
\caption{ Examples of FSEM backgrounds selected in final stage; tagged track is shown in solid red, tagged vertex is shown in cyan, 3d-projected candidate $\pi^0$ showers are shown in blue and green triangles, track-like hits are shown in black, shower-like hits are shown in red. Interactions here are shown only for CC, however NC interactions also populate this category. }
\label{fig:backgrounds_nc}
\end{figure}


\paragraph{Other}
This background contains everything that is left over after all other backgrounds have been assigned.  This includes $\nu_e$'s (Figure \ref{fig:backgrounds_other}a), $\overline{\nu_\mu}$'s (Figure \ref{fig:backgrounds_other}b), and $\nu_\mu$ interactions in which tracks are mis-reconstructed as showers (Figure \ref{fig:backgrounds_other}c).

\begin{figure}[H]
\centering
\includegraphics[scale=0.6]{Pi0_Cut_Section/backgrounds_other_v2.png}
\caption{ Examples of various additional backgrounds selected in final stage; tagged track is shown in solid red, tagged vertex is shown in cyan, 3d-projected candidate $\pi^0$ showers are shown in blue and green triangles, track-like hits are shown in black, shower-like hits are shown in red. }
\label{fig:backgrounds_other}
\end{figure}


\subsubsection{Tuning Selection for Events with Two Showers}
A few additional checks are applied before calculating the cross section.  First, the impact parameter of every two shower permutation is considered.  Small impact parameters (IP) between most shower pairs are expected because the candidate vertex is used to reconstruct 3D direction. However, sometimes mis-matches are made, mis-reconstruction occurs, or clusters are under-merged resulting in an incorrect reconstructed direction. The effect of IP on shower angular resolution can be seen in Figure \ref{fig:ex_cutjust_IP}a. The effect seems small at first glance, but when propagated through to the reconstructed two shower mass peak, it becomes clear that poorer angular resolution of high IP events smears the mass plot towards lower energies (Figure \ref{fig:ex_cutjust_IP}b). Thus, a cut on impact parameter between showers is important to ensure both spatial correlation between showers, and higher quality angular resolution.  To be conservative, shower-to-shower impact parameter must be $\leq$ 4cm.  This cut is chosen to maximize the quality of angular resolution in our pool of selected events, to increase the eventual power of $\pi^0$ peak calibration in data. A data-MC comparison of IP is shown in Figure \ref{fig:cutjust_pi0_IP}a. 

%We do this for the backgrounds described in the previous paragraph, and additionally for a simplified breakdown.  In this additional background, the signal sample is defined the same.  The remaining non-signal events are broken into 2 kinds of backgrounds.  The first defined as events that are not signal, but have any number of $\pi^0$'s in the final state induced by the neutrino interaction. The second no neutrino-induced $\pi^0$'s. 
\begin{figure}[H]
  \begin{subfigure}[t]{0.4\textwidth}
	\centering
\includegraphics[scale=0.75]{Pi0_Cut_Section/ex_2shower_angular_res_IP.png}
  \caption{ }
  \end{subfigure} 
  \hspace{15mm}
  \begin{subfigure}[t]{0.4\textwidth}
	\centering
\includegraphics[scale=0.75]{Pi0_Cut_Section/ex_2shower_mass_peak.png}
  \caption{ }
  \end{subfigure} 
\caption{ a) Angular resolution of reconstructed showers with and without a conservative 4 cm cut on IP; b) The effect of the poorer angular resolution events propagates through to the two-shower mass peak, smearing it towards lower energies. }
\label{fig:ex_cutjust_IP}
\end{figure}


\begin{figure}[H]
%  \begin{subfigure}[t]{0.35\textwidth}
%\includegraphics[scale=0.35]{Pi0_Cut_Section/CutJustify_pi0v2_eff_gamma_IP.png}
%  \hspace{20mm}
 \begin{subfigure}[t]{0.35\textwidth}
\includegraphics[scale=0.35]{Pi0_Cut_Section/Physics_showerPostSel2_onoffseparate__gamma_IP.png}
  \caption{ }
  \end{subfigure} 
\hspace{10mm}
 \begin{subfigure}[t]{0.35\textwidth}
    \includegraphics[scale=0.35]{Pi0_Cut_Section/Physics_showerPostSel2_onoffseparate__gamma_oangle.png}
  \caption{ }
  \end{subfigure} 
\caption{ Impact parameter of reconstructed showers in events with two or more reconstructed showers for a) Unscaled MicroBooNE simulation; b) Data to simulation comparison }
\label{fig:cutjust_pi0_IP}
\end{figure}


\par Opening angle between reconstructed showers is also considered. Figure \ref{fig:mcvar_pi0_onlyoangle}a shows the true opening angle between showers of signal events from the full Simulation (black), and after CC Inclusive selection (red). This distribution suggests that few events will have a small true opening angle. Additionally, $\pi^0$ candidate pairs that have small reconstructed opening angles tend to be separately reconstructed showers from the same original true shower (Figure \ref{fig:mcvar_pi0_onlyoangle}b). This point is emphasized in Figure \ref{fig:comp_cutjust_pi0_OA} where completeness of leading and subleading shower candidate pairs as a function of opening angle is plotted. Low opening angle has a high correlation with low subleading shower completeness (right plot), and with lower-than-average leading shower completeness (left plot). Thus, a lower bound on opening angle of 0.35 rad (20 degrees) is chosen to maximize the quality of the event sample without significantly damaging the efficiency. A data-MC comparison is shown in Figure \ref{fig:cutjust_pi0_IP}b. % Finally, we consider conversion distances of both showers (Figures \ref{fig:cutjust_pi0_RL} \ref{fig:cutjust_pi0_low_RL}). We note the that there is very little separation power in this variable and also no reconstruction-related justification for a conversion distance cut. We thus do not employ a conversion distance cut at this time. We also do not include an energy or mass peak cut.


\begin{figure}[H]
\centering
  \begin{subfigure}[t]{0.35\textwidth}
    \centering
     \includegraphics[scale=0.35]{XSection_Calc_Section/MCVar_pi0_OnlyOangle.png}
     \caption{ }
  \end{subfigure} 
  \begin{subfigure}[t]{0.6\textwidth}
    \centering
    \includegraphics[scale=0.6]{Pi0_Cut_Section/ex_oangle_bad_signal.png}
    \caption{ }
  \end{subfigure} 
  \caption{ a) Opening angle distribution of signal CC single $\pi^0$ events in a 4.232e20 POT of MicroBooNE simulation; b) Example signal event where both reconstructed candidate showers originate from the same true shower. }
\label{fig:mcvar_pi0_onlyoangle}
\end{figure}

\begin{figure}[H]
\centering
  \begin{subfigure}[t]{0.35\textwidth}
    \centering
\includegraphics[scale=0.35]{Pi0_Cut_Section/comp_gamma_oangle_signal_Leading.png}
  \caption{ }
  \end{subfigure} 
  \hspace{20mm}
  \begin{subfigure}[t]{0.35\textwidth}
    \centering
    \includegraphics[scale=0.35]{Pi0_Cut_Section/comp_gamma_oangle_signal_Subleading.png}
  \caption{ }
  \end{subfigure} 
\caption{ Completeness of a) leading and b) subleading showers against opening angle. }
\label{fig:comp_cutjust_pi0_OA}
\end{figure}

Finally, the ``Cosmic + Neutrino" background contribution is considered. This contribution is 11\% of the sample after both IP and opening angle quality cuts are applied. Because the cosmic contribution is modeled via MC simulation, it is important to mitigate this background to avoid accruing a large uncertainty on the final cross section.  To begin, the raw reconstructed energy of both candidate showers for signal and Cosmic + Neutrino background is plotted; there is separation power between the two samples when both reconstructed candidate shower energies are $<$ 40 MeV, where a cut is imposed (Figure \ref{fig:cutjust_pi0_e}a).  Additionally, the conversion distance is relatively flat across the cosmic sample's candidate shower; thus, events where either the leading shower conversion distance is $>$ 80cm, or the subleading shower distance is $>$ 100cm (Figure \ref{fig:cutjust_pi0_e}b) are also cut. With these additional checks, the cosmic contribution to the two shower sample reduces to 6\%. This is a 50\% reduction is cosmic background, at the smaller cost of 6\% of signal. This reduction builds confidence at this stage that assessing a conservative 100\% uncertainty on this background will not overwhelm the final cross sectional uncertainties. A before and after of this cut is shown for shower conversion distance in Figure \ref{fig:physics_radl_before_after}a, b.


\begin{figure}[H]
\centering
  \begin{subfigure}[t]{0.4\textwidth}
    \centering
\includegraphics[scale=0.4]{Pi0_Cut_Section/CutJust_pi0_E.png}
  \caption{ }
  \end{subfigure} 
  \hspace{20mm}
  \begin{subfigure}[t]{0.4\textwidth}
    \centering
\includegraphics[scale=0.4]{Pi0_Cut_Section/CutJust_pi0_RL.png}
  \caption{ }
  \end{subfigure} 
\caption{ Here we attempt to mitigate the cosmic contribution to the final two shower sample. We do this by considering: a) Energy of subleading and leading candidate showers; b) Conversion distances of subleading and leading candidate showers. }
\label{fig:cutjust_pi0_e}
\end{figure}

\begin{figure}[H]
  \begin{subfigure}[t]{0.4\textwidth}
\includegraphics[scale=0.4]{Pi0_Cut_Section/Paper_pi0_onoffseparate_pi0_low_radL.png}
  \caption{ }
  \end{subfigure} 
  \hspace{10mm}
  \begin{subfigure}[t]{0.4\textwidth}
\includegraphics[scale=0.4]{Pi0_Cut_Section/Paper_pi0_onoffseparate_pi0_radL_w_cut.png}
  \caption{ }
  \end{subfigure} 
\caption{ Data to simulation comparisons of conversion distances for all candidate showers in the two shower sample a) before and b) after additional minimum energy and conversion distance cuts. }
\label{fig:physics_radl_before_after}
\end{figure}

\par If a pair of showers passes all the criteria described thus far, they are considered to be a $\pi^0$ candidate. If more than one candidate pair is found per event, the event is neglected.  Ten candidate events selected from data are shown in Figure \ref{fig:physics_pi0_onbeam_eventdisplays}.
\begin{figure}[h!]
\centering
\includegraphics[scale=0.54]{Pi0_Cut_Section/OnBeam_Data_pi0_0.png}
\includegraphics[scale=0.54]{Pi0_Cut_Section/OnBeam_Data_pi0_1.png}
\caption{ Examples of selected events from 4.92e19 BNB OnBeam Data in the collection plane. The vertical axis in all displays is time, while the horizontal axis is wires. All points on the display represent hits reconstructed, the red line is the 2D projection of the 3D track that is the CC Inclusive muon candidate, and the triangles are the 2D projections of the 3D reconstructed shower, the cyan circle is the CC Inclusive candidate vertex. }
\label{fig:physics_pi0_onbeam_eventdisplays}
\end{figure}

%%%%%%%%%%%%%%%%%%%%%%%
\clearpage
\subsubsection{Results for $\pi^0\rightarrow \geq 2 \gamma$}
\par A table with event rates scaled to OnBeam POT is shown for OnBeam, OffBeam, and Simulation in Table \ref{tab:2shpi0_event_rates}. Neutrino interaction production modes in the final simulation sample are shown in Figure \ref{fig:physics_2shower_inttype}. Note the data-MC agreement at the final stage of the selection, in addition to the negligible presence of MEC induced interactions. 

\begin{table}[H] 
 \centering
 \captionof{table}{Event counts at each stage of the full two shower CC $\pi^0$ selection with samples scaled to OnBeam POT. Note that all OffBeam events in the current sample are removed by the two-shower chain.  For now, an uncertainty for this sample is assessed on one event. \label{tab:2shpi0_event_rates}}
 \begin{tabular}{| l | l | l | l | l |}
  \hline
   & OnBeam & OffBeam & On - OffBeam & Simulation \\ [0.1ex] \hline
No Cuts & 544751 $\pm$ 738 & 462076 $\pm$ 1001 & 82675 $\pm$ 1244 & 48949 $\pm$ 76 \\ 
CC Inclusive & 3753 $\pm$ 61 & 564 $\pm$ 35 & 3189 $\pm$ 71 & 4268 $\pm$ 22 \\ 
2 Shower Cuts & 69 $\pm$ 8 & 0 $\pm$ 2 & 67 $\pm$ 9 & 74 $\pm$ 2  \\ \hline
\end{tabular}
 \end{table}



\begin{figure}[H]
\centering
\includegraphics[scale=0.5]{Pi0_Cut_Section/Misc_pi0_EventType_vs_NeutrinoMode_w_Numbers.png}
\caption{ Event type broken down by neutrino interaction mode; note the small contribution of MEC events to the final selected sample, in comparison with larger contributions at earlier stages. }
\label{fig:physics_2shower_inttype}
\end{figure}

\par A summary of the passing rates and composition for signal and all backgrounds are shown in Tables \ref{tab:2shpi0_passrates} and \ref{tab:2shpi0_purity}. The final two shower sample efficiency (5.6\%) and purity (67.1\%) are competitive with the results obtained by other experiments as shown in Table \ref{tab:history_ccpi0_results}. Signal and background distributions are shown for a variety of kinematic variables in Figures \ref{fig:physics_pi0_mu_len} - \ref{fig:physics_pi0signalonly_mass}. 

\begin{table}[H]
\centering
\captionof{table}{Evolution of passing rates through full analysis chain \label{tab:2shpi0_passrates}}
 \begin{tabular}{| l | l | l |l|l|l|l|}
 \hline
 & CC1$\pi^0$ & CC0$\pi^0$ & NC$\pi^0$ & NC0$\pi^0$ & Other & All \\ [0.1ex] \hline
No Cuts & - & - & - & - & - & -\\
CC Inclusive & 0.331 & 0.106 & 0.008 & 0.010 & 0.030 & 0.099 \\ 
2 Shower Cuts & 0.056 & 0.0003 & 0.001 & 0.0001 & 0.003 & 0.001 \\ \hline
\end{tabular}
\end{table}

\begin{table}[H]
\centering
\captionof{table}{Evolution of sample composition through full analysis chain \label{tab:2shpi0_purity}}
 \begin{tabular}{| l | l | l |l|l|l|l|l|}
 \hline
  & CC1$\pi^0$ & CC0$\pi^0$ & NC$\pi^0$ & NC0$\pi^0$ & Other& Cosmic+$\nu$& Cosmic \\ [0.1ex] \hline
No Cuts  & 0.018 &  0.695 & 0.046 & 0.194  & 0.047 & -&-\\
CC Inclusive & 0.060 & 0.743 & 0.004 & 0.020 & 0.014 & 0.042 & 0.117  \\ 
2 Shower Cuts  & 0.671 & 0.117 & 0.041 & 0.011 & 0.096 & 0.000 & 0.063 \\ \hline
\end{tabular}
\end{table}

\begin{figure}[H]
  \begin{subfigure}[t]{0.3\textwidth}
\includegraphics[scale=0.3]{Pi0_Cut_Section/Physics_pi0_onoffseparate_mult.png}
  \caption{ }
  \end{subfigure} 
  \hspace{34mm}
  \begin{subfigure}[t]{0.3\textwidth}
    \includegraphics[scale=0.3]{Pi0_Cut_Section/Physics_pi0_onoffseparate_mu_len.png}
  \caption{ }
  \end{subfigure} 
  \caption{ Data to simulation comparison of a) multiplicity and b) $\mu$ length after two shower filter }
\label{fig:physics_pi0_mu_len}
\end{figure}

\begin{figure}[H]
  \begin{subfigure}[t]{0.3\textwidth}
\includegraphics[scale=0.3]{Pi0_Cut_Section/Physics_pi0_onoffseparate_mu_phi.png}
  \caption{ }
  \end{subfigure} 
  \hspace{34mm}
  \begin{subfigure}[t]{0.3\textwidth}
\includegraphics[scale=0.3]{Pi0_Cut_Section/Physics_pi0_onoffseparate_mu_angle.png}
  \caption{ }
  \end{subfigure} 
\caption{ Data to simulation comparison of $\mu$ a) $\phi$ and b) $\theta$ after two shower filter }

\label{fig:physics_pi0_mu_phi}
\end{figure}

\begin{figure}[H]
  \begin{subfigure}[t]{0.3\textwidth}
\includegraphics[scale=0.3]{Pi0_Cut_Section/Physics_pi0_onoffseparate_mu_startx.png}
  \caption{ }
  \end{subfigure} 
  \hspace{34mm}
  \begin{subfigure}[t]{0.3\textwidth}
\includegraphics[scale=0.3]{Pi0_Cut_Section/Physics_pi0_onoffseparate_mu_endx.png}
  \caption{ }
  \end{subfigure} 
\caption{ Data to simulation comparison of $\mu$ a) start and b) end x after $\pi^0$ filter }
\label{fig:physics_pi0_mu_x}
\end{figure}

\begin{figure}[H]
  \begin{subfigure}[t]{0.3\textwidth}
\includegraphics[scale=0.3]{Pi0_Cut_Section/Physics_pi0_onoffseparate_mu_starty.png}
  \caption{ }
  \end{subfigure} 
  \hspace{34mm}
  \begin{subfigure}[t]{0.3\textwidth}
\includegraphics[scale=0.3]{Pi0_Cut_Section/Physics_pi0_onoffseparate_mu_endy.png}
  \caption{ }
  \end{subfigure} 
\caption{ Data to simulation comparison of $\mu$ a) start and b) end y after two shower filter }
\label{fig:physics_pi0_mu_y}
\end{figure}

\begin{figure}[H]
  \begin{subfigure}[t]{0.3\textwidth}
\includegraphics[scale=0.3]{Pi0_Cut_Section/Physics_pi0_onoffseparate_mu_startz.png}
  \caption{ }
  \end{subfigure} 
  \hspace{34mm}
  \begin{subfigure}[t]{0.3\textwidth}
\includegraphics[scale=0.3]{Pi0_Cut_Section/Physics_pi0_onoffseparate_mu_endz.png}
  \caption{ }
  \end{subfigure} 
\caption{ Data to simulation comparison of $\mu$ a) start and b) end z after two shower filter.  The dip in z around 700 cm is near a region of dead wires in the detector. }
\label{fig:physics_pi0_mu_z}
\end{figure}

\begin{figure}[H]
  \begin{subfigure}[t]{0.3\textwidth}
\includegraphics[scale=0.3]{Pi0_Cut_Section/Physics_pi0_onoffseparate_pi0_oangle.png} 
  \caption{ }
  \end{subfigure} 
  \hspace{34mm}
  \begin{subfigure}[t]{0.3\textwidth}
\includegraphics[scale=0.3]{Pi0_Cut_Section/Physics_pi0_onoffseparate_pi0_mom.png}
  \caption{ }
  \end{subfigure} 
\label{fig:physics_pi0_pi0_oangle}
\caption{ Data to simulation comparison of $\pi^0$ a) opening angle and b) momentum after two shower selection }
\end{figure}

\begin{figure}[H]
  \begin{subfigure}[t]{0.3\textwidth}
\includegraphics[scale=0.3]{Pi0_Cut_Section/Physics_pi0_onoffseparate_pi0_low_radL.png}
  \caption{ }
  \end{subfigure} 
  \hspace{34mm}
  \begin{subfigure}[t]{0.3\textwidth}
\includegraphics[scale=0.3]{Pi0_Cut_Section/Physics_pi0_onoffseparate_pi0_high_radL.png}
  \caption{ }
  \end{subfigure} 
\label{fig:physics_pi0_pi0_rl}
\caption{ Data to simulation comparison of a) low and b) high energy shower candidate conversion distances after two shower filter }
\end{figure}

\begin{figure}[H]
  \begin{subfigure}[t]{0.3\textwidth}
\includegraphics[scale=0.3]{Pi0_Cut_Section/Physics_pi0_onoffseparate_pi0_Easym.png}
  \caption{ }
  \end{subfigure} 
  \hspace{34mm}
  \begin{subfigure}[t]{0.3\textwidth}
\includegraphics[scale=0.3]{Pi0_Cut_Section/Physics_pi0_onoffseparate_pi0_IP.png}
  \caption{ }
  \end{subfigure} 
\label{fig:physics_pi0_pi0_IP}
\caption{ Data to simulation comparison of a) energy asymmetry of the selected showers. This is defined to be the difference between the higher and lower energy showers' reconstructed energies divided by the sum of these energies. b) impact parameter between showers.  }
\end{figure}

\begin{figure}[H]
  \begin{subfigure}[t]{0.3\textwidth}
\includegraphics[scale=0.3]{Pi0_Cut_Section/Physics_pi0_onoffseparate_pi0_low_shrE.png}
  \caption{ }
  \end{subfigure} 
  \hspace{34mm}
  \begin{subfigure}[t]{0.3\textwidth}
\includegraphics[scale=0.3]{Pi0_Cut_Section/Physics_pi0_onoffseparate_pi0_low_shrE_corr.png}
  \caption{ }
  \end{subfigure} 
\label{fig:physics_pi0_pi0_low_e}
\caption{  Data to simulation comparison of low energy showers from $\pi^0$ selection a) before and b) after energy correction.}
\end{figure}

\begin{figure}[H]
  \begin{subfigure}[t]{0.3\textwidth}
\includegraphics[scale=0.3]{Pi0_Cut_Section/Physics_pi0_onoffseparate_pi0_high_shrE.png}
  \caption{ }
  \end{subfigure} 
  \hspace{28mm}
  \begin{subfigure}[t]{0.3\textwidth}
\includegraphics[scale=0.3]{Pi0_Cut_Section/Physics_pi0_onoffseparate_pi0_high_shrE_corr.png}
  \caption{ }
  \end{subfigure} 
\label{fig:physics_pi0_pi0_e}
\caption{ Data to simulation comparison of high energy showers from $\pi^0$ selection a) before and b) after energy correction. }
\end{figure}


\begin{figure}[H]
  \begin{subfigure}[t]{0.3\textwidth}
\includegraphics[scale=0.3]{Pi0_Cut_Section/Physics_pi0_onoffseparate_pi0_mass.png}
 \caption{ }
 \end{subfigure} 
 \hspace{34mm}
  \begin{subfigure}[t]{0.3\textwidth}
\includegraphics[scale=0.3]{Pi0_Cut_Section/Physics_pi0_onoffseparate_pi0_mass_corr.png}
  \caption{ }
  \end{subfigure} 
\caption{ Data to simulation comparison of a) uncorrected and b) corrected mass peak after $\pi^0$ filter }
%\caption{ Data to simulation comparison of the raw reconstructed two-shower mass. }
\label{fig:physics_pi0_mass} 
\end{figure}


\par It is worth noting that the reconstructed $\pi^0$ mass peak is displaced from the expected 135 MeV in Figure \ref{fig:physics_pi0_mass}a. The sources of this bias were discussed in the previous chapter. When those effects are combined, a bias of 25\% on the energy is expected, along with a mass peak around 100 MeV, which is the case in Figure \ref{fig:physics_pi0_mass}a. The corrected peaks sits at the expected 135 MeV. 
\par As noted earlier, the energy resolution is affected non-uniformly by each source of energy loss.  Thus, the fact that the `corrected' peak sits at the expected $\pi^0$ location is not necessarily sufficient at this stage to indicate that the final sample contains $\pi^0$'s. This naturally raises the question: how do we know these reconstructed showers are both photons and originating from a $\pi^0$ decay? The first element of this question is addressed by considering the conversion distance in Figure \ref{fig:physics_pi0signalonly_mass}a. A fit to the data returns a conversion distance of 24 $\pm$ 12 cm, which is in agreement with the expected value of 25 cm for photons at MicroBooNE energies. This suggests that these candidate showers are in fact photons. To confirm these photons are originating from $\pi^0$'s, the uncorrected two shower mass peak is considered (Figure \ref{fig:physics_pi0signalonly_mass}b). To verify that the two-shower “mass” distribution follows expectation, a sample of single-particle $\pi^0$’s with and without cosmics is reconstructed and passed through the selection chain. The peak location and shape of the single particle and data samples agree within statistical uncertainty; additionally, agreement improves from the sample without to the sample with cosmics. Together, these pieces indicate that we are selecting $\pi^0$'s from data. 



\begin{figure}[H]
  \begin{subfigure}[t]{0.4\textwidth}
\includegraphics[scale=0.4]{Pi0_Cut_Section/Paper_pi0signalonly_onoffseparate_pi0_low_radL.png}
  \caption{ }
  \end{subfigure} 
  \hspace{14mm}
  \begin{subfigure}[t]{0.4\textwidth}
\includegraphics[scale=0.4]{Pi0_Cut_Section/Paper_pi0signalonly_onoffseparate_pi0_mass.png}
  \caption{ }
  \end{subfigure} 

\caption{ Comparison of OnBeam - OffBeam - MC Backgrounds to signal only distribution for a) conversion distance distribution of all showers and b) two shower reconstructed mass peak.  }
\label{fig:physics_pi0signalonly_mass} 
\end{figure}


\clearpage
\subsubsection{A Closer Look at Sample Composition for $\pi^0\rightarrow\geq 2 \gamma$}
A detailed breakdown of the final selected sample is described in Table \ref{tab:pi0_obnox_breakdown}.  As discussed earlier, all interaction types with cosmic-origin showers are assigned to the cosmic category.  However, thus far we have not otherwise investigated the origin of the `showers' in the candidate pool. The shower origin content of each non-Cosmic `Sample Composition Category' is considered in Figure \ref{fig:physics_showerOriginBreakdown_mass}.  In Figure \ref{fig:physics_showerOriginBreakdown_mass}a, 93\% of the signal is composed of events with two neutrino-induced $\pi^0$ showers (salmon). There is a similar trend in the $\pi^0$-dominated backgrounds ($\nu_\mu$ CC CEx, $\nu_\mu$ Mult $\pi^0$, $\nu_\mu$ NC $\pi^0$) in Figures \ref{fig:physics_showerOriginBreakdown_mass}c-e.  In Figure \ref{fig:physics_showerOriginBreakdown_mass}b, there is a mix of shower origins. For example, the two $\nu$-induced showers shown in green are caused by $\eta$ decays, while the two neutrino-induced $\pi^0$ shower events are due to CC 1$\pi^0$ out of FV, NC Charge exchange, and Kaon decay. Finally, the `Other' category shows a mix.  The two $\pi^0$ shower contribution here is due to $\nu_e$ and $\overline{\nu_\mu}$, while the the remaining yellow background is due to tracks mis-reconstructed as showers in CC events.

\begin{figure}[H]
\centering
  \begin{subfigure}[t]{0.25\textwidth}
    \centering
\includegraphics[scale=0.25]{Pi0_Cut_Section/Physics_showerOriginBreakdown_Signal_pi0_mass.png}
  \caption{ }
  \end{subfigure} 
  \hspace{5mm}
  \begin{subfigure}[t]{0.25\textwidth}
    \centering
\includegraphics[scale=0.25]{Pi0_Cut_Section/Physics_showerOriginBreakdown_FSEM_pi0_mass.png}
  \caption{ }
  \end{subfigure} 
  \hspace{5mm}
  \begin{subfigure}[t]{0.25\textwidth}
    \centering
\includegraphics[scale=0.25]{Pi0_Cut_Section/Physics_showerOriginBreakdown_CCCex_pi0_mass.png}
  \caption{ }
  \end{subfigure} 
  \hspace{5mm}
  \begin{subfigure}[t]{0.25\textwidth}
    \centering
\includegraphics[scale=0.25]{Pi0_Cut_Section/Physics_showerOriginBreakdown_Multpi0_pi0_mass.png}
  \caption{ }
  \end{subfigure} 
  \hspace{5mm}
  \begin{subfigure}[t]{0.25\textwidth}
    \centering
\includegraphics[scale=0.25]{Pi0_Cut_Section/Physics_showerOriginBreakdown_NCpi0_pi0_mass.png}
  \caption{ }
  \end{subfigure} 
  \hspace{5mm}
  \begin{subfigure}[t]{0.25\textwidth}
    \centering
\includegraphics[scale=0.25]{Pi0_Cut_Section/Physics_showerOriginBreakdown_Other_pi0_mass.png}
  \caption{ }
  \end{subfigure} 
\caption{ Breakdown of origin of both showers after two shower selection for each sample.  From left to right, and then top to bottom: a) $\nu_{\mu}$ CC 1 $\pi^0$; b) $\nu_\mu$ CC and NC Final State Electromagnetic Activity; c) $\nu_{\mu}$ CC Charge Exchange; d) $\nu_\mu$ CC Multiple $\pi^0$; e) NC $\pi^0$; f) Other.  }

\label{fig:physics_showerOriginBreakdown_mass}
\end{figure}



\begin{table}[H]
\centering
\captionof{table}{Detailed breakdown of sample composition at final $\pi^0$ selection stage \label{tab:pi0_obnox_breakdown}}
 \begin{tabular}{|l|l|l|}
 \hline
Category & Interaction & Composition \\ [0.1ex] \hline
$\nu_\mu$ Signal & $\nu_\mu$ CC 1$\pi^0$ in FV & 0.671 \\ \hline
$\nu_\mu$ CC Cex & $\nu_\mu$ CC pion charge exchange & 0.039 \\ \hline
$\nu_\mu$ Multiple $\pi^0$ & $\nu_\mu$ Multiple $\pi^0$ & 0.068 \\ \hline
$\nu_\mu$ NC $\pi^0$ & $\nu_\mu$ NC $\pi^0$ & 0.041 \\ \hline
$\nu_\mu$ FSEM & $\nu_\mu$ CC 1$\pi^0$ out of FV & 0.022 \\
& $\nu_\mu$ $N-\gamma$ & 0.028 \\
& $\nu_\mu$ Kaon Decay & 0.005 \\
& $\nu_\mu$ NC pion charge exchange & 0.005 \\ 
&$\nu_\mu$ Brem + $\mu$ capture at rest & 0.046 \\ \hline
Other & $\nu_e$ &0.005 \\
&$\overline{\nu_\mu}$ & 0.002 \\
& Misreconstruction & 0.005 \\ \hline
Cosmic & Cosmic + Neutrino& 0.063 \\
& Cosmic & 0.000 \\ \hline
\end{tabular}
\end{table}

\par The two shower sample is now complete and ready to be analyzed for a cross section measurement. First, a complimentary sample of events with at least one reconstructed shower is prepared. This selection is discussed in the next section.


\clearpage
\subsection{Candidates With At Least One Shower}
In this section, we consider the hypothesis that any neutrino-induced photon originating from the vertex indicates the presence of a $\pi^0$.  This strategy will allow us access to the single-reconstructed shower events, and give us a chance to regain a significant chunk of events. More detail on the one-shower hypothesis exists in internal Ref. \cite{bib:timb_singleshower}.
\par As mentioned in an earlier section, requiring at least two showers reduces the sample size by 95\%.  However, this 95\% reduction includes roughly 50\% of the remaining signal after the CC Inclusive Selection filter has completed. This chunk of signals with only one reconstructed shower is largely the result of the low shower reconstruction efficiency for sub-leading photons, shown earlier in Figure \ref{fig:shower_reco_efficiency}. Below 100 MeV of deposited shower energy, the shower reconstruction efficiency drops below 50\%, which results in a number of signal events with only one reconstructed shower. 
\par When only one shower is required to be reconstructed in an event, the pass rate and sample compositions change.  These values are shown Tables \ref{tab:pi0_1shower_eventrates} and \ref{tab:pi0_1shower_composition}. From these tables, and before any additional checks or cuts, the one shower sample signal efficiency and purity are 26\% and 40\% respectively; this is in contrast to 11\% and 62\% efficiency and purity in the corresponding two-shower Tables \ref{tab:pi0_2showers_eventrates} and \ref{tab:pi0_2showers_composition}. This expected trade off is the result of the additional population of events now considered in the one shower bin.  


\begin{table}[H]
\centering
\captionof{table}{ Pass rate breakdown after one shower requirement scaled to OnBeam POT \label{tab:pi0_1shower_eventrates}}
 \begin{tabular}{| l | l | l |l|l|l|l|l|}
 \hline
 & CC1$\pi^0$ & CC0$\pi^0$ & NC$\pi^0$ & NC0$\pi^0$ & Other & Total \\ [0.1ex] \hline
CC Inclusive & 0.331 & 0.106 & 0.008 & 0.010 & 0.030 & 0.087 \\
$\geq$ 1 Shower & 0.257 & 0.006 & 0.005 & 0.001 & 0.015 & 0.010 \\ \hline
\end{tabular}
\end{table}

\begin{table}[H]
\centering
\captionof{table}{ Sample composition after one shower requirement scaled to OnBeam POT \label{tab:pi0_1shower_composition}}
 \begin{tabular}{| l | l | l |l|l|l|l|l|}
 \hline
 & CC1$\pi^0$ & CC0$\pi^0$ & NC$\pi^0$ & NC0$\pi^0$ & Other & Cosmic+$\nu$& Cosmic\\ [0.1ex] \hline
CC Inclusive & 0.060 & 0.743 & 0.004 & 0.020 & 0.014 & 0.042 & 0.117 \\
$\geq$ 1 Shower & 0.402 & 0.335 & 0.021 & 0.016 & 0.061 & 0.031 & 0.135\\ \hline
\end{tabular}
\end{table}

%\subsection{Signal and Backgrounds for $\pi^0$ \rightarrow $\geq$ 2$\gamma$}
\subsubsection{Tuning Selection for Events with Single Shower}

The one shower sample is next subjected to quality checks. First, the impact parameter of all showers with the vertex is considered.  As noted earlier, small impact parameters are expected, as the vertex is used to reconstruct 3D direction. However, due to the effect on shower direction noted in the two-shower section, an impact parameter cut is imposed to improve the angular resolution of the one shower sample. To be conservative, an impact parameter cut of $\leq$ 4cm is imposed, as shown in Figure \ref{fig:cutjust_pi0_1shower_IP}a.   Next, the conversion distance of the shower from the reconstructed vertex is considered. Because there is some separation power in this variable, a conversion distance cut of 62 cm is employed (Figure \ref{fig:cutjust_pi0_1shower_IP}b).  Finally, if more than one reconstructed shower passes both previous cut, the higher energy shower is selected as the shower candidate. This is in contrast to the two shower path in which multiple-candidate-pair events were removed.  Events selected from OnBeam data via the one shower path are shown in Figure \ref{fig:one_shower_event_displays}. 


\begin{figure}[H]
%  \begin{subfigure}[t]{0.35\textwidth}
%    \centering
%\includegraphics[scale=0.35]{Pi0_Cut_Section/CutJustify_pi0_1shower_eff_gamma_vtx_IP.png}
%  \caption{ }
%  \end{subfigure} 
%  \hspace{30mm}
  \begin{subfigure}[t]{0.35\textwidth}
\includegraphics[scale=0.35]{Pi0_Cut_Section/Physics_showerPostSel2_onoffseparate__gamma_vtx_IP.png}
  \caption{ }
  \end{subfigure} 
  \hspace{9mm}
  \begin{subfigure}[t]{0.35\textwidth}
\includegraphics[scale=0.35]{Pi0_Cut_Section/Physics_showerPostSel2_onoffseparate__gamma_RL.png}
  \caption{ }
  \end{subfigure} 
\caption{ Impact parameter of shower axis with vertex for a) Unscaled MicroBooNE simulation; b) Data to simulation comparison }
\label{fig:cutjust_pi0_1shower_IP}
\end{figure}

\begin{figure}[H]
    \centering
\includegraphics[scale=0.9]{Pi0_Cut_Section/one_shower_event_displays.png}
\caption{ Examples of selected events from data in the one shower sample.}
\label{fig:one_shower_event_displays}
\end{figure}



\subsubsection{Results for $\pi^0\rightarrow \geq 1\gamma$}
\par A table with event rates scaled to OnBeam POT is shown for OnBeam, OffBeam, and Simulation in Table \ref{tab:pi0_event_rates}. This can be compared to the two shower selection results summarized in Table \ref{tab:2shpi0_event_rates}. Note again the good data-MC agreement, and the minimal contribution of MEC events at this final stage in Figure \ref{fig:physics_singleshower_inttype}.

\begin{table}[H] 
 \centering
 \captionof{table}{Event counts at each selection stage for the Single Shower-CC$\pi^0$ selection chain with MCC8.3 samples scaled to OnBeam POT.  Uncertainties shown are statistical. \label{tab:pi0_event_rates}}
 \begin{tabular}{| l | l | l | l | l |}
  \hline
   & OnBeam & OffBeam & On - OffBeam & Simulation \\ [0.1ex] \hline
No Cuts & 544751 $\pm$ 738 & 462076 $\pm$ 1001 & 82675 $\pm$ 1244 & 48949 $\pm$ 76 \\ 
CC Inclusive & 3753 $\pm$ 61 & 564 $\pm$ 35  & 3189 $\pm$ 71 & 4268 $\pm$ 22  \\ 
1 Shower Cuts & 257 $\pm$ 16 & 15 $\pm$ 6 & 242 $\pm$ 17 & 252 $\pm$ 5  \\ \hline
\end{tabular}
 \end{table}


\begin{figure}[H]
\centering
\includegraphics[scale=0.5]{Pi0_Cut_Section/Misc_singleshower_EventType_vs_NeutrinoMode_w_Numbers.png}
\caption{ Event type broken down by neutrino interaction mode for Single Shower cuts; note the contribution of MEC events to the final selected sample. }
\label{fig:physics_singleshower_inttype}
\end{figure}


\par A summary of the passing rates for signal and all backgrounds is shown in Table \ref{tab:pi0_passrates}.  Note that the signal efficiency is significantly higher here in the one shower approach than in the two. This suggests that the one shower approach is a valuable handle on this additional pool of untapped events for extracting physics information. While the purity of the sample does take a hit for this selection, it is still competitive with measurements made by previous experiments (Table \ref{tab:history_ccpi0_results}). Sample composition is shown in Table \ref{tab:pi0_purity}.  Signal and background distributions are shown for a variety of kinematic variables in Figures \ref{fig:physics_singleshower_mulen} - \ref{fig:physics_singleshower_ip}.


\begin{table}[H]
\centering
\captionof{table}{Evolution of passing rates through full analysis chain \label{tab:pi0_passrates}}
 \begin{tabular}{| l | l | l |l|l|l|l|}
 \hline
 & CC1$\pi^0$ & CC0$\pi^0$ & NC$\pi^0$ & NC0$\pi^0$ & Other & All \\ [0.1ex] \hline
No Cuts & - & - & - & - & - & -\\
CC Inclusive & 0.331 & 0.105 & 0.008 & 0.010 & 0.030 & 0.087 \\ 
1 Shower Cuts & 0.170 & 0.002 & 0.003 & 0.0003 & 0.009 & 0.005 \\ \hline
\end{tabular}
\end{table}

\begin{table}[H]
\centering
\captionof{table}{Evolution of sample composition through full analysis chain \label{tab:pi0_purity}}
 \begin{tabular}{| l | l | l |l|l|l|l|l|}
 \hline
  & CC1$\pi^0$ & CC0$\pi^0$ & NC$\pi^0$ & NC0$\pi^0$ & Other& Cosmic+$\nu$ & Cosmic\\ [0.1ex] \hline
No Cuts  & 0.018 &  0.695 & 0.046 & 0.194  & 0.047 & -&-\\
CC Inclusive & 0.060 & 0.743 & 0.004 & 0.020 & 0.014 & 0.042 & 0.117  \\ 
1 Shower Cuts & 0.561 & 0.205 & 0.025 & 0.012 & 0.080 & 0.060 & 0.057 \\ \hline
\end{tabular}
\end{table}

\begin{figure}[H]
  \begin{subfigure}[t]{0.3\textwidth}
\includegraphics[scale=0.3]{Pi0_Cut_Section/Physics_singleshower_onoffseparate_mult.png}
  \caption{ }
  \end{subfigure} 
  \hspace{34mm}
  \begin{subfigure}[t]{0.3\textwidth}
\includegraphics[scale=0.3]{Pi0_Cut_Section/Physics_singleshower_onoffseparate_mu_len.png}
  \caption{ }
  \end{subfigure} 
\caption{ Data to simulation comparison of a) multiplicity and b) $\mu$ length after single shower filter }
\label{fig:physics_singleshower_mulen}
\end{figure}

\begin{figure}[H]
  \begin{subfigure}[t]{0.3\textwidth}
\includegraphics[scale=0.3]{Pi0_Cut_Section/Physics_singleshower_onoffseparate_mu_phi.png}
  \caption{ }
  \end{subfigure} 
  \hspace{34mm}
  \begin{subfigure}[t]{0.3\textwidth}
\includegraphics[scale=0.3]{Pi0_Cut_Section/Physics_singleshower_onoffseparate_mu_angle.png}
  \caption{ }
  \end{subfigure} 
\caption{ Data to simulation comparison of $\mu$ a) $\phi$ and b) $\theta$ length after single shower filter }
\label{fig:physics_singleshower_muphi}
\end{figure}

\begin{figure}[H]
  \begin{subfigure}[t]{0.3\textwidth}
\includegraphics[scale=0.3]{Pi0_Cut_Section/Physics_singleshower_onoffseparate_mu_startx.png}
  \caption{ }
  \end{subfigure} 
  \hspace{34mm}
  \begin{subfigure}[t]{0.3\textwidth}
\includegraphics[scale=0.3]{Pi0_Cut_Section/Physics_singleshower_onoffseparate_mu_endx.png}
  \caption{ }
  \end{subfigure} 
\caption{ Data to simulation comparison of $\mu$ a) start and b) end in x after single shower filter }
\label{fig:physics_singleshower_x}
\end{figure}

\begin{figure}[H]
  \begin{subfigure}[t]{0.3\textwidth}
\includegraphics[scale=0.3]{Pi0_Cut_Section/Physics_singleshower_onoffseparate_mu_starty.png}
  \caption{ }
  \end{subfigure} 
  \hspace{34mm}
  \begin{subfigure}[t]{0.3\textwidth}
\includegraphics[scale=0.3]{Pi0_Cut_Section/Physics_singleshower_onoffseparate_mu_endy.png}
  \caption{ }
  \end{subfigure} 
\caption{ Data to simulation comparison of $\mu$ a) start and b) end in y after single shower filter }
\label{fig:physics_singleshower_y}
\end{figure}

\begin{figure}[H]
  \begin{subfigure}[t]{0.3\textwidth}
\includegraphics[scale=0.3]{Pi0_Cut_Section/Physics_singleshower_onoffseparate_mu_startz.png}
  \caption{ }
  \end{subfigure} 
  \hspace{34mm}
  \begin{subfigure}[t]{0.3\textwidth}
\includegraphics[scale=0.3]{Pi0_Cut_Section/Physics_singleshower_onoffseparate_mu_endz.png}
  \caption{ }
  \end{subfigure} 
\caption{ Data to simulation comparison of $\mu$ a) start and b) end in z after single shower filter }
\label{fig:physics_singleshower_z}
\end{figure}

\begin{figure}[H]
  \begin{subfigure}[t]{0.3\textwidth}
\includegraphics[scale=0.3]{Pi0_Cut_Section/Physics_singleshower_onoffseparate_gamma_E.png}
  \caption{ }
  \end{subfigure} 
  \hspace{34mm}
  \begin{subfigure}[t]{0.3\textwidth}
\includegraphics[scale=0.3]{Pi0_Cut_Section/Physics_singleshower_onoffseparate_gamma_E_corr.png}
  \caption{ }
  \end{subfigure} 
\caption{ Data to simulation comparison of tagged shower a) energy and b) corrected energy }
\label{fig:physics_singleshower_e}
\end{figure}

\begin{figure}[H]
  \begin{subfigure}[t]{0.3\textwidth}
\includegraphics[scale=0.3]{Pi0_Cut_Section/Physics_singleshower_onoffseparate_gamma_RL.png}
  \caption{ }
  \end{subfigure} 
  \hspace{34mm}
  \begin{subfigure}[t]{0.3\textwidth}
\includegraphics[scale=0.3]{Pi0_Cut_Section/Physics_singleshower_onoffseparate_gamma_IP_w_vtx.png}
  \caption{ }
  \end{subfigure} 
\caption{ Data to simulation comparison of tagged shower a) conversion distance and b) impact parameter with the vertex  }
\label{fig:physics_singleshower_ip}
\end{figure}

\subsubsection{A Closer Look at Sample Composition for $\pi^0\rightarrow\geq 1 \gamma$} 
A detailed breakdown of the final selected one shower sample is considered.  This breakdown is described in Table \ref{tab:singleshower_obnox_breakdown} and corresponds to the data-MC comparisons in this section. From this table, the one shower sample is 56\% pure; this contrasts to the 67\% purity observed previously along the two shower selection path.  The quality of this 56\% is further examined by considering the origin of each candidate shower in Figure \ref{fig:physics_showerOriginBreakdown_E}a.  From this figure, 97\% of the selected signal sample has a selected candidate shower that matches to a true $\nu$-induced $\pi^0$, with the majority of remaining tags due to tracks mis-reconstructed as showers. This, in conjunction with the comparable two shower results in the previous section, indicates that the shower reconstruction employed here for both chains is not only automated, but also produces a set of high quality final events.  Plots of background samples as described in the `Sample Composition Category' in Table \ref{tab:singleshower_obnox_breakdown} are also shown in Figure \ref{fig:physics_showerOriginBreakdown_E}b-f.  %Example event displays of reconstructed showers that are the result of true $\nu$-induced showers and tracks are shown in Figure \ref{fig:ccoth_bkgd_nu_cos}.  An example of a `noise'-originating shower is shown in Figure \ref{fig:ccoth_bkgd_noise}, and finally a $\nu$-induced $\pi^0$ shower in Figure \ref{fig:cc1pi0_pi0}.


\begin{table}[H]
\centering
\captionof{table}{Detailed breakdown of sample composition at final one shower selection stage \label{tab:singleshower_obnox_breakdown}}
 \begin{tabular}{|l|l|l|}
 \hline
Category & Interaction & Composition \\ [0.1ex] \hline
$\nu_\mu$ Signal & $\nu_\mu$ CC 1$\pi^0$ in FV & 0.561 \\ \hline
$\nu_\mu$ CC Cex & $\nu_\mu$ CC pion charge exchange & 0.036 \\ \hline
$\nu_\mu$ Multiple $\pi^0$ & $\nu_\mu$ Multiple $\pi^0$ & 0.047 \\ \hline
$\nu_\mu$ NC $\pi^0$ & $\nu_\mu$ NC $\pi^0$ & 0.025 \\ \hline
$\nu_\mu$ FSEM & $\nu_\mu$ CC 1$\pi^0$ out of FV & 0.022 \\
& $\nu_\mu$ $N-\gamma$ & 0.022 \\
& $\nu_\mu$ Kaon Decay & 0.004 \\
& $\nu_\mu$ NC pion charge exchange & 0.005 \\ 
&$\nu_\mu$ Brem + $\mu$ capture at rest & 0.075 \\ \hline
Other & $\nu_e$ &0.007 \\
&$\overline{\nu_\mu}$ & 0.003 \\
& Misreconstruction & 0.075 \\ \hline
Cosmic & Cosmic + $\nu$ & 0.060 \\
& Cosmic (Data) & 0.057 \\ \hline
\end{tabular}
\end{table}

%\begin{table}[H]
%\centering
%\captionof{table}{Detailed background breakdown of the CC 1$\pi^0$ sample from the single shower sample \label{tab:singleshower_obnox_ccoth_breakdown}}
% \begin{tabular}{|l|l|l|}
% \hline
%Candidate $\pi^0$ `Shower' Description & CC 1$\pi^0$ \\ [0.1ex] \hline
%$\nu$-Induced Track & 0.023 \\ 
%$\nu$-Induced Shower (non-$\pi^0$) & 0.007 \\ 
%$\nu$-Induced Shower ($\pi^0$) & 0.970 \\ 
%Noise & 0.031  \\ \hline
%\end{tabular}
%\end{table}

\begin{figure}[H]
\centering
  \begin{subfigure}[H]{0.25\textwidth}
    \centering
\includegraphics[scale=0.25]{Pi0_Cut_Section/Physics_showerOriginBreakdown_1gamma_Signal_gamma_E.png}
  \caption{ }
  \end{subfigure} 
  \hspace{4mm}
  \begin{subfigure}[H]{0.25\textwidth}
    \centering
\includegraphics[scale=0.25]{Pi0_Cut_Section/Physics_showerOriginBreakdown_1gamma_FSEM_gamma_E.png}
  \caption{ }
  \end{subfigure} 
  \hspace{4mm}
  \begin{subfigure}[H]{0.25\textwidth}
    \centering
\includegraphics[scale=0.25]{Pi0_Cut_Section/Physics_showerOriginBreakdown_1gamma_CCCex_gamma_E.png}
  \caption{ }
  \end{subfigure} 
  \hspace{4mm}
  \begin{subfigure}[H]{0.25\textwidth}
    \centering
\includegraphics[scale=0.25]{Pi0_Cut_Section/Physics_showerOriginBreakdown_1gamma_Multpi0_gamma_E.png}
  \caption{ }
  \end{subfigure} 
  \hspace{4mm}
  \begin{subfigure}[H]{0.25\textwidth}
    \centering
\includegraphics[scale=0.25]{Pi0_Cut_Section/Physics_showerOriginBreakdown_1gamma_NCpi0_gamma_E.png}
  \caption{ }
  \end{subfigure} 
  \hspace{4mm}
  \begin{subfigure}[H]{0.25\textwidth}
    \centering
\includegraphics[scale=0.25]{Pi0_Cut_Section/Physics_showerOriginBreakdown_1gamma_Other_gamma_E.png}
  \caption{ }
  \end{subfigure} 
\caption{ Breakdown of origin of both showers after one shower selection for each sample shown for uncorrected shower energy.  From left to right, and then top to bottom: a) $\nu_{\mu}$ CC 1 $\pi^0$; b) $\nu_\mu$ CC and NC Final State Electromagnetic Activity; c) $\nu_{\mu}$ CC Charge Exchange; d) $\nu_\mu$ CC Multiple $\pi^0$; e) NC $\pi^0$; f) Other.  }
\label{fig:physics_showerOriginBreakdown_E}
\end{figure}

%\begin{figure}[H]
%\centering
%\includegraphics[scale=0.8]{Pi0_Cut_Section/CCOther_nu_cos.png}
%\hspace{1 mm}
%\caption{ Example of $\nu$ and cosmic-induced CCOther backgrounds. The cyan circle is the reconstructed vertex and the green triangle is the candidate $\pi^0$ shower.  The candidate muon track is also shown in solid red in the $\nu$-induced track example to differentiate from the $\pi^0$ candidate `shower'.  }
%\label{fig:ccoth_bkgd_nu_cos}
%\end{figure}
%
%\begin{figure}[H]
%\centering
%\includegraphics[scale=0.9]{Pi0_Cut_Section/CCOther_noise.png}
%\caption{ Example of a `Noise' CCOther background. MCClusters are energy depositions built into clusters based on the deposition’s true origin particle. Each mccluster color represents a new particle (though the event display color wheel is finite, so there is some cycling of colors).  Gaushits are reconstructed from waveforms into individual points (hits) with time and wire coordinates, and an associated charge. MCClusters are shown on the left, while reconstructed gaushits and the candidate reconstructed shower is shown on the right.
% Note that the candidate `shower' has no corresponding mccluster in the left panel. The cyan circle is the reconstructed vertex and the green triangle is the candidate $\pi^0$ shower.  }
%\label{fig:ccoth_bkgd_noise}
%\end{figure}
%
%\begin{figure}[H]
%\centering
%\includegraphics[scale=0.8]{Pi0_Cut_Section/CC1pi0_nu_cos.png}
%\caption{ Example of $\nu$ and cosmic-induced CC 1-$\pi^0$ backgrounds. The cyan circle is the reconstructed vertex, the green triangle is the candidate $\pi^0$ shower, and the solid red line is the $\mu$ candidate. Track-like hits are shown in black, and shower-like hits are shown in faint red. }
%\label{fig:cc1pi0_nu_cos}
%\end{figure}
%
%\begin{figure}[H]
%\centering
%\includegraphics[scale=0.8]{Pi0_Cut_Section/CC1pi0_pi0.png}
%\caption{ Example of a true CC 1-$\pi^0$ event. The cyan circle is the reconstructed vertex, the green triangle is the candidate $\pi^0$ shower, and the solid red line is the $\mu$ candidate. Track-like hits are shown in black, and shower-like hits are shown in faint red. }
%\label{fig:cc1pi0_pi0}
%\end{figure}
