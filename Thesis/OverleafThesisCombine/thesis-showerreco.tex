\clearpage
%\section{Track-like Hit Removal}
\section{Shower Reconstruction}
During past efforts, 2D-clustering has been a bottle neck in the reconstruction chain where many events are lost due to the complexity of reconstructing complicated topologies. To mitigate this problem, 2D-clustering is broken into 2 stages: track-like hit removal and clustering. Both stages are discussed in this chapter.

\subsection{Identifying Electromagnetic Activity}
Hit Removal is a suite of algorithms developed by a MicroBooNE collaborator to identify electromagnetic activity at the hit level and store this information for use by downstream algorithms. While the package is called ``Hit Removal", no hits are ever actually removed from the event; rather, a track or shower-like designation is stored per hit. Thus, at the end of all Hit Removal algorithms, both the identified electromagnetic hits and all original hits can be accessed. A brief overview is discussed in this section; more detail on all algorithms is provided in internal document \cite{bib:davidc_hitremoval}.

\par The goal of these algorithms is to identify charge which is both neutrino-induced and shower-like. This hit-identification is broken into cosmic-induced and neutrino-induced stages.  In both stages, only hits within a 1~m Region Of Interest (ROI) of the reconstructed vertex are considered. An example ROI is depicted in Figure \ref{fig:roi}. The cosmic-induced hit identification algorithms are run first.  Here, the hits associated with Pandora cosmic-tagged tracks are `removed'.  Similarly, groups of hits that are poorly aligned with the neutrino vertex are removed, whether or not that charge already has a Pandora cosmic tag. Finally, remaining hits near in 2D space to already-removed cosmic clusters are considered. If this activity resembles a delta ray, these hits are also removed. 

\begin{figure}[h!]
\centering
\includegraphics[scale=0.35]{Cluster_Shower/Misc_ROI.png}
\caption{$\pi^0$ Region Of Interest (ROI) around $\nu_\mu$ CC CC Inclusive Selection tagged  vertex. The bounds extend 1m away from the vertex in each direction. }
\label{fig:roi}
\end{figure}

\par The cosmic-targeted algorithms are followed by a suite of algorithms aimed at neutrino-induced activity. First, Pandora clusters that have associated hits within 3.5 cm of the vertex are required to satisfy an aggressive linearity requirement. The goal of this tight cut on linearity is to preserve the population of photons that convert very near the vertex, while also removing some vertex-related track activity. Following this requirement, cuts are imposed on each cluster's number of hits and 2D slope, in addition to a looser linearity cut which aims to remove $\mu$ and charged $\pi$ activity. Finally, all charge within 3.5 cm of the vertex is removed. This is motivated by the fact that hits are generally crowded near the vertex; this frequently leads to over-merging near the vertex at the clustering stage. Once these algorithms have completed, the remaining electromagnetic hit candidates are passed onto clustering. The remaining hits after Hit Removal has completed are shown in red in Figure \ref{fig:hitremoval} for two separate example events. 
\begin{figure}[h!]
\centering
\includegraphics[scale=0.4]{Cluster_Shower/HR_mcc8.png}
\caption{ Example display after Hit Removal. Candidate electromagnetic activity is shown in red for two separate events. Note in the interaction on the right that the relatively linear shower (circled) has been removed. The effects of energy loss at all stages are quantified in internal Ref. \cite{bib:davidc_missingE}.} 
\label{fig:hitremoval}
\end{figure}

\subsection{OpenCV Software Package}
OpenCV is an open source computer vision library with functionality to aid in pattern recognition and image processing \cite{bib:opencv}. These functions have been developed by a world-wide community and are designed to be both efficient and fast. OpenCV has Python, C++, and Java interfaces, and is easy to use. A number of OpenCV image-manipulation algorithms are employed here to reconstruct $\pi^0$ shower candidates.  

\subsection{LArOpenCV Framework}
LArOpenCV is a C++ framework with an interface to the OpenCV API. The LArOpenCV framework allows for event-by-event image creation, application of image manipulation algorithms, two dimensional contour matching between three physical wire planes, and cluster creation and output. The input into LArOpenCV is either a full readout or region of interest (ROI) reconstructed hit object. Input data products are first translated into an ``event image'' and then handed over to a separate manager class for image processing. Electromagnetic hit candidates identified during it Removal are used in this clustering stage. 

\subsection{Event Image}
\par In order to process hit objects as input within the LArOpenCV framework, data products first need to be translated into a data format compatible with OpenCV. To do this, the LArOpenCV framework creates a single channel image for each plane determined by the wire and time tick ranges. Each row pixel represents a single wire and each column pixel a single time tick. The integral charge of each hit is scaled to 8 bits (integer 0 to 255) and assigned to that hit's peak time and wire. The result is a single channel grey-scale image. Note that this scaling is solely for the purposes of identifying clusters at the OpenCV stage. All analysis post-clustering (including energy reconstruction) is entirely independent of OpenCV and the LArOpenCV framework. 
\par Finally, the image is ``pooled". In MicroBooNE, a single time tick corresponds to 0.05 cm and a single wire spacing to 0.3 cm (roughly 6 time ticks).  In pooling, 6 ticks are combined into 1 tick pixel with the summed pixel value at that location.  Pooling is beneficial as the image manipulation kernels described in the coming sections perform better when they act uniformly in both directions. At this point, an image is prepared and ready for clustering.

\subsection{Clustering with LArOpenCV }

\par The job of the first algorithm in the clustering chain is to identify candidate clusters for eventual particle identification. The algorithm first transforms image information into polar coordinates using the OpenCV linearPolar function \cite{bib:linearPolar}, with the origin at the reconstructed vertex location and the radius set to the ROI width.  Any pixels outside this region (for example, if the radius extends outside the TPC) are set to 0. This polar strategy has the advantage over Cartesian algorithms in that it enforces an image blur in the direction of showering which prevents lateral over-merging. Examples of the polar transformation are shown in Figure \ref{fig:polar} and \ref{fig:pi0_polar}. Following translation, a series of OpenCV image manipulations as depicted in Figure \ref{fig:sbc} are employed. First, the image is dilated \cite{bib:dilate} with an elliptic structuring element \cite{bib:structuringElement} 5 pixels in radius. During dilation, pixels surrounding each hit within the dilation radius acquire the grey scale value of the hit; the fundamental pixel dimensions, however, remain the same. In this way, hits are connected without changing the size of the image. Next, a blurring function \cite{bib:blur} with kernel width 10 and height 5 is applied. Blurring smears the dilated hits together with a Gaussian filter and smooths image edges. This process, similar to dilation, only changes the grey scale values of various pixels, not the size of the original pixels. Finally, contour finding \cite{bib:contourFinding} is run on the blurred image.  Once contours are stored, we translate back into Cartesian coordinates and assign hits to the contours that they fall inside of \cite{bib:pointPolygon}. These groups of hits become preliminary shower candidate clusters.  Clusters are required to have at least 10 hits to pass to the next algorithm. The motivation for this cut is discussed in Section 1.6.

\begin{figure}[H]
\centering
\includegraphics[scale=0.6]{Cluster_Shower/Misc_opencv_polar.png}
\caption{ Example from OpenCV manual depicting polar transformation algorithm \cite{bib:linearPolar}}
\label{fig:polar}
\end{figure}

\begin{figure}[h!]
\centering
\includegraphics[scale=0.83]{Cluster_Shower/Misc_opencv_cluster_linear.png}
\includegraphics[scale=0.83]{Cluster_Shower/Misc_opencv_cluster_polar.png}
\caption{ Example of a single particle $\pi^0$ event in linear (left) and polar (right)}
\label{fig:pi0_polar}
\end{figure}


\begin{figure}[H]
\centering
\includegraphics[scale=0.6]{Cluster_Shower/Misc_image_manipulation.png}
\caption{ a) Original hits that make up a single cluster; b) Dilate pixels to ``connect" nearby hits; c) Blur image to smooth edges and smear charge; d) OpenCV contour finding on blurred image produces a polygon which now encloses the original hits.}
\label{fig:sbc}
\end{figure}

%This corresponds to five radiation lengths beyond the first conversion length which means that little energy will pierce this boundary (Figure \ref{fig:davidc_conversion_distance}). Thus, reconstructed clusters extending outside the ROI are unlikely candidates for our $\gamma$ search. We only consider information in this ROI around the CC Inclusive Selection candidate vertex throughout the remainder of the clustering chain.

%\begin{figure}[H]
%\centering
%\includegraphics[scale=0.4]{Cluster_Shower/conversion_distance.png}
%\caption{ Mean free path of photons as a function of energy, courtesy of David Caratelli. }
%\label{fig:davidc_conversion_distance}
%\end{figure}

\par Next, a start point and direction are assigned to each cluster. Start point calculation is performed in the following way: each OpenCV-calculated contour has an associated minimum bounding box \cite{bib:minAreaRect} that surrounds it (outer red rectangle in Figure \ref{fig:misc_opencv_startpoint}a). The start point finding module segments this bounding box into 2 equal segments long ways (red center line in Figure \ref{fig:misc_opencv_startpoint}a). The algorithm then locates the hit of the cluster that is closest to the ROI vertex (green circle).  Next, it take the segment this hit belongs to (green box in Figure \ref{fig:misc_opencv_startpoint}b) and searches for the hit furthest from the center (yellow dot) of the minimum bounding box to assign a start point (purple star in Figure \ref{fig:misc_opencv_startpoint}c). Finally, the hit furthest from the center in the adjacent segment is assigned to be the end point (orange star in Figure \ref{fig:misc_opencv_startpoint}c). The cluster's direction is assigned to be the direction of the minimum bounding box.

\begin{figure}[h!]
\centering
\includegraphics[width=0.95\textwidth]{Cluster_Shower/Misc_opencv_startpoint_explanation.png}
\caption{ a) The start point finding algorithm uses the ROI vertex (green circle) to determine the
cluster’s nearest hit to the vertex; b) The segment containing this hit (green) is then selected; c) The
start point (purple star) is assigned to the hit furthest from the center of the bounding box (yellow
dot), and likewise the end point in the other segment (orange star). }
\label{fig:misc_opencv_startpoint}
\end{figure}

\par Now that the clusters have start points and directions we attempt to combine charge which was not clustered together during polar clustering. The algorithm used to do the merging proceeds in a series of steps.  First, the minimum bounding box for each cluster is again considered. The bounding points farthest from the reconstructed vertex are designated as the `end' of the rectangle, where as the points closest are the `start'.  A trapezoid of height 13 cm (h) with 20 degree fan ($\theta$) away from the direction of the box is built onto the end of original bounding box (Figure \ref{fig:misc_flashlights}b); these values were chosen empirically to maximize first, the purity and second, the completeness on a sample of CC $\pi^0$ interactions.  The results of this operation are objects that resemble flash lights around each cluster per plane (Figure \ref{fig:misc_flashlights}c). Clusters are next sorted based on their proximity to the vertex and looped over. Per cluster in this outer loop, the overlap between it and all other thus-unassociated clusters is considered in an inner loop.  This is done by looking for overlaps between the flash piece (segments 2-3-4-5 from Figure \ref{fig:misc_flashlights}b) of the closer-to-vertex cluster with the base (segments 5-0-1-2 from Figure \ref{fig:misc_flashlights}b) of the further cluster. Care is taken to keep neighbor associations unique, and not double count cluster merges. Once all overlaps have been considered, the convex hull \cite{bib:convexHull} of all newly-associated cluster hits is calculated and stored (Figures \ref{fig:misc_flashlights}d and e). The start point of the old cluster closest to the ROI vertex is used as the start point for the new combined cluster, while the end point is reassigned to the new cluster hit furthest from the start point.

\begin{figure}[h!]
\centering
\includegraphics[scale=0.5]{Cluster_Shower/flashlights.png}
\caption{ a) The flashlight merging algorithm is fed the vertex (cyan) and a set of clusters identified in the polar clustering stage; b) A trapezoid of angle $\theta$ and height h is attached to the end of the cluster, with the end being the side furthest from the vertex.  The new contour bounding box points are labeled for convenience; c) Flashlights are constructed only for identified clusters in the plane; d) Associations are built between flashlights based on overlap of their segments; e) The final bounding contour is the minimum contour that encloses all points.  }
\label{fig:misc_flashlights}
\end{figure}

%\par Finally, two simple filters are applied to reduce the number of bad/uninteresting clusters passing on to matching. First, clusters which are not aligned well with the vertex are removed.  We do this by considering the dot product of the cluster's direction with the direction vector from the vertex to the cluster's start point.  If this dot product is $>$ 0.71 or if the cluster start point is closer than 12 pixels to the vertex, the cluster is kept, where these values were determined empirically to minimize the removal of visually well-aligned clusters.  The latter condition prevents removing clusters that appear to have very misaligned directions by function of being near the vertex. This algorithm additionally removes lingering cosmic rays and some mis-clustered hits. Second, we filter clusters whose surrounding contour contains the CC Inclusive Selection vertex (Figure \ref{fig:vertex_in_hull}). Occasionally when events are busy (e.g. a crossing cosmic muon through the interaction, a large amount of charge deposited, etc.) the flashlight algorithm will over-merge. A contained vertex is nearly always a good sign that over-merging has occurred.  

%\begin{figure}[H]
%\centering
%\includegraphics[width=0.35\textwidth]{Cluster_Shower/vertex_in_hull.png}
%\caption{Example event targeted by the second simple filter we apply after merging. The selected vertex is shown as a yellow star, and the convex hull formed after merging is shown in red. Here, merging has joined two short tracks at the vertex, which has resulted in a vertex-contained-hull. }
%\label{fig:vertex_in_hull}
%\end{figure}


\subsection{Cluster Matching}
The final step before shower reconstruction is cluster matching across planes. This is a necessary step to identify the $y$ and $z$ coordinates of the interaction, which are needed to extract 3D information about the showers. Only two planes are required here in order for matched pairs to be saved. This decision is based on the fact that clustering generally only goes well consistently in two of the three planes; dead wires, events with extra activity (noise), and poor reconstruction are often present in at least one plane. Additionally, one of the matched clusters must come from the collection plane, as this is the plane where the energy scale is currently best understood.  In order to choose the second plane (U or V), an event-by-event score is assigned to the U and V plane based on the percentage of the ROI (determined to be 1 m by the Hit Removal stage) that is covered by ``dead'' or poorly-functioning wires.  
 This is done by assigning the width of the ROI as the denominator and the sum of all dead wire ranges within the ROI as the numerator. The score is computed by subtracting this percentage-of-dead-wires-in-ROI ratio from 1; thus ROI's with few dead wires will have high scores, while ROI's with many will have low scores.  The clusters in the U or V plane with the highest plane score are passed on to matching. 

\par Matching utilizes the fact that time is a shared coordinate across planes and assigns scores to cluster pairs based on their agreement in time. This score is quantified using a measure denoted as the \texttt{Intersection over Union}, \texttt{IoU} and defined as:
\begin{equation}
  {\rm IoU} = \frac{ \Delta t_1 \cap \Delta t_2  }{ \Delta t_1 \cup \Delta t_2 }
\end{equation}

\noindent where $\Delta t$ denotes the time-range associated to the peak time of each hit in a given cluster.  Clusters which do not overlap are assigned a score of -1, while those that do are assigned a score between 0 and 1, with 1 being perfect overlap. At the end of the consideration of all match permutations, the highest scores are used to create matched pairs until no clusters or viable matches remain. A minimum of 25\% agreement in time and 20\% agreement in number of cluster hits is required in order for a match to be made. 
\par As noted earlier, an early cut on clusters with less than 10 hits before the flashlight merge was applied. An example of the situation this cut addresses is shown in Figure \ref{fig:matching_ex_0}. The top panel shows the event before any merging has occurred. The bottom panel shows the event after merging if a cut on hit size is not included. On the bottom, a cluster in one plane merges with a small, distant blob of charge and its true partner-cluster from the other plane does not merge with that same small blob of charge. As a result, the clusters are no longer well-matched in the shared time coordinate.  This occasionally leads to clusters being mismatched with another cluster with which they now share a better time overlap (middle black cluster in Plane 0 with green cluster in plane 2). Losing some charge and correcting for the energy loss later is better than dealing with mis-matched pairs.
%Here planes 0 and 2 are considered right after polar clustering. There is 1 true shower, with the majority of the shower (green triangle) separated in space from the remaining bit of the shower (blue triangle).  Time is a shared coordinate across the planes, so while the angle is different in different views, time will always line up (t0 and t1 on the green cluster).  Because the green cluster has a steeper angle in plane 2 than in plane 0, the blue cluster appears closer to the green in plane 2.  These two clusters will likely be combined during flashlight merging.  However, the blue and green clusters in plane 0 are unlikely to be merged,  because they appear further apart. After merging completes, plane 0’s green cluster goes from t0 $\rightarrow$ t1, and plane 2’s green cluster goes from t0 $\rightarrow$ t2. Now it appears that plane 2’s cluster has a larger time overlap with the middle track than with the green plane 0 cluster.  Thus, if this track has not been removed during hit removal, matching will likely go awry unless the tiny cluster is prevented from entered the pool of clusters to-be-merged.

\begin{figure}[H]
\centering
\includegraphics[width=0.8\textwidth]{Cluster_Shower/matching_ex_0.png}
\includegraphics[width=0.8\textwidth]{Cluster_Shower/matching_ex_1.png}
\caption{Example event targeted by the cut on small clusters.}
\label{fig:matching_ex_0}
\end{figure}


\subsection{3D Shower Reconstruction}
Shower reconstruction uses 2D information created during the previous matching stage to create one 3D object. 
\paragraph{3D Direction}  The reconstructed 3D candidate vertex from the CC Inclusive selection is used here to reconstruct 2D shower directions. These 2D direction are computed as the charge-weighted average vector sum of the 2D distance from the vertex (projected into each plane) to each hit in the cluster
\begin{equation}
  \hat{p}_{\rm 2D} = \frac{1}{Q_{tot}} \sum_{i=0}^{N} (r_i - r_{\rm vtx}) \times q_i 
\end{equation}
where N denotes the number of hits in the cluster, $r_i$ the position of the hit, $q_i$ its charge, and $r_{\rm vtx}$ the position of the projected vertex. Given two 2D weighted directions, the 3D direction is calculated using geometric relations between the planes and clusters \cite{bib:larliteGeoHelper}. 

\paragraph{3D Start Point Reconstruction} The 3D start point is calculated using the OpenCV reconstructed 2D start points of the matched pair of clusters. The time tick coordinates from each cluster are averaged to calculate a 3D shared time coordinate. The (Y,Z) coordinates are identified by the intersection between the wires associated with the 2D start points.  Wires must intersect inside the TPC for a shower to be reconstructed.  An example of successful 3D shower reconstruction (projected back into 2D) is shown in Figure \ref{fig:showers}.

\begin{figure}[h!] %H]
\centering
\fbox{\includegraphics[width=0.4\textwidth]{Cluster_Shower/Misc_opencv_showerreco.png}}
\caption{3D reconstructed showers are projected back into 2D as a visualization tool to indicate whether or not shower reconstruction is successful. }
\label{fig:showers}
\end{figure}

\subsection{Shower Energy Reconstruction}
\label{sec:ereco}
\par This section briefly describes the series of correction factors and constants applied to the raw charge to account for physical processes, detector, and electronics effects. All correction factors are derived using information only from the collection plane. % More detail on the motivation for these constants can be found in internal documents \cite{bib:davidc_energycalibration} \cite{bib:davidc_missingE}.

%\begin{itemize}
\paragraph{Electronics Gain} 
Collection plane hits associated to the shower are considered.  Each hit's area represents the integrated ADC charge deposited on that wire at that point in time.  The integrated charge is converted from ADC to $e^-$'s by applying an electronics gain factor of 198 $e^-$ / ADC. More detail on the calculation of this constant can be found in MicroBooNE's noise paper ~\cite{bib:noise}.  
\noindent The gain calibration value for data 243 $e^-$ / ADC.  This factor is extracted from a sample of stopping muons and applied for both data samples.   
\paragraph{ Lifetime Correction} No lifetime correction is applied due to the high measured argon purity and electron lifetime in MicroBooNE data \cite{bib:purity}. 
\paragraph{Argon Ionization Work Function} Deposited electrons are next converted into an energy scale. To do this, the number of deposited electrons is multiplied by the 23.6 $\frac{eV}{e}$ it takes to ionize a single electron in argon \cite{bib:ionization_per_electron}. 
\paragraph{Ion Recombination}  A charged particle that traverses the liquid argon medium leaves a trail of ionized electrons and argon ions in its wake.  Some number of these electrons will recombine with argon ions rather than making it to the wire planes. This number depends on the local electric field and on $\frac{dE}{dx}$ of the ionizing particle. Here, a single, constant recombination correction of 0.423 is applied. This factor was obtained by assuming a fixed $\frac{dE}{dx}$ of 2.3 MeV/cm and utilizing the Modified Box recombination model as parametrized by the ArgoNeuT collaboration \cite{bib:argoneut_recomb}, applied at MicroBooNE's electric field of 273 V/cm. More detail on the extraction of this constant can be found in internal Ref. \cite{bib:davidc_missingE}.
%\end{itemize}
\paragraph{Summary - Calibration Constants}
The calibration constants used in shower energy reconstruction for the rest of the note are summarized below:
\begin{equation}
  C_{MC} = 198 \frac{e^-}{\rm ADC} \times 23.6 \times 10^{-6} \frac{MeV}{e^-} \times \frac{1}{1-0.423} = 8.10 \times 10^{-2} \frac{\rm MeV}{\rm ADC}
\end{equation}

\begin{equation}
  C_{Data} = 243 \frac{e^-}{\rm ADC} \times 23.6 \times 10^{-6} \frac{MeV}{e^-} \times \frac{1}{1-0.423} = 9.94 \times 10^{-2} \frac{\rm MeV}{\rm ADC}
\end{equation}


\subsection{Shower Quality and Energy Correction}
%We begin by studying the reconstructed shower energy (via the shower's associated collection plane cluster).
In this section, the quality of reconstructed showers in the simulated neutrino + cosmics sample is evaluated. To begin, collection plane hits are combined into 2D `mcclusters' in each plane using truth information (Figure \ref{fig:mcclusters}) and used to assign a purity and completeness to each reconstructed cluster. A pictorial example is shown in Figure \ref{fig:showerquality_purcompex}. In this example, the purity of the purple reconstructed cluster is 100\% because the reconstruction has only included hits from one truth cluster (green). On the other hand, the red reconstructed cluster was over-merged, and ended up with hits from both the green and blue truth clusters.  Thus, the purity of the red cluster is less than 1. Likewise, the completeness of each reconstructed cluster is considered. The completeness indicates how much of the truth cluster the reconstruction has captured.  In both the case of the purple and the red reconstructed clusters, hits were missed during reconstruction. Thus the completeness for both is less than 1.  The purity and completeness of our current sample of reconstructed showers are shown in Figure \ref{fig:showerquality_purcomp}. Note that a purity of zero indicates that a reconstructed shower shares no hits with any neutrino-induced $\pi^0$ cluster in the event. 
% Reconstructed cluster `purity' is defined to be the number of hits in the reconstructed cluster associated to the mccluster with the largest overlap in hits over the total hits of the reconstructed cluster. Cluster `completeness' is defined to be the number of hits in the reconstructed cluster associated to the mccluster with the largest overlap in hits over the total hits of the mccluster. 

\begin{figure}[h!]
\centering
\fbox{\includegraphics[scale=0.28]{Cluster_Shower/mcclusters.png}}
\caption{ Example of `mcclusters' in a neutrino interaction in the collection plane. Each color ideally represents an individual truth level particle in cluster form. Note however that some of the colors repeat in this example, as the event display has a finite color wheel. }
\label{fig:mcclusters}
\end{figure}

\begin{figure}[H]
\centering
  \begin{subfigure}[t]{0.4\textwidth}
    \centering
\includegraphics[scale=0.4]{Cluster_Shower/ShowerQuality_purComp.png}
  \caption{ }
  \end{subfigure} 
  \hspace{5mm}
  \begin{subfigure}[t]{0.45\textwidth}
    \centering
\includegraphics[scale=0.45]{Cluster_Shower/ShowerQuality_purComp_bothex.png}
  \caption{ }
  \end{subfigure} 
\caption{a) Example $\pi^0$ event with both the mcclusters and reconstructed clusters pictured; b) Purity and completeness results for both reconstructed showers }
\label{fig:showerquality_purcompex}
\end{figure}


\begin{figure}[H]
\centering
  \begin{subfigure}[t]{0.45\textwidth}
    \centering
\includegraphics[scale=0.45]{Cluster_Shower/ShowerQuality_purity.png}
  \caption{ }
  \end{subfigure} 
  \hspace{3mm}
  \begin{subfigure}[t]{0.45\textwidth}
    \centering
\includegraphics[scale=0.45]{Cluster_Shower/ShowerQuality_complete.png}
  \caption{ }
  \end{subfigure} 
\caption{Collection plane a) purity and b) completeness of reconstructed showers. }
\label{fig:showerquality_purcomp}
\end{figure}

\par  Next, each reconstructed cluster is matched to an mccluster based on its purity. These matches allow us to study of the energy bias and resolution of the reconstructed showers in the candidate pool.  With an understanding of bias, reconstructed shower products can be corrected and used as a calibration tool for the detector (such as the reconstructed $\pi^0$ mass peak).  The energy resolution is shown in Figure \ref{fig:showerquality_eres} with an empirical fit. The x = y line is included for reference, and to emphasize the reco bias away from truth.  The contributions to this bias are mainly twofold. For one, roughly 10\% of the reconstructed energy is lost to hit thresholding and containment effects. An additional 15\% loss is due to both the hit removal stage, where pieces of showers are sometimes excluded from clustering consideration, and clustering, where hits are sometimes missed. These bias are considered in detail externally in internal Refs. \cite{bib:davidc_hitthresholding} and \cite{bib:davidc_missingE}.  Note that the population of events above the line are instances of over-merging. This occurs when a reconstructed cluster is made up of two or more mcclusters and thus appears to have a higher energy than its associated truth cluster. 

\begin{figure}[H]
\centering
\includegraphics[scale=0.5]{Cluster_Shower/ShowerQuality_eres2d.png}
\caption{True energy vs reconstructed energy for each reco-MC matched pair}
\label{fig:showerquality_eres}
\end{figure}


\par It's clear from the fit that a simple linear energy correction is insufficient to recover all of the charge lost at various stages of shower reconstruction.  Nevertheless, this correction fits enough of the reconstructed information to provide a sense of where distributions will lie when more sophisticated corrections are applied. Thus, this first-pass empirical fit will be used to correct the reconstructed shower energies in data and MC for the remainder of this document. These corrected energies will be calculated according to the equation
\begin{equation}
\label{eq:ecorr}
E_{corr} = \frac{E_{reco}}{0.77} .
\end{equation}

\noindent Note that the shower energies play no role in the calculation of the cross section, and only affect the kinematic distributions that are shown in later sections. Corrected plots will always be shown next to their uncorrected counterparts as a guide. 
\par The energy resolution is additionally considered in 1D in Figures \ref{fig:showerquality_eres_corr}a and b, with and without the corrected energy as described by Equation \ref{eq:ecorr}. Here it is clear that there are two populations.  The secondary bump away from 0 is explored by considering single particle $\pi^0$ samples without and with cosmics.  Figure \ref{fig:showerquality_eres_series} (left) shows that the bump is absent from the resolution curve for single particle $\pi^0$ without cosmics.  The bump appears when cosmics are added in with the single particles (right) and is accentuated further when we integrate neutrino-related track activity (Figure\ref{fig:showerquality_eres_corr}a). This indicates that the bump is related to the presence of track-related activity.  During hit removal, hits associated with showers may be removed if they overlap with tracks.  The more track activity present in the event the more often this will happen. %, which is seen when we look at the CC $\pi^0$ + cosmics sample.  

\begin{figure}[H]
  \begin{subfigure}[t]{0.35\textwidth}
\includegraphics[scale=0.35]{Cluster_Shower/ShowerQuality_eres.png}
  \caption{ }
  \end{subfigure} 
  \hspace{15mm}
  \begin{subfigure}[t]{0.35\textwidth}
\includegraphics[scale=0.35]{Cluster_Shower/ShowerQuality_eres_corr.png}
  \caption{ }
  \end{subfigure} 
\caption{a) Energy resolution in 1D, where $E_{MC}$ is the true shower energy; b) Corrected energy resolution in 1D using the extracted correction factor. The quoted mean and sigma related to the fitted Gaussian.}
\label{fig:showerquality_eres_corr}
\end{figure}

\begin{figure}[H]
\centering
 \includegraphics[scale=0.8]{Cluster_Shower/ShowerQuality_eres_progression.png}
 \caption{Energy resolution of reconstructed showers in single particle $\pi^0$ samples without cosmics (left) and with cosmics (right).  As cosmic track activity is added to the interaction, a secondary energy resolution bump begins to grow.  }
\label{fig:showerquality_eres_series}
\end{figure}


To finish up the discussion of reconstructed shower quality, shower start point resolution is considered.  The resulting resolutions are shown in Figure \ref{fig:showerquality_xyzres}. Note that start point resolution by itself is an imperfect metric for assessing the quality of shower reconstruction. Sometimes the trunk of the shower is very linear and removed during hit removal; in this case, the result is often a poor start point resolution, despite the shower being usable to identify the presence of a $\pi^0$. An example of this type of event is shown in Figure \ref{fig:showerquality_startpointoff}.  Note that this effect also contributes to the worse resolution seen in the Z coordinate in Figure \ref{fig:showerquality_xyzres}. As showers are generally forward going in z, but uniformly distributed in x and y, a removed trunk will most affect the z coordinate..

\begin{figure}[h!]
\centering
\includegraphics[scale=0.7]{Cluster_Shower/ShowerQuality_startpointoff.png}
\caption{Example of an event in which the beginning of the shower has been removed during hit removal, but which still may be used to identify the presence of a $\pi^0$ }
\label{fig:showerquality_startpointoff}
\end{figure}

\begin{figure}[h!]
\centering
  \begin{subfigure}[t]{0.45\textwidth}
    \centering
\includegraphics[scale=0.45]{Cluster_Shower/ShowerQuality_diff_x.png}
  \caption{ }
  \end{subfigure} 
  \hspace{5mm}
  \begin{subfigure}[t]{0.45\textwidth}
    \centering
\includegraphics[scale=0.45]{Cluster_Shower/ShowerQuality_diff_y.png}
  \caption{ }
  \end{subfigure} 
  \hspace{5mm}
  \begin{subfigure}[t]{0.45\textwidth}
    \centering
\includegraphics[scale=0.45]{Cluster_Shower/ShowerQuality_diff_z.png}
  \caption{ }
  \end{subfigure} 
  \hspace{5mm}
  \begin{subfigure}[t]{0.45\textwidth}
    \centering
\includegraphics[scale=0.45]{Cluster_Shower/ShowerQuality_tot.png}
  \caption{ }
  \end{subfigure} 
\caption{ Matched showers resolution in a) X, b) Y, c) Z, and d) total 3D. The values displayed in the text box are the mean and standard deviation of the distribution. }
\label{fig:showerquality_xyzres}
\end{figure}

\clearpage
\subsection{Shower Reconstruction Efficiency}
Finally, reconstruction efficiency in the same neutrino + cosmic simulation is considered.  The efficiency is defined to be the number of reconstructed showers that are matched to a true signal shower divided by the number of true signal showers. The results are shown in Figure \ref{fig:shower_reco_efficiency}.  The majority of the subleading shower population is below 100 MeV, where shower reconstruction efficiency drops below 50\%.  In the past, a second pass reconstruction to increase efficiency for subleading photons was explored, but there was limited success (internal Ref. \cite{bib:jz_catch_subleading}).  

\begin{figure}[h!]
\centering
\includegraphics[scale=0.7]{Cluster_Shower/shower_reco_efficiency.png}
\caption{Shower reconstruction efficiency, with subleading and leading energy distributions arbitrarily normalized.  }
\label{fig:shower_reco_efficiency}
\end{figure}


\subsection{Data - MC Shower Comparison}
Reconstructed showers are compared between data and simulation in this section beginning with the number of reconstructed showers per event.  In Figure \ref{fig:physics_sel2_nshrs}, there is a large discrepancy in the 0 reconstructed showers bin, and better agreement for higher numbers of showers. It is difficult to draw a strong conclusion on the 0-bin disagreement due to the data-MC discrepancy observed at the CC Inclusive preselection stage; however, events with at least 1 shower will be used to tag $\pi^0$ candidates for the remainder of this note, where the agreement is considerably better.  An event mode breakdown for events with at least 1 reconstructed shower is shown in Figure \ref{fig:physics_gt0shower_eventtype}. Note that the MEC contribution to this remaining population of events is minimal.
 
\begin{figure}[H]
  \begin{subfigure}[t]{0.3\textwidth}
\includegraphics[scale=0.3]{Selection_II_Section/Physics_sel2_onoffseparate_nshrs.png}
  \caption{ }
  \end{subfigure} 
  \hspace{35 mm}
  \begin{subfigure}[t]{0.3\textwidth}
\includegraphics[scale=0.3]{Cluster_Shower/Physics_sel2_onoffseparate_nshrs_log.png}
  \caption{ }
  \end{subfigure} 
\caption{ Data to simulation comparison of number of reconstructed showers per event in a) linear scale and b) log scale. }
\label{fig:physics_sel2_nshrs}
\end{figure}

\begin{figure}[H]
\centering
\includegraphics[scale=0.35]{Cluster_Shower/Misc_gt0shower_EventType_vs_NeutrinoMode_w_Numbers.png}
\caption{ Event type breakdown for all events with at least 1 reconstructed shower. }
\label{fig:physics_gt0shower_eventtype}
\end{figure}


\par Next, reconstructed shower variables are compared. Because this comparison naturally excludes events falling into the 0-shower bin in Figure \ref{fig:physics_sel2_nshrs}, better data-MC agreement is expected in these comparisons than at the CC Inclusive selection. The rates and shapes of reconstructed showers agree relatively well in start point and direction (Figures \ref{fig:physics_sel2_shr_x} - \ref{fig:physics_sel2_shr_z}), opening angle of the shower, the distance from the vertex to the shower start point (Figures \ref{fig:physics_sel2_shr_rl}a, b), energy and corrected energy (Figures \ref{fig:physics_sel2_shr_e}a, b).  Note that the conversion distance distribution in Figure \ref{fig:physics_sel2_shr_rl}b drops off at the ROI boundary of 1 m.  Because the ROI is a rectangle rather than a circle, spill over above 1 m occurs where showers have converted in the corners of the ROI. 

\begin{figure}[H]
  \begin{subfigure}[t]{0.3\textwidth}
\includegraphics[scale=0.3]{Cluster_Shower/Physics_sel2_onoffseparate_shr_startx.png}
  \caption{ }
  \end{subfigure} 
  \hspace{34mm}
  \begin{subfigure}[t]{0.3\textwidth}
\includegraphics[scale=0.3]{Cluster_Shower/Physics_sel2_onoffseparate_shr_dirx.png}
  \caption{ }
  \end{subfigure} 
\caption{ Data to simulation comparisons of shower a) start point and b) direction in x.}
\label{fig:physics_sel2_shr_x}
\end{figure}


\begin{figure}[H]
  \begin{subfigure}[t]{0.3\textwidth}
\includegraphics[scale=0.3]{Cluster_Shower/Physics_sel2_onoffseparate_shr_starty.png}
  \caption{ }
  \end{subfigure} 
  \hspace{34mm}
  \begin{subfigure}[t]{0.3\textwidth}
\includegraphics[scale=0.3]{Cluster_Shower/Physics_sel2_onoffseparate_shr_diry.png}
  \caption{ }
  \end{subfigure} 
\caption{ Data to simulation comparisons of shower a) start point and b) direction in y.}
\label{fig:physics_sel2_shr_y}
\end{figure}


\begin{figure}[H]
  \begin{subfigure}[t]{0.3\textwidth}
\includegraphics[scale=0.3]{Cluster_Shower/Physics_sel2_onoffseparate_shr_startz.png}
  \caption{ }
  \end{subfigure} 
  \hspace{34mm}
  \begin{subfigure}[t]{0.3\textwidth}
\includegraphics[scale=0.3]{Cluster_Shower/Physics_sel2_onoffseparate_shr_dirz.png}
  \caption{ }
  \end{subfigure} 
\caption{ Data to simulation comparisons of shower a) start point and b) direction in z.}
\label{fig:physics_sel2_shr_z}
\end{figure}


\begin{figure}[H]
  \begin{subfigure}[t]{0.3\textwidth}
\includegraphics[scale=0.3]{Cluster_Shower/Physics_sel2_onoffseparate_shr_oangle.png}
  \caption{ }
  \end{subfigure} 
  \hspace{34mm}
  \begin{subfigure}[t]{0.3\textwidth}
\includegraphics[scale=0.3]{Cluster_Shower/Physics_sel2_onoffseparate_shr_vtx_dist.png}
  \caption{ }
  \end{subfigure} 
\caption{ Data to simulation comparison of shower in a) opening angle and b) start point to vertex distance. The cutoff at around 100~cm is due the ROI bound and the requirement that a shower cannot pierce or lay outside this bound, the tail beyond 1~m is due to the ROI being rectangular.}
\label{fig:physics_sel2_shr_rl}
\end{figure}


\begin{figure}[H]
  \begin{subfigure}[t]{0.3\textwidth}
\includegraphics[scale=0.3]{Cluster_Shower/Physics_sel2_onoffseparate_shr_energy.png}
  \caption{ }
  \end{subfigure} 
  \hspace{34mm}
  \begin{subfigure}[t]{0.3\textwidth}
\includegraphics[scale=0.3]{Cluster_Shower/Physics_sel2_onoffseparate_shr_energy_corr.png}
  \caption{ }
  \end{subfigure} 
\caption{ Data to simulation comparison of shower in a) energy and b) corrected energy.  }
\label{fig:physics_sel2_shr_e}
\end{figure}
