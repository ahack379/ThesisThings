\clearpage
\section{ $\nu_{\mu}$ CC Inclusive Selection(II) Cuts }

%\begin{table*} 
% \centering
% \captionof{table}{Corresponding POT (e20) in anatree and art root files used in this note.  Recall that OffBeam POT is less for Reco2 files due to failed grid jobs during processing of Selection II at this stage. \label{tab:pot}}
% \begin{tabular}{| l | l | l | l |}
%  \hline
%  POT (E20) & OnBeam & OffBeam & MCBNBCos \\ [0.1ex] \hline
%Anatree & 0.489 & 0.400 & 3.389 \\ 
%Reco2 & 0.492 & 0.227 & 4.232 \\ \hline
% \end{tabular}
%\end{table*}
\par In this section we begin to search for candidate events to perform the cross section analysis. This search is broken into three stages.  In the first stage, we search for charged current interactions. In this document, we refer to this as the "CC Inclusive" filter. The role of CC Inclusive is largely to reduce the cosmic background and select neutrino interactions in time with the beam spill that contain a forward-going $\mu$ candidate. When we've filted events with no candidate track that meets all CC Inclusive criteria, we run shower reconstruction on the remaining events.  This involves first identifying electromagnetic-like candidate hit activity, clustering these candidate hits together with open-source computer vision tools, matching clusters across planes and performing reconstructing the clusters into shower candidate.  Finally, we consider two parallel $\pi^0$ candidate selections: those that require two reconstructed showers in the interactions, and those that require one.  We perform our final cross section analysis on both samples.

%\par Throughout the next few sections, we will show data-MC comparisons at each stage of the selection chain.  These comparisons

\subsection{Signal and Backgrounds Descriptions}
\par We begin our discussion of the CC Inclusive filter with a set of signal and background definitions, and their evolutions across the different filters we employ.  These categories are used to tune analysis cuts and present resulting distributions after the filter is applied.  These backgrounds will be grouped in tabular form throughout this document into the following categories: Signal, $\nu_\mu$ CC-0$\pi^0$, $\nu_\mu$ NC-0$\pi^0$, $\nu_\mu$ NC-$\pi^0$, Other, Cosmic - $\nu$ Coincident, and Cosmic - In Time (Data). Backgrounds will be further broken down into subcategories in data-MC comparison histograms to give the reader a more in-depth look at the selected interactions.  These categories (and sub-categories for the CC Inclusive portion of the selection) are described here. Note that all signal and background categories except for the data-extracted ``Cosmic" are defined by the GENIE final state information stored for each interaction.

%We do this for the first part of the analysis (Selection II) by using truth information to track the origin of the candidate $\mu$. We will revisit the background breakdown once we've reconstructed showers and have this additional truth information.

\paragraph{Signal - $\nu_\mu$ CC 1$\pi^0$} The Signal category describes events with a single $\mu^-$, single final state $\pi^0$, any additional particles (except a second final state $\pi^0$), and a FV-contained true-interaction vertex.  Note that this is the only category with a FV requirement.

\paragraph{ $\nu_\mu$ CC 0$\pi^0$}
The CC 0$\pi^0$ background describes charged current interactions with a single $\mu^-$, and 0 observable $\pi^0$'s originating from the interaction vertex.  Sub-categories of this background include CC $\pi^{\pm}$ charge exchange (`CC Cex'), interactions with 1 or more photons in the final state (`X$\rightarrow$ $\geq$ 1$\gamma$'), and other charged current (`CC Other') interactions. CC 0$\pi^0$ is the dominant background at the Selection II stage as we haven't yet reconstructed and selected for events with $\pi^0$'s.

\paragraph{  $\nu_\mu$ NC 0$\pi^0$}
The NC 0$\pi^0$ background describes neutral current events with no observable $\pi^0$ originating from the vertex.  NC 0$\pi^0$ will be shown in lime green in Selection II data-MC stacked histograms. 

%electromagnetic activity at the vertex, charge exchange, $K^\pm$ decay, $\eta$ decay, proton and neutron inelastic events. 

\paragraph{  $\nu_\mu$ NC $\pi^0$}
The NC $\pi^0$ background includes neutral current events in which 1 or more $\pi^0$'s originate at the neutrino vertex.  This background will be shown in dark green in data-MC stacked histograms.

%This is the dominant background at the final stage of the analysis.

\paragraph{ Other}
This category of events includes CC 1$\pi^0$ events with true vertex outside the fiducial volume, CC events with multiple $\pi^0$'s originating from the vertex, $\overline{\nu}_\mu$ and $\nu_e$-induced interactions.  These backgrounds will be shown in shades of purple and yellow in data-MC stacked histograms.

\paragraph{ Cosmic + Neutrino }
In this class of events, we have a neutrino interacting somewhere in the cryostat, but we have instead selected a cosmic ray as our candidate $\mu$.  This background is shown in dark grey in data-MC stacked histograms.  

\paragraph{Cosmic }
Data cosmics are selected and scaled from External BNB (EXTBNB) data.  This background describes events in which a cosmic interacts in the detector during the beam spill window when there is no neutrino interaction anywhere in the TPC.  This background will be shown in light grey in all data-MC stacked histograms.


%\subsection{Sample Normalization}
%In order to compare MC and data after various cut stages, we must normalize the POT of the MC and OffBeam samples to the OnBeam sample. Starting sample POT's are summarized in Table \ref{tab:pot}.  We calculate these POT's and other scaling factors for On and OffBeam with a script provided by Zarko \cite{bib:zarkopot}; these factors are summarized in Table \ref{tab:ext_1D}. The scalings are described below, and will be updated using full Run1 data and MC when available.\\
%
%\noindent \textbf{ONBEAM POT} 
%
%\noindent \textbf{Anatree:} 
%%\begin{equation}
%%  POT_{ON} = \frac{N_{ON}}{N^{4.95E19}_{ON}} * 0.495 
%%\end{equation}
%
%%\noindent where $N_{ON}$ is the number of OnBeam events in the utilized sample (540817) and $N^{4.95E19}_{ON}$ is the number of OnBeam events equivalent to $4.95E19$ (567157).
%The Anatree OnBeam POT is calculated using the OnBeam file list passed to the anatree-version of SelectionII. This list can be found here: 
%\par /uboone/app/users/ahack/MCC8\_3/OnBeam.txt 
%
%\noindent The Anatree OnBeam POT is found to be 0.487e20.
%
%\noindent \textbf{Reco2:} The Reco2 OnBeam POT is calculated using the json file output from grid processing. The OnBeam POT is found to be 0.492e20.
%
%\noindent \textbf{OFF to ONBEAM POT} 
%
%\noindent \textbf{Anatree}: \\
%To scale the OffBeam sample to OnBeam, we divide the number of triggers in our OnBeam sample (10917359) by the number of triggers in our OffBeam sample (8926439). The resulting scale factor to go from Off to OnBeam is 1.2230.
%
%The OffBeam file list can be found here: 
%\par /uboone/app/users/ahack/MCC8\_3/OffBeam.txt 

%\noindent \textbf{Reco2:}\\
%To scale the OffBeam sample to OnBeam, we divide the number of triggers in our OnBeam sample (10993739) by the number of triggers in our OffBeam sample (5069967). The resulting scale factor for Reco2 files is 2.1684.
%
%\noindent \textbf{MC to ONBEAM POT} 
%\noindent \textbf{Anatree:} 
%\noindent POT is currently not stored in anatree.  Thus, to calculate MC POT for the anatree portion of the analysis, we use the scaling factor derived externally \cite{bib:normdatamc} :
%%\begin{equation}
%\begin{align}
%  POT_{MC} &= \frac{N_{MC}}{99035.2} \\
%  &= 3.389e20
% \end{align}
%%\end{equation}
%
%where $N_{MC}$ is the number of events in the original sample before any cuts are performed. Note that the anatree stage is only used for cut tuning, not for extracting final efficiencies, purities or event rates.  Thus, this method is sufficient for our needs.
%
%\noindent \textbf{Reco2:}
%For the Reco2 portion of the analysis, the MC POT is calculated by summing over the POT in all utilized subruns.  The MC POT is 4.232e20.
%
%\noindent  In both files types, the scaling factor for MC to OnBeam is given by:
%\begin{equation}
%  f_{MC} = \frac{POT_{ON}}{POT_{MC}} 
%\end{equation}

%\begin{table*} 
% \centering
% \captionof{table}{Results of running Zarko's Script for Anatree and Reco2 Corresponding POT (e20) in anatree and art root files used in this note. \label{tab:ext_1D}}
% \begin{tabular}{| l | l | l | l | l | l | l |  }
%  \hline
% Sample  & \multicolumn{1}{c}{} & \multicolumn{1}{c}{Anatree}  & &\multicolumn{1}{c}{}  & \multicolumn{1}{c}{Reco2} &   \\ \hline
% & EXT & E1DCNT\_wcut & tor875\_wcut & EXT  & E1DCNT\_wcut & tor875\_wcut \\ [0.1ex] \hline
%OnBeam & - & 10917359 & 4.89E19 & - & 10993739 & 4.92E19 \\ 
%OffBeam & 8926439 & - & - & 5069967 & - & - \\ \hline
% \end{tabular}
%\end{table*}

\subsection{Tuning Selection II}
In this section we describe first the Selection II stage of the analysis.  Here we consider the following producers: simpleFlashBeam flashes, pandoraNu tracks, pandoraNu vertices, and pandoraNucalo calorimetry.

\noindent \paragraph{Beam Windows} An important first step is to understand the beam window locations. These boundaries are different for each sample and are necessary for checking that a 40 PE flash has occurred in coincidence with the beam. The window is found by plotting the flash times with respect to the trigger and empirically setting the beam gate boundaries by the leading edge. The OnBeam and OffBeam windows are offset due to differing trigger condition configurations. Specifically, OffBeam has a configured delay between the trigger time and the beginning of the waveform of 400 ns larger than OnBeam. Note that the cosmic samples contain an exponential feature near the trigger. This is due to late light from cosmic interactions occurring before the trigger which causes a flash pile up near the beginning of the window. More detail on distribution features is discussed externally \cite{bib:davidcpot}. These windows are shown in Figure \ref{fig:misc_beamwindows} and summarized in Table \ref{tab:windows}.

\begin{figure}[h!]
%\centering
\begin{subfigure}[b]{0.3\textwidth}
\includegraphics[scale=0.28]{Selection_II_Section/CutJustify_sel2_MCBNBCos_beamwindow.png}
\caption{ }
\end{subfigure}
\hspace{3mm}
\begin{subfigure}[b]{0.3\textwidth}
\includegraphics[scale=0.28]{Selection_II_Section/CutJustify_sel2_OnBeam_beamwindow.png}
\caption{ }
\end{subfigure}
\hspace{3mm}
\begin{subfigure}[b]{0.3\textwidth}
\includegraphics[scale=0.28]{Selection_II_Section/CutJustify_sel2_OffBeam_beamwindow.png}
\caption{ }
\end{subfigure}
\caption{Time associated with a flash since the trigger time in the ``beam readout''. Using these distributions we can isolate when the beam windows occur for a) MC BNB + Cosmics simulation, b) On Beam data, and c) Off Beam data.}
\label{fig:misc_beamwindows}
\end{figure}

\begin{table} 
 \centering
 \captionof{table}{MCC8 Beam Window Locations\label{tab:windows}}
 \begin{tabular}{| l | l | l |}
  \hline
 Sample & Lower Bound [$\mu$s] & Upper Bound [$\mu$s] \\ [0.1ex] \hline
MCBNBCos & 3.2 & 4.8 \\ 
OnBeam & 3.3 & 4.9   \\ 
OffBeam & 3.65 & 5.25 \\ \hline
\end{tabular}
\end{table}

%neutrino interactions without electromagnetic activity in the final state, notably $\nu_\mu$ CC and NC 0-$\pi^0$.

% Need data-MC flash cut comparison plot
\paragraph{Flash Cuts}
Once we've found the beam windows, we can begin to filter down our samples. The first cut in Selection II requires that there be a flash of 40 PE or greater in the beam window. The highest PE flash per event is shown per neutrino interaction in an MCBNB + Cosmics sample in Figure \ref{fig:cutjust_sel2_numPE}a. A zoom into the low PE region in Figure \ref{fig:cutjust_sel2_numPE}b shows that this cut has a stronger effect on our background distributions. This effect is largely due to the FV requirement: interactions tend to have higher reconstructed PE flashes when they occur directly in front of the PMT array, then in the surrounding FV regions. To minimize the impact of the systematic effect of our scintillation mismodeling \cite{bib:kazu_optical_bug} we have opted for a conservatively loose cut of 40 PE. A data-MC comparison is also shown in Figure \ref{fig:cutjust_sel2_numPE_datamc}.
\par We next move to matching candidate pandoraNu tracks to the flash we just identified.  Flash matching considers the weighted-PE $z$ position of the highest PE flash found in the event as well as the start and end $z$ points of all candidate tracks. The distance between the weighted PE flash $z$ position and the start or end of the track must be $\leq$ 70 cm in order for the track to pass. \textbf{If the weighted $z$ position of the flash is between the start and end point of a candidate track, the distance is set to 0 cm}. A data-MC comparison at this cut can be seen in Figure \ref{fig:cutjust_sel2_flashtrkdist}a, with background distributions in Figure \ref{fig:cutjust_sel2_flashtrkdist}b.  A zoom in on the lower-distance region can be found in Figures \ref{fig:cutjust_sel2_flashtrkdist_zoom}a and b. 

%A more sophisticated flash-matching technique of minimizing the observed flash PE distribution with respect to optical simulation hypothesis for each candidate track is in progress and will be used in future iterations of CC Selection \cite{bib:marco_selection}.

\begin{figure}[t!]
  \centering
  \begin{subfigure}[t]{0.4\textwidth}
    \centering
    \includegraphics[scale=0.4]{Selection_II_Section/CutJustify_sel2_numPE.png}
    \caption{ }
  \end{subfigure}
    \hspace{10 mm}
   \begin{subfigure}[t]{0.4\textwidth}
    \centering
    \includegraphics[scale=0.4]{Selection_II_Section/CutJustify_sel2_numPE_zoom.png}
    \caption{ }
  \end{subfigure}
\caption{ MicroBooNE simulation of broken down into signal and background categories for a) Flash distribution per neutrino interaction; b) A zoom into low PE region. }
\label{fig:cutjust_sel2_numPE}
\end{figure}


\begin{figure}[t!]
  \centering
  \begin{subfigure}[t]{0.4\textwidth}
    \centering
    \includegraphics[scale=0.4]{Selection_II_Section/CutJustify_sel2_datamc_numPE.png}
    \caption{ }
  \end{subfigure} 
  \hspace{10 mm}
  \begin{subfigure}[t]{0.4\textwidth}
    \centering
    \includegraphics[scale=0.4]{Selection_II_Section/CutJustify_sel2_datamc_numPE_zoom.png}
    \caption{ }
  \end{subfigure}
\caption{ MicroBooNE data to simulation comparison of the a) flash distribution and b) A zoom into the interesting region. }
\label{fig:cutjust_sel2_numPE_datamc}
\end{figure}

\begin{figure}[h!]
\centering
  \begin{subfigure}[t]{0.4\textwidth}
    \centering
\includegraphics[scale=0.4]{Selection_II_Section/CutJustify_sel2_datamc_flashtrkdist.png}
    \caption{ }
  \end{subfigure} 
  \hspace{10 mm}
  \begin{subfigure}[t]{0.4\textwidth}
    \centering
\includegraphics[scale=0.4]{Selection_II_Section/CutJustify_sel2_flashtrkdist.png}
    \caption{ }
  \end{subfigure} 
\caption{ a) Data to simulation comparison for the flash match cut; b) Breakdown of backgrounds at flash match stage. }
\label{fig:cutjust_sel2_flashtrkdist}
\end{figure}

\begin{figure}[h!]
\centering
  \begin{subfigure}[t]{0.4\textwidth}
    \centering
\includegraphics[scale=0.35]{Selection_II_Section/CutJustify_sel2_datamc_flashtrkdist_zoom.png}
    \caption{ }
  \end{subfigure} 
  \hspace{10 mm}
  \begin{subfigure}[t]{0.4\textwidth}
    \centering
\includegraphics[scale=0.35]{Selection_II_Section/CutJustify_sel2_flashtrkdist_zoom.png}
   \caption{ }
  \end{subfigure} 
\caption{ Zoom in to the a) Data to simulation comparison for the flash match cut and b) breakdown of backgrounds at flash match stage. }
\label{fig:cutjust_sel2_flashtrkdist_zoom}
\end{figure}

\paragraph{Track-Vertex Association + $\frac{dE}{dx}$ Correction}
At this point we have a preliminary set of flash-matched candidate $\mu^-$ tracks. We next want to associate these flash-match candidates with pandoraNu candidate vertices in the event. This is done by comparing the distance between the start (or end) of each track with each pandoraNu candidate vertex. If this distance is less than 3 cm, a track-vertex association is built.  The number of tracks associated to each vertex at the end of this procedure is called the ``vertex multiplicity''; a data-MC comparison and a break down of backgrounds at this vertex-association stage are shown in Figures \ref{fig:cutjust_sel2_vtxtrackdist}a and b.  The (unstacked) multiplicity distributions after this association are shown in Figures \ref{fig:cutjust_sel2_mult} a (normal) and b (log).

\begin{figure}[h!]
\centering
  \begin{subfigure}[t]{0.4\textwidth}
    \centering
\includegraphics[scale=0.4]{Selection_II_Section/CutJustify_sel2_datamc_vtxtrkdist.png}
    \caption{ }
  \end{subfigure} 
  \hspace{10 mm}
  \begin{subfigure}[t]{0.4\textwidth}
    \centering
\includegraphics[scale=0.4]{Selection_II_Section/CutJustify_sel2_vtxtrackdist.png}
    \caption{ }
  \end{subfigure} 
\caption{ a) Data to simulation comparison for the vertex-track distance ; b) Breakdown of backgrounds at vertex-track distance cut. }
\label{fig:cutjust_sel2_vtxtrackdist}
\end{figure}

\begin{figure}[h!]
\centering
  \begin{subfigure}[t]{0.4\textwidth}
    \centering
\includegraphics[scale=0.4]{Selection_II_Section/CutJustify_sel2_mult.png}
    \caption{ }
  \end{subfigure} 
  \hspace{10 mm}
  \begin{subfigure}[t]{0.4\textwidth}
    \centering
\includegraphics[scale=0.4]{Selection_II_Section/CutJustify_sel2_multlog.png}
    \caption{ }
  \end{subfigure} 
\caption{ Track multiplicity distribution after vertex-track association in a) linear and b) log form. }
\label{fig:cutjust_sel2_mult}
\end{figure}

\par  Next we consider the mean energy deposition over distance of our tracks.  For convenience, we first describe the calorimetric calculation procedure. The track $\frac{dE}{dx}$ and $\frac{dQ}{dx}$ are calculated via the larreco Calorimetry module during production.  The $\frac{dQ}{dx}$ per track-associated-hit, per plane is calculated by taking the integral of the hit (ADC's) and dividing it by the pitch (wire spacing divided by dot product of the 3D track direction and wire plane direction). To convert this $\frac{dQ}{dx}$ from $\frac{ADC}{cm}$ to $\frac{e^-}{cm}$, we next apply the Calibrations-Team-extracted conversion constants in the default calorimetry fcl file in the v06\_26\_01\_06 version of uboonecode (calorimetry\_microboone.fcl). The MC constants are 5.029e-3, 5.088e-3 and 5.183e-3 $\frac{ADC}{e^-}$ for planes 0, 1 and 2 respectively.  The data constants are 4.746e-3, 4.7e-3 and 4.688e-3 $\frac{ADC}{e^-}$ for planes 0, 1 and 2.  These constants were extracted using MCC7 files, as noted in a fcl-file comment above the constants.  The calibrations group is actively working to extract new calorimetry constants for MCC8.3 that we will integrate once they are released. 
To get to $\frac{dE}{dx}$, a lifetime correction is first applied to the $\frac{dQ}{dx}$ computed in the previous step with electron lifetime 1 s. When lower electron lifetime data is added, this will be modified.  A recombination correction derived from the ``Box Model'' at a field of 273 $\frac{V}{cm}$ is also applied \cite{bib:argoneut_recomb}. % {\color{red} \bf REFERENCE NEEDED}.

%can consider each interaction by a set of individual multiplicity cuts we need to consider two additional items. First, 
\par Before we use the $\frac{dE}{dx}$ described above, we consider two additional items. First, we must choose a plane in which to consider the $\frac{dE}{dx}$.  We do this by selecting the plane which has the most track-related hits; we call this the ``best plane''. A breakdown of backgrounds and data-MC comparison for best plane is shown in Figures \ref{fig:cutjust_sel2_bestplane}a and b. Second, we correct the $\frac{dE}{dx}$ in the selected plane for any variation over drift distance. It is beneficial to shift things as little as possible in these distributions while we perform the corrections. Thus, we choose to shift to a value that already aligns fairly well with the distributions to avoid distorting our results more than necessary.  These corrections are found by first considering the MC $\frac{dE}{dx}$ vs x distributions in Figure \ref{fig:cutjust_sel2_mc_dedx_v_x}. In particular, we look at the collection plane. We empirically determine the peak value of the uncorrected, but mostly-flat-over-x, collection plane view to be 1.63 $\frac{MeV}{cm}$; this value was also used in the original work that was done by Tingjun on improving Selection II after Neutrino 2016. With our chosen 1.63 $\frac{MeV}{cm}$ baseline in hand, we proceed by fitting a line to the $\frac{dE}{dx}$ vs $x$ distributions and shifting the distributions using the baseline in conjunction with this fit. These correction equations as a function of x are summarized for all three samples in Table \ref{tab:dedx_corr}. These corrections were previously shown to improve the performance of Selection II \cite{bib:6172}. Examples of these before- and after-correction distributions for all three planes of the MCBNBCos sample are shown in Figure \ref{fig:cutjust_sel2_mc_dedx_v_x}, OnBeam in Figure \ref{fig:cutjust_sel2_onbeam_dedx_v_x}, and OffBeam in Figure \ref{fig:cutjust_sel2_offbeam_dedx_v_x}. The corresponding corrected 1D distributions are shown in Figure \ref{fig:cutjust_sel2_1d_dedx_v_x}.  

\begin{table} 
 \centering
 \captionof{table}{$\frac{dE}{dx}$ correction equations as a function of $x$.  Note that the corrections for On and OffBeam are the same per plane. \label{tab:dedx_corr}}
 \begin{tabular}{| l | l | l |}
  \hline
  & OnBeam and OffBeam & MCBNB+Cosmics \\ [0.1ex] \hline
Plane 0 & $\frac{1.63}{-0.0003125x + 1.53}$ & $\frac{1.63}{0.0009766x + 1.70}$ \\ \hline
Plane 1 & $\frac{1.63}{0.0027344x + 2.50}$ & $\frac{1.63}{0.0009766x + 1.70}$ \\ \hline 
Plane 2 & $\frac{1.63}{0.0001953x + 1.40}$ & $\frac{1.63}{0.0007812x + 1.60}$ \\  \hline
\end{tabular}
\end{table}


\begin{figure}[t!]
\centering
  \begin{subfigure}[t]{0.3\textwidth}
    \centering
\includegraphics[scale=0.3]
{Selection_II_Section/CutJustify_sel2_bestplane.png}
    \caption{ }
  \end{subfigure} 
  \hspace{15mm}
  \begin{subfigure}[t]{0.35\textwidth}
    \centering
\includegraphics[scale=0.35]
{Selection_II_Section/CutJustify_sel2_datamc_bestplane.png}
    \caption{ }
  \end{subfigure} 
\caption{Best selected plane shown for a) signal and all backgrounds and b) MC-data comparison }
\label{fig:cutjust_sel2_bestplane}
\end{figure}

\begin{figure}[h!]
\centering
\includegraphics[scale=0.25]{Selection_II_Section/CutJustify_sel2_MC_hdEdxVsX_2_0_0.png}
\hspace{1 mm}
\includegraphics[scale=0.25]{Selection_II_Section/CutJustify_sel2_MC_hdEdxVsX_2_0_1.png}
\hspace{1 mm}
\includegraphics[scale=0.25]{Selection_II_Section/CutJustify_sel2_MC_hdEdxVsX_2_0_2.png}
\hspace{1 mm}
\includegraphics[scale=0.25]{Selection_II_Section/CutJustify_sel2_MC_hdEdxVsXCor_2_0_0.png}
\hspace{1 mm}
\includegraphics[scale=0.25]
{Selection_II_Section/CutJustify_sel2_MC_hdEdxVsXCor_2_0_1.png}
\hspace{1 mm}
\includegraphics[scale=0.25]
{Selection_II_Section/CutJustify_sel2_MC_hdEdxVsXCor_2_0_2.png}
\caption{MCBNB+Cosmics candidate tracks' $\frac{dE}{dx}$ at each point along track before (top row) and after (bottom row) $\frac{dE}{dx}$ correction for planes 0, 1, 2 (left to right).  The empirical line extracted to do this intermediate calibration is shown in the top row. }
\label{fig:cutjust_sel2_mc_dedx_v_x}
\end{figure}


\begin{figure}[h!]
\centering
\includegraphics[scale=0.25]{Selection_II_Section/CutJustify_sel2_OnBeam_hdEdxVsX_0_0_0.png}
\hspace{1 mm}
\includegraphics[scale=0.25]{Selection_II_Section/CutJustify_sel2_OnBeam_hdEdxVsX_0_0_1.png}
\hspace{1 mm}
\includegraphics[scale=0.25]{Selection_II_Section/CutJustify_sel2_OnBeam_hdEdxVsX_0_0_2.png}
\hspace{1 mm}
\includegraphics[scale=0.25]{Selection_II_Section/CutJustify_sel2_OnBeam_hdEdxVsXCor_0_0_0.png}
\hspace{1 mm}
\includegraphics[scale=0.25]
{Selection_II_Section/CutJustify_sel2_OnBeam_hdEdxVsXCor_0_0_1.png}
\hspace{1 mm}
\includegraphics[scale=0.25]
{Selection_II_Section/CutJustify_sel2_OnBeam_hdEdxVsXCor_0_0_2.png}
\caption{OnBeam candidate tracks' $\frac{dE}{dx}$ at each point along track before (top row) and after (bottom row) $\frac{dE}{dx}$ correction for planes 0, 1, 2 (left to right).  The empirical line extracted to do this intermediate calibration is shown in the top row. }
\label{fig:cutjust_sel2_onbeam_dedx_v_x}
\end{figure}

\begin{figure}[h!]
\centering
\includegraphics[scale=0.25]{Selection_II_Section/CutJustify_sel2_OffBeam_hdEdxVsX_1_0_0.png}
\hspace{1 mm}
\includegraphics[scale=0.25]{Selection_II_Section/CutJustify_sel2_OffBeam_hdEdxVsX_1_0_1.png}
\hspace{1 mm}
\includegraphics[scale=0.25]{Selection_II_Section/CutJustify_sel2_OffBeam_hdEdxVsX_1_0_2.png}
\hspace{1 mm}
\includegraphics[scale=0.25]{Selection_II_Section/CutJustify_sel2_OffBeam_hdEdxVsXCor_1_0_0.png}
\hspace{1 mm}
\includegraphics[scale=0.25]
{Selection_II_Section/CutJustify_sel2_OffBeam_hdEdxVsXCor_1_0_1.png}
\hspace{1 mm}
\includegraphics[scale=0.25]
{Selection_II_Section/CutJustify_sel2_OffBeam_hdEdxVsXCor_1_0_2.png}
\caption{OffBeam candidate tracks' $\frac{dE}{dx}$ at each point along track before (top row) and after (bottom row) $\frac{dE}{dx}$ correction for planes 0, 1, 2 (left to right).  The empirical line extracted to do this intermediate calibration is shown in the top row. }
\label{fig:cutjust_sel2_offbeam_dedx_v_x}
\end{figure}


\begin{figure}[h!]
\centering
\includegraphics[scale=0.25]{Selection_II_Section/CutJustify_sel2_hdEdxVsX_0_0_0_plane_0.png}
\hspace{1 mm}
\includegraphics[scale=0.25]{Selection_II_Section/CutJustify_sel2_hdEdxVsX_0_0_1_plane_1.png}
\hspace{1 mm}
\includegraphics[scale=0.25]{Selection_II_Section/CutJustify_sel2_hdEdxVsX_0_0_2_plane_2.png}
\hspace{1 mm}
\includegraphics[scale=0.25]{Selection_II_Section/CutJustify_sel2_hdEdxVsXCor_0_0_0_plane_0.png}
\hspace{1 mm}
\includegraphics[scale=0.25]
{Selection_II_Section/CutJustify_sel2_hdEdxVsXCor_0_0_1_plane_1.png}
\hspace{1 mm}
\includegraphics[scale=0.25]
{Selection_II_Section/CutJustify_sel2_hdEdxVsXCor_0_0_2_plane_2.png}
\caption{One dimensional area-normalized distributions for MC (BNB+Cosmics) and Data (OnBeam). The uncorrected distributions (top) have aligned peaks after intermediate calibration (bottom) in planes 0, 1, 2 (left to right). }
\label{fig:cutjust_sel2_1d_dedx_v_x}
\end{figure}


At this point, we have a set of candidate interactions with different multiplicities.  We now subject these interactions to a series of cuts determined by multiplicity, in order to take advantage of multiplicity-specific features. Unless otherwise noted, these cuts are determined empirically to maximize CC single $\pi^0$ selection while also minimizing cosmic contamination. These cuts are described in detail below.

\paragraph{Multiplicity 1 Cuts}
% Why so many single track cosmics passing the FV cut? Are they mainly broken/ mis-reconstructed?
Multiplicity 1 events are most affected by the cosmic background because they only have 1 track associated to the reconstructed vertex. To account for this, we require that the candidate track be fully contained in a FV; this removes most exiting and entering cosmics that are well reconstructed. The FV used here is 20 cm from the boundaries in X and Y, and 10 cm from the boundaries in Z; this is the FV used in the Neutrino2016 CC Inclusive analysis. To further limit cosmic contamination from broken tracks, tracks that pass dead regions of the detector, and mis-reconstructed tracks, we next consider the track length vs track y directional component. We first cut all interactions with a cos($\theta$) of $>$ 0.4 and a length of candidate track $<$ 15cm. 1D distributions of these cut are shown in Figures \ref{fig:cutjust_sel2_mult1_cosy}a, b and Figures \ref{fig:cutjust_sel2_mult1_len}a, b. Note that these cuts, particularly that on cos($\theta$), remove a significant fraction of cosmic contamination, and shift the dominant background in the multiplicity 1 category to CC-0$\pi^0$. 2D distributions showing the effect of these cuts for CC 1$\pi^0$ signal and various backgrounds are shown in Figure \ref{fig:cutjust_sel2_mult1_len_v_cosy}. On and OffBeam distributions are shown in Figure \ref{fig:cutjust_sel2_onbeam_mult1_len_v_cosy}. 
\par Finally, we compare the ratio of the track's starting average $\frac{dE}{dx}$ to ending average $\frac{dE}{dx}$ to the projected y length of the track. These comparisons are motivated by cosmics, the primary background of multiplicity 1 events. Cosmic muons generally downward going, they leave a long projection in the vertical direction. Additionally, because there is a containment criterion on the candidate vertex, the majority of this cosmic background consists of muons that stop or decay inside the FV. These tracks deposit more energy near the $\mu$'s end (the Bragg peak) than along the rest of the length of the track (Figure \ref{fig:cut_ex_0}). Thus, we can use the projected length and $\frac{dE}{dx}$ ratio as handles to identify cosmics. The $\frac{dE}{dx}$ is averaged (with scaling applied for drift distance, as discussed earlier) over the first 10 hits of the track in the best plane to calculate the start $<\frac{dE}{dx}>$. A similar calculation is performed to calculate end $<\frac{dE}{dx}>$.  If a track has less than 20 associated hits, the number of hits used for each calculation is half the total hits available.  Next, a start to end $<\frac{dE}{dx}>$ ratio is taken; if the track's start point occurs higher in y than the end point, the ratio is flipped. We remove candidates with either a) a ratio $>$ 1.5 or b) a ratio $<=$ 1.5 and a projected track length less than 25cm.  This cut is illustrated in Fig. 31.

\begin{figure}[h!]
\centering
\includegraphics[scale=0.9]{Selection_II_Section/cut_ex_0.png}
\caption{An example of the motivating event topology for the ratio cut on the start $<\frac{dE}{dx}>$ to end $<\frac{dE}{dx}>$.  The goal of this cut is to remove cosmic events with crossing muons that stop and decay in the detector. If the Michel electron is not reconstructed, the stopping muon’s candidate vertex will have multiplicity 1.  In this case, because we require the vertex be contained in the FV, the vertex/start point will be at the muon stop point such that the Bragg peak sits at the vertex. }
\label{fig:cut_ex_0}
\end{figure}

These 1D distributions are shown in Figures \ref{fig:cutjust_sel2_mult1_ratiodedx}-\ref{fig:cutjust_sel2_mult1_projylen}; note that as expected, CC-0$\pi^0$ dominates at this cut stage. 2D distributions of all backgrounds are shown in Figure \ref{fig:cutjust_sel2_mult1_dedxratio_v_leny}, and of On and OffBeam in Figure \ref{fig:cutjust_sel2_onbeam_mult1_dedxratio_v_leny}.

\begin{figure}[h!]
\centering
  \begin{subfigure}[t]{0.4\textwidth}
    \centering
\includegraphics[scale=0.4]{Selection_II_Section/CutJustify_sel2_mult1_dcosy.png}
    \caption{ }
  \end{subfigure} 
  \hspace{20mm}
  \begin{subfigure}[t]{0.4\textwidth}
    \centering
\includegraphics[scale=0.4]{Selection_II_Section/CutJustify_sel2_datamc_mult1_dcosy.png}
    \caption{ }
  \end{subfigure} 
\caption{Multiplicity 1 candidate directional y component shown for a) signal and all backgrounds and b) MC-data comparison.  Multiplicity 1 candidates are required to have a y-directional component less than or equal to 0.4. }
\label{fig:cutjust_sel2_mult1_cosy}
\end{figure}

\begin{figure}[h!]
\centering
  \begin{subfigure}[t]{0.4\textwidth}
    \centering
\includegraphics[scale=0.4]{Selection_II_Section/CutJustify_sel2_mult1_tracklen.png}
    \caption{ }
  \end{subfigure} 
  \hspace{20mm}
  \begin{subfigure}[t]{0.4\textwidth}
    \centering
\includegraphics[scale=0.4]{Selection_II_Section/CutJustify_sel2_datamc_mult1_tracklen.png}
    \caption{ }
  \end{subfigure} 
\caption{Multiplicity 1 candidate track length shown for a) signal and all backgrounds and b) MC-data comparison }
\label{fig:cutjust_sel2_mult1_len}
\end{figure}

\begin{figure}[t!]
\centering
  \begin{subfigure}[t]{0.25\textwidth}
    \centering
    \includegraphics[scale=0.25]{Selection_II_Section/CutJustify_sel2_mult1_length_v_cosy_All.png}
    \caption{ }
  \end{subfigure} 
  \hspace{1mm}
  \begin{subfigure}[t]{0.25\textwidth}
    \centering
    \includegraphics[scale=0.25]{Selection_II_Section/CutJustify_sel2_mult1_length_v_cosy_Cosmic.png}
    \caption{ }
  \end{subfigure} 
  \hspace{1mm}
  \begin{subfigure}[t]{0.25\textwidth}
    \centering
    \includegraphics[scale=0.25]{Selection_II_Section/CutJustify_sel2_mult1_length_v_cosy_CC1pi0.png}
    \caption{ }
  \end{subfigure} 
  \hspace{1 mm}
  \begin{subfigure}[t]{0.25\textwidth}
    \centering
\includegraphics[scale=0.25]{Selection_II_Section/CutJustify_sel2_mult1_length_v_cosy_CC0pi0.png}
    \caption{ }
  \end{subfigure} 
  \hspace{1 mm}
  \begin{subfigure}[t]{0.25\textwidth}
    \centering
\includegraphics[scale=0.25]{Selection_II_Section/CutJustify_sel2_mult1_length_v_cosy_NC1pi0.png}
    \caption{ }
  \end{subfigure} 
  \hspace{1 mm}
  \begin{subfigure}[t]{0.25\textwidth}
    \centering
\includegraphics[scale=0.25]{Selection_II_Section/CutJustify_sel2_mult1_length_v_cosy_NC0pi0.png}
    \caption{ }
  \end{subfigure} 
  \hspace{1 mm}
  \begin{subfigure}[t]{0.25\textwidth}
    \centering
\includegraphics[scale=0.25]{Selection_II_Section/CutJustify_sel2_mult1_length_v_cosy_Other.png}
    \caption{ }
  \end{subfigure} 

\caption{Multiplicity 1 track length vs y directional component distribution with cut shown in pink for a) All candidates; b) Cosmics; c) CC 1$\pi^0$; d) CC 0$\pi^0$; e) NC $\pi^0$; f) NC 0$\pi^0$; g) Other }
\label{fig:cutjust_sel2_mult1_len_v_cosy}
\end{figure}

\begin{figure}[h!]
\centering
  \begin{subfigure}[t]{0.25\textwidth}
    \centering
\includegraphics[scale=0.25]{Selection_II_Section/CutJustify_sel2_mult1_length_v_cosy_OnBeam.png}
    \caption{ }
  \end{subfigure} 
  \hspace{20mm}
  \begin{subfigure}[t]{0.25\textwidth}
    \centering
    \includegraphics[scale=0.25]{Selection_II_Section/CutJustify_sel2_mult1_length_v_cosy_OffBeam.png}
    \caption{ }
  \end{subfigure} 

\caption{Multiplicity 1 track length vs y directional component distribution with cut shown in pink for a) OnBeam and b) OffBeam }
\label{fig:cutjust_sel2_onbeam_mult1_len_v_cosy}
\end{figure}


\begin{figure}[h!]
\centering
  \begin{subfigure}[t]{0.4\textwidth}
    \centering
\includegraphics[scale=0.4]{Selection_II_Section/CutJustify_sel2_mult1_ratiodedx.png}
    \caption{ }
  \end{subfigure} 
  \hspace{20mm}
  \begin{subfigure}[t]{0.4\textwidth}
    \centering
\includegraphics[scale=0.4]{Selection_II_Section/CutJustify_sel2_datamc_mult1_ratiodedx.png}
    \caption{ }
  \end{subfigure} 
\caption{Multiplicity 1 ratio of high to low dEdx shown for a) signal and all backgrounds and b) MC-data comparison }
\label{fig:cutjust_sel2_mult1_ratiodedx}
\end{figure}

\begin{figure}[h!]
\centering
  \begin{subfigure}[t]{0.4\textwidth}
    \centering
\includegraphics[scale=0.4]{Selection_II_Section/CutJustify_sel2_mult1_ratiodedx_log.png}
    \caption{ }
  \end{subfigure} 
  \hspace{20mm}
  \begin{subfigure}[t]{0.4\textwidth}
    \centering
\includegraphics[scale=0.4]{Selection_II_Section/CutJustify_sel2_datamc_mult1_ratiodedx_log.png}
   \caption{ }
  \end{subfigure} 
\caption{Multiplicity 1 ratio of high to low dEdx shown in log for a) signal and all backgrounds and b) MC-data comparison }
\label{fig:cutjust_sel2_mult1_ratiodedx_log}
\end{figure}

\begin{figure}[h!]
\centering
  \begin{subfigure}[t]{0.4\textwidth}
    \centering
\includegraphics[scale=0.4]{Selection_II_Section/CutJustify_sel2_mult1_projylen.png}
    \caption{ }
  \end{subfigure} 
  \hspace{20mm}
  \begin{subfigure}[t]{0.4\textwidth}
    \centering
\includegraphics[scale=0.4]{Selection_II_Section/CutJustify_sel2_datamc_mult1_projylen.png}
   \caption{ }
  \end{subfigure} 
\caption{Multiplicity 1 projected candidate length in Y for a) signal and all backgrounds and b) MC-data comparison }
\label{fig:cutjust_sel2_mult1_projylen}
\end{figure}

\begin{figure}[t!]
\centering
  \begin{subfigure}[t]{0.25\textwidth}
    \centering
\includegraphics[scale=0.25]{Selection_II_Section/CutJustify_sel2_mult1_dedxratio_v_projylen_All.png}    
  \caption{ }
  \end{subfigure} 
  \hspace{1 mm}
  \begin{subfigure}[t]{0.25\textwidth}
    \centering
\includegraphics[scale=0.25]{Selection_II_Section/CutJustify_sel2_mult1_dedxratio_v_projylen_Cosmic.png}
  \caption{ }
  \end{subfigure} 
  \hspace{1 mm}
  \begin{subfigure}[t]{0.25\textwidth}
    \centering
\includegraphics[scale=0.25]{Selection_II_Section/CutJustify_sel2_mult1_dedxratio_v_projylen_CC1pi0.png}  
  \caption{ }
  \end{subfigure} 
  \hspace{1 mm}
  \begin{subfigure}[t]{0.25\textwidth}
    \centering
\includegraphics[scale=0.25]{Selection_II_Section/CutJustify_sel2_mult1_dedxratio_v_projylen_CC0pi0.png}
  \caption{ }
  \end{subfigure} 
  \hspace{1 mm}
  \begin{subfigure}[t]{0.25\textwidth}
    \centering
\includegraphics[scale=0.25]{Selection_II_Section/CutJustify_sel2_mult1_dedxratio_v_projylen_NC1pi0.png}
  \caption{ }
  \end{subfigure} 
  \hspace{1 mm}
  \begin{subfigure}[t]{0.25\textwidth}
    \centering
\includegraphics[scale=0.25]{Selection_II_Section/CutJustify_sel2_mult1_dedxratio_v_projylen_NC0pi0.png}
  \caption{ }
  \end{subfigure} 
  \hspace{1 mm}
  \begin{subfigure}[t]{0.25\textwidth}
    \centering
\includegraphics[scale=0.25]{Selection_II_Section/CutJustify_sel2_mult1_dedxratio_v_projylen_Other.png}
  \caption{ }
  \end{subfigure}
  
\caption{Multiplicity 1 cut on $\frac{dE}{dx}$ ratio vs track length projection with cut shown in pink for a) All candidates; b) Cosmics; c) CC 1$\pi^0$; d) CC 0$\pi^0$; e) NC $\pi^0$; f) NC 0$\pi^0$; g) Other }
\label{fig:cutjust_sel2_mult1_dedxratio_v_leny}
\end{figure}

\begin{figure}[t!]
\centering
\begin{subfigure}[t]{0.25\textwidth}
  \centering
  \includegraphics[scale=0.25]{Selection_II_Section/CutJustify_sel2_mult1_dedxratio_v_projylen_OnBeam.png}  
  \caption{ }
  \end{subfigure} 
  \hspace{1 mm}
  \begin{subfigure}[t]{0.25\textwidth}
    \centering
\includegraphics[scale=0.25]{Selection_II_Section/CutJustify_sel2_mult1_dedxratio_v_projylen_OffBeam.png}
  \caption{ }
  \end{subfigure} 
\caption{Multiplicity 1 cut on $\frac{dE}{dx}$ ratio vs track length projection with cut shown in pink for a) OnBeam and b) OffBeam }
\label{fig:cutjust_sel2_onbeam_mult1_dedxratio_v_leny}

\end{figure}

\clearpage
\paragraph{Multiplicity $>$ 1 Cuts}
Cosmic tracks can be mistakenly reconstructed as higher multiplicity events when they are ``broken'' (two recob::Tracks are derived for a single particle). To minimize the contribution of these events to our sample, the first cut we consider is on the angle between the two longest tracks. Selection II bookkeeping ensures that all track directions reflect the location of the current candidate vertex; thus, if the two tracks are part of the same particle, the angle between them should be close to 180 degrees. This distribution is shown in Figure \ref{fig:cutjust_sel2_cosangle}; we cut candidates with a cos($\theta$) $>$ 0.9.  To further mitigate broken cosmic track contamination we consider the end point.  If the candidate $\mu$ track end point is higher in Y than the second longest track end point in Y, we employ additional cuts. Our goal with these additional cuts is to reduce cosmic-induced events of the topology shown in Figure \ref{fig:cut_ex_2}. This cut, being contingent on the Y end point, has the effect of producing a sample with a higher concentration of candidate end points at low y than at high Y.  This will be seen later when we consider cuts for Multiplicity 2 events.  In this case, we cut candidates with a second longest track length $<$ 30 cm and a directional y component of the candidate track of $>$ 0.65. The 1D distributions of each variable are shown in Figures \ref{fig:cutjust_sel2_multgt1_tracklen1} and \ref{fig:cutjust_sel2_multgt1_dcosy}.  2D distributions of signal and background are shown in Figure \ref{fig:cutjust_sel2_multgt1_len_v_cosy}; corresponding data distributions are shown in Figure \ref{fig:cutjust_sel2_onbeam_multgt1_len_v_cosy}. 

\begin{figure}[t!]
  \centering
  \includegraphics[scale=0.8]{Selection_II_Section/cut_ex_2.png}
  \caption{ Example topology targeted by multiplicity $>$ 1 cuts contingent on candidate end point in Y. }
\label{fig:cut_ex_2}
\end{figure}

\begin{figure}[t!]
\centering
  \begin{subfigure}[t]{0.4\textwidth}
    \centering
\includegraphics[scale=0.4]{Selection_II_Section/CutJustify_sel2_cosangle_OnBeam.png}
  \caption{ }
  \end{subfigure} 
  \hspace{20mm}
  \begin{subfigure}[t]{0.4\textwidth}
    \centering
\includegraphics[scale=0.4]{Selection_II_Section/CutJustify_sel2_cosangle.png}
  \caption{ }
  \end{subfigure} 
\caption{Cosine of the angle between largest 2 tracks associated to vertex by a) Data to simulation comparison and b) Background breakdown. }
\label{fig:cutjust_sel2_cosangle}
\end{figure}

\begin{figure}[t!]
\centering
 \begin{subfigure}[t]{0.4\textwidth}
    \centering
\includegraphics[scale=0.4]{Selection_II_Section/CutJustify_sel2_multgt1_tracklen1.png}
 \caption{ }
  \end{subfigure} 
  \hspace{20mm}
  \begin{subfigure}[t]{0.4\textwidth}
    \centering
\includegraphics[scale=0.4]{Selection_II_Section/CutJustify_sel2_datamc_multgt1_tracklen1.png}
 \caption{ }
  \end{subfigure} 
\caption{Multiplicity $>$ 1 length of second longest track of a) signal and all backgrounds and b) MC-data comparison }
\label{fig:cutjust_sel2_multgt1_tracklen1}
\end{figure}

\begin{figure}[t!]
\centering
  \begin{subfigure}[t]{0.4\textwidth}
    \centering
\includegraphics[scale=0.4]{Selection_II_Section/CutJustify_sel2_multgt1_cosy0.png}
 \caption{ }
  \end{subfigure} 
  \hspace{20mm}
  \begin{subfigure}[t]{0.4\textwidth}
    \centering
\includegraphics[scale=0.4]{Selection_II_Section/CutJustify_sel2_datamc_multgt1_cosy0.png}
 \caption{ }
  \end{subfigure} 
\caption{Multiplicity $>$ 1 directional Y component of candidate $\mu$ for a) signal and all backgrounds and b) MC-data comparison }
\label{fig:cutjust_sel2_multgt1_dcosy}
\end{figure}

\begin{figure}[t!]
\centering
  \begin{subfigure}[t]{0.25\textwidth}
    \centering
\includegraphics[scale=0.25]{Selection_II_Section/CutJustify_sel2_multgt1_len_v_cosy_All.png}
  \caption{ }
  \end{subfigure} 
  \hspace{1 mm}
  \begin{subfigure}[t]{0.25\textwidth}
    \centering
\includegraphics[scale=0.25]{Selection_II_Section/CutJustify_sel2_multgt1_len_v_cosy_Cosmic.png}
  \caption{ }
  \end{subfigure} 
  \hspace{1 mm}
  \begin{subfigure}[t]{0.25\textwidth}
    \centering
\includegraphics[scale=0.25]{Selection_II_Section/CutJustify_sel2_multgt1_len_v_cosy_CC1pi0.png}
  \caption{ }
  \end{subfigure} 
  \hspace{1 mm}
  \begin{subfigure}[t]{0.25\textwidth}
    \centering
\includegraphics[scale=0.25]{Selection_II_Section/CutJustify_sel2_multgt1_len_v_cosy_CC0pi0.png}
  \caption{ }
  \end{subfigure} 
  \hspace{1 mm}
  \begin{subfigure}[t]{0.25\textwidth}
    \centering
\includegraphics[scale=0.25]{Selection_II_Section/CutJustify_sel2_multgt1_len_v_cosy_NC1pi0.png}
  \caption{ }
  \end{subfigure} 
  \hspace{1 mm}
  \begin{subfigure}[t]{0.25\textwidth}
    \centering
\includegraphics[scale=0.25]{Selection_II_Section/CutJustify_sel2_multgt1_len_v_cosy_NC0pi0.png}  \caption{ }
  \end{subfigure} 
  \hspace{1 mm}
  \begin{subfigure}[t]{0.25\textwidth}
    \centering
\includegraphics[scale=0.25]{Selection_II_Section/CutJustify_sel2_multgt1_len_v_cosy_Other.png}
 \caption{ }
  \end{subfigure} 
\caption{ Multiplicity $>$ 1 event cut on length of shorter track vs the y directional component of the longer track with cut shown in pink for a) All candidates; b) Cosmics; c) CC 1$\pi^0$; d) CC 0$\pi^0$; e) NC $\pi^0$; f) NC 0$\pi^0$; g) Other }
\label{fig:cutjust_sel2_multgt1_len_v_cosy}
\end{figure}

\begin{figure}[t!]
\centering
  \begin{subfigure}[t]{0.25\textwidth}
    \centering
\includegraphics[scale=0.25]{Selection_II_Section/CutJustify_sel2_multgt1_len_v_cosy_OnBeam.png}
 \caption{ }
  \end{subfigure} 
  \hspace{20mm}
  \begin{subfigure}[t]{0.25\textwidth}
    \centering
  \includegraphics[scale=0.25]{Selection_II_Section/CutJustify_sel2_multgt1_len_v_cosy_OffBeam.png}
   \caption{ }
  \end{subfigure} 
\caption{ Multiplicity $>$ 1 event cut on length of shorter track vs the y directional component of the longer track with cut shown in pink for a) OnBeam and b) OffBeam }
\label{fig:cutjust_sel2_onbeam_multgt1_len_v_cosy}
\end{figure}

\clearpage
\paragraph{Multiplicity 2 Cuts}
Multiplicity 2 events pass first through the previous set of cuts for mult $>$ 1, and are thus already a reduced sample. We now apply an additional set of cuts to mitigate the remaining population of multiplicity 2 candidates formed by stopping cosmic muons and their Michel electrons. In these background events, the vertex candidate is at the stopping point of the muon, and is assumed to be the start point of a neutrino interaction. We require one of two conditions be met: in the first condition, we check that the smaller of the two tracks be greater than 30 cm (Figure \ref{fig:cutjust_sel2_mult2_secondtrklen}). If the smaller track satisfies this condition, the event passes the multiplicity 2 cuts. If the event does not pass the track length cut, we consider an additional set of conditions before filtering the event.  The second set of conditions considers the $<\frac{dE}{dx}>$ at the assumed start of the track (Figure \ref{fig:cutjust_sel2_mult2_dedxst}), $<\frac{dE}{dx}>$ at the assumed end (Figure \ref{fig:cutjust_sel2_mult2_dedxend}), and the end point of the candidate $\mu$ (Figure \ref{fig:cutjust_sel2_mult2_endy}).  In the case of a stopping muon, the start $<\frac{dE}{dx}>$ will be larger than the end $<\frac{dE}{dx}>$, and the end point of the candidate track will be high in Y in the TPC (Figure \ref{fig:cut_ex_1}).  Thus, in the second set of conditions we require that the end point of the candidate $\mu$ track is $<=$ 96.5 cm in Y and that either 1) the start $<\frac{dE}{dx}>$ of the candidate track is less than the end, 2) the start $<\frac{dE}{dx}>$ is $<=$ 2.5 $\frac{MeV}{cm}$ or 3) the end $<\frac{dE}{dx}>$ is $>=$ 4 $\frac{MeV}{cm}$. The corresponding 2D distributions are shown in Figures \ref{fig:cutjust_sel2_mult2_dedx_v_dedx} and \ref{fig:cutjust_sel2_onbeam_mult2_dedx_v_dedx}.

\begin{figure}[h!]
  \centering
  \includegraphics[scale=0.8]{Selection_II_Section/cut_ex_1.png}
  \caption{Example topology targeted by the multiplicity 2 cut on start and end $\frac{dE}{dx}$.  In the case of a Michel decay, the track's starting $\frac{dE}{dx}$ will be larger than that of the end, compared to a neutrino interaction where the opposite is true. }
\label{fig:cut_ex_1}
\end{figure}

\begin{figure}[h!]
\centering
  \begin{subfigure}[t]{0.4\textwidth}
    \centering
    \includegraphics[scale=0.4]{Selection_II_Section/CutJustify_sel2_mult2_trklen.png}
    \caption{ }
  \end{subfigure} 
  \hspace{20mm}
  \begin{subfigure}[t]{0.4\textwidth}
    \centering
    \includegraphics[scale=0.4]{Selection_II_Section/CutJustify_sel2_datamc_mult2_trklen.png}
    \caption{ }
  \end{subfigure} 

\caption{Multiplicity 2 length of second longest track for a) signal and all backgrounds and b) MC-data comparison }
\label{fig:cutjust_sel2_mult2_secondtrklen}
\end{figure}

\begin{figure}[h!]
\centering
  \begin{subfigure}[t]{0.4\textwidth}
    \centering
\includegraphics[scale=0.4]{Selection_II_Section/CutJustify_sel2_mult2_dedx_st.png}
    \caption{ }
  \end{subfigure} 
  \hspace{20mm}
  \begin{subfigure}[t]{0.4\textwidth}
    \centering
\includegraphics[scale=0.4]{Selection_II_Section/CutJustify_sel2_datamc_mult2_dedxStart.png}
    \caption{ }
  \end{subfigure} 
\caption{Multiplicity 2 mean start dEdx on best plane of candidate $\mu$ for a) signal and all backgrounds and b) MC-data comparison }
\label{fig:cutjust_sel2_mult2_dedxst}
\end{figure}

\begin{figure}[h!]
\centering
  \begin{subfigure}[t]{0.4\textwidth}
    \centering
\includegraphics[scale=0.4]{Selection_II_Section/CutJustify_sel2_mult2_dedx_end.png}
    \caption{ }
  \end{subfigure} 
  \hspace{20mm}
  \begin{subfigure}[t]{0.4\textwidth}
    \centering
\includegraphics[scale=0.4]{Selection_II_Section/CutJustify_sel2_datamc_mult2_dedxEnd.png}
  \caption{ }
  \end{subfigure} 
  
\caption{Multiplicity 2 mean end dEdx on best plane of candidate $\mu$ for a) signal and all backgrounds and b) MC-data comparison }
\label{fig:cutjust_sel2_mult2_dedxend}
\end{figure}

\begin{figure}[h!]
\centering
  \begin{subfigure}[t]{0.4\textwidth}
    \centering
\includegraphics[scale=0.4]{Selection_II_Section/CutJustify_sel2_mult2_longesttrk_endy.png}
    \caption{ }
  \end{subfigure} 
  \hspace{20mm}
  \begin{subfigure}[t]{0.4\textwidth}
    \centering
\includegraphics[scale=0.4]{Selection_II_Section/CutJustify_sel2_datamc_mult2_longesttrk_endy.png}
    \caption{ }
  \end{subfigure} 

\caption{Multiplicity 2 end point in Y of the candidate $\mu$ track for a) signal and all backgrounds and b) MC-data comparison }
\label{fig:cutjust_sel2_mult2_endy}
\end{figure}

\begin{figure}[h!]
\centering
  \begin{subfigure}[t]{0.25\textwidth}
    \centering
\includegraphics[scale=0.25]{Selection_II_Section/CutJustify_sel2_mult2_dedx_v_dedx_All.png}
    \caption{ }
  \end{subfigure} 
  \hspace{1mm}
  \begin{subfigure}[t]{0.25\textwidth}
    \centering
\includegraphics[scale=0.25]{Selection_II_Section/CutJustify_sel2_mult2_dedx_v_dedx_Cosmic.png}
    \caption{ }
  \end{subfigure} 
  \hspace{1mm}
  \begin{subfigure}[t]{0.25\textwidth}
    \centering
\includegraphics[scale=0.25]{Selection_II_Section/CutJustify_sel2_mult2_dedx_v_dedx_CC1pi0.png}    
  \caption{ }
  \end{subfigure} 
  \hspace{1mm}
  \begin{subfigure}[t]{0.25\textwidth}
    \centering
\includegraphics[scale=0.25]{Selection_II_Section/CutJustify_sel2_mult2_dedx_v_dedx_CC0pi0.png}
    \caption{ }
  \end{subfigure} 
  \begin{subfigure}[t]{0.25\textwidth}
    \centering
\includegraphics[scale=0.25]{Selection_II_Section/CutJustify_sel2_mult2_dedx_v_dedx_NC1pi0.png}
    \caption{ }
  \end{subfigure} 
  \hspace{1mm}
  \begin{subfigure}[t]{0.25\textwidth}
    \centering
\includegraphics[scale=0.25]{Selection_II_Section/CutJustify_sel2_mult2_dedx_v_dedx_NC0pi0.png}
    \caption{ }
  \end{subfigure} 
  \begin{subfigure}[t]{0.25\textwidth}
    \centering
\includegraphics[scale=0.25]{Selection_II_Section/CutJustify_sel2_mult2_dedx_v_dedx_Other.png}
    \caption{ }
  \end{subfigure} 

\caption{ Multiplicity 2 event cut on $\frac{dE}{dx}$ at end of track vs start a) All candidates; b) Cosmics; c) CC 1$\pi^0$; d) CC 0$\pi^0$; e) NC $\pi^0$; f) NC 0$\pi^0$; g) Other }
\label{fig:cutjust_sel2_mult2_dedx_v_dedx}
\end{figure}

\begin{figure}[h!]
\centering
\begin{subfigure}[t]{0.25\textwidth}
    \centering
\includegraphics[scale=0.25]{Selection_II_Section/CutJustify_sel2_mult2_dedx_v_dedx_OnBeam.png}
    \caption{ }
  \end{subfigure} 
  \hspace{1mm}
  \begin{subfigure}[t]{0.25\textwidth}
    \centering
    \includegraphics[scale=0.25]{Selection_II_Section/CutJustify_sel2_mult2_dedx_v_dedx_OffBeam.png}
    \caption{ }
  \end{subfigure} 

\caption{ Multiplicity 2 event cut on $\frac{dE}{dx}$ at end of track vs start for a) OnBeam and b) OffBeam }
\label{fig:cutjust_sel2_onbeam_mult2_dedx_v_dedx}
\end{figure}

\clearpage
\subsection{Selection II Results}
\par A table with event rates scaled to OnBeam POT is shown for OnBeam, OffBeam, and MCBNB + Cosmic in Table \ref{tab:sel2_event_rates}. We focus specifically here on the final two columns, where we can draw conclusions from direct Data to simulation comparison. There are a number of interesting features in this table.  First, the scaled number of ``No Cuts'' interactions before any cuts have been applied show a disagreement in row 1 of Table \ref{tab:sel2_event_rates}.  Note that the ``No Cuts'' events in the MC column do not include a requirement on the software trigger, while the data do.  We do no believe this will cause problems later in the analysis because we apply a 40 PE cut during Selection II.  Another feature worth noting in the last two columns of this table is the excess of MC over Data at the final stage of the Selection. To understand this further, an external study showed that the MC and data can be brought into agreement by modifying our GENIE simulation by including scaling for quasi-elastic events in $q_0 -- q_3$ space to account for nuclear screening (roughly 5\% reduction in QE events), suppressed non-resonant single charged pion production down by 75\%, reduced resonant single charged pion production by 10\%, and scale down by 50\% the empirical MEC contribution~\cite{bib:jaz_datamc_agreement}.  A check was additionally performed to verify the scaling of the OffBeam to the OnBeam data in Ref.~\cite{bib:davidcpot}. 

\begin{table*} 
 \centering
 \captionof{table}{Normalized event numbers at different stages of Selection2 for MCC8.3 samples using anatree files \label{tab:sel2_event_rates}}
 \begin{tabular}{| l | l | l | l | l |}
  \hline
   & OnBeam & OffBeam & On - OffBeam & MCBNBCos \\ [0.1ex] \hline
No Cuts & 540769 $\pm$ 735 & 461683 $\pm$ 751 & 79086 $\pm$ 1051 & 48230 $\pm$ 83 \\ 
40 PE Cut & 150214 $\pm$ 388 & 102822 $\pm$ 355 & 47392 $\pm$ 525 & 37392 $\pm$ 73  \\ 
Flash Match & 95435 $\pm$ 309 & 61363 $\pm$ 274 & 34072 $\pm$ 413 & 30839 $\pm$ 67 \\ \hline
Mult 1 & 2195 $\pm$ 47 & 587 $\pm$ 27 & 1608 $\pm$ 54  & 1660 $\pm$ 15  \\ 
Mult $>$ 1 & 3998 $\pm$ 63  & 618 $\pm$ 27 & 3380 $\pm$ 69 & 4577 $\pm$ 26 \\ 
Final & 6193 $\pm$ 79 & 1205 $\pm$ 38 & 4988 $\pm$ 88 & 6237 $\pm$ 30  \\ \hline
 \end{tabular}
 \end{table*}

\par A summary of the passing rates for signal and all backgrounds is shown in Table \ref{tab:sel2_passrates}.  Recall that ``Other'' events include CC 1$\pi^0$ events with vertex outside the fiducial volume, multiple $\pi^0$ events, $\overline{\nu}_\mu$ and $\nu_e$ induced interactions.  Note that we have maintained a relatively high efficiency for the signal with respect to other listed backgrounds.  
\par Sample composition is shown in Table \ref{tab:sel2_purity}. The high contribution of CC 0$\pi^0$ in the selected sample is expected as we have not yet attempted to enhance an exclusive final state beyond CC Inclusive. In the case of most NC backgrounds, a proton or charged pion has been tagged as the muon candidate.

\begin{table*}
\centering
\captionof{table}{Evolution of passing rates.  The rate is defined as the total number of event per category at that cut stage over the total number of event in that category before any cuts. \label{tab:sel2_passrates}}
 \begin{tabular}{| l | l | l |l|l|l|l|}
 \hline
 & CC 1$\pi^0$ & CC 0$\pi^0$ & NC $\pi^0$ & NC 0$\pi^0$ & Other & All \\ [0.1ex] \hline
No Cuts & - & - & - & - & - & -\\
40 PE Cut & 0.970 & 0.810 & 0.820 & 0.602 & 0.816 & 0.775 \\ 
Flash Match & 0.940 & 0.690 & 0.615 & 0.433 & 0.642 & 0.639 \\ \hline
Mult 1 & 0.138 & 0.029 & 0.056 & 0.026 & 0.023 & 0.034 \\ 
Mult $>$ 1 & 0.358 & 0.113 & 0.015 & 0.021 & 0.025 & 0.095 \\ 
Final & 0.496 & 0.142 & 0.071 & 0.047 & 0.048 & 0.129 \\ \hline
\end{tabular}
\end{table*}


\begin{table*}
\centering
\captionof{table}{Evolution of sample composition. The composition is defined as the total number of event per category at that cut stage over the number of all (scaled) events at that cut stage  \label{tab:sel2_purity}}
 \begin{tabular}{| l | l | l |l|l|l|l|l|}
 \hline
 & CC 1$\pi^0$ & CC 0$\pi^0$ & NC $\pi^0$ & NC 0$\pi^0$ & Other & Cosmic, $\nu$ Coincident & Cosmic, In time (Data) \\ [0.1ex] \hline
No Cuts & 0.018 & 0.684 & 0.045 & 0.189 & 0.064 & - & -\\ 
50 PE Cut & 0.023 & 0.715 & 0.047 & 0.147 & 0.068 & -& - \\
Flash Match & 0.027 & 0.738 & 0.043 & 0.128 & 0.065 & -&- \\ \hline
Mult 1 & 0.054 & 0.422 & 0.054 & 0.105 & 0.031 & 0.073 & 0.261 \\ 
Mult $>$ 1 & 0.060 & 0.720 & 0.006 & 0.037 & 0.015 & 0.042 & 0.119 \\ 
Final & 0.058 & 0.630 & 0.021 & 0.058 & 0.020 & 0.051 & 0.162 \\ \hline
\end{tabular}
\end{table*}


\clearpage
\subsection{ MIP Consistency Cuts}
Selection II up until this point selects too many $\nu_e$ interactions for us to unblind more than the open 5e19 POT of data according to the blinding policy.  We thus consider two additional cuts here to comply with blinding policy for a run over Run 1.  These cuts are considered in extensive detail externally, and only discussed briefly here \cite{bib:jz_unblinding_note}.  These cuts are exclusively applied to the candidate muon track selected by Selection II.  We will be using reco2 files processed through Selection II for these studies and for the rest of the note.  Note that we also include here the optical precut filter developed by the Deep Learning group (without the PMT fraction cut) \cite{bib:jarrett_opticalprecut}.
\par First, we consider track length and truncated mean $\frac{dQ}{dx}$ on the collection plane of the candidate $\mu$ track. These cuts aim to identify events in which the candidate $\mu$ is not consistent with a MIP hypothesis. The tagged track is generally inconsistent with the MIP hypothesis when it is actually a proton. 1D distribution of these cuts are shown in a format similar to the previous section in Figures \ref{fig:cutjust_sel2_multall_len} and \ref{fig:cutjust_sel2_multall_dqdx}. Note that there is a bump in the $\frac{dQ}{dx}$ distribution at low values of $\frac{elec}{cm}$; this is understood to originate from crossing tracks that are near-perpendicular to the anode plane.  2D distributions are shown for signal and all backgrounds in Figure \ref{fig:cutjust_mip_2d}.  Note that the majority of NC events lie in the region we're cutting; these cuts thus play a large role in mitigating the NC background.

\begin{figure}[h!]
\centering
  \begin{subfigure}[t]{0.3\textwidth}
    \centering
\includegraphics[scale=0.3]{Selection_II_Section/CutJustify_MIPAngle_mip_len.png}
    \caption{ }
  \end{subfigure} 
  \hspace{20mm}
  \begin{subfigure}[t]{0.3\textwidth}
    \centering
\includegraphics[scale=0.3]{Selection_II_Section/CutJustify_datamc_MIPAngle_mip_len.png}
    \caption{ }
  \end{subfigure} 

\caption{ Candidate $\mu$ length shown for a) All backgrounds and b) MC data comparison. }
\label{fig:cutjust_sel2_multall_len}
\end{figure}


\begin{figure}[h!]
\centering
  \begin{subfigure}[t]{0.3\textwidth}
    \centering
\includegraphics[scale=0.3]{Selection_II_Section/CutJustify_MIPAngle_mip_dqdx.png}
    \caption{ }
  \end{subfigure} 
  \hspace{20mm}
  \begin{subfigure}[t]{0.3\textwidth}
    \centering
\includegraphics[scale=0.3]{Selection_II_Section/CutJustify_datamc_MIPAngle_mip_dqdx.png}
    \caption{ }
  \end{subfigure} 
\caption{ Candidate $\mu$ dQdx in [electrons/cm] shown for a) All backgrounds and b) MC data comparison. }
\label{fig:cutjust_sel2_multall_dqdx}
\end{figure}

\begin{figure}[t!]
\centering
  \begin{subfigure}[t]{0.35\textwidth}
    \centering
\includegraphics[scale=0.35]{Selection_II_Section/CutJustify_MIPAngle_mip_len_vs_mip_dqdx_Cosmic.png}
    \caption{ }
  \end{subfigure} 
  \hspace{3mm}
  \begin{subfigure}[t]{0.35\textwidth}
    \centering
\includegraphics[scale=0.35]{Selection_II_Section/CutJustify_MIPAngle_mip_len_vs_mip_dqdx_CC1pi0.png}
    \caption{ }
  \end{subfigure} 
  \hspace{3mm}
  \begin{subfigure}[t]{0.35\textwidth}
    \centering
\includegraphics[scale=0.35]{Selection_II_Section/CutJustify_MIPAngle_mip_len_vs_mip_dqdx_CC0pi0.png}
    \caption{ }
  \end{subfigure} 
    \hspace{3mm}
  \begin{subfigure}[t]{0.35\textwidth}
    \centering
\includegraphics[scale=0.35]{Selection_II_Section/CutJustify_MIPAngle_mip_len_vs_mip_dqdx_NC1pi0.png}
    \caption{ }
  \end{subfigure} 
  \hspace{3mm}
  \begin{subfigure}[t]{0.35\textwidth}
    \centering
\includegraphics[scale=0.35]{Selection_II_Section/CutJustify_MIPAngle_mip_len_vs_mip_dqdx_NC0pi0.png}
    \caption{ }
  \end{subfigure}
    \hspace{3mm}
  \begin{subfigure}[t]{0.35\textwidth}
    \centering
\includegraphics[scale=0.35]{Selection_II_Section/CutJustify_MIPAngle_mip_len_vs_mip_dqdx_Other.png}
    \caption{ }
  \end{subfigure} 
\caption{ MIP consistency event cut on truncated mean dQdx vs track length a) Cosmics; c) CC 1$\pi^0$; d) CC 0$\pi^0$; e) NC $\pi^0$; f) NC 0$\pi^0$; g) Other }
\label{fig:cutjust_mip_2d}
\end{figure}

We consider one additional cut before proceeding to the results of this augmented Selection II filter.  Much of the $\nu_e$ population remaining after the MIP consistency cuts have a candidate $\mu$ track that is mis-reconstructed across a shower or across several particles.  This causes the angular deviation of the track to be higher on average than a well-reconstructed candidate.  We thus consider a cut on maximum angular deviation of the candidate track.  This angular distribution for all events left after the default Selection II is shown in Figure \ref{fig:cutjust_sel2_multall_deviation}.  The implementation of this cut and the MIP consistency cut brings our selected $\nu_e$ below the required number in the blinding scheme \cite{bib:jz_unblinding_note}.  For the rest of this note, we will refer to the usual Selection II pass described previously and these additional MIP consistency and angle cuts collectively as the new `Selection II'.

\begin{figure}[t!]
\centering
  \begin{subfigure}[t]{0.35\textwidth}
    \centering
\includegraphics[scale=0.35]{Selection_II_Section/CutJustify_MIPAngle_deviation.png}
    \caption{ }
  \end{subfigure} 
  \hspace{30mm}
  \begin{subfigure}[t]{0.35\textwidth}
    \centering
\includegraphics[scale=0.35]{Selection_II_Section/CutJustify_datamc_MIPAngle_deviation.png}
    \caption{ }
  \end{subfigure} 
\caption{ Candidate $\mu$ angular deviation shown for a) All backgrounds and b) MC data comparison. }
\label{fig:cutjust_sel2_multall_deviation}
\end{figure}

\clearpage

\subsection{New Selection II Results}
Here we evaluate the effect of the new MIP and angle cuts on our remaining sample of events.  We will perform these evaluations, and all future evaluations in this note, using reco2 files. This is in contrast to the anatree files used to perform the Selection II pass rate and composition studies discussed previously. Thus we expect some small, statistical variance in pass rate and composition for the various samples.  More details on the separate samples are discussed in the introduction.
\par A table with event rates scaled to OnBeam POT is shown for OnBeam, OffBeam, and MCBNBCosmic in Table \ref{tab:sel2_w_mip_event_rates} for this new Selection II. Note that the MC excess over Data we saw in the previous section remains after the MIP consistency and angular cuts. Event break downs by interaction mode before and after Selection II are shown in Figures \ref{fig:physics_sel2_inttype}a and b.  
\begin{figure}[t!]
\centering
  \begin{subfigure}[t]{0.35\textwidth}
    \centering
\includegraphics[scale=0.35]{Selection_II_Section/Misc_full_EventType_vs_NeutrinoMode_w_Numbers.png}
    \caption{ }
  \end{subfigure} 
  \hspace{20 mm}
  \begin{subfigure}[t]{0.35\textwidth}
    \centering
\includegraphics[scale=0.35]{Selection_II_Section/Misc_sel2_EventType_vs_NeutrinoMode_w_Numbers.png}
    \caption{ }
  \end{subfigure} 

\caption{ Event type broken down by neutrino interaction mode a) before and b) after Selection II cuts. Note that these blocks contain raw MC numbers that are not scaled to data. }
\label{fig:physics_sel2_inttype}
\end{figure}


\begin{table*} 
 \centering
 \captionof{table}{Normalized event numbers at different stages of Selection2 for MCC8.3 samples using reco2 files \label{tab:sel2_w_mip_event_rates}}
 \begin{tabular}{| l | l | l | l | l |}
  \hline
   & OnBeam & OffBeam & On - OffBeam & MCBNBCos \\ [0.1ex] \hline
No Cuts & 544751 $\pm$ 738 & 462076 $\pm$ 1001 & 82675 $\pm$ 1244 & 48972 $\pm$76 \\ 
Selection II & 6031 $\pm$ 738 & 1086 $\pm$ 49 & 4945 $\pm$ 92 & 6150 $\pm$ 27  \\ \hline
MIP Consistency & 4162 $\pm$ 65 & 661 $\pm$ 38 & 3501 $\pm$ 75 & 4570 $\pm$ 23  \\ 
Angular Deviation & 3753 $\pm$ 61 & 564 $\pm$ 35 & 3189 $\pm$ 71 & 4268 $\pm$ 22  \\ \hline

\end{tabular}
 \end{table*}

\par A summary of the event totals for signal and all backgrounds is shown in Table \ref{tab:passrates}.  Recall that ``Other'' events include CC 1$\pi^0$ events with vertex outside the fiducial volume, multiple $\pi^0$ events, $\overline{\nu}_\mu$ and $\nu_e$ induced interactions.  Note that we have maintained a relatively high efficiency for the signal with respect to other listed backgrounds. Sample composition is shown in Table \ref{tab:purity}. 

\begin{table*}
\centering
\captionof{table}{Evolution of passing rates for various stages of the analysis. ``MIP Consistency '' refers to cuts on the Selection II candidate muon track length and the truncated mean $\frac{dQ}{dx}$, while ``Angular Deviation'' refers the track straightness cut that requires there is not angular deflection in the Selection II candidate muon greater than 8$^{o}$. \label{tab:passrates}}
 \begin{tabular}{| l | l | l |l|l|l|l|}
 \hline
 & CC 1$\pi^0$ & CC 0$\pi^0$ & NC $\pi^0$ & NC 0$\pi^0$ & Other & All \\ [0.1ex] \hline
No Cuts & - & - & - & - & - & -\\
Selection II & 0.478 & 0.136 & 0.067 & 0.044 & 0.084 & 0.126 \\ \hline
MIP Consistency & 0.362 & 0.111 & 0.018 & 0.011 & 0.036 & 0.093 \\ 
Angular Deviation & 0.331 & 0.105 & 0.008 & 0.010 & 0.030 & 0.087 \\ \hline
\end{tabular}
\end{table*}


\begin{table*}
\centering
\captionof{table}{Evolution of sample composition.  ``MIP Consistency '' refers to cuts on the Selection II candidate muon track length and the truncated mean $\frac{dQ}{dx}$, while ``Angular Deviation'' refers the track straightness cut that requires there is not angular deflection in the Selection II candidate muon greater than $8^{o}$.  \label{tab:purity}}
 \begin{tabular}{| l | l | l |l|l|l|l|l|}
 \hline
 & CC 1$\pi^0$ & CC 0$\pi^0$ & NC $\pi^0$ & NC 0$\pi^0$ & Other & Cosmic, $\nu$ Flash & Cosmic, In time (Data) \\ [0.1ex] \hline
No Cuts & 0.018 & 0.695 & 0.046 & 0.194 & 0.047 & - & -\\ 
Selection II & 0.058 & 0.639 & 0.021 & 0.058 & 0.027 & 0.046 & 0.150 \\ \hline
MIP Consistency & 0.061 & 0.725 & 0.007 & 0.020 & 0.016 & 0.044 & 0.126 \\ 
Angular Deviation & 0.060 & 0.743 & 0.004 & 0.020 & 0.014 & 0.042 & 0.117 \\ \hline
\end{tabular}
\end{table*}

Vertex distributions are shown in XY, XZ and YZ views for OnBeam data in Figures \ref{fig:ll_sel2_vertices_onbeam}a-c, OffBeam data in Figures \ref{fig:ll_sel2_vertices_offbeam}a-c, and MCBNB + Cosmics in Figures \ref{fig:ll_sel2_vertices_mc}a-c.  Vertex resolution is shown in Figure \ref{fig:physics_sel2_vtxres}a in all three dimensions and Figure \ref{fig:physics_sel2_vtxres}b in total. 

\begin{figure}[h!]
\centering
  \begin{subfigure}[t]{0.26\textwidth}
    \centering
\includegraphics[scale=0.26]{Selection_II_Section/LL_sel2_vtxx_vtxy_2Donbeam.png}
    \caption{ }
  \end{subfigure} 
  \hspace{10 mm}
  \begin{subfigure}[t]{0.26\textwidth}
    \centering
\includegraphics[scale=0.26]{Selection_II_Section/LL_sel2_vtxz_vtxx_2Donbeam.png}
    \caption{ }
  \end{subfigure} 
  \hspace{10 mm}
  \begin{subfigure}[t]{0.26\textwidth}
    \centering
\includegraphics[scale=0.26]{Selection_II_Section/LL_sel2_vtxz_vtxy_2Donbeam.png}
    \caption{ }
  \end{subfigure} 

\caption{ Vertex distributions in OnBeam data for a) XY, b) XZ and c) YZ views. }
\label{fig:ll_sel2_vertices_onbeam}
\end{figure}

\begin{figure}[t!]
\centering
  \begin{subfigure}[t]{0.26\textwidth}
    \centering
\includegraphics[scale=0.26]{Selection_II_Section/LL_sel2_vtxx_vtxy_2Doffbeam.png}
    \caption{ }
  \end{subfigure} 
  \hspace{10 mm}
  \begin{subfigure}[t]{0.26\textwidth}
    \centering
\includegraphics[scale=0.26]{Selection_II_Section/LL_sel2_vtxz_vtxx_2Doffbeam.png}
    \caption{ }
  \end{subfigure} 
  \hspace{10 mm}
  \begin{subfigure}[t]{0.26\textwidth}
    \centering
\includegraphics[scale=0.26]{Selection_II_Section/LL_sel2_vtxz_vtxy_2Doffbeam.png}
    \caption{ }
  \end{subfigure} 

\caption{ Vertex distributions in OffBeam data for a) XY, b) XZ and c) YZ views. }
\label{fig:ll_sel2_vertices_offbeam}
\end{figure}

\begin{figure}[h!]
\centering
  \begin{subfigure}[t]{0.26\textwidth}
    \centering
\includegraphics[scale=0.26]{Selection_II_Section/LL_sel2_vtxx_vtxy_2Dmcbnbcos.png}
    \caption{ }
  \end{subfigure} 
  \hspace{10 mm}
  \begin{subfigure}[t]{0.26\textwidth}
    \centering
\includegraphics[scale=0.26]{Selection_II_Section/LL_sel2_vtxz_vtxx_2Dmcbnbcos.png}
    \caption{ }
  \end{subfigure} 
  \hspace{10 mm}
  \begin{subfigure}[t]{0.26\textwidth}
    \centering
\includegraphics[scale=0.26]{Selection_II_Section/LL_sel2_vtxz_vtxy_2Dmcbnbcos.png}
    \caption{ }
  \end{subfigure} 
\caption{ Vertex distributions in MCBNB+Cosmics for a) XY, b) XZ and c) YZ views. }
\label{fig:ll_sel2_vertices_mc}
\end{figure}


\begin{figure}[t!]
\centering
 \begin{subfigure}[t]{0.35\textwidth}
    \centering
\includegraphics[scale=0.35]{Selection_II_Section/LL_sel2_vtx_res.png}
    \caption{ }
  \end{subfigure} 
  \hspace{10 mm}
  \begin{subfigure}[t]{0.35\textwidth}
    \centering
\includegraphics[scale=0.35]{Selection_II_Section/LL_sel2_vtx_mc_reco_dist.png}
    \caption{ }
  \end{subfigure} 

\caption{Distance from true to reconstructed vertex in a) the 3 separate 1-D projections and b) the 3D resolution as measured in 2D by projecting onto the XZ (collection) plane. }
\label{fig:physics_sel2_vtxres}
\end{figure}

Signal and background distributions are shown for a variety of kinematic variables in Figures \ref{fig:physics_sel2_mulen} - \ref{fig:physics_sel2_z} (uncertainties are purely statistical at this point). The cross hatched region corresponds to the statistical uncertainty from the MC sample combined in quadrature with the offbeam statistical uncertainty.

\begin{figure}[h!]
\centering
  \begin{subfigure}[t]{0.3\textwidth}
    \centering
\includegraphics[scale=0.3]{Selection_II_Section/Physics_sel2_onoffseparate_mult.png}
    \caption{ }
  \end{subfigure} 
  \hspace{30 mm}
  \begin{subfigure}[t]{0.3\textwidth}
    \centering
\includegraphics[scale=0.3]{Selection_II_Section/Physics_sel2_onoffseparate_mu_len.png}
    \caption{ }
  \end{subfigure} 
 
\caption{ Data to simulation comparison of a) observed track multiplicity and b) $\mu$ contained length after Selection II filter.  Note that Selection II includes un-contained and contained candidate $\mu$'s, but we are only able to observe the length contained within the TPC.  }
\label{fig:physics_sel2_mulen}
\end{figure}

\begin{figure}[h!]
\centering
  \begin{subfigure}[t]{0.3\textwidth}
    \centering
\includegraphics[scale=0.3]{Selection_II_Section/Physics_sel2_onoffseparate_mu_angle.png}
   \caption{ }
  \end{subfigure} 
  \hspace{30 mm}
  \begin{subfigure}[t]{0.3\textwidth}
    \centering
    \includegraphics[scale=0.3]{Selection_II_Section/Physics_sel2_onoffseparate_mu_phi.png}
  \caption{ }
  \end{subfigure} 
\caption{ Data to simulation comparison of $\mu$ a) $\cos\theta$  and b) $\phi$ after Selection II filter }
\label{fig:physics_sel2_muphi}
\end{figure}

\begin{figure}[t!]
\centering
  \begin{subfigure}[t]{0.3\textwidth}
    \centering
\includegraphics[scale=0.3]{Selection_II_Section/Physics_sel2_onoffseparate_mu_startx.png}
   \caption{ }
  \end{subfigure} 
  \hspace{30 mm}
  \begin{subfigure}[t]{0.3\textwidth}
    \centering
\includegraphics[scale=0.3]{Selection_II_Section/Physics_sel2_onoffseparate_mu_endx.png}
   \caption{ }
  \end{subfigure} 
\caption{ Data to simulation comparison of $\mu$ a) start and b) end in x after Selection II filter }
\label{fig:physics_sel2_x}
\end{figure}

\begin{figure}[t!]
\centering
  \begin{subfigure}[t]{0.3\textwidth}
    \centering
\includegraphics[scale=0.3]{Selection_II_Section/Physics_sel2_onoffseparate_mu_starty.png}
   \caption{ }
  \end{subfigure} 
  \hspace{30 mm}
  \begin{subfigure}[t]{0.3\textwidth}
    \centering
\includegraphics[scale=0.3]{Selection_II_Section/Physics_sel2_onoffseparate_mu_endy.png}
   \caption{ }
  \end{subfigure} 
\caption{ Data to simulation comparison of $\mu$ a) start and b) end in y after Selection II filter }
\label{fig:physics_sel2_y}
\end{figure}

\begin{figure}[t!]
\centering
  \begin{subfigure}[t]{0.3\textwidth}
    \centering
\includegraphics[scale=0.3]{Selection_II_Section/Physics_sel2_onoffseparate_mu_startz.png}
   \caption{ }
  \end{subfigure} 
  \hspace{30mm}
  \begin{subfigure}[t]{0.3\textwidth}
    \centering
\includegraphics[scale=0.3]{Selection_II_Section/Physics_sel2_onoffseparate_mu_endz.png}
   \caption{ }
  \end{subfigure} 

\caption{ Data to simulation comparison of $\mu$ a) start and b) end in z after Selection II filter }
\label{fig:physics_sel2_z}
\end{figure}




\clearpage
%\section{Track-like Hit Removal}
\section{Shower Reconstruction}
During past efforts, 2D-clustering has been a bottle neck in the reconstruction chain where we lose many events due to the complexity of reconstructing complicated topologies. To mitigate this problem, we have broken 2D-clustering into 2 stages: track-like hit removal and clustering. We discuss both stages in this section.

\subsection{Identifying Electromagnetic Activity}
%At this point, we have identified a sample of CC $\nu_\mu$ candidate events. We must next look for CC events that also contain a single $\pi^0$ originating from the candidate vertex. $\pi^0$ identification is performed over several stages: first we identify electromagnetic activity at the hit level. The goal of this stage is to clean up the event. Second, we run image-based clustering and shower reconstruction algorithms on the remaining electromagnetic-like hits. Finally, once our events have 3D reconstructed showers, we apply a series of correlation and parameter cuts to identify candidate $\gamma$ daughters from the original $\pi^0$.  These stages will be discussed across the next 2 sections.

Hit Removal is a suite of algorithms developed by David Caratelli to identify electromagnetic activity at the hit level and store this information for use by downstream algorithms. While the package is called ``Hit Removal", no hits are ever actually removed from the event; rather, a track or shower-like designation is stored per hit in the hit's ``GoodnessOfFit'' class variable. Thus, at the end of all Hit Removal algorithms, both the identified electromagnetic hits and all original hits can be accessed. A brief overview is discussed in this section; extensive detail on all algorithms is provided externally \cite{bib:davidc_hitremoval}.

\par The goal of these algorithms is to identify induced charge which is both shower-like and originating at the candidate vertex in the event. This hit-identification is broken into cosmic-induced and neutrino-induced stages.  In both stages, only hits within a 1~m ROI of the reconstructed vertex are considered. The cosmic-induced hit identification algorithms are run first.  Here, the hits associated with pandoraCosmic cosmic-tagged tracks are `removed'.  Similarly, we remove groups of hits that are poorly aligned with the neutrino vertex, whether or not that charge already has a pandoraCosmic cosmic tag. Finally, we search for remaining electromagnetic hits near in 2D space to already-removed pandoraCosmic clusters. If the detected activity resembles a delta ray, these hits are also removed. 
\par We follow these cosmic-targeted algorithms with a suite of algorithms aimed at neutrino-induced activity. First, pandoraCosmic clusters that have associated hits within 3.5 cm of the vertex are required to satisfy an aggressive tight truncated local linearity (TLL) requirement. The goal of this tight cut on linearity is to preserve the population of photons that convert very near the vertex, while also removing some vertex-related track activity. Following this aggressive TLL requirement, we consider number of cluster hits, cluster 2D slope, and the TLL variable with aim to remove $\mu$ and charged $\pi$ activity. Note that the TLL cut used in this case is looser than in the previous algorithm. Finally, we remove all charge within 3.5 cm of the vertex. Hits are generally crowded near the vertex, which leads to potential over-merging at the clustering stage. In this context, this step cleans the area around the vertex, and prepares us for downstream clustering. Once these algorithms have completed, the remaining electromagnetic hit candidates are passed onto clustering. The remaining hits after this hit removal pass are shown in red in Figure \ref{fig:hitremoval} for two separate example events. Data to simulation comparisons of all hits in the event and track-like hits after hit removal are shown in Figures \ref{fig:datamc_total_hits} and \ref{fig:datamc_tracklike_hits}.  Note that the offset between data and MC observed in the total hits plots is in agreement with MCC8.3 validation studies \cite{bib:mcc83_validation_plots}. Area-normalized versions of these plots can be found in Appendix \ref{sec:AppB} -- we conclude from there that the On and OffBeam distributions are largely comparable within statistical variations in planes 0 and 2, and less so in plane 1.   
\begin{figure}[h!]
\centering
\fbox{\includegraphics[scale=0.4]{Cluster_Shower/HR_mcc8.png}}
\caption{ Example display after Hit Removal, remaining candidate electromagnetic activity in red for two separate events. Note in the interaction on the right that the relatively linear shower (circled) has been removed. The effects of energy loss at all stages are quantified externally \cite{bib:davidc_missingE}.} 
\label{fig:hitremoval}
\end{figure}

\subsection{OpenCV Software Package}
OpenCV is an open source computer vision library with functionality to aid in pattern recognition and image processing \cite{bib:opencv}. These functions have been developed by a world-wide community and are designed to be both efficient and fast. OpenCV has Python, C++, and Java interfaces, and is easy to use. Here we describe algorithms developed to improve the efficiency and quality of the selected sample from Neutrino 2016 efforts \cite{bib:5864}. More on OpenCV tools used specifically for Neutrino 2016 can be found in Ref.~\cite{bib:5856}.  All fcl parameters associated with this chain are referenced in Appendix \ref{sec:AppA}.

\subsection{LArOpenCV Framework}
LArOpenCV is a C++ framework application built on top of the LArLite event processing framework with an interface to the OpenCV API. The LArOpenCV framework allows for event-by-event image creation, application of image manipulation algorithms, two dimensional contour matching between three physical wire planes, and cluster and PFParticle creation and output. The input into LArOpenCV is either a full readout or region of interest (ROI) reconstructed hit object. Input data products are first translated into an ``event image'' (described in next section), then handed over to a separate manager class for image processing. In this work, we consider reconstructed information inside an ROI around the Selection II candidate vertex.  More detail on translation of event into image coordinates is discussed in Ref. \cite{bib:5856}.

\subsection{Event Image}
\par In order to process hit objects as input with our LArOpenCV framework, we first need to translate our data products into a data format compatible with OpenCV. To do this, the LArOpenCV framework creates a single channel image for each plane determined by the wire and time tick ranges. Each row pixel represents a single wire and each column pixel a single time tick. The integral charge of each hit is scaled to 8 bits (integer 0 to 255) and assigned to that hit's peak time and wire. The result is a single channel grey-scale image. Note that this scaling is solely for the purposes of identifying clusters at the OpenCV stage. All analysis post-clustering (including energy reconstruction) is entirely independent of OpenCV and the LArOpenCV framework. 
\par Finally, we ``pool" the image. In MicroBooNE, a single time tick corresponds to 0.05 cm and a single wire spacing to 0.3 cm (or roughly 6 time ticks).  In ``pooling'', we combine 6 ticks into 1 tick pixel with the summed pixel value at that location. We do this because the image manipulation kernels we will describe in the coming sections perform better for our purposes if they can act uniformly in both directions. Now that the image is prepared, we can proceed with clustering.
\subsection{Clustering with LArOpenCV }

\begin{figure}[h!]
\centering
\includegraphics[scale=0.6]{Cluster_Shower/Misc_opencv_polar.png}
\caption{ Example from OpenCV manual depicting polar transformation algorithm \cite{bib:linearPolar}}
\label{fig:polar}
\end{figure}

\begin{figure}[h!]
\centering
\includegraphics[scale=0.8]{Cluster_Shower/Misc_opencv_cluster_linear.png}
\includegraphics[scale=0.8]{Cluster_Shower/Misc_opencv_cluster_polar.png}

\caption{ Example of single $\pi^0$ event in linear (left) and polar (right)}
\label{fig:pi0_polar}
\end{figure}

%{\LARGE \color{red} \bf Reference?}
\par The job of the first algorithm in the clustering chain (``PolarCluster") is to identify candidate clusters for eventual particle identification. The algorithm first transforms image information into polar coordinates using the OpenCV linearPolar function \cite{bib:linearPolar}, with the origin at the reconstructed vertex location and the radius set to the ROI width (Figure \ref{fig:polar} and \ref{fig:pi0_polar}). Any pixels outside this region (for example, if the radius extends outside the TPC) are set to 0. This polar strategy has the advantage over Cartesian algorithms in that it enforces an image blur in the direction of showering which prevents lateral over-merging. Following translation, we perform a series of OpenCV image manipulations as depicted in Figure \ref{fig:sbc}. First, we dilate the image \cite{bib:dilate}. We do this using an elliptic structuring element \cite{bib:structuringElement} 5 pixels in radius as our kernel. During dilation, pixels surrounding each hit within the dilation radius acquire the grey scale value of the hit; the fundamental pixel dimensions, however, remain the same. In this way, we begin to connect hits without changing the size of the image. We then apply a blur function \cite{bib:blur} with kernel width 10 and height 5. Blurring smears these dilated hits together with a Gaussian filter and smooths image edges. This process, similar to dilation, only changes the grey scale values of various pixels, not the size of the original pixels. Finally, we run contour finding \cite{bib:contourFinding} on the blurred image, and initialize our cluster definitions based on these contours' spatial locations and geometric properties.  Once contours are found, we translate back into Cartesian coordinates and pass the acquired cluster information to the next algorithm.

\begin{figure}[H]
\centering
\includegraphics[scale=0.6]{Cluster_Shower/Misc_image_manipulation.png}
\caption{ a) Original hits that make up a single cluster; b) Dilate pixels to ``connect" nearby hits; c) Blur image to smooth edges and smear charge; d) OpenCV contour finding on blurred image produces a polygon which now encloses the original hits.}
\label{fig:sbc}
\end{figure}

\par Once our first candidate clusters have been identified, we run a set of simple filters.  First we remove very small clusters ($<$ 10 hits); the rational for this cut is motivated by the later matching stage and will be discussed later. Next, we remove any cluster that pierces or lies outside of a Region of Interest (ROI)  (Figure \ref{fig:roi}).  This corresponds to five radiation lengths beyond the first conversion length which means that little energy will pierce this boundary (Figure \ref{fig:davidc_conversion_distance}). Thus, reconstructed clusters extending outside the ROI are unlikely candidates for our $\gamma$ search. We only consider information in this ROI around the Selection II candidate vertex throughout the remainder of the clustering chain.

\begin{figure}[H]
\centering
\includegraphics[scale=0.4]{Cluster_Shower/conversion_distance.png}
\caption{ Mean free path of photons as a function of energy, courtesy of David Caratelli. }
\label{fig:davidc_conversion_distance}
\end{figure}

\begin{figure}[h!]
\centering
\includegraphics[scale=0.35]{Cluster_Shower/Misc_ROI.png}
\caption{$\pi^0$ Region Of Interest (ROI) around $\nu_\mu$ CC Selection II tagged  vertex. The bounds extend 1m away from the vertex in each direction. }
\label{fig:roi}
\end{figure}


\par For the remaining clusters a start point and direction is assigned. Start point calculation is performed in the following way: each OpenCV-calculated contour has an associated minimum bounding box \cite{bib:minAreaRect} that surrounds it (outer red rectangle in Figure \ref{fig:misc_opencv_startpoint}a). The start point finding module segments this bounding box into 2 equal segments long ways (red center line in Figure \ref{fig:misc_opencv_startpoint}a). The algorithm then locates the hit of the cluster that is closest to the ROI vertex (green circle).  Next, we take the segment this hit belongs to (green box in Figure \ref{fig:misc_opencv_startpoint}b) and search for the hit furthest from the center (yellow dot) of the minimum bounding box to assign a start point (purple star in Figure \ref{fig:misc_opencv_startpoint}c). Finally, the hit furthest from the center in the adjacent segment is assigned to be the end point (orange star in Figure \ref{fig:misc_opencv_startpoint}c). We assign the cluster's direction to be the direction of the cluster's minimum bounding box.

\begin{figure}[h!]
\centering
\includegraphics[width=0.95\textwidth]{Cluster_Shower/Misc_opencv_startpoint_explanation.png}
\caption{ a) The start point finding algorithm uses the ROI vertex (green circle) to determine the
cluster’s nearest hit to the vertex; b) The segment containing this hit (green) is then selected; c) The
start point (purple star) is assigned to the hit furthest from the center of the bounding box (yellow
dot), and likewise the end point in the other segment (orange star). }
\label{fig:misc_opencv_startpoint}
\end{figure}

\par Now that our clusters have start points and directions we can attempt to combine charge which was not clustered together during polar clustering. The algorithm we use to do the merging proceeds in a series of steps.  First, we consider all remaining clusters per plane (Figure \ref{fig:misc_flashlights}a) and again calculate the minimum bounding rectangles. The bounding points farthest from the reconstructed vertex are considered to be the `end' of this rectangle, where as the points closest are the `start'.  We then build a trapezoid of height 13 cm (h) that fans out 20 degrees ($\theta$) away from the direction of the box, and stick it on the end of the original bounding box (Figure \ref{fig:misc_flashlights}b); these values were chosen empirically to maximize first, the purity and second, the completeness on a sample of CC $\pi^0$ interactions.  The results of this operation are objects that resemble flash lights around each cluster per plane (Figure \ref{fig:misc_flashlights}c). We next sort the clusters based on their proximity to the vertex and loop over the sorted clusters. Per cluster in this outer loop, we then consider the overlap between it and all other thus-unassociated clusters in an inner loop.  We do this by looking for overlaps between the flash piece (segments 2-3-4-5 from Figure \ref{fig:misc_flashlights}b) of the closer-to-vertex cluster with the base (segments 5-0-1-2 from Figure \ref{fig:misc_flashlights}b) of the further cluster. Care is taken to keep neighbor associations unique, and not double count cluster merges. Once all overlaps have been considered, the convex hull \cite{bib:convexHull} of all newly-associated cluster hits is calculated and stored (Figures \ref{fig:misc_flashlights}d and e). The start point of the old cluster closest to the ROI vertex is used as the start point for the new combined cluster, while the end point is reassigned to the new cluster hit furthest from the start point.

\begin{figure}[h!]
\centering
\includegraphics[scale=0.5]{Cluster_Shower/flashlights.png}
\caption{ a) The flashlight merging algorithm is fed the vertex (cyan) and a set of clusters identified in the polar clustering stage; b) A trapezoid of angle $\theta$ and height h is attached to the end of the cluster, with the end being the side furthest from the vertex.  The new contour bounding box points are labeled for convenience; c) Flashlights are constructed only for identified clusters in the plane; d) Associations are built between flashlights based on overlap of their segments; e) The final bounding contour is the minimum contour that encloses all points.  }
\label{fig:misc_flashlights}
\end{figure}

\par Finally, we apply 2 simple filters to reduce the number of bad/uninteresting clusters passing on to matching. First, we remove any clusters which are not aligned well with the vertex.  We do this by considering the dot product of the cluster's direction with the direction vector from the vertex to the cluster's start point.  If this dot product is $>$ 0.71 or if the cluster start point is closer than 12 pixels to the vertex, the cluster is kept, where these values were determined empirically to minimize the removal of visually well-aligned clusters.  The latter condition prevents removing clusters that appear to have very misaligned directions by function of being near the vertex. This algorithm removes lingering cosmic rays and some mis-clustered hits. Second, we filter clusters whose surrounding contour contains the Selection II vertex (Figure \ref{fig:vertex_in_hull}). Occasionally when events are busy (e.g. a crossing cosmic muon through the interaction, a large amount of charge deposited, etc.) the flashlight algorithm will over-merge. A contained vertex is nearly always a good sign that over-merging has occurred.  

\begin{figure}[H]
\centering
\includegraphics[width=0.3\textwidth]{Cluster_Shower/vertex_in_hull.png}
\caption{Example event targeted by the second simple filter we apply after merging. The selected vertex is shown as a yellow star, and the convex hull formed after merging is shown in red. Here, merging has joined two short tracks at the vertex, which has resulted in a vertex-contained-hull. }
\label{fig:vertex_in_hull}
\end{figure}


\subsection{Cluster Matching}
Before we can reconstruct showers and search for $\pi^{0}$'s, we must match clusters across planes.  In order to reconstruct showers, we need matched 2D clusters from at least two different planes. Here we require only two rather than three planes for matched pairs to be saved, where one of those planes must be the collection plane. This decision is based on the fact that clustering generally only goes well consistently in two of the three planes; dead wires, events with extra activity (noise), and poor reconstruction are often present in at least one plane. We require that one of the matched clusters come from the collection plane.  In order to choose the second plane (U or V), we assign an event-by-event score to the U and V plane correlated with the percentage of the ROI which is covered by ``dead'' or poorly-functioning wires. We use the channel information from uboonecode release v06\_26\_01\_10 in Ref. \cite{bib:calibration_ref} to configure our dead wire ranges for this algorithm. 
 To do this, we take the width of the ROI as the denominator and the sum of all dead wire ranges within the ROI as the numerator. The ranges of dead wires closer to the vertex have a stronger weight ($\times$2) than wires near the ROI boundaries.  Here, “closer to the vertex” referred specifically to wires within $\frac{1}{3}$ of the ROI width ($\frac{1}{3} \times$ 1m, or 33~cm) away from the vertex. The score is computed by subtracting this percentage-of-dead-wires-in-ROI ratio from 1; thus ROI's with few dead wires will have high scores, while ROI's with many will have low scores.  Because we give more weight to wires closest to the vertex, we have the potential to create scores that are less than 0; in this case, we just set the score of the plane to 0. The clusters in the remaining U or V plane with the highest plane score are passed on to matching. 

\par Matching utilizes the fact that time is a shared coordinate across planes and assigns scores to cluster pairs based on their agreement in time. We quantify this score using a measure denoted as the \texttt{Intersection over Union}, \texttt{IoU}. This quantity is defined as:
\begin{equation}
  {\rm IoU} = \frac{ \Delta t_1 \cap \Delta t_2  }{ \Delta t_1 \cup \Delta t_2 }
\end{equation}

\noindent where $\Delta t$ denotes the time-range associated to the peak time of each hit in a given cluster.  Clusters which do not overlap are assigned a score of -1, while those that do are assigned a score between 0 and 1, with 1 being perfect overlap. At the end of the consideration of all match permutations, the highest scores are used to create matched pairs until no clusters or viable matches remain. We require that there be at minimum a 25\% agreement in time, and a 20\% agreement in number of cluster hits in order for a match to be made. As noted earlier, we applied a cut on clusters with less than 10 hits. An example of the situation we're trying to avoid is shown in Figures \ref{fig:matching_ex_0} and \ref{fig:matching_ex_1}.  We want to prevent the scenario in which a cluster in one plane merges with a small, distant blob of charge, its true partner-cluster from the other plane does not merge with that same small blob of charge, and as a result the clusters are mismatched with some other cluster with which they now share a better time overlap (Figure \ref{fig:matching_ex_1}). Losing some charge and correcting for the energy loss later is better than dealing with mis-matched pairs.

\begin{figure}[H]
\centering
\includegraphics[width=0.7\textwidth]{Cluster_Shower/matching_ex_0.png}
\caption{Example event targeted by the cut on small clusters. Here we are considering planes 0 and 2 right after polar clustering (before any merging algorithms are run).  There is 1 shower, with the majority of the shower (green triangle) separated in space from the remaining bit of the shower (blue triangle).  Time is a shared coordinate across the planes, so while the angle may be different in different views, time will always line up (demonstrated below with t0 and t1 on the green cluster).  Now, because the green cluster has a steeper angle in plane 2 than in plane 0, the blue cluster appears closer to the green in plane 2.  These two clusters will likely be combined during flashlight merging.  However, the blue and green clusters in plane 0 are unlikely to be merged,  because they appear further apart. }
\label{fig:matching_ex_0}
\end{figure}

\begin{figure}[H]
\centering
\includegraphics[width=0.7\textwidth]{Cluster_Shower/matching_ex_1.png}
\caption{After merging completes, plane 0’s green cluster goes from t0 $\rightarrow$ t1, and plane 2’s green cluster goes from t0 $\rightarrow$ t2. Now it appears that plane 2’s cluster has a larger time overlap with the middle track than with the green plane 0 cluster.  Thus, if this track has not been removed during hit removal, matching will likely go awry unless the tiny cluster is prevented from entered the pool of clusters to-be-merged.
 }
\label{fig:matching_ex_1}
\end{figure}


\subsection{3D Shower Reconstruction}
Shower reconstruction uses 2D information created during the previous matching stage to create one 3D object. 
\paragraph{3D Direction}  We rely here on the reconstructed 3D interaction vertex to reconstruct the 2D projections. The 2D direction is computed as the charge-weighted average vector sum of the 2D distance from the vertex to each hit in the cluster.
\begin{equation}
  \hat{p}_{\rm 2D} = \frac{1}{Q_{tot}} \sum_{i=0}^{N} (r_i - r_{\rm vtx}) \times q_i 
\end{equation}
With N denoting the number of hits in the cluster, $r_i$ the position of the hit, $q_i$ its charge, and $r_{\rm vtx}$ the position of the projected vertex. Given two 2D weighted directions, the 3D direction is calculated using geometric relations between the planes and clusters \cite{bib:larliteGeoHelper}. 

\paragraph{3D Start Point Reconstruction} We calculate the 3D start point by finding the 3D overlap position of the OpenCV reconstructed 2D start points of the matched pair of clusters. The time tick coordinates from each cluster are averaged to calculate a 3D shared time coordinate. The (Y,Z) coordinates are identified by the intersection between the wires associated with 2D start points.  Wires must intersect inside the TPC for a shower to be reconstructed.  An example of successful 3D shower reconstruction (projected back into 2D) is shown in Figure \ref{fig:showers}.

\begin{figure}[h!] %H]
\centering
\fbox{\includegraphics[width=0.4\textwidth]{Cluster_Shower/Misc_opencv_showerreco.png}}
\caption{3D reconstructed showers are projected back into 2D as a visualization tool to indicate whether or not shower reconstruction is successful. }
\label{fig:showers}
\end{figure}

\subsection{Shower Energy Reconstruction}
\label{sec:ereco}
\par In this section we will briefly describe the series of correction factors and constants applied to the raw charge to account for physical processes, detector and electronics effects. All correction factors are derived using information only from the collection plane.  More detail on the motivation for these constants can be found externally \cite{bib:davidc_energycalibration}\cite{bib:davidc_recomb}.

%\begin{itemize}
\paragraph{Electronics Gain} 
%We begin by considering all collection plane hits associated to the shower.  Each hit's area represents the integrated ADC charge deposited on that wire at that point in time.  We convert that integrated charge from ADC to $e^-$'s by applying an electronics gain factor of 198 $e^-$ / ADC. More detail on the calculation of this constant can be found externally in MicroBooNE's noise paper ~\cite{bib:noise}.  
%\noindent The gain calibration values are different for data. In this case, we convert to $e^-$s by applying a value of 243 $e^-$ / ADC.  This factor is extracted from a sample of stopping muons and applied for both data samples.   
Here we utilize David C’s extracted constants from a sample of stopping muons.  The gain factor is 198 $\frac{e}{ADC}$ in MC and 243 $\frac{e}{ADC}$ in data. See Ref.~\cite{bib:davidc_energycalibration_gain} for more detail on the extraction of this factor.
\paragraph{ Lifetime Correction}: No lifetime correction is applied due to the high measured Ar purity and electron lifetime ~\cite{bib:purity}. 
\paragraph{Argon Ionization Work Function} We next convert our deposited electrons into an energy scale. To do this, we multiply our deposited electrons by the 23.6 $\frac{eV}{e}$ it takes to ionize a single electron in argon \cite{bib:ionization_per_electron}. 
\paragraph{Ion Recombination} Finally, we consider recombination.  A charged particle that traverses the liquid argon medium leaves a trail of ionized electrons and argon ions in its wake.  Some number of these electrons will recombine with argon ions rather than making it to the wire planes. This number depends on the local electric field and on $\frac{dE}{dx}$ of the ionizing particle. Here we apply a single, constant recombination correction of 0.423. This factor was obtained by assuming a fixed $\frac{dE}{dx}$ of 2.3 MeV/cm  and utilizing the Modified Box recombination model as parametrized by the ArgoNeuT collaboration ~\cite{bib:argoneut_recomb}, applied at MicroBooNE's electric field of 273 V/cm.  More detail on the extraction of this constant can be found externally \cite{bib:davidc_recomb}.
%\end{itemize}
\paragraph{Summary - Calibration Constants}
The calibration constants used in shower energy reconstruction for the rest of the note are summarized below:
\begin{equation}
  C_{MC} = 198 \frac{e^-}{\rm ADC} \times 23.6 \times 10^{-6} \frac{MeV}{e^-} \times \frac{1}{1-0.423} = 8.10 \times 10^{-2} \frac{\rm MeV}{\rm ADC}
\end{equation}

\begin{equation}
  C_{Data} = 243 \frac{e^-}{\rm ADC} \times 23.6 \times 10^{-6} \frac{MeV}{e^-} \times \frac{1}{1-0.423} = 9.94 \times 10^{-2} \frac{\rm MeV}{\rm ADC}
\end{equation}


\subsection{Shower Quality and Energy Correction}
%We begin by studying the reconstructed shower energy (via the shower's associated collection plane cluster).
We next consider the quality of the post-Selection II shower reconstruction discussed above.  Specifically, we study the reconstructed energy of the shower via its associated collection-plane cluster. To begin, we use the BackTracker module to identify and store all true particle energy depositions in the TPC at the hit level.  An example of the resulting `mcclusters' is shown in Figure \ref{fig:mcclusters}. At this point, we narrow our consideration to mcclusters that are produced by a neutrino-induced $\pi^0$. Using this set of $\pi^0$-related mcclusters, we define two metrics: Cluster `purity' is defined to be the number of hits in the reconstructed cluster associated to the mccluster with the largest overlap in hits over the total hits of the reco cluster. Cluster `completeness' is defined to be the number of hits in the reconstructed cluster associated to the mccluster with the largest overlap in hits over the total hits of the mccluster. A pictorial example is shown in Figure \ref{fig:showerquality_purcompex}. In this example, each reconstructed cluster has both a purity and a completeness.  The purity of the purple reconstructed cluster is 100\% because the reconstruction has only included hits from one truth cluster (green). On the other hand, the red reconstructed cluster was over-merged, and ended up with hits from both the green and blue truth clusters.  Thus, the purity of the red cluster is less than 1. We likewise consider the completeness of each reconstructed cluster. The completeness tells us how much of the truth cluster the reconstruction has captured.  In both the case of the purple and the red reconstructed clusters, hits were missed during reconstruction. Thus the completeness for both is less than 1.  The purity and completeness of our sample of reconstructed showers are shown in Figure \ref{fig:showerquality_purcomp}. Note that a purity of zero indicates that a reconstructed shower shares no hits with any neutrino-induced $\pi^0$ cluster in the event. 

\begin{figure}[h!]
\centering
\fbox{\includegraphics[scale=0.35]{Cluster_Shower/mcclusters.png}}
\caption{ Example of BackTracker-identified `mcclusters', zoomed in to the neutrino interaction in the Collection Plane. Each color ideally represents an individual truth level particle in cluster form. Note however that some of the colors repeat in this example, as the event display has a finite color wheel. }
\label{fig:mcclusters}
\end{figure}

\begin{figure}[t!]
\centering
  \begin{subfigure}[t]{0.4\textwidth}
    \centering
\includegraphics[scale=0.4]{Cluster_Shower/ShowerQuality_purComp.png}
  \caption{ }
  \end{subfigure} 
  \hspace{5mm}
  \begin{subfigure}[t]{0.45\textwidth}
    \centering
\includegraphics[scale=0.45]{Cluster_Shower/ShowerQuality_purComp_bothex.png}
  \caption{ }
  \end{subfigure} 
\caption{a) Example $\pi^0$ event with both the mcclusters and reco clusters pictured; b) Purity and completeness results for both reconstructed showers }
\label{fig:showerquality_purcompex}
\end{figure}


\begin{figure}[t!]
\centering
  \begin{subfigure}[t]{0.45\textwidth}
    \centering
\includegraphics[scale=0.45]{Cluster_Shower/ShowerQuality_purity.png}
  \caption{ }
  \end{subfigure} 
  \hspace{3mm}
  \begin{subfigure}[t]{0.45\textwidth}
    \centering
\includegraphics[scale=0.45]{Cluster_Shower/ShowerQuality_complete.png}
  \caption{ }
  \end{subfigure} 
\caption{Collection plane a) purity and b) completeness of reconstructed showers. }
\label{fig:showerquality_purcomp}
\end{figure}

\par  Finally, each reconstructed shower is matched to the `mccluster' it shares the greatest overlap with in hits. We are now in a position to compare reconstructed shower energy to truth. This is an important step towards understanding both our resolution and bias. With an understanding of the bias, we can correct our energies to get a sense of scale in the downstream analysis. The results of this comparison are shown in Figure \ref{fig:showerquality_eres} with an empirical fit. We will use this empirical fit to correct our reconstructed shower energies in data and MC for the remainder of this analysis; these corrected energies will be calculated according to the equation
\begin{equation}
\label{eq:ecorr}
E_{corr} = \frac{E_{reco}}{0.77} .
\end{equation}

\noindent We note that the corrected shower energy plays no role in the calculation of the cross section and only affects the kinematic distributions that are shown in later sections. Corrected plots will always be shown next to their uncorrected counterparts as a guide. 
\par The x = y line is included for reference and emphasizes the reconstructed energy bias away from truth. The contributions to this bias are (mainly) twofold. For one, roughly 10\% of the reconstructed energy is lost to hit thresholding and containment effects. The additional 15 \% is due to both the hit removal stage, where pieces of showers are sometimes excluded from clustering consideration, and clustering, where we sometimes miss hits. These bias are considered in detail externally \cite{bib:davidc_hitthresholding} \cite{bib:davidc_missingE}. Most events lie below the x = y line. The population of events above the line are instances of over-merging. This occurs when a reconstructed cluster is made up of two or more mcclusters and thus appears to have a higher energy than its associated truth cluster. We perform the same check in 1D in Figures \ref{fig:showerquality_eres_corr}a and b, with and without the corrected energy as described by Equation \ref{eq:ecorr}. The energy resolution observed here has two populations.  We explore the secondary bump away from 0 by considering single particle $\pi^0$ samples without and with cosmics.  When we do this, we see in the left of Figure \ref{fig:showerquality_eres_series} that the bump is absent from the resolution curve for single particle $\pi^0$ without cosmics.  The bump appears when we add cosmics in with the single particles (middle) and is accentuated further when we integrate neutrino-related track activity (right). This tells us that the bump is related to the presence of track-related activity.  During hit removal, hits associated with showers may be removed if they overlap with tracks.  The more track activity present in the event the more often this will happen, which is seen when we look at the CC $\pi^0$ + cosmics sample.  

\begin{figure}[h!]
\centering
\includegraphics[scale=0.5]{Cluster_Shower/ShowerQuality_eres2d.png}
\caption{True energy vs reconstructed energy for each reco-MC matched pair}
\label{fig:showerquality_eres}
\end{figure}

\begin{figure}[t!]
\centering
  \begin{subfigure}[t]{0.4\textwidth}
    \centering
\includegraphics[scale=0.4]{Cluster_Shower/ShowerQuality_eres.png}
  \caption{ }
  \end{subfigure} 
  \hspace{10mm}
  \begin{subfigure}[t]{0.4\textwidth}
    \centering
\includegraphics[scale=0.4]{Cluster_Shower/ShowerQuality_eres_corr.png}
  \caption{ }
  \end{subfigure} 
\caption{a) Energy resolution in 1D, where $E_{MC}$ is the true shower energy; b) Corrected energy resolution in 1D using the extracted correction factor. The quoted mean and sigma related to the fitted Gaussian.}
\label{fig:showerquality_eres_corr}
\end{figure}

\begin{figure}[t!]
\centering
 \includegraphics[scale=0.7]{Cluster_Shower/ShowerQuality_eres_progression.png}
 \caption{Energy resolution of reconstructed showers in single particle $\pi^0$ samples without cosmics (left), with cosmics (middle), and CC $\pi^0$ + cosmics (right). Note that $E_{MC}$ is the true shower energy; }
\label{fig:showerquality_eres_series}
\end{figure}


To finish up our discussion of shower quality, we additionally consider shower start point resolution using the same MC-reco shower matches described above. We consider the MC shower start point to be the point of first energy deposition in the detector. The resulting resolutions are shown in Figure \ref{fig:showerquality_xyzres}. Note that start point resolution by itself is an imperfect metric for assessing the quality of shower reconstruction. Sometimes the trunk of the shower is very linear and removed during hit removal the result is often a poor start point resolution, despite the shower being usable to identify the presence of a $\pi^0$. An example is shown in Figure \ref{fig:showerquality_startpointoff}.  Note that this effect also contributes to the worse resolution seen in the Z coordinate in Figure \ref{fig:showerquality_xyzres}.

\begin{figure}[h!]
\centering
\includegraphics[scale=0.7]{Cluster_Shower/ShowerQuality_startpointoff.png}
\caption{Example of an event in which the beginning of the shower has been removed during hit removal, but which still may be used to identify the presence of a $\pi^0$ }
\label{fig:showerquality_startpointoff}
\end{figure}

\begin{figure}[h!]
\centering
  \begin{subfigure}[t]{0.45\textwidth}
    \centering
\includegraphics[scale=0.45]{Cluster_Shower/ShowerQuality_diff_x.png}
  \caption{ }
  \end{subfigure} 
  \hspace{5mm}
  \begin{subfigure}[t]{0.45\textwidth}
    \centering
\includegraphics[scale=0.45]{Cluster_Shower/ShowerQuality_diff_y.png}
  \caption{ }
  \end{subfigure} 
  \hspace{5mm}
  \begin{subfigure}[t]{0.45\textwidth}
    \centering
\includegraphics[scale=0.45]{Cluster_Shower/ShowerQuality_diff_z.png}
  \caption{ }
  \end{subfigure} 
  \hspace{5mm}
  \begin{subfigure}[t]{0.45\textwidth}
    \centering
\includegraphics[scale=0.45]{Cluster_Shower/ShowerQuality_tot.png}
  \caption{ }
  \end{subfigure} 
\caption{ Matched showers resolution in a) X, b) Y, c) Z, and d) total 3D. The values displayed in the text box are the mean and standard deviation of the distribution. }
\label{fig:showerquality_xyzres}
\end{figure}

\clearpage
\subsection{Shower Reconstruction Efficiency}
Finally, we consider our shower reconstruction efficiency. This information will give us a sense of what kind of overall efficiency to expect at the final stage of selection. To do this, we consider all signal CC 1$\pi^0$ events that passed the Selection II + MIP consistency cuts. Of this sample, 3\% of the photons that originate from the $\pi^0$ decay convert outside the active TPC volume. Of the remaining showers that convert inside the TPC, we look at the shower energy dependent reconstruction efficiency. We define this efficiency to be the number of reconstructed showers that are matched to a true signal shower divided by the number of true signal showers. The results are shown in Figure \ref{fig:shower_reco_efficiency}.  The majority of the subleading shower population is below 100 MeV, where shower reconstruction efficiency drops below 50\%.  This is the main reason for the large number of signal events that have only one signal shower reconstructed, as we will see in the next subsection. In the past, we explored methods to perform a second pass reconstruction to increase our efficiency for subleading photons, but there was limited success \cite{bib:jz_catch_subleading}.  

\begin{figure}[h!]
\centering
\includegraphics[scale=0.7]{Cluster_Shower/shower_reco_efficiency.png}
\caption{Shower reconstruction efficiency after Selection II, with the subleading and leading energy distributions normalized to fit in the same plot.  }
\label{fig:shower_reco_efficiency}
\end{figure}


\subsection{Data - MC Shower Comparison}

In this section, we compare absolutely normalized shower reconstruction parameters between data and simulation. An important first step is to consider the number of showers reconstructed per event.  In Figure \ref{fig:physics_sel2_nshrs}, we see a large discrepancy in the 0 reconstructed showers bin, and better agreement for higher numbers of showers. It is difficult to draw a strong conclusion on the 0-bin disagreement due to the data-MC discrepancy observed at the Selection II stage, however, we note that we will only be focusing on events with at least 1 shower to tag $\pi^0$ candidates for the remainder of this note where the agreement is considerably better. Kinematic distributions related specifically to events with 0-showers reconstructed can be found in Appendix \ref{sec:AppC}. We additionally consider the origin of the reconstructed showers in each bin in Figure \ref{fig:physics_sel2_nshrs}. These breakdowns are shown for 1 - 7 reconstructed showers in Figure \ref{fig:physics_sel2_nshr_background_breakdown}. As expected, most identified electromagnetic activity is due to a $\pi^0$. Note that sometimes more than 1 reconstructed shower will correspond to the same truth shower; we discuss a cut on opening angle later to prevent these kinds of shower pairs from entering our final sample.  An event mode breakdown is shown in Figure \ref{fig:physics_gt0shower_eventtype}.

% 
\begin{figure}[t!]
\centering
  \begin{subfigure}[t]{0.3\textwidth}
    \centering
\includegraphics[scale=0.3]{Selection_II_Section/Physics_sel2_onoffseparate_nshrs.png}
  \caption{ }
  \end{subfigure} 
  \hspace{30 mm}
  \begin{subfigure}[t]{0.3\textwidth}
    \centering
\includegraphics[scale=0.3]{Selection_II_Section/Physics_sel2_onoffseparate_nshrs_log.png}
  \caption{ }
  \end{subfigure} 

\caption{ Data to simulation comparison of number of reconstructed showers per event in a) linear scale and b) log scale. }
\label{fig:physics_sel2_nshrs}
\end{figure}

\begin{figure}[t!]
\centering
  \begin{subfigure}[t]{0.25\textwidth}
    \centering
\includegraphics[scale=0.25]{Cluster_Shower/Physics_showerOriginBreakdown_nshrs_1shr.png}
  \caption{ }
  \end{subfigure} 
  \hspace{4mm}
  \begin{subfigure}[t]{0.25\textwidth}
    \centering
\includegraphics[scale=0.25]{Cluster_Shower/Physics_showerOriginBreakdown_nshrs_2shr.png}
  \caption{ }
  \end{subfigure} 
  \hspace{4mm}
  \begin{subfigure}[t]{0.25\textwidth}
    \centering
\includegraphics[scale=0.25]{Cluster_Shower/Physics_showerOriginBreakdown_nshrs_3shr.png}
  \caption{ }
  \end{subfigure} 
  \hspace{4mm}
  \begin{subfigure}[t]{0.25\textwidth}
    \centering
\includegraphics[scale=0.25]{Cluster_Shower/Physics_showerOriginBreakdown_nshrs_4shr.png}
  \caption{ }
  \end{subfigure} 
  \hspace{4mm}
  \begin{subfigure}[t]{0.25\textwidth}
    \centering
\includegraphics[scale=0.25]{Cluster_Shower/Physics_showerOriginBreakdown_nshrs_5shr.png}
  \caption{ }
  \end{subfigure} 
  \hspace{4mm}
  \begin{subfigure}[t]{0.25\textwidth}
    \centering
\includegraphics[scale=0.25]{Cluster_Shower/Physics_showerOriginBreakdown_nshrs_6shr.png}
  \caption{ }
  \end{subfigure} 
  \hspace{4mm}
  \begin{subfigure}[t]{0.25\textwidth}
    \centering
\includegraphics[scale=0.25]{Cluster_Shower/Physics_showerOriginBreakdown_nshrs_7shr.png}
  \caption{ }
  \end{subfigure} 
\caption{ Breakdown of shower origin for each number of shower per event bin in Figure \ref{fig:physics_sel2_nshrs}. }
\label{fig:physics_sel2_nshr_background_breakdown}
\end{figure}

\begin{figure}[h!]
\centering
\includegraphics[scale=0.4]{Cluster_Shower/Misc_gt0shower_EventType_vs_NeutrinoMode_w_Numbers.png}
\caption{ Event type breakdown for all events with at least 1 reconstructed shower. }
\label{fig:physics_gt0shower_eventtype}
\end{figure}

\noindent We next consider a series of variables for all reconstructed showers. Because this comparison naturally excludes events falling into the 0-shower bin in Figure \ref{fig:physics_sel2_nshrs}, we expect better data-MC agreement in these comparisons. We find the rates and shapes of reconstructed showers agree relatively well in start point and direction (Figures \ref{fig:physics_sel2_shr_x} - \ref{fig:physics_sel2_shr_z}), opening angle of the shower, the distance from the vertex to the shower start point (Figures \ref{fig:physics_sel2_shr_rl}a, b), energy and corrected energy (Figures \ref{fig:physics_sel2_shr_e}a, b).  Note that the conversion distance distribution in Figure \ref{fig:physics_sel2_shr_rl}b drops off at the ROI boundary of 1 m.  Because the ROI is a rectangle rather than a circle, we also get some spill over 1 m where showers have converted in the corners of the ROI. 

\begin{figure}[t!]
\centering
  \begin{subfigure}[t]{0.3\textwidth}
    \centering
\includegraphics[scale=0.3]{Cluster_Shower/Physics_sel2_onoffseparate_shr_startx.png}
  \caption{ }
  \end{subfigure} 
  \hspace{30mm}
  \begin{subfigure}[t]{0.3\textwidth}
    \centering
\includegraphics[scale=0.3]{Cluster_Shower/Physics_sel2_onoffseparate_shr_dirx.png}
  \caption{ }
  \end{subfigure} 
\caption{ Data to simulation comparisons of shower a) start point and b) direction in x.}
\label{fig:physics_sel2_shr_x}
\end{figure}


\begin{figure}[h!]
\centering
  \begin{subfigure}[t]{0.3\textwidth}
    \centering
\includegraphics[scale=0.3]{Cluster_Shower/Physics_sel2_onoffseparate_shr_starty.png}
  \caption{ }
  \end{subfigure} 
  \hspace{30mm}
  \begin{subfigure}[t]{0.3\textwidth}
    \centering
\includegraphics[scale=0.3]{Cluster_Shower/Physics_sel2_onoffseparate_shr_diry.png}
  \caption{ }
  \end{subfigure} 

\caption{ Data to simulation comparisons of shower a) start point and b) direction in y.}
\label{fig:physics_sel2_shr_y}
\end{figure}


\begin{figure}[t!]
\centering
  \begin{subfigure}[t]{0.3\textwidth}
    \centering
\includegraphics[scale=0.3]{Cluster_Shower/Physics_sel2_onoffseparate_shr_startz.png}
  \caption{ }
  \end{subfigure} 
  \hspace{30mm}
  \begin{subfigure}[t]{0.3\textwidth}
    \centering
\includegraphics[scale=0.3]{Cluster_Shower/Physics_sel2_onoffseparate_shr_dirz.png}
  \caption{ }
  \end{subfigure} 
\caption{ Data to simulation comparisons of shower a) start point and b) direction in z.}
\label{fig:physics_sel2_shr_z}
\end{figure}


\begin{figure}[h!]
\centering
  \begin{subfigure}[t]{0.3\textwidth}
    \centering
\includegraphics[scale=0.3]{Cluster_Shower/Physics_sel2_onoffseparate_shr_oangle.png}
  \caption{ }
  \end{subfigure} 
  \hspace{20mm}
  \begin{subfigure}[t]{0.3\textwidth}
    \centering
\includegraphics[scale=0.3]{Cluster_Shower/Physics_sel2_onoffseparate_shr_vtx_dist.png}
  \caption{ }
  \end{subfigure} 
\caption{ Data to simulation comparison of shower in a) opening angle and b) start point to vertex distance. The cutoff at around 100~cm is due the ROI bound and the requirement that a shower cannot pierce or lay outside this bound, the tail beyond 1~m is due to the ROI being rectangular.}
\label{fig:physics_sel2_shr_rl}
\end{figure}


\begin{figure}[h!]
\centering
  \begin{subfigure}[t]{0.4\textwidth}
    \centering
\includegraphics[scale=0.4]{Cluster_Shower/Physics_sel2_onoffseparate_shr_energy.png}
  \caption{ }
  \end{subfigure} 
  \hspace{10mm}
  \begin{subfigure}[t]{0.4\textwidth}
    \centering
\includegraphics[scale=0.4]{Cluster_Shower/Physics_sel2_onoffseparate_shr_energy_corr.png}
  \caption{ }
  \end{subfigure} 
\caption{ Data to simulation comparison of shower in a) energy and b) corrected energy.  }
\label{fig:physics_sel2_shr_e}
\end{figure}


\clearpage
\section{$\pi^0$ Selection} 

In this section we will use the reconstructed shower information we just created to identify events that contain a $\pi^0$ candidate. This is broken into two separate but complementary paths: one that considers all events post-SelectionII with at least two reconstructed showers and another that considers all events post-SelectionII with at least one reconstructed shower. There is a large overlap between these two final samples, but one is not a subset of the other.  
We begin our search for 2-shower candidates by considering the events with at least 2 reconstructed showers (Figure \ref{fig:physics_pi0_nshrs}).  We see that by requiring at least 2 showers we reduce our sample size by 95\%, while also creating a highly signal-enriched selection. Pass rate and composition breakdowns after this at-least-2-showers requirement are shown in Tables \ref{tab:pi0_2showers_eventrates} and \ref{tab:pi0_2showers_composition}. 

\subsection{Candidates With At Least Two Showers}
Now that we have a highly signal-enriched sample, we're almost in a position to make a set of final selection comparison plots for the two shower sample.  Before we do this, we consider two additional items. First, because we have now performed shower reconstruction and have access to the backtracked information about those showers, we re-classify background interaction types to give the reader a more precise sense of what we're selecting. Once we've done this, we apply a series of checks and cuts to ensure that our selected sample maintains a low cosmic background and is well-reconstructed.

\begin{figure}[H]
\centering
  \begin{subfigure}[t]{0.3\textwidth}
    \centering
\includegraphics[scale=0.3]{Pi0_Cut_Section/Physics_sel2gt1shower_onoffseparate_nshrs.png}
  \caption{ }
  \end{subfigure} 
  \hspace{30mm}
  \begin{subfigure}[t]{0.3\textwidth}
    \centering
    \includegraphics[scale=0.3]{Pi0_Cut_Section/Physics_sel2gt1shower_onoffseparate_nshrs_log.png}
  \caption{ }
  \end{subfigure} 
\caption{ Data to simulation comparison of number of reconstructed showers in a) linear and b) log scale. }
\label{fig:physics_pi0_nshrs}
\end{figure}

\begin{table}[H]
\centering
\captionof{table}{ Pass rate breakdown after 2 shower requirement scaled to OnBeam POT 
\label{tab:pi0_2showers_eventrates}}
 \begin{tabular}{| l | l | l |l|l|l|l|l|}
 \hline
 & CC 1$\pi^0$ & CC 0$\pi^0$ & NC $\pi^0$ & NC 0$\pi^0$ & Other & Total \\ [0.1ex] \hline
Selection II & 0.331 & 0.106 & 0.008 & 0.010 & 0.030 & 0.087 \\
At Least Two Showers & 0.106 & 0.001 & 0.002 & 0.0002 & 0.008 & 0.003 \\ \hline
\end{tabular}
\end{table}

\begin{table}[H]
\centering
\captionof{table}{ Sample composition after 2 shower requirement scaled to OnBeam POT 
\label{tab:pi0_2showers_composition}}
 \begin{tabular}{| l | l | l |l|l|l|l|l|}
 \hline
 & CC 1$\pi^0$ & CC 0$\pi^0$ & NC $\pi^0$ & NC 0$\pi^0$ & Other & Cosmic (MC) & Cosmic (Data) \\ [0.1ex] \hline
Selection II & 0.060 & 0.743 & 0.004 & 0.020 & 0.014 & 0.042 & 0.117 \\
At Least Two Showers & 0.624 & 0.179 & 0.037 & 0.010 & 0.120 & 0.016 & 0.014\\ \hline
\end{tabular}
\end{table}

\subsubsection{Signal and Backgrounds for $\pi^0\rightarrow\geq 2 \gamma$}
The background breakdowns discussed throughout the note thus far were chosen to give the reader a sense of sample composition at each selection stage given the information at hand. This was done by backtracking the origin of the candidate $\mu$ to first identify cosmic- and noise-originating events. For the remaining $\nu$-origin candidates, we assigned each interaction to a background category based on its truth information. At this point, we have additional information about our reconstructed shower candidates.  Because of this new information, it's important that we re-classify all interactions in which at least 1 of the showers is of cosmic origin into a 'cosmic' category to maintain a pure sample of signals. We will describe these new backgrounds below and use them in the following data-MC comparison plots, however we will continue to present pass rate and composition tables using the backgrounds described at the beginning of the note.  This is because there isn't a good way to translate these new background definitions, that use reconstructed shower information, back to Selection II or the No Cuts stage, before that information existed. Thus, for continuity with the rest of the note, we choose to present the original backgrounds in those tables.  

\paragraph{ Cosmics}
This category remains un-changed from the previously described background breakdowns.  It represents the OffBeam data contribution to the stack. 

\paragraph{ Cosmic + $\nu$}
Events previously tagged as ``Cosmic - $\nu$ Coincident" enter this category automatically.  We next consider the backtracked origin of each candidate shower in our remaining sample.  If either candidate shower is of cosmic origin, we add the event to this category. An example of this type of event is shown in Figure \ref{fig:backgrounds_cosmic_nuflash}.

\paragraph{ $\nu_\mu$ CC 1 $\pi^0$ } Our signal event classification in these plots is the same as before with one additional requirement: both reconstructed showers must be neutrino-induced.  All signals with cosmic-origin showers are reclassified as `Cosmic + $\nu$' under this scheme.

\paragraph{$\nu_\mu$ CC Charge Exchange}
This background contains all instances of $\nu_\mu$-induced CC charge exchange outside of the initial neutrino interaction point in which both showers are of $\nu$-origin. An example of this interaction type in our final sample is shown in Figure \ref{fig:backgrounds_cc}a. 

\paragraph{$\nu_\mu$ CC Multiple $\pi^0$}
This background contains all instances of $\nu_\mu$-induced CC multiple $\pi^0$ production in which both candidate showers are of $\nu$-origin.  An example of this interaction type in our final sample is shown in Figure \ref{fig:backgrounds_other}b.

\paragraph{$\nu_\mu$ NC $\pi^0$}
This background contains all instances of $\nu_\mu$-induced NC $\pi^0$ production in which both candidate showers are of $\nu$-origin. Note that NC multiple $\pi^0$ events are included in this category.  An example of this interaction type in our final sample is shown in Figure \ref{fig:backgrounds_nc}d. 

\paragraph{ $\nu_\mu$ Final State Electro-Magnetic Activity ($\nu_\mu$ FSEM)}
Here we attempt to separate the remaining events into classifications of with and without electromagnetic activity in the final state.  Final State EM events are $\nu_\mu$ events in which both reconstructed showers are of $\nu$-origin, and at least one of these reconstructed showers is due to a $\nu$-induced true shower. This category contains, for example, proton and neutron inelastic scatters which produce EM showers (Figures \ref{fig:backgrounds_cc}b ,c and Figures \ref{fig:backgrounds_nc}a,b), kaon decays (Figure \ref{fig:backgrounds_cc}d)),  interactions with 1 or more photons in the final state (Figure \ref{fig:backgrounds_cc}e), mis-reconstructed events where at least one reconstructed candidate is from a $\nu$-induced shower (Figure \ref{fig:backgrounds_cc}f), NC Charge Exchange (Figure \ref{fig:backgrounds_nc}a) and $\nu_\mu$ CC 1$\pi^0$ out of the FV (Figure \ref{fig:backgrounds_other}a).

\paragraph{Other}
This background contains all of what's left over after all other backgrounds have been assigned.  This includes $\nu_e$'s (Figure \ref{fig:backgrounds_other}c), $\overline{\nu_\mu}$'s, and all $\nu_\mu$ CC events with no EM activity in the final state.

\begin{figure}[h!]
\centering
\includegraphics[scale=0.65]{Pi0_Cut_Section/backgrounds_cosmic_neutrino_flash.png}
\caption{ Example of background selected in final stage for Cosmic + $\nu$ category; tagged track is shown in solid red, tagged vertex is shown in cyan, 3d-projected candidate $\pi^0$ showers are shown in blue and green triangles, track-like hits are shown in black, shower-like hits are shown in red. }
\label{fig:backgrounds_cosmic_nuflash}
\end{figure}


\begin{figure}[h!]
\centering
\includegraphics[scale=0.65]{Pi0_Cut_Section/backgrounds_cc.png}
\caption{ Examples of various CC background events in our final selected sample; tagged track is shown in solid red, tagged vertex is shown in cyan, 3d-projected candidate $\pi^0$ showers are shown in blue and green triangles, track-like hits are shown in black, shower-like hits are shown in red. }
\label{fig:backgrounds_cc}
\end{figure}

\begin{figure}[h!]
\centering
\includegraphics[scale=0.65]{Pi0_Cut_Section/backgrounds_nc.png}
\caption{ Examples of various NC backgrounds selected in final stage; tagged track is shown in solid red, tagged vertex is shown in cyan, 3d-projected candidate $\pi^0$ showers are shown in blue and green triangles, track-like hits are shown in black, shower-like hits are shown in red. }
\label{fig:backgrounds_nc}
\end{figure}

\begin{figure}[h!]
\centering
\includegraphics[scale=0.65]{Pi0_Cut_Section/backgrounds_other.png}
\caption{ Examples of various additional backgrounds selected in final stage; tagged track is shown in solid red, tagged vertex is shown in cyan, 3d-projected candidate $\pi^0$ showers are shown in blue and green triangles, track-like hits are shown in black, shower-like hits are shown in red. }
\label{fig:backgrounds_other}
\end{figure}

\subsubsection{Tuning Selection for Events with Two Showers}
Now that we've re-organized our backgrounds based on the new reconstructed shower information, we want to clean up the sample with a few additional checks.  We first break remaining events (events with at least 2 showers) into signal and background samples. We do this to get a sense of whether there is any clear separation between the sample variables we will consider in this section. The signal sample is defined to be all CC events that have a single $\pi^0$ originating from the neutrino vertex in the FV (same as our signal sample throughout the note thus far).  The remaining non-signal events are broken into 2 kinds of backgrounds.  The first kind of background is defined as events that are not signal, but have any number of $\pi^0$'s in the final state induced by the neutrino interaction. The second kind of background has no neutrino-induced $\pi^0$'s. 

\begin{figure}[H]
\centering
  \begin{subfigure}[t]{0.35\textwidth}
    \centering
\includegraphics[scale=0.35]{Pi0_Cut_Section/CutJustify_pi0v2_eff_gamma_IP.png}
  \caption{ }
  \end{subfigure} 
  \hspace{20mm}
  \begin{subfigure}[t]{0.35\textwidth}
    \centering
\includegraphics[scale=0.35]{Pi0_Cut_Section/Physics_showerPostSel2_onoffseparate__gamma_IP.png}
  \caption{ }
  \end{subfigure} 
\caption{ Impact parameter of reconstructed showers in events with 2 or more reconstructed showers for a) Unscaled MCBNB + Cosmics; b) Data to simulation comparison }
\label{fig:cutjust_pi0_IP}
\end{figure}

\begin{figure}[H]
\centering
  \begin{subfigure}[t]{0.25\textwidth}
    \centering
\includegraphics[scale=0.7]{Pi0_Cut_Section/ex_2shower_angular_res_IP.png}
  \caption{ }
  \end{subfigure} 
  \hspace{40mm}
  \begin{subfigure}[t]{0.25\textwidth}
    \centering
\includegraphics[scale=0.7]{Pi0_Cut_Section/ex_2shower_mass_peak.png}
  \caption{ }
  \end{subfigure} 
\caption{ a) Angular resolution of reconstructed showers with and without a conservative 4 cm cut on IP; b) The effect of the poorer angular resolution events propagates through to the two-shower mass peak, smearing it towards lower energies. }
\label{fig:ex_cutjust_IP}
\end{figure}

Next we examine potential variables to use for separation between these signal and background samples  and improve the overall quality of the reconstructed $\pi^{0}$.  First, we consider the impact parameter of every 2 shower permutation.  This is shown broken down by the sample definitions described in the previous paragraph in Figure \ref{fig:cutjust_pi0_IP}a, and with a data-MC comparison in Figure \ref{fig:cutjust_pi0_IP}b.  Note that we expect small impact parameters between most shower pairs, as the vertex is used to reconstruct 3D direction. However, sometimes mis-matches are made, mis-reconstruction occurs, or clusters are under-merged resulting in an incorrect reconstructed direction; the effect of IP on angular resolution of our reconstructed showers can be seen in Figure \ref{fig:ex_cutjust_IP}a. The effect seems small at first glance, but when we propagate the effect through to the reconstructed 2-shower mass peak, we see that the poorer angular resolution of high IP events smears the mass plot towards lower energies (Figure \ref{fig:ex_cutjust_IP}b). Thus, a cut on impact parameter between showers is important to ensure both spatial correlation between showers, and higher quality angular resolution.  To be conservative, we choose here an impact parameter $\leq$ 4cm.  
We next consider a cut on small opening angle between reconstructed showers. Figure \ref{fig:mcvar_pi0_onlyoangle}a shows the true opening angle between showers of signal events from our original 420k MCBNB+Cosmics sample (black), and after SelectionII (red). This distribution suggests that we expect few events to have a small true opening angle. Additionally, $\pi^0$ candidate pairs that have very small reconstructed opening angles tend to be separately reconstructed showers from the same original true shower (Figure \ref{fig:mcvar_pi0_onlyoangle}b). This point is emphasized in Figure \ref{fig:comp_cutjust_pi0_OA} where we consider the completeness of leading and subleading shower candidate pairs as a function of opening angle. We see that this leads to subleading showers with very low completeness (middle plot), and leading showers with lower-than-average completeness (right plot). Thus, we choose a lower bound of 0.35 rad (20 degrees) to maximize the quality of our event sample without significantly damaging the efficiency. Opening angle is shown broken down by the sample definitions described in the previous paragraph in Figure \ref{fig:cutjust_pi0_OA}a, and with a data-MC comparison in Figure \ref{fig:cutjust_pi0_OA}b. % Finally, we consider conversion distances of both showers (Figures \ref{fig:cutjust_pi0_RL} \ref{fig:cutjust_pi0_low_RL}). We note the that there is very little separation power in this variable and also no reconstruction-related justification for a conversion distance cut. We thus do not employ a conversion distance cut at this time. We also do not include an energy or mass peak cut.


\begin{figure}[H]
\centering
  \begin{subfigure}[t]{0.35\textwidth}
    \centering
     \includegraphics[scale=0.35]{XSection_Calc_Section/MCVar_pi0_OnlyOangle.png}
     \caption{ }
  \end{subfigure} 
  \begin{subfigure}[t]{0.6\textwidth}
    \centering
    \includegraphics[scale=0.6]{Pi0_Cut_Section/ex_oangle_bad_signal.png}
    \caption{ }
  \end{subfigure} 
  \caption{ a) Opening angle distribution of signal CC single $\pi^0$ events in a 420k sample of MCBNB + Cosmics; b) Example signal event where both reconstructed candidate showers originate from the same true shower. }
\label{fig:mcvar_pi0_onlyoangle}
\end{figure}

\begin{figure}[H]
\centering
  \begin{subfigure}[t]{0.35\textwidth}
    \centering
\includegraphics[scale=0.35]{Pi0_Cut_Section/comp_gamma_oangle_signal_Leading.png}
  \caption{ }
  \end{subfigure} 
  \hspace{20mm}
  \begin{subfigure}[t]{0.35\textwidth}
    \centering
    \includegraphics[scale=0.35]{Pi0_Cut_Section/comp_gamma_oangle_signal_Subleading.png}
  \caption{ }
  \end{subfigure} 
\caption{ Completeness of a) leading and b) subleading showers against opening angle. Here we see clearly that at low opening angle, the subleading completeness is very low. }
\label{fig:comp_cutjust_pi0_OA}
\end{figure}

\begin{figure}[H]
\centering
  \begin{subfigure}[t]{0.35\textwidth}
    \centering
\includegraphics[scale=0.35]{Pi0_Cut_Section/CutJustify_pi0v2_eff_gamma_oangle.png}
  \caption{ }
  \end{subfigure} 
  \hspace{20mm}
  \begin{subfigure}[t]{0.35\textwidth}
    \centering
    \includegraphics[scale=0.35]{Pi0_Cut_Section/Physics_showerPostSel2_onoffseparate__gamma_oangle.png}
  \caption{ }
  \end{subfigure} 
\caption{ Opening angle of reconstructed showers in events with 2 or more reconstructed showers for a) Unscaled MCBNB + Cosmics; b) Data to simulation comparison }
\label{fig:cutjust_pi0_OA}
\end{figure}

% \begin{figure}[H]
% \centering
%   \begin{subfigure}[t]{0.35\textwidth}
%     \centering
% \includegraphics[scale=0.35]{Pi0_Cut_Section/CutJustify_pi0v2_eff_gamma_high_RL.png}
%   \caption{ }
%   \end{subfigure} 
%   \hspace{20mm}
%   \begin{subfigure}[t]{0.35\textwidth}
%     \centering
% \includegraphics[scale=0.35]{Pi0_Cut_Section/Physics_showerPostSel2_onoffseparate__gamma_high_RL.png}
%   \caption{ }
%   \end{subfigure} 
% \caption{ Conversion distance of leading shower in candidate pair for a) Unscaled MCBNBC + Cosmics; b) Data to simulation comparison }
% \label{fig:cutjust_pi0_RL}
% \end{figure}

% \begin{figure}[H]
% \centering
%   \begin{subfigure}[t]{0.35\textwidth}
%     \centering
% \includegraphics[scale=0.35]{Pi0_Cut_Section/CutJustify_pi0v2_eff_gamma_low_RL.png}
%   \caption{ }
%   \end{subfigure} 
%   \hspace{20mm}
%   \begin{subfigure}[t]{0.35\textwidth}
%     \centering
% \includegraphics[scale=0.35]{Pi0_Cut_Section/Physics_showerPostSel2_onoffseparate__gamma_low_RL.png}
%   \caption{ }
%   \end{subfigure} 
% \caption{  Conversion distance of subleading shower in candidate pair for a) Unscaled MCBNBC + Cosmics; b) Data to simulation comparison }
% \label{fig:cutjust_pi0_low_RL}
% \end{figure}

Finally, we consider the "Cosmic + Neutrino" background contribution to our current sample.  Figure \ref{fig:physics_radl_before_after}a shows that the cosmic contribution at this point is 11 \%. Because the cosmic contributions is modeled via MC systematics, we believe it is important to mitigate this background to avoid accruing a large uncertainty on the final cross section.

We begin by considering the raw reconstructed energy of both candidate showers for signal and cosmic+$\nu$ background. We find that there is separation power between the two samples when both reconstructed candidate shower energies are $<$ 40 MeV (Figure \ref{fig:cutjust_pi0_e}a).  We also notice that the conversion distance is relatively flat across the cosmic sample's candidate shower; thus, we additionally cut events where either the leading shower conversion distance is $>$ 80cm, or the subleading shower distance is $>$ 100cm (Figure \ref{fig:cutjust_pi0_e}b). When we implement these cuts, we find that the cosmic contribution to the 2-shower sample reduces to 5\% (Figure \ref{fig:physics_radl_before_after}b).  This is a 50\% reduction is cosmic background, at the smaller cost of 6\% of signal. At this point, we are more confident that assessing a conservative 100\% uncertainty on this background will not overwhelm our final cross sectional uncertainties.
 
% If we assess a 100\% uncertainty on this background, we find that it adds an additional 18\% final fractional uncertainty on the cross section.
 
\begin{figure}[t!]
\centering
  \begin{subfigure}[t]{0.4\textwidth}
    \centering
\includegraphics[scale=0.4]{Pi0_Cut_Section/Paper_pi0_onoffseparate_pi0_low_radL.png}
  \caption{ }
  \end{subfigure} 
  \hspace{20mm}
  \begin{subfigure}[t]{0.4\textwidth}
    \centering
\includegraphics[scale=0.4]{Pi0_Cut_Section/Paper_pi0_onoffseparate_pi0_radL_w_cut.png}
  \caption{ }
  \end{subfigure} 
\caption{ Data to simulation comparisons of conversion distances for all candidate showers in the two shower sample a) before and b) after additional minimum energy and conversion distance cuts. }
\label{fig:physics_radl_before_after}
\end{figure}


\begin{figure}[t!]
\centering
  \begin{subfigure}[t]{0.4\textwidth}
    \centering
\includegraphics[scale=0.4]{Pi0_Cut_Section/CutJust_pi0_E.png}
  \caption{ }
  \end{subfigure} 
  \hspace{20mm}
  \begin{subfigure}[t]{0.4\textwidth}
    \centering
\includegraphics[scale=0.4]{Pi0_Cut_Section/CutJust_pi0_RL.png}
  \caption{ }
  \end{subfigure} 
\caption{ Data to simulation comparison of a) multiplicity and b) $\mu$ length after single shower filter }
\label{fig:cutjust_pi0_e}
\end{figure}

\par If a pair of showers passes all these criteria, they are considered to be a $\pi^0$ candidate. If more than one candidate pair is found per event, the event is neglected.

\clearpage
\subsubsection{Results for $\pi^0\rightarrow \geq 2 \gamma$}
\par A table with event rates scaled to OnBeam POT is shown for OnBeam, OffBeam, and MCBNBCosmic in Table \ref{tab:2shpi0_event_rates}. Backgrounds by neutrino interaction mode are shown in Figure \ref{fig:physics_pi0_inttype}. 

\begin{figure}[h!]
\centering
\includegraphics[scale=0.5]{Pi0_Cut_Section/Misc_pi0_EventType_vs_NeutrinoMode_w_Numbers.png}
\caption{ Event type broken down by neutrino interaction mode; note the small contribution of MEC events to the final selected sample. }
\label{fig:physics_pi0_inttype}
\end{figure}

\begin{table*} 
 \centering
 \captionof{table}{Event counts at each stage of the full 2-shower CC $\pi^0$ selection chain with MCC8.3 samples scaled to OnBeam POT. Note that all OffBeam events in our current sample are removed by the two-shower chain.  For now, we assess an uncertainty for this sample on 1 event. \label{tab:2shpi0_event_rates}}
 \begin{tabular}{| l | l | l | l | l |}
  \hline
   & OnBeam & OffBeam & On - OffBeam & MCBNBCos \\ [0.1ex] \hline
No Cuts & 544751 $\pm$ 738 & 462076 $\pm$ 1001 & 82675 $\pm$ 1244 & 48949 $\pm$ 76 \\ 
Selection II & 3753 $\pm$ 61 & 564 $\pm$ 35 & 3189 $\pm$ 71 & 4268 $\pm$ 22 \\ 
2 Shower Cuts & 69 $\pm$ 8 & 0 $\pm$ 2 & 67 $\pm$ 9 & 74 $\pm$ 2  \\ \hline
\end{tabular}
 \end{table*}

\par A summary of the passing rates for signal and all backgrounds is shown in Table \ref{tab:2shpi0_passrates}.  Note that we have maintained a relatively high efficiency for the signal with respect to other listed backgrounds. Sample composition is shown in Table \ref{tab:2shpi0_purity}. 

\begin{table*}
\centering
\captionof{table}{Evolution of passing rates through full analysis chain \label{tab:2shpi0_passrates}}
 \begin{tabular}{| l | l | l |l|l|l|l|}
 \hline
 & CC 1$\pi^0$ & CC 0$\pi^0$ & NC $\pi^0$ & NC 0$\pi^0$ & Other & All \\ [0.1ex] \hline
No Cuts & - & - & - & - & - & -\\
SelectionII & 0.331 & 0.106 & 0.008 & 0.010 & 0.030 & 0.099 \\ 
2 Shower Cuts & 0.056 & 0.0003 & 0.001 & 0.0001 & 0.003 & 0.001 \\ \hline
\end{tabular}
\end{table*}

\begin{table*}
\centering
\captionof{table}{Evolution of sample composition through full analysis chain \label{tab:2shpi0_purity}}
 \begin{tabular}{| l | l | l |l|l|l|l|l|}
 \hline
  & CC 1$\pi^0$ & CC 0$\pi^0$ & NC $\pi^0$ & NC 0$\pi^0$ & Other& Cosmic (MC) & Cosmic (Data) \\ [0.1ex] \hline
No Cuts  & 0.018 &  0.695 & 0.046 & 0.194  & 0.047 & -&-\\
SelectionII & 0.060 & 0.743 & 0.004 & 0.020 & 0.014 & 0.042 & 0.117  \\ 
2 Shower Cuts  & 0.671 & 0.117 & 0.041 & 0.011 & 0.096 & 0.000 & 0.063 \\ \hline
\end{tabular}
\end{table*}

Signal and background distributions are shown for a variety of kinematic variables in Figures \ref{fig:physics_pi0_mu_len} - \ref{fig:physics_pi0signalonly_mass}. Note that the $\pi^0$ mass peak is displaced from the expected 135 MeV; this is well understood to be the result of the multiple ways that we can lose energy during shower reconstruction. First, some energy depositions are not large enough to be detected during hit reconstruction. Thus, energy is missed during this stage. This (in conjunction with containment) has a roughly 10\% effect on biasing the mass peak \cite{bib:davidc_hitthresholding}. Additionally, some amount of charge is lost during the hit removal and clustering stages of the chain.  This has been investigated extensively externally \cite{bib:davidc_missingE} to have a roughly 15\% combined effect. When these effects are combined, we expect to see a peak with a roughly 25\% bias, and to sit around 100 MeV. The combination of these effects can be seen in the shower energy resolution (Figure \ref{fig:showerquality_eres_corr}b), where it is clear that a simple linear energy scale correction is not sufficient to recover most of the lost charge. For that reason we propose to not correct the shower energy. One consequence of this is that we will not be able to produce a $M_{\pi^0}$ distribution. This naturally raises the question: how do we know that the showers we are reconstructing are both photons and originating from a $\pi^0$ decay? The first element of this question pertains to how we know the `showers’ we are reconstructing are truly photons. We can answer this question by considering the conversion distance in Figure \ref{fig:physics_pi0signalonly_mass}b.  Here we see that a fit to the data returns a conversion distance of 24 $\pm$ 12 cm; this is in agreement with the expected value of 25 cm for photons at our energies, suggesting that we are in fact selecting photons. We can confirm we are selecting $\pi^0$'s by considering the reconstructed $\pi^0$ mass peak (Figures \ref{fig:physics_pi0signalonly_mass}a).  As noted earlier, the mass peak is offset due to the multiple sources of energy loss in our current reconstruction chain. Thus, to verify that the two-shower “mass” distribution follows our expectation we consider the result of reconstructing a sample of single-particle $\pi^0$’s with and without cosmics. For an apples to apples comparison, we additionally smear the vertex location of the single particle samples according to the vertex resolution after Selection II.  The agreement between the peak location and shape between the signal and the single particle samples agrees within statistical uncertainty, and improves as we move from the sample without to the sample with cosmics. 
Ten events selected from data are shown in Figure \ref{fig:physics_pi0_onbeam_eventdisplays}.

\begin{figure}[h!]
\centering
  \begin{subfigure}[t]{0.3\textwidth}
    \centering
\includegraphics[scale=0.3]{Pi0_Cut_Section/Physics_pi0_onoffseparate_mult.png}
  \caption{ }
  \end{subfigure} 
  \hspace{10mm}
  \begin{subfigure}[t]{0.3\textwidth}
    \centering
    \includegraphics[scale=0.3]{Pi0_Cut_Section/Physics_pi0_onoffseparate_mu_len.png}
  \caption{ }
  \end{subfigure} 
  \caption{ Data to simulation comparison of a) multiplicity and b) $\mu$ length after $\pi^0$ filter }
\label{fig:physics_pi0_mu_len}
\end{figure}

\begin{figure}[h!]
\centering
  \begin{subfigure}[t]{0.3\textwidth}
    \centering
\includegraphics[scale=0.3]{Pi0_Cut_Section/Physics_pi0_onoffseparate_mu_angle.png}
  \caption{ }
  \end{subfigure} 
  \hspace{10mm}
  \begin{subfigure}[t]{0.3\textwidth}
    \centering
\includegraphics[scale=0.3]{Pi0_Cut_Section/Physics_pi0_onoffseparate_mu_phi.png}
  \caption{ }
  \end{subfigure} 
\caption{ Data to simulation comparison of $\mu$ a) $\theta$ and b) $\phi$ after $\pi^0$ filter }

\label{fig:physics_pi0_mu_phi}
\end{figure}

\begin{figure}[t!]
\centering
  \begin{subfigure}[t]{0.3\textwidth}
    \centering
\includegraphics[scale=0.3]{Pi0_Cut_Section/Physics_pi0_onoffseparate_mu_startx.png}
  \caption{ }
  \end{subfigure} 
  \hspace{30mm}
  \begin{subfigure}[t]{0.3\textwidth}
    \centering
\includegraphics[scale=0.3]{Pi0_Cut_Section/Physics_pi0_onoffseparate_mu_endx.png}
  \caption{ }
  \end{subfigure} 
\caption{ Data to simulation comparison of $\mu$ a) start and b) end x after $\pi^0$ filter }
\label{fig:physics_pi0_mu_x}
\end{figure}

\begin{figure}[t!]
\centering
  \begin{subfigure}[t]{0.3\textwidth}
    \centering
\includegraphics[scale=0.3]{Pi0_Cut_Section/Physics_pi0_onoffseparate_mu_starty.png}
  \caption{ }
  \end{subfigure} 
  \hspace{30mm}
  \begin{subfigure}[t]{0.3\textwidth}
    \centering
\includegraphics[scale=0.3]{Pi0_Cut_Section/Physics_pi0_onoffseparate_mu_endy.png}
  \caption{ }
  \end{subfigure} 
\caption{ Data to simulation comparison of $\mu$ a) start and b) end y after $\pi^0$ filter }
\label{fig:physics_pi0_mu_y}
\end{figure}

\begin{figure}[t!]
\centering
  \begin{subfigure}[t]{0.3\textwidth}
    \centering
\includegraphics[scale=0.3]{Pi0_Cut_Section/Physics_pi0_onoffseparate_mu_startz.png}
  \caption{ }
  \end{subfigure} 
  \hspace{30mm}
  \begin{subfigure}[t]{0.3\textwidth}
    \centering
\includegraphics[scale=0.3]{Pi0_Cut_Section/Physics_pi0_onoffseparate_mu_endz.png}
  \caption{ }
  \end{subfigure} 
\caption{ Data to simulation comparison of $\mu$ a) start and b) end z after $\pi^0$ filter }
\label{fig:physics_pi0_mu_z}
\end{figure}

\begin{figure}[h!]
\centering
  \begin{subfigure}[t]{0.3\textwidth}
    \centering
\includegraphics[scale=0.3]{Pi0_Cut_Section/Physics_pi0_onoffseparate_pi0_mom.png}
  \caption{ }
  \end{subfigure} 
  \hspace{30mm}
  \begin{subfigure}[t]{0.3\textwidth}
    \centering
\includegraphics[scale=0.3]{Pi0_Cut_Section/Physics_pi0_onoffseparate_pi0_oangle.png} 
  \caption{ }
  \end{subfigure} 
\label{fig:physics_pi0_pi0_oangle}
\caption{ Data to simulation comparison of $\pi^0$ a) momentum and b) opening angle after $\pi^0$ filter }
\end{figure}

\begin{figure}[t!]
\centering
  \begin{subfigure}[t]{0.3\textwidth}
    \centering
\includegraphics[scale=0.3]{Pi0_Cut_Section/Physics_pi0_onoffseparate_pi0_low_radL.png}
  \caption{ }
  \end{subfigure} 
  \hspace{30mm}
  \begin{subfigure}[t]{0.3\textwidth}
    \centering
\includegraphics[scale=0.3]{Pi0_Cut_Section/Physics_pi0_onoffseparate_pi0_high_radL.png}
  \caption{ }
  \end{subfigure} 
\label{fig:physics_pi0_pi0_rl}
\caption{ Data to simulation comparison of a) low and b) high energy shower candidate conversion distances  after $\pi^0$ filter }
\end{figure}

\begin{figure}[h!]
\centering
  \begin{subfigure}[t]{0.3\textwidth}
    \centering
\includegraphics[scale=0.3]{Pi0_Cut_Section/Physics_pi0_onoffseparate_pi0_low_IP_w_vtx.png}
  \caption{ }
  \end{subfigure} 
  \hspace{30mm}
  \begin{subfigure}[t]{0.3\textwidth}
    \centering
\includegraphics[scale=0.3]{Pi0_Cut_Section/Physics_pi0_onoffseparate_pi0_high_IP_w_vtx.png}
  \caption{ }
  \end{subfigure} 
\label{fig:physics_pi0_pi0_IP_w_vtx}
\caption{ Data to simulation comparison of a) low and b) high energy shower candidate impact parameters with vertex after $\pi^0$ filter }
\end{figure}

\begin{figure}[h!]
\centering
  \begin{subfigure}[t]{0.3\textwidth}
    \centering
\includegraphics[scale=0.3]{Pi0_Cut_Section/Physics_pi0_onoffseparate_pi0_IP.png}
  \caption{ }
  \end{subfigure} 
  \hspace{30mm}
  \begin{subfigure}[t]{0.3\textwidth}
    \centering
\includegraphics[scale=0.3]{Pi0_Cut_Section/Physics_pi0_onoffseparate_pi0_Easym.png}
  \caption{ }
  \end{subfigure} 
\label{fig:physics_pi0_pi0_IP}
\caption{ Data to simulation comparison of a) impact parameter and b) energy asymmetry of the selected showers. This is defined to be the difference between the higher and lower energy showers' reconstructed energies divided by the sum of these energies. }
\end{figure}

\begin{figure}[h!]
\centering
  \begin{subfigure}[t]{0.3\textwidth}
    \centering
\includegraphics[scale=0.3]{Pi0_Cut_Section/Physics_pi0_onoffseparate_pi0_low_shrE.png}
  \caption{ }
  \end{subfigure} 
  \hspace{30mm}
  \begin{subfigure}[t]{0.3\textwidth}
    \centering
\includegraphics[scale=0.3]{Pi0_Cut_Section/Physics_pi0_onoffseparate_pi0_low_shrE_corr.png}
  \caption{ }
  \end{subfigure} 
\label{fig:physics_pi0_pi0_low_e}
\caption{  Data to simulation comparison of low energy showers from $\pi^0$ selection a) before and b) after energy correction.}
\end{figure}

\begin{figure}[h!]
\centering
  \begin{subfigure}[t]{0.3\textwidth}
    \centering
\includegraphics[scale=0.3]{Pi0_Cut_Section/Physics_pi0_onoffseparate_pi0_high_shrE.png}
  \caption{ }
  \end{subfigure} 
  \hspace{30mm}
  \begin{subfigure}[t]{0.3\textwidth}
    \centering
\includegraphics[scale=0.3]{Pi0_Cut_Section/Physics_pi0_onoffseparate_pi0_high_shrE_corr.png}
  \caption{ }
  \end{subfigure} 
\label{fig:physics_pi0_pi0_e}
\caption{ Data to simulation comparison of high energy showers from $\pi^0$ selection a) before and b) after energy correction. }
\end{figure}


\begin{figure}[h!]
\centering
  \begin{subfigure}[t]{0.3\textwidth}
    \centering
\includegraphics[scale=0.3]{Pi0_Cut_Section/Physics_pi0_onoffseparate_pi0_mass.png}
 \caption{ }
 \end{subfigure} 
 \hspace{30mm}
  \begin{subfigure}[t]{0.3\textwidth}
    \centering
\includegraphics[scale=0.3]{Pi0_Cut_Section/Physics_pi0_onoffseparate_pi0_mass_corr.png}
  \caption{ }
  \end{subfigure} 
\caption{ Data to simulation comparison of a) uncorrected and b) corrected mass peak after $\pi^0$ filter }
%\caption{ Data to simulation comparison of the raw reconstructed two-shower mass. }
\label{fig:physics_pi0_mass} 
\end{figure}

\begin{figure}[h!]
\centering
  \begin{subfigure}[t]{0.3\textwidth}
    \centering
\includegraphics[scale=0.3]{Pi0_Cut_Section/Paper_pi0signalonly_onoffseparate_pi0_mass.png}
  \caption{ }
  \end{subfigure} 
  \hspace{30mm}
  \begin{subfigure}[t]{0.3\textwidth}
    \centering
\includegraphics[scale=0.3]{Pi0_Cut_Section/Paper_pi0signalonly_onoffseparate_pi0_low_radL.png}
  \caption{ }
  \end{subfigure} 
\caption{ Comparison of OnBeam - OffBeam - MC Backgrounds to signal only distribution for a) two shower reconstructed mass peak and b) conversion distance distribution of all showers. }
\label{fig:physics_pi0signalonly_mass} 
\end{figure}


\begin{figure}[h!]
\centering
\includegraphics[scale=0.65]{Pi0_Cut_Section/OnBeam_Data_pi0_0.png}
\includegraphics[scale=0.65]{Pi0_Cut_Section/OnBeam_Data_pi0_1.png}
\caption{ Examples of selected events from 5e19 of Neutrino 2016 BNB Data in the collection plane. The vertical axis in all displays is time, while the horizontal axis is wires. All points on the display represent hits reconstructed, the red line is the 2D projection of the 3D track that is the Selection2 muon candidate, and the triangles are the 2D projections of the 3D reconstructed shower, the cyan circle is the Selection2 identified vertex. }
\label{fig:physics_pi0_onbeam_eventdisplays}
\end{figure}

\clearpage
\subsubsection{A Closer Look at Sample Composition for $\pi^0\rightarrow\geq 2 \gamma$}
Here we take a closer look at the composition of our current sample.  A detailed breakdown of the final selected sample is described in Table \ref{tab:pi0_obnox_breakdown}. The backgrounds described in this table correspond to the data-MC comparisons in this section. We see from this table that our sample is 67\% pure with $\nu_\mu$ CC 1$\pi^0$.  We further examine the quality of this 67\% by considering the origin of each candidate shower in the signal sample. This additional breakdown of the signal CC 1$\pi^0$ is shown in Table \ref{tab:pi0cuts_cc1pi0_obnox_breakdown}.  From here, we see that 93\% of our sample has 2 selected candidate showers that backtrack to true $\nu$-induced $\pi^0$'s, while nearly all of the remaining 7\% have at least 1 candidate showers that tracks to a true $\nu$-induced $\pi^0$.  Plots of signal and background as described in the `Sample Composition Category' in Table \ref{tab:pi0cuts_cc1pi0_obnox_breakdown} for the 2-shower mass peak are shown in in Figure \ref{fig:physics_showerOriginBreakdown_mass}. All references to `$\pi^0$ shower' in these plots and in table \ref{tab:pi0cuts_cc1pi0_obnox_breakdown} refer to $\nu$-induced $\pi^0$ showers.  In Figure \ref{fig:physics_showerOriginBreakdown_mass}a, we can see the majority of the peak is composed of events with 2 $\pi^0$ showers, as expected from Table \ref{tab:pi0cuts_cc1pi0_obnox_breakdown}. We see a similar trend in our set of $\pi^0$ backgrounds ($\nu_\mu$ CC CEx, $\nu_\mu$ Mult $\pi^0$, $\nu_\mu$ NC $\pi^0$) in Figures \ref{fig:physics_showerOriginBreakdown_mass}c-e.  In Figure \ref{fig:physics_showerOriginBreakdown_mass}b, we get a mix of shower origins-- the 2-$\nu$ showers shown in green are caused by $\eta$ decays from our `Final State EM' category, while the 2 $\pi^0$ shower events are due to CC 1$\pi^0$ out of FV, NC Charge exchange, and Kaon decay. Finally, the other category shows a mix.  The 2-$\pi^0$ shower contribution here is due to $\nu_e$ and $\overline{\nu_\mu}$, while the the remaining yellow background is due to tracks mis-reconstructed as showers in CC events.

\begin{table}[H]
\centering
\captionof{table}{Detailed breakdown of sample composition at final $\pi^0$ selection stage \label{tab:pi0_obnox_breakdown}}
 \begin{tabular}{|l|l|l|}
 \hline
Sample Composition Category & Interaction & Sample Composition \\ [0.1ex] \hline
$\nu_\mu$ Signal & $\nu_\mu$ CC 1$\pi^0$ in FV & 0.671 \\ \hline
$\nu_\mu$ CC pion charge exchange & $\nu_\mu$ CC pion charge exchange & 0.039 \\ \hline
$\nu_\mu$ Multiple $\pi^0$ & $\nu_\mu$ Multiple $\pi^0$ & 0.068 \\ \hline
$\nu_\mu$ NC $\pi^0$ & $\nu_\mu$ NC $\pi^0$ & 0.041 \\ \hline
$\nu_\mu$ FSEM & $\nu_\mu$ CC 1$\pi^0$ out of FV & 0.022 \\
& $\nu_\mu$ $N-\gamma$ & 0.028 \\
& $\nu_\mu$ Kaon Decay & 0.005 \\
& $\nu_\mu$ NC pion charge exchange & 0.005 \\ 
&$\nu_\mu$ Brem + $\mu$ capture at rest & 0.046 \\ \hline
Other & $\nu_e$ &0.005 \\
&$\overline{\nu_\mu}$ & 0.002 \\
& Misreconstruction & 0.005 \\ \hline
Cosmic & Cosmic + $\nu$ & 0.063 \\
& Cosmic (Data) & 0.000 \\ \hline
\end{tabular}
\end{table}

\begin{table}[H]
\centering
\captionof{table}{Detailed background breakdown of the CC 1$\pi^0$ sample \label{tab:pi0cuts_cc1pi0_obnox_breakdown}}
 \begin{tabular}{|l|l|l|}
 \hline
Candidate $\pi^0$ `Shower' 1 Description & Candidate $\pi^0$ `Shower' 2 Description & CC 1$\pi^0$ Composition \\ [0.1ex] \hline
$\nu$-Induced $\pi^0$ shower & $\nu$-Induced $\pi^0$ shower & 0.93 \\ 
& $\nu$-Induced non-$\pi^0$ shower & 0.02 \\ 
& $\nu$-Induced track & 0.05 \\ 
$\nu$-Induced track & $\nu$-Induced track & 0.002 \\ \hline
\end{tabular}
\end{table}

\begin{figure}[H]
\centering
  \begin{subfigure}[t]{0.25\textwidth}
    \centering
\includegraphics[scale=0.25]{Pi0_Cut_Section/Physics_showerOriginBreakdown_Signal_pi0_mass.png}
  \caption{ }
  \end{subfigure} 
  \hspace{5mm}
  \begin{subfigure}[t]{0.25\textwidth}
    \centering
\includegraphics[scale=0.25]{Pi0_Cut_Section/Physics_showerOriginBreakdown_FSEM_pi0_mass.png}
  \caption{ }
  \end{subfigure} 
  \hspace{5mm}
  \begin{subfigure}[t]{0.25\textwidth}
    \centering
\includegraphics[scale=0.25]{Pi0_Cut_Section/Physics_showerOriginBreakdown_CCCex_pi0_mass.png}
  \caption{ }
  \end{subfigure} 
  \hspace{5mm}
  \begin{subfigure}[t]{0.25\textwidth}
    \centering
\includegraphics[scale=0.25]{Pi0_Cut_Section/Physics_showerOriginBreakdown_Multpi0_pi0_mass.png}
  \caption{ }
  \end{subfigure} 
  \hspace{5mm}
  \begin{subfigure}[t]{0.25\textwidth}
    \centering
\includegraphics[scale=0.25]{Pi0_Cut_Section/Physics_showerOriginBreakdown_NCpi0_pi0_mass.png}
  \caption{ }
  \end{subfigure} 
  \hspace{5mm}
  \begin{subfigure}[t]{0.25\textwidth}
    \centering
\includegraphics[scale=0.25]{Pi0_Cut_Section/Physics_showerOriginBreakdown_Other_pi0_mass.png}
  \caption{ }
  \end{subfigure} 
\caption{ Breakdown of origin of both showers after 2 shower selection for each sample.  From left to right, and then top to bottom: a) $\nu_{\mu}$ CC 1 $\pi^0$; b) $\nu_\mu$ CC and NC Final State Electromagnetic Activity; c) $\nu_{\mu}$ CC Charge Exchange; d) $\nu_\mu$ CC Multiple $\pi^0$; e) NC $\pi^0$; f) Other.  }

\label{fig:physics_showerOriginBreakdown_mass}
\end{figure}


\clearpage
\subsection{Candidates With At Least One Showers}
In this section, we consider the hypothesis that any neutrino-induced photon originating from the vertex indicates the presence of a $\pi^0$.  This strategy will allow us access to the single-reconstructed shower events, and give us a chance to regain a significant chunk of events. More detail on the one-shower hypothesis exists externally \cite{bib:timb_singleshower}.
\par As mentioned in an earlier section, by requiring at least 2 showers we reduce our sample size by 95\%.  However, this 95\% reduction includes roughly 50\% of the remaining signal after the Selection II filter has completed. This chunk of signals with only one reconstructed shower is largely the result of our shower reconstruction efficiency, shown earlier in Figure \ref{fig:shower_reco_efficiency}. Below 100 MeV of deposited shower energy, the shower reconstruction efficiency drop below 50\%, which results in a number of signal events with only 1 reconstructed shower. When we include these events by only requiring 1 shower be reconstructed in an event, we see the pass rates and sample compositions change.  These values are shown Tables \ref{tab:pi0_1shower_eventrates} and \ref{tab:pi0_1shower_composition}. We see from these tables that, before any additional checks or cuts, our signal efficiency and purity are 26\% and 40\% respectively; this is in contrast to 11\% and 62\% efficiency and purity we saw in the corresponding two-shower Tables \ref{tab:pi0_2showers_eventrates} and \ref{tab:pi0_2showers_composition}. 

\begin{table}[H]
\centering
\captionof{table}{ Pass rate breakdown after 1 shower requirement scaled to OnBeam POT \label{tab:pi0_1shower_eventrates}}
 \begin{tabular}{| l | l | l |l|l|l|l|l|}
 \hline
 & CC 1$\pi^0$ & CC 0$\pi^0$ & NC $\pi^0$ & NC 0$\pi^0$ & Other & Total \\ [0.1ex] \hline
Selection II & 0.331 & 0.106 & 0.008 & 0.010 & 0.030 & 0.087 \\
$\geq$ 1 Shower & 0.257 & 0.006 & 0.005 & 0.001 & 0.015 & 0.010 \\ \hline
\end{tabular}
\end{table}

\begin{table}[H]
\centering
\captionof{table}{ Sample composition after 1 shower requirement scaled to OnBeam POT \label{tab:pi0_1shower_composition}}
 \begin{tabular}{| l | l | l |l|l|l|l|l|}
 \hline
 & CC 1$\pi^0$ & CC 0$\pi^0$ & NC $\pi^0$ & NC 0$\pi^0$ & Other & Cosmic (MC) & Cosmic (Data) \\ [0.1ex] \hline
Selection II & 0.060 & 0.743 & 0.004 & 0.020 & 0.014 & 0.042 & 0.117 \\
$\geq$ 1 Shower & 0.402 & 0.335 & 0.021 & 0.016 & 0.061 & 0.031 & 0.135\\ \hline
\end{tabular}
\end{table}

%\subsection{Signal and Backgrounds for $\pi^0$ \rightarrow $\geq$ 2$\gamma$}
\subsubsection{Tuning Selection for Events with Single Shower}

Similar to how we proceeded to tune our 2-shower cuts, we begin here by breaking events into signal and background samples.  The sample definitions and background breakdowns are the same as described in the 2-shower selection section.

\par We now examine potential variables to use in the selection.  First, we consider the impact parameter of all showers with the vertex.  This is shown in Figure \ref{fig:cutjust_pi0_1shower_IP}a for the cut-tuning backgrounds, and with data-MC comparison in Figure \ref{fig:cutjust_pi0_1shower_IP}b. As noted earlier, we expect small impact parameters, as the vertex is used to reconstruct 3D direction. However, due to exception noted in the two-shower section, we choose to impose an impact parameter cut to improve the angular resolution of our candidate shower sample. To be conservative, we choose here an impact parameter $\leq$ 4cm. Next, we consider the conversion distance of the shower from the reconstructed vertex (Figure \ref{fig:cutjust_pi0_1shower_RL}a and b). Because there is some separation power in this variable, we employ a conversion distance cut of 62 cm in the single shower filter. Finally, if more than one candidate shower passes both previous cut, we select the higher energy shower to be our candidate.  Recall that this is in contrast to the 2-shower path in which multiple-candidate-pair events were removed.

\begin{figure}[t!]
\centering
  \begin{subfigure}[t]{0.35\textwidth}
    \centering
\includegraphics[scale=0.35]{Pi0_Cut_Section/CutJustify_pi0_1shower_eff_gamma_vtx_IP.png}
  \caption{ }
  \end{subfigure} 
  \hspace{30mm}
  \begin{subfigure}[t]{0.35\textwidth}
    \centering
\includegraphics[scale=0.35]{Pi0_Cut_Section/Physics_showerPostSel2_onoffseparate__gamma_vtx_IP.png}
  \caption{ }
  \end{subfigure} 
\caption{ Impact parameter of shower axis with vertex for a) Unscaled MCBNBC + Cosmics; b) Data to simulation comparison }
\label{fig:cutjust_pi0_1shower_IP}
\end{figure}

\begin{figure}[h!]
\centering
  \begin{subfigure}[t]{0.35\textwidth}
    \centering
\includegraphics[scale=0.35]{Pi0_Cut_Section/CutJustify_pi0_1shower_eff_gamma_RL.png}
  \caption{ }
  \end{subfigure} 
  \hspace{30mm}
  \begin{subfigure}[t]{0.35\textwidth}
    \centering
\includegraphics[scale=0.35]{Pi0_Cut_Section/Physics_showerPostSel2_onoffseparate__gamma_RL.png}
  \caption{ }
  \end{subfigure} 
\caption{ Conversion distance of shower candidate from vertex for a) Unscaled MCBNBC + Cosmics; b) Data to simulation comparison }
\label{fig:cutjust_pi0_1shower_RL}
\end{figure}

\subsubsection{Results for $\pi^0\rightarrow \geq 1\gamma$}
\par A table with event rates scaled to OnBeam POT is shown for OnBeam, OffBeam, and MCBNBCosmic in Table \ref{tab:pi0_event_rates}. This can be compared to the two shower selection results summarized in Table \ref{tab:2shpi0_event_rates}.  We additionally note that the MEC contribution to the final sample for single shower cuts is $<$ 1\%; this is similar to what we observed for two shower cuts (Figure \ref{fig:physics_singleshower_inttype}). %This may  suggest a correlation between MC-excess over data and MEC events. 

\begin{table}[H] 
 \centering
 \captionof{table}{Event counts at each selection stage for the Single Shower-CC$\pi^0$ selection chain with MCC8.3 samples scaled to OnBeam POT.  Uncertainties shown are statistical. \label{tab:pi0_event_rates}}
 \begin{tabular}{| l | l | l | l | l |}
  \hline
   & OnBeam & OffBeam & On - OffBeam & MCBNBCos \\ [0.1ex] \hline
No Cuts & 544751 $\pm$ 738 & 462076 $\pm$ 1001 & 82675 $\pm$ 1244 & 48949 $\pm$ 76 \\ 
Selection II & 3753 $\pm$ 61 & 564 $\pm$ 35  & 3189 $\pm$ 71 & 4268 $\pm$ 22  \\ 
Single Shower Cuts & 257 $\pm$ 16 & 15 $\pm$ 6 & 242 $\pm$ 17 & 252 $\pm$ 5  \\ \hline
\end{tabular}
 \end{table}


\begin{figure}[H]
\centering
\includegraphics[scale=0.5]{Pi0_Cut_Section/Misc_singleshower_EventType_vs_NeutrinoMode_w_Numbers.png}
\caption{ Event type broken down by neutrino interaction mode for Single Shower cuts; note the contribution of MEC events to the final selected sample. }
\label{fig:physics_singleshower_inttype}
\end{figure}


\par A summary of the passing rates for signal and all backgrounds is shown in Table \ref{tab:pi0_passrates}.  Note that we have maintained a relatively high efficiency for the signal with respect to other listed backgrounds. Sample composition is shown in Table \ref{tab:pi0_purity}. 

\begin{table}[H]
\centering
\captionof{table}{Evolution of passing rates through full analysis chain \label{tab:pi0_passrates}}
 \begin{tabular}{| l | l | l |l|l|l|l|}
 \hline
 & CC 1$\pi^0$ & CC 0$\pi^0$ & NC $\pi^0$ & NC 0$\pi^0$ & Other & All \\ [0.1ex] \hline
No Cuts & - & - & - & - & - & -\\
SelectionII & 0.331 & 0.105 & 0.008 & 0.010 & 0.030 & 0.087 \\ 
Single Shower Cuts & 0.170 & 0.002 & 0.003 & 0.0003 & 0.009 & 0.005 \\ \hline
\end{tabular}
\end{table}

\begin{table}[H]
\centering
\captionof{table}{Evolution of sample composition through full analysis chain \label{tab:pi0_purity}}
 \begin{tabular}{| l | l | l |l|l|l|l|l|}
 \hline
  & CC 1$\pi^0$ & CC 0$\pi^0$ & NC $\pi^0$ & NC 0$\pi^0$ & Other& Cosmic (MC) & Cosmic (Data) \\ [0.1ex] \hline
No Cuts  & 0.018 &  0.695 & 0.046 & 0.194  & 0.047 & -&-\\
SelectionII & 0.060 & 0.743 & 0.004 & 0.020 & 0.014 & 0.042 & 0.117  \\ 
Single Shower Cuts & 0.561 & 0.205 & 0.025 & 0.012 & 0.080 & 0.060 & 0.057 \\ \hline
\end{tabular}
\end{table}

Signal and background distributions are shown more explicitly for a variety of kinematic variables in Figure \ref{fig:physics_singleshower_mulen} - Figure \ref{fig:physics_singleshower_ip}.

\begin{figure}[H]
\centering
  \begin{subfigure}[t]{0.3\textwidth}
    \centering
\includegraphics[scale=0.3]{Pi0_Cut_Section/Physics_singleshower_onoffseparate_mult.png}
  \caption{ }
  \end{subfigure} 
  \hspace{20mm}
  \begin{subfigure}[t]{0.3\textwidth}
    \centering
\includegraphics[scale=0.3]{Pi0_Cut_Section/Physics_singleshower_onoffseparate_mu_len.png}
  \caption{ }
  \end{subfigure} 

\caption{ Data to simulation comparison of a) multiplicity and b) $\mu$ length after single shower filter }
\label{fig:physics_singleshower_mulen}
\end{figure}

\begin{figure}[H]
\centering
  \begin{subfigure}[t]{0.3\textwidth}
    \centering
\includegraphics[scale=0.3]{Pi0_Cut_Section/Physics_singleshower_onoffseparate_mu_angle.png}
  \caption{ }
  \end{subfigure} 
  \hspace{20mm}
  \begin{subfigure}[t]{0.3\textwidth}
    \centering
\includegraphics[scale=0.3]{Pi0_Cut_Section/Physics_singleshower_onoffseparate_mu_phi.png}
  \caption{ }
  \end{subfigure} 
\caption{ Data to simulation comparison of $\mu$ a) $\theta$ and b) $\phi$ length after single shower filter }
\label{fig:physics_singleshower_muphi}
\end{figure}

\begin{figure}[H]
\centering
  \begin{subfigure}[t]{0.3\textwidth}
    \centering
\includegraphics[scale=0.3]{Pi0_Cut_Section/Physics_singleshower_onoffseparate_mu_startx.png}
  \caption{ }
  \end{subfigure} 
  \hspace{30mm}
  \begin{subfigure}[t]{0.3\textwidth}
    \centering
\includegraphics[scale=0.3]{Pi0_Cut_Section/Physics_singleshower_onoffseparate_mu_endx.png}
  \caption{ }
  \end{subfigure} 
\caption{ Data to simulation comparison of $\mu$ a) start and b) end in x after single shower filter }
\label{fig:physics_singleshower_x}
\end{figure}

\begin{figure}[H]
\centering
  \begin{subfigure}[t]{0.3\textwidth}
    \centering
\includegraphics[scale=0.3]{Pi0_Cut_Section/Physics_singleshower_onoffseparate_mu_starty.png}
  \caption{ }
  \end{subfigure} 
  \hspace{30mm}
  \begin{subfigure}[t]{0.3\textwidth}
    \centering
\includegraphics[scale=0.3]{Pi0_Cut_Section/Physics_singleshower_onoffseparate_mu_endy.png}
  \caption{ }
  \end{subfigure} 
\caption{ Data to simulation comparison of $\mu$ a) start and b) end in y after single shower filter }
\label{fig:physics_singleshower_y}
\end{figure}

\begin{figure}[H]
\centering
  \begin{subfigure}[t]{0.3\textwidth}
    \centering
\includegraphics[scale=0.3]{Pi0_Cut_Section/Physics_singleshower_onoffseparate_mu_startz.png}
  \caption{ }
  \end{subfigure} 
  \hspace{30mm}
  \begin{subfigure}[t]{0.3\textwidth}
    \centering
\includegraphics[scale=0.3]{Pi0_Cut_Section/Physics_singleshower_onoffseparate_mu_endz.png}
  \caption{ }
  \end{subfigure} 
\caption{ Data to simulation comparison of $\mu$ a) start and b) end in z after single shower filter }
\label{fig:physics_singleshower_z}
\end{figure}

\begin{figure}[H]
\centering
  \begin{subfigure}[t]{0.3\textwidth}
    \centering
\includegraphics[scale=0.3]{Pi0_Cut_Section/Physics_singleshower_onoffseparate_gamma_E.png}
  \caption{ }
  \end{subfigure} 
  \hspace{30mm}
  \begin{subfigure}[t]{0.3\textwidth}
    \centering
\includegraphics[scale=0.3]{Pi0_Cut_Section/Physics_singleshower_onoffseparate_gamma_E_corr.png}
  \caption{ }
  \end{subfigure} 
\caption{ Data to simulation comparison of tagged shower a) energy and b) corrected energy }
\label{fig:physics_singleshower_e}
\end{figure}

\begin{figure}[H]
\centering
  \begin{subfigure}[t]{0.3\textwidth}
    \centering
\includegraphics[scale=0.3]{Pi0_Cut_Section/Physics_singleshower_onoffseparate_gamma_IP_w_vtx.png}
  \caption{ }
  \end{subfigure} 
  \hspace{20mm}
  \begin{subfigure}[t]{0.3\textwidth}
    \centering
\includegraphics[scale=0.3]{Pi0_Cut_Section/Physics_singleshower_onoffseparate_gamma_RL.png}
  \caption{ }
  \end{subfigure} 
\caption{ Data to simulation comparison of tagged shower a) impact parameter with the vertex and b) conversion distance }
\label{fig:physics_singleshower_ip}
\end{figure}

\subsubsection{A Closer Look at Sample Composition for $\pi^0\rightarrow\geq 1 \gamma$} 
We again consider a detailed breakdown of the final selected sample of the single shower sample.  This breakdown is described in Table \ref{tab:singleshower_obnox_breakdown} and corresponds to the data-MC comparisons in this section. We see from this table that our sample is 56\% pure with our $\nu_\mu$ CC 1$\pi^0$ signal; this contrasts to the 67\% purity we observed previously along the 2-shower selection path.  We further examine the quality of this 56\% by considering the origin of each candidate shower in the signal sample in Table \ref{tab:singleshower_obnox_ccoth_breakdown}.  From row 3 in this table, we see that 97\% of our sample has a selected candidate shower that backtracks to a true $\nu$-induced $\pi^0$, with the majority of remaining tags due to tracks mis-reconstructed as showers. This indicates that we have selected a set of signal events with high quality, automated shower reconstruction.  Plots of signal and background as described in the `Sample Composition Category' in Table \ref{tab:singleshower_obnox_ccoth_breakdown} are shown in Figure \ref{fig:physics_showerOriginBreakdown_E}a-f. Examples event displays of reconstructed showers that backtrack to $\nu$-induced showers and $\nu$-induced tracks are shown in Figure \ref{fig:ccoth_bkgd_nu_cos}.  We additionally include an example of a `noise'-originating shower in Figure \ref{fig:ccoth_bkgd_noise}, and finally a $\nu$-induced $\pi^0$ shower in Figure \ref{fig:cc1pi0_pi0}.
\clearpage
\begin{figure}[H]
\centering
  \begin{subfigure}[H]{0.25\textwidth}
    \centering
\includegraphics[scale=0.25]{Pi0_Cut_Section/Physics_showerOriginBreakdown_1gamma_Signal_gamma_E.png}
  \caption{ }
  \end{subfigure} 
  \hspace{4mm}
  \begin{subfigure}[H]{0.25\textwidth}
    \centering
\includegraphics[scale=0.25]{Pi0_Cut_Section/Physics_showerOriginBreakdown_1gamma_FSEM_gamma_E.png}
  \caption{ }
  \end{subfigure} 
  \hspace{4mm}
  \begin{subfigure}[H]{0.25\textwidth}
    \centering
\includegraphics[scale=0.25]{Pi0_Cut_Section/Physics_showerOriginBreakdown_1gamma_CCCex_gamma_E.png}
  \caption{ }
  \end{subfigure} 
  \hspace{4mm}
  \begin{subfigure}[H]{0.25\textwidth}
    \centering
\includegraphics[scale=0.25]{Pi0_Cut_Section/Physics_showerOriginBreakdown_1gamma_Multpi0_gamma_E.png}
  \caption{ }
  \end{subfigure} 
  \hspace{4mm}
  \begin{subfigure}[H]{0.25\textwidth}
    \centering
\includegraphics[scale=0.25]{Pi0_Cut_Section/Physics_showerOriginBreakdown_1gamma_NCpi0_gamma_E.png}
  \caption{ }
  \end{subfigure} 
  \hspace{4mm}
  \begin{subfigure}[H]{0.25\textwidth}
    \centering
\includegraphics[scale=0.25]{Pi0_Cut_Section/Physics_showerOriginBreakdown_1gamma_Other_gamma_E.png}
  \caption{ }
  \end{subfigure} 
\caption{ Breakdown of origin of both showers after 1 shower selection for each sample shown for uncorrected shower energy.  From left to right, and then top to bottom: a) $\nu_{\mu}$ CC 1 $\pi^0$; b) $\nu_\mu$ CC and NC Final State Electromagnetic Activity; c) $\nu_{\mu}$ CC Charge Exchange; d) $\nu_\mu$ CC Multiple $\pi^0$; e) NC $\pi^0$; f) Other.  }
\label{fig:physics_showerOriginBreakdown_E}
\end{figure}

\begin{table}[H]
\centering
\captionof{table}{Detailed breakdown of sample composition at final one shower selection stage \label{tab:singleshower_obnox_breakdown}}
 \begin{tabular}{|l|l|l|}
 \hline
Sample Composition Category & Interaction & Sample Composition \\ [0.1ex] \hline
$\nu_\mu$ Signal & $\nu_\mu$ CC 1$\pi^0$ in FV & 0.561 \\ \hline
$\nu_\mu$ CC pion charge exchange & $\nu_\mu$ CC pion charge exchange & 0.036 \\ \hline
$\nu_\mu$ Multiple $\pi^0$ & $\nu_\mu$ Multiple $\pi^0$ & 0.047 \\ \hline
$\nu_\mu$ NC $\pi^0$ & $\nu_\mu$ NC $\pi^0$ & 0.025 \\ \hline
$\nu_\mu$ FSEM & $\nu_\mu$ CC 1$\pi^0$ out of FV & 0.022 \\
& $\nu_\mu$ $N-\gamma$ & 0.022 \\
& $\nu_\mu$ Kaon Decay & 0.004 \\
& $\nu_\mu$ NC pion charge exchange & 0.005 \\ 
&$\nu_\mu$ Brem + $\mu$ capture at rest & 0.075 \\ \hline
Other & $\nu_e$ &0.007 \\
&$\overline{\nu_\mu}$ & 0.003 \\
& Misreconstruction & 0.075 \\ \hline
Cosmic & Cosmic + $\nu$ & 0.060 \\
& Cosmic (Data) & 0.057 \\ \hline
\end{tabular}
\end{table}

\begin{table}[H]
\centering
\captionof{table}{Detailed background breakdown of the CC 1$\pi^0$ sample from the single shower sample \label{tab:singleshower_obnox_ccoth_breakdown}}
 \begin{tabular}{|l|l|l|}
 \hline
Candidate $\pi^0$ `Shower' Description & CC 1$\pi^0$ \\ [0.1ex] \hline
$\nu$-Induced Track & 0.023 \\ 
$\nu$-Induced Shower (non-$\pi^0$) & 0.007 \\ 
$\nu$-Induced Shower ($\pi^0$) & 0.970 \\ 
Noise & 0.031  \\ \hline
\end{tabular}
\end{table}

\begin{figure}[h!]
\centering
\includegraphics[scale=0.8]{Pi0_Cut_Section/CCOther_nu_cos.png}
\hspace{1 mm}
\caption{ Example of $\nu$ and cosmic-induced CCOther backgrounds. The cyan circle is the reconstructed vertex and the green triangle is the candidate $\pi^0$ shower.  The candidate muon track is also shown in solid red in the $\nu$-induced track example to differentiate from the $\pi^0$ candidate `shower'.  }
\label{fig:ccoth_bkgd_nu_cos}
\end{figure}

\begin{figure}[h!]
\centering
\includegraphics[scale=0.9]{Pi0_Cut_Section/CCOther_noise.png}
\caption{ Example of a `Noise' CCOther background. MCClusters are energy depositions built into clusters based on the deposition’s true origin particle. Each mccluster color represents a new particle (though the event display color wheel is finite, so there is some cycling of colors).  Gaushits are reconstructed from waveforms into individual points (or `hits') with time and wire coordinates, and an associated charge. MCClusters are shown on the left, while reconstructed gaushits and the candidate reconstructed shower is shown on the right.
 Note that the candidate `shower' has no corresponding mccluster in the left panel. The cyan circle is the reconstructed vertex and the green triangle is the candidate $\pi^0$ shower.  }
\label{fig:ccoth_bkgd_noise}
\end{figure}

\begin{figure}[h!]
\centering
\includegraphics[scale=0.8]{Pi0_Cut_Section/CC1pi0_nu_cos.png}
\caption{ Example of $\nu$ and cosmic-induced CC 1-$\pi^0$ backgrounds. The cyan circle is the reconstructed vertex, the green triangle is the candidate $\pi^0$ shower, and the solid red line is the $\mu$ candidate. Track-like hits are shown in black, and shower-like hits are shown in faint red. }
\label{fig:cc1pi0_nu_cos}
\end{figure}

\begin{figure}[h!]
\centering
\includegraphics[scale=0.8]{Pi0_Cut_Section/CC1pi0_pi0.png}
\caption{ Example of a true CC 1-$\pi^0$ event. The cyan circle is the reconstructed vertex, the green triangle is the candidate $\pi^0$ shower, and the solid red line is the $\mu$ candidate. Track-like hits are shown in black, and shower-like hits are shown in faint red. }
\label{fig:cc1pi0_pi0}
\end{figure}



\clearpage
\section{Cross Section Analysis}
\par At this point we are in a position to calculate the cross section of CC 1-$\pi^0$ on argon.  Before we do so, it is important to understand whether we have introduced kinematical bias into the analysis at any stage.  In the following sections we will examine the analysis efficiencies across all stages and a variety of variables, as well as gathering all the pieces we need and calculating the final cross section.

\subsection{Truth and Efficiency Distributions}
This section considers the distributions and selection efficiencies across a variety of kinematic variables across the selection process.  The events considered in these plots are CC events with a single neutrino-induced $\pi^0$ originating from a true vertex contained in the FV. The efficiency is defined here to be the number of this kind of events at some stage (Selection II or $\pi^0$ cuts) divided by the number of this kind of event before any cuts have been applied.  Figures \ref{fig:pi0_effs_0} - \ref{fig:pi0_effs_6} show various kinematic distributions and their corresponding efficiencies at each stage of our CC$\pi^0$ selection up through the 2-shower set of cuts.  We show a similar set of plots for the single shower set of cuts.  These results are shown in Figures \ref{fig:pi0_effs_7} - \ref{fig:pi0_effs_13}. Given the flatness of our efficiencies we conclude that we will not be kinematically biasing an absolute cross section measurement. Only statistical uncertainties are considered here. 

\begin{figure}[h!]
\centering

 \begin{subfigure}[t]{0.39\textwidth}
    \centering
    \includegraphics[scale=0.39]{XSection_Calc_Section/MCVar_pi0_true_nu_e.png}
  \caption{ }
  \end{subfigure} 
  \hspace{20mm}
  \begin{subfigure}[t]{0.39\textwidth}
    \centering
\includegraphics[scale=0.39]{XSection_Calc_Section/MCVar_eff_pi0_true_nu_e.png}
  \caption{ }
  \end{subfigure} 
\caption{a) $E_\nu$ distribution before and after selection; b) Efficiency as a function of $E_\nu$. }
\label{fig:pi0_effs_0}
\end{figure}
%; note that the efficiency is negligible below 0.275 GeV

\begin{figure}[t!]
\centering
  \begin{subfigure}[t]{0.39\textwidth}
    \centering
\includegraphics[scale=0.39]{XSection_Calc_Section/MCVar_pi0_true_pi0_mom.png}
  \caption{ }
  \end{subfigure} 
  \hspace{20mm}
  \begin{subfigure}[t]{0.39\textwidth}
    \centering
\includegraphics[scale=0.39]{XSection_Calc_Section/MCVar_eff_pi0_true_pi0_mom.png}
  \caption{ }
  \end{subfigure} 
\caption{a) $\pi^0$ momentum distribution across all stages of CC $\pi^0$ selection; b) Efficiency as a function of momentum. }
\label{fig:pi0_effs_1}
\end{figure}

\begin{figure}[h!]
\centering
\includegraphics[scale=0.39]{XSection_Calc_Section/MCVar_pi0_true_gamma_e_min.png}
\hspace{1 mm}
\includegraphics[scale=0.39]{XSection_Calc_Section/MCVar_eff_pi0_true_gamma_e_min.png}
\caption{a) Lower energy shower distribution across all stages of CC-$\pi^0$ selection; b) Efficiency as a function of energy. }
\label{fig:pi0_effs_2}
\end{figure}

\begin{figure}[h!]
\centering
\includegraphics[scale=0.39]{XSection_Calc_Section/MCVar_pi0_true_gamma_e_max.png}
\hspace{1 mm}
\includegraphics[scale=0.39]{XSection_Calc_Section/MCVar_eff_pi0_true_gamma_e_max.png}
\caption{a) Higher energy shower distribution across all stages of CC-$\pi^0$ selection; b) Efficiency as a function of energy. }
\label{fig:pi0_effs_3}
\end{figure}

\begin{figure}[h!]
\centering
\includegraphics[scale=0.39]{XSection_Calc_Section/MCVar_pi0_true_RL_minE.png}
\hspace{1 mm}
\includegraphics[scale=0.39]{XSection_Calc_Section/MCVar_eff_pi0_true_RL_minE.png}
\caption{a) Lower energy shower conversion distances across all stages of CC-$\pi^0$ selection; b) Efficiency as a function of conversion distance. }
\label{fig:pi0_effs_4}
\end{figure}

\begin{figure}[h!]
\centering
\includegraphics[scale=0.39]{XSection_Calc_Section/MCVar_pi0_true_RL_maxE.png}
\hspace{1 mm}
\includegraphics[scale=0.39]{XSection_Calc_Section/MCVar_eff_pi0_true_RL_maxE.png}
\caption{a) Higher energy shower conversion distances across all stages of CC-$\pi^0$ selection; b) Efficiency as a function of energy. }
\label{fig:pi0_effs_5}
\end{figure}


\begin{figure}[h!]
\includegraphics[scale=0.39]{XSection_Calc_Section/MCVar_pi0_true_angle.png}
\hspace{3 mm}
\includegraphics[scale=0.39]{XSection_Calc_Section/MCVar_eff_pi0_true_angle.png}
\caption{a) $\pi^0$ opening angle distribution across all stages of CC $\pi^0$ selection; b) Efficiency as a function of opening angle. }
\label{fig:pi0_effs_6}
\end{figure}



\begin{figure}[h!]
\centering
\includegraphics[scale=0.39]{XSection_Calc_Section/MCVar_singleshower_true_nu_e.png}
\hspace{1 mm}
\includegraphics[scale=0.39]{XSection_Calc_Section/MCVar_eff_singleshower_true_nu_e.png}
\caption{a) $E_\nu$ distribution before and after selection with single shower selection at final stage; b) Efficiency as a function of $E_\nu$. }
\label{fig:pi0_effs_7}
\end{figure}
%; note that the efficiency is negligible below 0.275 GeV

\begin{figure}[h!]
\centering
\includegraphics[scale=0.39]{XSection_Calc_Section/MCVar_singleshower_true_pi0_mom.png}
\hspace{1 mm}
\includegraphics[scale=0.39]{XSection_Calc_Section/MCVar_eff_singleshower_true_pi0_mom.png}
\caption{a) $\pi^0$ momentum distribution across all stages of CC $\pi^0$ selection with single shower selection at final stage; b) Efficiency as a function of momentum. }
\label{fig:pi0_effs_8}
\end{figure}

\begin{figure}[h!]
\centering
\includegraphics[scale=0.39]{XSection_Calc_Section/MCVar_singleshower_true_gamma_e_min.png}
\hspace{1 mm}
\includegraphics[scale=0.39]{XSection_Calc_Section/MCVar_eff_singleshower_true_gamma_e_min.png}
\caption{a) Lower energy shower distribution across all stages of CC-$\pi^0$ selection with single shower selection at final stage; b) Efficiency as a function of energy. }
\label{fig:pi0_effs_9}
\end{figure}

\begin{figure}[h!]
\centering
\includegraphics[scale=0.39]{XSection_Calc_Section/MCVar_singleshower_true_gamma_e_max.png}
\hspace{1 mm}
\includegraphics[scale=0.39]{XSection_Calc_Section/MCVar_eff_singleshower_true_gamma_e_max.png}
\caption{a) Higher energy shower distribution across all stages of CC-$\pi^0$ selection with single shower selection at final stage; b) Efficiency as a function of energy. }
\label{fig:pi0_effs_10}
\end{figure}

\begin{figure}[h!]
\centering
\includegraphics[scale=0.39]{XSection_Calc_Section/MCVar_singleshower_true_RL_minE.png}
\hspace{1 mm}
\includegraphics[scale=0.39]{XSection_Calc_Section/MCVar_eff_singleshower_true_RL_minE.png}
\caption{a) Lower energy shower conversion distances across all stages of CC-$\pi^0$ selection with single shower selection at final stage; b) Efficiency as a function of conversion distance. }
\label{fig:pi0_effs_11}
\end{figure}

\begin{figure}[h!]
\centering
\includegraphics[scale=0.39]{XSection_Calc_Section/MCVar_singleshower_true_RL_maxE.png}
\hspace{1 mm}
\includegraphics[scale=0.39]{XSection_Calc_Section/MCVar_eff_singleshower_true_RL_maxE.png}
\caption{a) Higher energy shower conversion distances across all stages of CC-$\pi^0$ selection with single shower selection at final stage; b) Efficiency as a function of energy. }
\label{fig:pi0_effs_12}
\end{figure}


\begin{figure}[h!]
\includegraphics[scale=0.39]{XSection_Calc_Section/MCVar_singleshower_true_angle.png}
\hspace{3 mm}
\includegraphics[scale=0.39]{XSection_Calc_Section/MCVar_eff_singleshower_true_angle.png}
\caption{a) $\pi^0$ opening angle distribution across all stages of CC $\pi^0$ selection with single shower selection at final stage; b) Efficiency as a function of opening angle. }
\label{fig:pi0_effs_13}
\end{figure}

\clearpage
\subsection{Cross Section}

Our first step is to calculate the GENIE flux-averaged cross section on argon for our simulation. The cross section can be calculated according to the following equation:

\begin{equation}
  \sigma = \frac{N_{tagged} - N_{bkgd}}{\epsilon*N_{targ}*\phi}
\end{equation}

\noindent where $N_{tagged}$, $N_{bkgd}$ are the number of tagged events and background events respectively, $\epsilon$ is the efficiency, $N_{targ}$ the number of Argon nuclei targets and $\phi$ the flux. 
\par We use the previously described 420k MC BNB + Cosmics sample to perform this initial calculation.  Because we are using MC information to calculate a true value here, we use $\epsilon$ = 1, $N_{bkgd}$=0 and $N_{tagged}$=$N_{signal}$.  To calculate $N_{signal}$, we choose our volume of interest to be the Fiducial Volume (FV) used by Selection II, with 20cm from the wall in X and Y, and 10cm from the wall in Z. We find $N_{tagged}$ = 7567, for signal interaction vertices inside the FV. Note that while 420k events are simulated, more than half of these interactions occur outside the FV. 
\par Our next job is to calculate the number of targets in our FV:

\begin{equation} \label{eq:1}
  N_{targ} = \frac{\rho_{Ar} * V * Avogadro}{m_{mol}} 
\end{equation}
\noindent where $\rho_{Ar}$ is the density of Liquid argon, V is the volume of interest, and $m_{mol}$ is the number of grams per mole of argon.  Using the NIST database at our temperature and pressure for the density of liquid Argon, and the FV as our volume of interest, we find: 

\begin{align}
N_{targ} &= \frac{1.3836 [\frac{g}{cm^3}] * 4.246e7 [cm^3] * 6.022e23 [\frac{molec}{mol}]}{39.95 [\frac{g}{mol}]} \\\\
&= 8.855\times10^{29}~\text{molecular targets}
\end{align}

\par Our final step is to calculate the integrated flux.  We do this by integrating over the $\nu_\mu$ flux histogram (Figure \ref{fig:flux}a) provided by the Beam Working Group \cite{bib:flux}, and normalizing by the POT in our sample. The POT is calculated by integrating over the POT of all subruns under consideration; in this case, our MC POT is 4.232e20.  We calculate a total integrated flux of 3.02e11 $cm^{-2}$ over the full energy range of the flux histograms with $<E>$ = 824 MeV. 
\noindent Putting it all together we find:


\begin{align}
\sigma^{\text{MC}}_{CC\pi^0} &= \frac{7567}{3.02e11 \frac{1}{cm^2} * 8.855e29 Ar } \\\\
&= (2.83 \pm 0.03) *10^{-38} \frac{cm^2}{Ar}
\end{align}

\begin{figure}[h!]
\centering
\includegraphics[scale=0.4]{XSection_Calc_Section/Misc_numu_MC_flux.png}
\includegraphics[scale=0.4]{XSection_Calc_Section/Misc_numu_flux.png}

\caption{$\nu_\mu$ Flux from Booster Neutrino Beam (BNB) at 470m scaled to a) MCBNB+Cosmics sample 4.23e20 POT; b) OnBeam 5e19 POT}
\label{fig:flux}
\end{figure}

\noindent where the uncertainty presented is purely statistical and dependent only on the number of signal events.  

\par We perform a similar integration over the $\nu_\mu$ flux histogram (Figure \ref{fig:flux}b) in order to calculate the flux we'll need to measure a cross section on data. This integrated flux is 3.51e10 $cm^{-2}$ over the full energy range of the flux histograms with $<E>$ = 824 MeV. 

\begin{table*}
\centering
\captionof{table}{Summary of all pieces that feed into the cross section calculation.  The corresponding Table number that each piece of information can be found in is included in parenthesis next to the entry.  Note that the parameters reported here are not rounded as they are in earlier Tables, in order to give the reader all info needed to reproduce the calculations.  \label{tab:summary_of_xsec_params}}
 \begin{tabular}{|l|l|l|l|l|l|l|}
 \hline
 & $N_{OnBeam}$ & $N_{OffBeam}$ & $N_{MCBkgd}$ & $\epsilon~[\%]$ & $N_{targ}$ & $\phi ~[cm^{-2}]$  \\ [0.1ex] \hline
2 Shower & 69 $\pm$ 9 (\ref{tab:2shpi0_event_rates}) & 0.00 $\pm$ 2.17 (\ref{tab:2shpi0_event_rates}) & 24.18 $\pm$ 1.68 (\ref{tab:2shpi0_event_rates}, \ref{tab:pi0_2showers_composition}) & 5.6 $\pm$ 0.3 (\ref{tab:pi0_2showers_eventrates}) & $8.855\times10^{29}$ & $3.51\times10^{10}$\\ \hline
1 Shower & 257 $\pm$ 16(\ref{tab:pi0_event_rates}) & 15.18 $\pm$ 5.74 (\ref{tab:pi0_event_rates}) & 102.07 $\pm$ 3.44(\ref{tab:pi0_event_rates}, \ref{tab:pi0_1shower_composition} ) & 17.0 $\pm$ 0.5 (\ref{tab:pi0_1shower_eventrates}) & $8.855\times10^{29}$ & $3.51\times10^{10}$\\ \hline

\end{tabular}
\end{table*}

\par At this point, we summarize the information we'll need to calculate the cross section from data via both selection paths in Table \ref{tab:summary_of_xsec_params}. We can calculate the cross section using the 2-shower results described in the previous section where $N_{tag}$ is the number of OnBeam events at the final stage, and $N_{bkgd}$ is the number of scaled OffBeam and MC backgrounds :
\begin{align}
\sigma^{\text{Data}}_{CC\pi^0_{2\gamma}} &= \frac{69_{OnBeam} - 0_{OffBeam} - 24.18_{MCBkgd}}{0.056 * 3.51e10 \frac{1}{cm^2} * 8.855e29 Ar} \\\\
&= (2.56 \pm 0.50) *10^{-38} \frac{cm^2}{Ar}
\end{align}

This result is within statistical uncertainties of the true calculated cross section above. 

\par Finally, we can calculate the cross section using 1 shower-selection results on data:
\begin{align}
\sigma^{\text{Data}}_{CC\pi^0_{1\gamma}} &= \frac{257_{OnBeam} - 15.18_{OffBeam} - 102.07_{MCBkgd}}{0.17 * 3.51e10 \frac{1}{cm^2} * 8.855e29 Ar} \\\\
&= (2.64 \pm 0.33) *10^{-38} \frac{cm^2}{Ar}
\end{align}

This result is also within statistical uncertainties of the true calculated cross section above. These results are plotted in Figure \ref{fig:genie_uboone_xsec} on the GENIE extracted cross section plot along with MiniBooNE and SciBooNE measured cross sections for comparison.  Only statistical uncertainties are shown for now.

\begin{figure}[h!]
\centering
\includegraphics[width=1\textwidth]{FinalCrossSectionPlots/Final_stat.png}
%\includegraphics[scale=0.35]{XSection_Calc_Section/GenieTruth_stat_sys.png}
\caption{ $\nu_{\mu}+\text{Ar}$ charged current single pion production cross section extracted from GENIE with the MicroBooNE measured cross section using the two- and one-shower paths shown with statistical uncertainties only. }
\label{fig:genie_uboone_xsec}
\end{figure}

\clearpage
\section{Systematic Uncertainties}
The precision and sensitivity of an experimental measurement depends exactly on how well we understand the contributing models and detector limitations. In the case of MicroBooNE, the nominal cross section uncertainties in the GENIE neutrino generator, modeling of the beam flux, and detector systematics all affect the final measured cross section. The total uncertainty will then ideally be the combination of independent matrices corresponding to each systematic source, as shown below:
% cuts-based variation
\begin{equation}
\label{eq:sys_error}
M^{syst} =  M^{genie} + M^{flux} + M^{detector}
\end{equation}

\noindent In this section, we explore the degree to which each of these sources contributes to the final uncertainty. 
\subsection{Uncertainty Propagation - Context} 
There is a simple prescription to follow when approaching systematic uncertainty evaluation. We must first identify contributing parameters and their corresponding degrees of uncertainty.  Then, we vary these parameters randomly across many iterations, each time recalculating the cross section.
\par To assess these variations as a 1$\sigma$ uncertainty we can approach this in two ways. If a variation in parameter assignment affects only the rates of event production, a reweighing scheme can be applied to the final distributions rather than re-doing the full simulation for each generated universe.  This re-weighting strategy can be applied to uncertainty sources such as beam flux and GENIE cross sections.  For example, if the CC-neutrino interaction rate is halved by a parameter adjustment to the underlying neutrino interaction models, we can simply apply a factor of one-half to our final calculations.  On the other hand, if parameter adjustments affect event topology, re-weighting will not be sufficient, and a full generation of detector MC must be performed. One example of such a parameter is space charge. Space charge refers to the presence of positively charged ions that are formed when the argon is ionized.  These ions influence the recombination of ionization electrons from new interactions, and can cause distortions in the readout. To handle space charge and other detector effects, we must generate samples for each effect and study the impact on our selection. A framework exists to do these variations, and studies here are on going. For now, we limit our systematic evaluations to the flux and GENIE cross sections.

\subsection{GENIE Cross Sections}
The impact of each GENIE parameter variation can be tested by reweighing event distributions using the built-in GENIE event reweighting framework. Here, each physical parameter P is varied by $\pm$1$\sigma$. The output of this variation is a set of event weights per parameter that represent the output had you run with the varied parameter from the beginning. A summary of parameters utilized for this study is summarized in Table \ref{tab:genie_parameters}. We note the absence of MEC parameter variation in the reweighting framework used here, though a method to assess the uncertainty has been introduced externally \cite{bib:jaz_datamc_agreement}.  This uncertainty will contribute negligibly due to the small contribution of MEC events at the final stage of both selection paths.

\begin{table*}
\centering
\captionof{table}{Table of GENIE parameters reproduced for convenience from the GENIE manual and CC Inclusive Internal Note \cite{bib:genie} \cite{bib:numucc} \label{tab:genie_parameters}}
\begin{tabular}{| l | l | l | l |}
\hline
   Parameter P & Description of P & Central Value  & $\delta$P/P \\ [0.1ex] \hline
 $M_A^{NCEL}$  & Axial mass for NC elastic & 0.990 GeV & $\pm$25\% \\
 $\eta^{NCEL}$  & Strange axial form factor $\eta$ for NC elastic & 0.120 GeV & $\pm$30\% \\
$M_A^{CCQE}$  & Axial mass for CC quasi-elastic & 0.990 GeV & -15\% +25\% \\
$M_A^{CCRES}$  & Axial mass for CC res $\nu$ production & 1.120 GeV & $\pm$20\% \\
$M_V^{CCRES}$  & Vector mass for CC res $\nu$ production & 0.840 GeV & $\pm$10\% \\
$M_A^{NCRES}$  & Axial mass for NC res $\nu$ production & 1.120 GeV & $\pm$20\% \\
$M_V^{NCRES}$  & Vector mass for NC res $\nu$ production & 0.840 GeV & $\pm$10\% \\
$M_A^{COH}\pi$  & Axial mass for CC and NC coherent pion production & 1.000 GeV & $\pm$50\% \\
$R_0^{COH}\pi$  & Nuc param. controlling $\pi$ absorption in RS model & 1.000 fm & $\pm$10\% \\ \hline

$A_{HT}^{BY}$  & $A_{HT}$ twist param in scaling variable $\xi_\omega$ & 0.5380 & $\pm$25\% \\
$B_{HT}^{BY}$  & $B_{HT}$ twist param in scaling variable $\xi_\omega$ & 0.305 & $\pm$25\% \\
$C_{V1u}^{BY}$  & $C_{V1u}$ u valence GRV98 PDF correction param & 0.291 & $\pm$30\% \\
$C_{V2u}^{BY}$  & $C_{V2u}$ u valence GRV98 PDF correction param & 0.189 & $\pm$40\% \\ \hline

FormZone  & Hadron formation zone & 0.342 fm & $\pm$50\% \\
%FZ ($\pi$)  & Hadron formation zone & 0.342 fm & $\pm$50\% \\
%FZ (nucleon)  & Hadron formation zone & 2.300 fm & $\pm$50\% \\
BR ($\gamma$)  & Branch ratio for radiative resonance decays & - & $\pm$50\% \\
BR ($\eta$)  & Branch ratio for single-$\eta$ resonance decays & - & $\pm$50\% \\ \hline

$RR_{\nu p}^{CC1\pi}$ & Non-resonance bkg in $\nu$p CC1$\pi$ reactions & - & $\pm$50\% \\ 
$RR_{\nu p}^{CC2\pi}$ & Non-resonance bkg in $\nu$p CC2$\pi$ reactions & - & $\pm$50\% \\ 
$RR_{\nu n}^{CC1\pi}$ & Non-resonance bkg in $\nu$n CC1$\pi$ reactions & - & $\pm$50\% \\ 
$RR_{\nu n}^{CC2\pi}$ & Non-resonance bkg in $\nu$n CC2$\pi$ reactions & - & $\pm$50\% \\ 
$RR_{\nu p}^{NC1\pi}$ & Non-resonance bkg in $\nu$p NC1$\pi$ reactions & - & $\pm$50\% \\ 
$RR_{\nu p}^{NC2\pi}$ & Non-resonance bkg in $\nu$p NC2$\pi$ reactions & - & $\pm$50\% \\ 
$RR_{\nu n}^{NC1\pi}$ & Non-resonance bkg in $\nu$n NC1$\pi$ reactions & - & $\pm$50\% \\ 
$RR_{\nu n}^{NC2\pi}$ & Non-resonance bkg in $\nu$n NC2$\pi$ reactions & - & $\pm$50\% \\  \hline

$x_{abs}^{N}$ & Nucleon absorption probability & - & $\pm$20\% \\ 
$x_{cex}^{N}$ & Nucleon charge exchange probability & - & $\pm$50\% \\ 
$x_{el}^{N}$ & Nucleon elastic reaction probability & - & $\pm$30\% \\ 
$x_{inel}^{N}$ & Nucleon inelastic reaction probability & - & $\pm$40\% \\ 
$x_{mfp}^{N}$ & Nucleon mean free path & - & $\pm$20\% \\
$x_{\pi}^{N}$ & Nucleon $\pi$-production probability & - & $\pm$20\% \\
$x_{abs}^{\pi}$ & $\pi$ absorption probability & - & $\pm$20\% \\
$x_{cex}^{\pi}$ & $\pi$ charge exchange probability & - & $\pm$50\% \\
$x_{el}^{\pi}$ & $\pi$ elastic reaction probability & - & $\pm$10\% \\
$x_{inel}^{\pi}$ & $\pi$ inelastic reaction probability & - & $\pm$40\% \\
$x_{mfp}^{\pi}$ & $\pi$ mean free path & - & $\pm$20\% \\
$x_{\pi}^{\pi}$ & $\pi$ $\pi$-production probability & - & $\pm$20\% \\ \hline
\end{tabular}
\end{table*}

\par Once we have obtained our event re-weighting results, we calculate the cross section variation by considering the variation to the background B and to the efficiency $\epsilon$, as shown in Equation \ref{eq:genie_xsec_var}. We do not consider variation to total tagged events N, as this quantity will be measured directly from data.  To calculate the cross section percentile difference due to a single parameter variation, we then compare $\sigma^\pm$ to nominal $\sigma$ calculated in earlier $\pi^0$ Cuts section as shown in Equation \ref{eq:genie_xsec_percentdiff}.

\par Event re-weighting as described was run over the 420k MCBNB+Cosmic events described in Table \ref{tab:nevents}. The results of this reweighting scheme is shown in Table \ref{tab:genie_results} for the higher variation (i.e., +$\sigma$ or -$\sigma$). In summary, the CC 1-$\pi^0$ analysis chain is most affected by the axial mass for CC resonance production;  this result is supported by Figure \ref{fig:physics_pi0_inttype}, in which resonant processes dominate the neutrino interaction modes of selected events. The final sample is also sensitive to the form zone parameter, nucleon elastic reaction probability, and the nucleon mean free path.   Finally, we see a significant contribution from the $\pi$ absorption probability. We also expect this; variation in the probability for $\pi$ absorption in the nucleus is correlated with the resulting number of $\pi^0$'s originating from the nucleus and entering our signal sample.  

\par Finally, we consider the contribution of each parameter weighting to individual signal and backgrounds.  We do this by applying reweighting ONLY to the signal or to the individual  background we're focusing on, and then comparing to the nominal as in Equation \ref{eq:genie_xsec_percentdiff}. These results are shown in Table \ref{tab:genie_bkgd_2shower_var_results} for the 2-shower sample and in Table \ref{tab:genie_bkgd_2shower_var_results} for the 1-shower sample. To interpret the information in this table, we consider the example of $M_A^{CCRES}$ in Table \ref{tab:genie_bkgd_2shower_var_results}. We expect variation of this parameter to dominantly affect CC Resonant processes.  What we see is that variations to this individual parameter have no impact on the NC backgrounds (0.00\%) and a maximum impact on FSEM, which contains all $\nu_\mu$ CC events with electromagnetic activity in the final state. 
Note that while the breakdowns Tables \ref{tab:genie_bkgd_2shower_var_results} and \ref{tab:genie_bkgd_2shower_var_results} gives us an idea of the contribution of each background to the overall uncertainty, we can not quadratically sum these breakdowns to estimate the final uncertainty. 

\begin{equation} \label{eq:genie_xsec_var}
  \sigma^\pm \propto \frac{N - B^\pm}{\epsilon^\pm} 
\end{equation}

\begin{equation} \label{eq:genie_xsec_percentdiff}
  Percentile\ Difference = \frac{| \sigma - \sigma^\pm |}{\sigma} 
\end{equation}

\begin{table*}
\centering
\captionof{table}{Results from event reweighting on CC-$\pi^0$ analysis with 2- and 1-shower selection chain \label{tab:genie_results}}
 \begin{tabular}{| l | l | l |}
 \hline
  Parameter P & Percentile Variation - 2 Shower & Percentile Variation - 1 Shower  \\ [0.1ex] \hline
$M_A^{NCEL}$ &  0.40\% & 0.35\% \\
$\eta^{NCEL}$  & 0.00\% & 0.01\% \\
$M_A^{CCQE}$  & 1.13\% & 2.66\% \\
$M_V^{CCQE}$  & 0.12\% & 0.16\% \\
$M_A^{CCRES}$  & 5.25\% & 7.00\% \\
$M_V^{CCRES}$  & 3.22\% & 4.19\% \\
$M_A^{NCRES}$  & 1.78\% & 1.24\% \\
$M_V^{NCRES}$  & 0.54\% & 0.31\%\\
$M_A^{COH}\pi$  & 0.00\% & 0.49\% \\
$R_0^{COH}\pi$  & 0.00\% & 0.49\%\\

AGKYpT & 0.00\% & 0.00\% \\
AGKYxF & 0.00\% & 0.00\% \\
DISAth & 0.18\% & 0.14\% \\
DISBth & 0.27\% & 0.21\% \\
DISC$\nu$1u & 0.16\% & 0.09\% \\
DISC$\nu$2u & 0.14\% & 0.08\% \\ \hline

FormZone  & 5.05\% & 4.92\% \\
BR ($\gamma$)  & 0.02\% & 0.27\% \\
BR ($\eta$)  & 3.17\% & 2.24\% \\
BR ($\theta$)  & 2.44\% & 2.00\% \\ \hline

$RR_{\nu p}^{CC1\pi}$ & 0.65\% & 0.30\% \\ 
$RR_{\nu p}^{CC2\pi}$ & 1.00\% & 2.00\% \\
$RR_{\nu n}^{CC1\pi}$ & 2.94\% & 3.24\% \\ 
$RR_{\nu n}^{CC2\pi}$ & 1.06\% & 1.41\% \\ \hline

$x_{abs}^{N}$ & 1.41\% & 1.72\% \\
$x_{cex}^{N}$ & 1.49\% & 1.22\%\\
$x_{el}^{N}$ & 2.52\% & 3.36\% \\
$x_{inel}^{N}$ & 0.79\% & 0.20\% \\
$x_{mfp}^{N}$ & 3.79\% & 3.59\% \\
$x_{\pi}^{N}$ & 0.22\% & 0.46\% \\
$x_{abs}^{\pi}$ & 3.96\% & 3.89\% \\
$x_{cex}^{\pi}$ & 1.49\% & 0.03\% \\
$x_{el}^{\pi}$ & 0.15\% & 0.08\% \\
$x_{inel}^{\pi}$ & 2.57\% & 4.48\% \\
$x_{mfp}^{\pi}$ & 0.40\% & 0.44\% \\
$x_{\pi}^{\pi}$ & 0.21\% & 0.33\% \\
\hline
Total Combined Uncertainty & 12.08\% & 13.80\% \\ \hline
\end{tabular}
\end{table*}

\begin{table*}
\centering
\captionof{table}{Results from event reweighting each individual background alone on CC-$\pi^0$ analysis with 2-shower selection chain. Recall that the Cosmic ($\nu$) background contains candidate $\mu$ tracks that are cosmic in origin; these cosmic events are neutrino-coincident.  \label{tab:genie_bkgd_2shower_var_results}}
 \begin{tabular}{| l | l | l | l | l | l | l | l | l | l | l | l | }
 \hline
  Param P &Signal& FSEM & CCCex & CC$>$1$\pi^0$ & NC$\pi^0$& Other & Cosmics \\ [0.1ex] \hline
$M_A^{NCEL}$ & 0.00\% &  0.40\% &  0.00\% &  0.00\% &  0.00\% &  0.00\% &  0.00\%   \\
$\eta^{NCEL}$  &0.00\% &  0.00\% &  0.00\% &  0.00\% &  0.00\% &  0.00\% &  0.00\%  \\
$M_A^{CCQE}$  & 0.21\% &  2.32\% &  0.64\% &  0.97\% &  0.00\% &  0.20\% &  1.13\% \\
$M_V^{CCQE}$  & 0.01\% &  0.03\% &  0.00\% &  0.00\% &  0.00\% &  0.02\% &  0.06\% \\
$M_A^{CCRES}$  & 0.21\% &  2.32\% &  0.64\% &  0.97\% &  0.00\% &  0.20\% &  1.13\% \\
$M_V^{CCRES}$ & 0.04\% &  1.51\% &  0.38\% &  0.55\% &  0.00\% &  0.09\% &  0.67\% \\
$M_A^{NCRES}$ & 0.00\% &  0.43\% &  0.00\% &  0.00\% &  1.35\% &  0.00\% &  0.00\%  \\
$M_V^{NCRES}$ &0.00\% &  0.13\% &  0.00\% &  0.00\% &  0.41\% &  0.00\% &  0.00\%\\
$M_A^{COH}\pi$ & 0.00\% &  0.00\% &  0.00\% &  0.00\% &  0.00\% &  0.00\% &  0.00\% \\
$R_0^{COH}\pi$ & 0.00\% &  0.00\% &  0.00\% &  0.00\% &  0.00\% &  0.00\% &  0.00\% \\

AGKYpT & 0.00\% &  0.00\% &  0.00\% &  0.00\% &  0.00\% &  0.00\% &  0.00\% \\ %
AGKYxF & 0.00\% &  0.00\% &  0.00\% &  0.00\% &  0.00\% &  0.00\% &  0.00\% \\ %
DISAth & 0.04\% &  0.03\% &  0.02\% &  0.05\% &  0.02\% &  0.02\% &  0.00\% \\ %
DISBth & 0.05\% &  0.05\% &  0.04\% &  0.07\% &  0.03\% &  0.02\% &  0.00\%  \\ %
DISC$\nu$1u & 0.01\% &  0.05\% &  0.01\% &  0.06\% &  0.01\% &  0.03\% &  0.00\%\\ %
DISC$\nu$2u & 0.01\% &  0.04\% &  0.01\% &  0.05\% &  0.01\% &  0.02\% &  0.00\% \\ \hline

FormZone  & 1.30\% &  0.29\% &  0.38\% &  1.36\% &  1.18\% &  0.21\% &  0.36\% \\
BR ($\gamma$) &0.01\% &  0.01\% &  0.00\% &  0.01\% &  0.01\% &  0.00\% &  0.01\%\\
BR ($\eta$) &0.31\% &  2.26\% &  0.02\% &  0.82\% &  0.00\% &  0.00\% &  0.21\% \\
BR ($\theta$) & 1.56\% &  0.66\% &  0.01\% &  0.22\% &  0.12\% &  0.00\% &  0.11\% \\ \hline

$RR_{\nu p}^{CC1\pi}$ & 0.00\% &  0.13\% &  0.26\% &  0.13\% &  0.13\% &  0.00\% &  0.00\%\\ 
$RR_{\nu p}^{CC2\pi}$ & 0.09\% &  0.13\% &  0.13\% &  0.13\% &  0.39\% &  0.13\% &  0.00\% \\
$RR_{\nu n}^{CC1\pi}$ & 0.85\% &  0.39\% &  0.52\% &  0.13\% &  0.13\% &  0.13\% &  0.78\%\\ 
$RR_{\nu n}^{CC2\pi}$ & 0.65\% &  0.13\% &  0.26\% &  0.78\% &  0.13\% &  0.26\% &  0.13\%  \\ \hline

$x_{abs}^{N}$ & 0.72\% &  0.01\% &  0.00\% &  0.28\% &  0.20\% &  0.02\% &  0.23\%\\
$x_{cex}^{N}$ &1.08\% &  0.03\% &  0.08\% &  0.11\% &  0.08\% &  0.04\% &  0.23\%\\
$x_{el}^{N}$ & 1.91\% &  0.09\% &  0.15\% &  0.17\% &  0.14\% &  0.06\% &  0.47\% \\
$x_{inel}^{N}$ & 0.49\% &  0.10\% &  0.12\% &  0.08\% &  0.13\% &  0.07\% &  0.15\% \\
$x_{mfp}^{N}$ & 1.84\% &  0.77\% &  0.43\% &  0.31\% &  0.18\% &  0.10\% &  0.14\%  \\
$x_{\pi}^{N}$ &0.07\% &  0.07\% &  0.03\% &  0.17\% &  0.07\% &  0.03\% &  0.03\%  \\
$x_{abs}^{\pi}$ &0.65\% &  0.54\% &  0.46\% &  1.17\% &  0.41\% &  0.13\% &  0.93\% \\
$x_{cex}^{\pi}$ & 0.48\% &  0.01\% &  0.12\% &  0.85\% &  0.09\% &  0.12\% &  0.22\% \\
$x_{el}^{\pi}$ &0.55\% &  0.01\% &  0.18\% &  0.17\% &  0.04\% &  0.03\% &  0.14\%\\
$x_{inel}^{\pi}$ & 0.91\% &  0.60\% &  0.46\% &  0.45\% &  0.66\% &  0.23\% &  0.60\%  \\
$x_{mfp}^{\pi}$ & 1.41\% &  0.25\% &  0.27\% &  0.09\% &  0.54\% &  0.16\% &  0.14\%  \\
$x_{\pi}^{\pi}$ &0.13\% &  0.08\% &  0.03\% &  0.07\% &  0.04\% &  0.03\% &  0.02\%\\
\hline
%Total Combined Uncertainty & 10.44\% & 13.40\% \\ \hline
\end{tabular}
\end{table*}

%%%%%%%% Here we're adding background + signal variation specifically
\begin{table*}
\centering
\captionof{table}{Results from event reweighting each individual background alone on CC-$\pi^0$ analysis with 1-shower selection chain. Recall that the Cosmic ($\nu$) background contains candidate $\mu$ tracks that are cosmic in origin; these cosmic events are neutrino-coincident. \label{tab:genie_bkgd_1shower_var_results}}
 \begin{tabular}{| l | l | l | l | l | l | l | l | l | l | l | l | }
 \hline
  Param P &Signal& FSEM & CCCex & CC$>$1$\pi^0$ & NC$\pi^0$& Other & Cosmics \\ [0.1ex] \hline
$M_A^{NCEL}$ & 0.00\% &  0.13\% &  0.00\% &  0.00\% &  0.06\% &  0.00\% &  0.16\%  \\
$\eta^{NCEL}$  & 0.00\% &  0.00\% &  0.00\% &  0.00\% &  0.00\% &  0.00\% &  0.01\% \\
$M_A^{CCQE}$  & 0.17\% &  0.84\% &  0.05\% &  0.00\% &  0.00\% &  0.60\% &  1.01\% \\
$M_V^{CCQE}$  & 0.02\% &  0.07\% &  0.01\% &  0.00\% &  0.00\% &  0.06\% &  0.01\% \\
$M_A^{CCRES}$  & 0.63\% &  3.03\% &  0.87\% &  1.02\% &  0.00\% &  2.15\% &  0.56\% \\
$M_V^{CCRES}$ & 0.41\% &  1.80\% &  0.48\% &  0.63\% &  0.00\% &  1.29\% &  0.29\% \\
$M_A^{NCRES}$ & 0.00\% &  0.40\% &  0.00\% &  0.00\% &  0.60\% &  0.12\% &  0.12\% \\
$M_V^{NCRES}$ & 0.00\% &  0.13\% &  0.00\% &  0.00\% &  0.18\% &  0.01\% &  0.00\%\\
$M_A^{COH}\pi$ & 0.00\% &  0.08\% &  0.24\% &  0.00\% &  0.00\% &  0.08\% &  0.08\%  \\
$R_0^{COH}\pi$ & 0.00\% &  0.08\% &  0.24\% &  0.00\% &  0.00\% &  0.08\% &  0.08\%  \\

AGKYpT & 0.00\% &  0.00\% &  0.00\% &  0.00\% &  0.00\% &  0.00\% &  0.00\% \\ %
AGKYxF & 0.00\% &  0.00\% &  0.00\% &  0.00\% &  0.00\% &  0.00\% &  0.00\% \\ %
DISAth & 0.01\% &  0.06\% &  0.01\% &  0.04\% &  0.01\% &  0.01\% &  0.00\% \\ %
DISBth & 0.00\% &  0.08\% &  0.02\% &  0.06\% &  0.02\% &  0.02\% &  0.00\% \\ %
DISC$\nu$1u & 0.02\% &  0.03\% &  0.02\% &  0.02\% &  0.02\% &  0.02\% &  0.00\%  \\ %
DISC$\nu$2u & 0.01\% &  0.03\% &  0.02\% &  0.02\% &  0.02\% &  0.02\% &  0.00\%  \\ \hline

FormZone  & 3.21\% &  0.81\% &  0.36\% &  0.86\% &  0.45\% &  0.58\% &  0.21\% \\
BR ($\gamma$) & 0.01\% &  0.28\% &  0.00\% &  0.01\% &  0.00\% &  0.00\% &  0.00\% \\
BR ($\eta$) &0.03\% &  1.52\% &  0.11\% &  1.08\% &  0.04\% &  0.26\% &  0.01\% \\
BR ($\theta$) & 0.56\% &  0.26\% &  0.23\% &  0.27\% &  0.09\% &  0.46\% &  0.12\% \\ \hline

$RR_{\nu p}^{CC1\pi}$ & 0.15\% &  0.12\% &  0.08\% &  0.04\% &  0.04\% &  0.08\% &  0.08\% \\ 
$RR_{\nu p}^{CC2\pi}$ & 0.46\% &  0.54\% &  0.04\% &  0.12\% &  0.33\% &  0.33\% &  0.17\% \\
$RR_{\nu n}^{CC1\pi}$ & 0.53\% &  0.92\% &  0.42\% &  0.04\% &  0.17\% &  0.75\% &  0.42\% \\ 
$RR_{\nu n}^{CC2\pi}$ & 0.24\% &  0.00\% &  0.17\% &  0.42\% &  0.29\% &  0.21\% &  0.08\% \\ \hline

$x_{abs}^{N}$ & 1.15\% &  0.25\% &  0.03\% &  0.11\% &  0.09\% &  0.01\% &  0.08\%\\
$x_{cex}^{N}$ & 0.88\% &  0.15\% &  0.02\% &  0.09\% &  0.07\% &  0.20\% &  0.03\% \\
$x_{el}^{N}$ & 2.01\% &  0.74\% &  0.01\% &  0.18\% &  0.01\% &  0.38\% &  0.12\% \\
$x_{inel}^{N}$ & 0.61\% &  0.36\% &  0.03\% &  0.13\% &  0.03\% &  0.26\% &  0.05\% \\
$x_{mfp}^{N}$ & 1.16\% &  0.98\% &  0.22\% &  0.28\% &  0.11\% &  0.45\% &  0.45\% \\
$x_{\pi}^{N}$ & 0.06\% &  0.09\% &  0.06\% &  0.16\% &  0.06\% &  0.14\% &  0.00\%  \\
$x_{abs}^{\pi}$ & 0.21\% &  1.02\% &  0.39\% &  1.01\% &  0.56\% &  0.70\% &  0.05\% \\
$x_{cex}^{\pi}$ & 0.15\% &  0.07\% &  0.11\% &  0.36\% &  0.04\% &  0.11\% &  0.08\% \\
$x_{el}^{\pi}$ &0.10\% &  0.15\% &  0.16\% &  0.03\% &  0.01\% &  0.06\% &  0.11\%\\
$x_{inel}^{\pi}$ & 0.53\% &  1.48\% &  0.40\% &  0.52\% &  0.68\% &  1.14\% &  0.03\%  \\
$x_{mfp}^{\pi}$ & 0.87\% &  0.24\% &  0.20\% &  0.37\% &  0.17\% &  0.23\% &  0.11\% \\
$x_{\pi}^{\pi}$ & 0.10\% &  0.02\% &  0.04\% &  0.14\% &  0.01\% &  0.02\% &  0.00\% \\
\hline
\end{tabular}
\end{table*}


We conclude here by combining the uncertainties in quadrature and calculating a 12.08\% uncertainty for the 2-shower path and a 13.80\% uncertainty for the 1-shower path from the GENIE model.  The higher uncertainty on the 1-shower sample is due to the lower purity of the single shower selection path.

\clearpage
\subsection{Flux Systematics}
When considering uncertainty due to flux, we are primarily interested in variations of particle production at the target ($\pi^+$, $\pi^-$, $K^+$, $K^-$, $K^0$) and POT.  The uncertainty due to POT counting (roughly 2\%, measured in the beam hall) is considerably smaller than that due to the flux itself. 
\par To evaluate the flux systematic contribution, we consider variations on a series of parameters.  These parameter variations cover uncertainty on the depth by which the current penetrates into the horn conductor (the ‘skin effect’), the current that the horn is pulsed with, pion and nucleon cross sections
(total, inelastic, and quasi-elastic) on aluminum and beryllium and hadron production. To report our results, we combine all non-hadron production uncertainty contributions into one reported value called `FluxUnisim', while keeping the hadronic parameters separated. With these parameter variations in mind, the EventWeight calculator produces 1000 multisims using a unique random number generator seed for each flux systematic uncertainty. Per event, we then proceed to calculate 1000 cross sections using the information stored in these 1000 universes.  We do this using the Equation \ref{eq:flux_xsec_var_0} for i in 0 - 1000 universes:

\begin{equation} \label{eq:flux_xsec_var_0}
  \sigma_i \propto \frac{N - B_i}{\epsilon_i \phi_i} 
\end{equation}

Where `N' is the On - OffBeam value from the final stage of each selection path, $B_i$ is the weighted background contribution in the i'th universe, $\epsilon_i$ is the i'th universe efficiency, and $\phi_i$ is the flux renormalization factor for the i'th universe. To create the variations in the flux renormalization we take the beam group produced gsimple files and convert them into art events by filling the MCFlux data product. The resulting artroot files are then passed through EventWeight and with the same random number seeds as was used to produce the event weights. This leads to a one-to-one mapping of the universes of the event weight to the reweighted flux. Using these modified fluxes we integrate from 0 to 3 GeV to determine the flux through the detector. This integration across the full energy range increases our overall systematic uncertainty due to the behavior of spline fits to the HARP $\pi^{+}$ production data in regions where there is no HARP data~\cite{bib:flux_uncertainty_tn}. This uncertainty could be reduced by utilizing a physically driven constraint on the pion production in these regions.   

Once 1000 cross sections are calculated, we calculate 1000 corresponding percentile differences with respect to the nominal according to Equation \ref{eq:genie_xsec_percentdiff}. We then use the percentile variations to extract a 1 $\sigma$ uncertainty from the nominal value at 0 for each varied flux parameter.  These distributions are shown for the 2 shower sample in Figures \ref{fig:flux_2shower_unc_plots_0}-\ref{fig:flux_2shower_unc_plots_2} and for the 1 shower sample in Figures \ref{fig:flux_1shower_unc_plots_0}-\ref{fig:flux_1shower_unc_plots_2}. The uncertainty extracted from each sample is displayed on these plots and additionally summarized in Table \ref{tab:flux_results}.  

%\noindent More details on the event weight framework and these parameters can be found externally \cite{bib:flux_uncertainty_tn}.  

 \begin{table}[H]
 \centering
 \captionof{table}{ \label{tab:flux_results} Summary of contributions of the flux systematic uncertainty for both 2 and 1 shower paths. }
  \begin{tabular}{| l | l | l |}
  \hline
   Parameter & Percent Variation - 2 Shower & Percent Variation - 1 Shower  \\ [0.1ex] \hline
 FluxUnisim & 8.3 \% & 9.16 \%  \\
 $K^-$ Production & 0.03\% & 0.34\%\\
 $K^+$ Production &  1.01 \% & 1.07\% \\
 $K^0$ Production & 0.14\% & 0.33\% \\
 $\pi^-$ Production & 0.05\% & 0.22\%\\
 $\pi^+$ Production &  11.75\% & 11.16 \% \\ \hline
 Total Combined Uncertainty & 14.42\% & 14.49\%\\ \hline
\end{tabular}
\end{table}

\begin{figure}[h!]
\centering
\includegraphics[scale=0.35]{XSection_Calc_Section/Flux_perc_var_pi0_FluxUnisim.png}
\includegraphics[scale=0.35]{XSection_Calc_Section/Flux_perc_var_pi0_K0.png}
\caption{ Uncertainty contributions broken down by function for the two shower sample. On the left is the uncertainty contribution from all the non-hadronic processes; on the right are variations due to $K^0$ production. }
\label{fig:flux_2shower_unc_plots_0}
\end{figure}

\begin{figure}[h!]
\centering
\includegraphics[scale=0.35]{XSection_Calc_Section/Flux_perc_var_pi0_K+.png}
\includegraphics[scale=0.35]{XSection_Calc_Section/Flux_perc_var_pi0_K-.png}
\caption{ Uncertainty contributions broken down by function for the 2 shower sample. On the left is the uncertainty contribution from variations on $K^+$ production; on the right is the uncertainty due to $K^-$ production. }
\label{fig:flux_2shower_unc_plots_1}
\end{figure}

\begin{figure}[h!]
\centering
\includegraphics[scale=0.35]{XSection_Calc_Section/Flux_perc_var_pi0_pi+.png}
\includegraphics[scale=0.35]{XSection_Calc_Section/Flux_perc_var_pi0_pi-.png}
\caption{ Uncertainty contributions broken down by function for the 2 shower sample. On the left is the uncertainty contribution from variations on $\pi^+$ production; on the right is the uncertainty due to $\pi^-$ production. }
\label{fig:flux_2shower_unc_plots_2}
\end{figure}


\begin{figure}[h!]
\centering
\includegraphics[scale=0.35]{XSection_Calc_Section/Flux_perc_var_singleshower_FluxUnisim.png}
\includegraphics[scale=0.35]{XSection_Calc_Section/Flux_perc_var_singleshower_K0.png}
\caption{ Uncertainty contributions broken down by function for 1 shower sample. On the left is the uncertainty contribution from all the non-hadronic processes; on the right are variations due to $K^0$ production. }
\label{fig:flux_1shower_unc_plots_0}
\end{figure}

\begin{figure}[h!]
\centering
\includegraphics[scale=0.35]{XSection_Calc_Section/Flux_perc_var_singleshower_K+.png}
\includegraphics[scale=0.35]{XSection_Calc_Section/Flux_perc_var_singleshower_K-.png}
\caption{ Uncertainty contributions broken down by function for the 1 shower sample. On the left is the uncertainty contribution from variations on $K^+$ production; on the right is the uncertainty due to $K^-$ production. }
\label{fig:flux_1shower_unc_plots_1}
\end{figure}

\begin{figure}[h!]
\centering
\includegraphics[scale=0.35]{XSection_Calc_Section/Flux_perc_var_singleshower_pi+.png}
\includegraphics[scale=0.35]{XSection_Calc_Section/Flux_perc_var_singleshower_pi-.png}
\caption{ Uncertainty contributions broken down by function for the 1 shower sample. On the left is the uncertainty contribution from variations on $\pi^+$ production; on the right is the uncertainty due to $\pi^-$ production. }
\label{fig:flux_1shower_unc_plots_2}
\end{figure}

\clearpage
\subsection{Detector Simulation Systematic Uncertainties}

There are three classes of detector simulation based systematic uncertainties that we are taking into account in this section of the note. Those due to the overall simulation of detector effects (known colloquially as ``detector systematics''), those due to the modeling of the CORSIKA cosmic simulation, and finally those associated with the number of targets that are contained in the fiducial volume.    

\subsubsection{Detector Simulation Unisims}
To assess systematic uncertainties on our modeling of various detector effects we will generate a complementary set of MC with variations to a given parameter. By using the same set of base events we can minimize the statistical variation in the sample and hone in directly on the systematic due to the model change. One caveat to this method is that it is currently not a functionality within LArSoft to fix the GEANT random number seeds. Therefore, if one of our variations requires us to rerun GEANT, this may lead to a different behavior of the particles traversing our detector. This could enhance the estimated effect of varying the underlaying parameter. Using these varied simulations, unisims, we will assess the uncertainty by remeasuring the cross section given a central value (CV) sample with an high statistics CC1$\pi^{0}$ MC and a sample of BNB+cosmic events. We will then remeasure the cross section for each unisim under the assumption that each unisim represents a 1$\sigma$ shift in a gaussian distributed uncertainty. Since this is not in principle true, we will create conservative variations in the underlaying parameters that will produce a conservative estimate of these uncertainties. 

Currently we are producing the following unisims through production and will process them in the near future:
\begin{itemize}
\item Shut off space-charge
\item Turn on a simulation of the dynamic induced charge effect 
\item Fix a bug in the scintillation light generated by different particles
\item Stretch the reconstruction response function by 20\% in time 
\item Mask out misconfigured channels
\item Mask out channels that are prone to having their ASICs saturate
\item Turn off the PMT single PE noise
\item Turn off the data-drive signal response (effectively a perfect calibration)
\item Switch to using a white-noise model for the electronics noise simulation
\item Increase the visibility of the region outside the active TPC volume by 50\%
\item Set the electron lifetime to 10ms 
\item Turn off lateral diffusion 
\item Turn off transverse diffusion 
\item Switch to using a simulation of the recombination based on the Birk’s model 
\end{itemize}

\noindent Additionally, we are studying the effect of adjusting the following GEANT parameters:
\begin{itemize}
\item Turn off delta ray production 
\item Turn off hadronic interactions in the bulk
\end{itemize}

\subsubsection{MC-based Cosmic Simulation}\label{sec:cosuncert}
The ``cosmic+neutrino'' background makes up 6\% and 6\% of the two- and one-shower selected events, respectively. This background is simulated by the CORSIKA simulation of the cosmics flux through our detector. Currently to assess an uncertainty on these events we are taking an overall 100\% normalization uncertainty on these events. The showers that are selected typically originate from cosmic muons, either delta-rays or bremsstrahlung or Michel electrons, we know that our simulation of the overall cosmic muon flux is not more than 100\% off based on past external measurements inside the MicroBooNE hall. Using scintillator counters it was measured that the muon flux was $141\pm21~\text{Hz~m}^2$~\cite{datacosflux}, while the CORSIKA simulation generates a flux of $160.9\pm0.3~\text{Hz~m}^2$  through the detector~\cite{mccosflux}. Given this variation we find a resulting systematic uncertainty on the final cross section of 10\% and 11\% for the two- and one-shower selection, respectively. 
 
\subsubsection{Uncertainty on the Number of Targets}\label{sec:ntarguncert}

When we want to assess and uncertainty on the number of targets that we have inside our fiducial volume there are two main aspects that can lead to variations. First, the density of the argon and second the fiducial volume. For the first we can rely on the sensors that we have oriented throughout the detector and the cryogenic system to estimate the central value and expected variation in the temperature and pressure of the cryostat. Using slow controls and discussing the accuracy of the various devices with the cryogenic experts we find that throughout Run 1 during stable periods the temperature and pressure inside the cryostat are $89.2  \pm 0.3$~K and $1.241 \pm 0.004$~bar, respectively. These uncertainties are taken outside the RMS spread of these variables to create a conservative estimate. Using these variations we find that the uncertainty on the density of the liquid argon is $1.3836^{+0.0019}_{-0.0002}~\frac{\text{g}}{\text{cm}^3}$, which is a 0.1\% effect. This is safely a negligible effect considering the size of the other uncertainties we are taking into account.

The second uncertainty we are taking into account is the fact that the fiducial volume that we have defined is affected by the presence of the space charge effect. The electric field distortions has the effect making regions of the detector appear to be within our fiducial volume and vice-versa. To assess an uncertainty on this we take the entire active volume of the detector and divide it into $0.0619~\text{cm}^3$ voxels. We then utilize the simulated space charge displacement maps and compute the voxels that moved into and out of our defined fiducial volume. We find that this leads to a change in the fiducial volume of 6.3\%. Taking this as the 1$\sigma$ variation we find a fractional change in the cross section measurement to be 6.3\% for both selections. 


\subsubsection{Total Detector Simulation Uncertainty}
	
We assess the overall effect on the final cross section by taking the quadratic sum of the each total of each of the systematic effects. Currently we are only integrating the values from Sec.~\ref{sec:cosuncert} and~\ref{sec:ntarguncert} which leads to an overall uncertainty on the final cross section of 12\% and 13\% for the two- and one-shower selection respectively. 

\clearpage 
 \section{Conclusion}
In this note, we described a procedure that culminated in a total integrated charged current single $\pi^0$ cross section on argon. We began by defining our signal as neutrino-induced charged current interactions with a single $\pi^0$ in the final state originating from a FV-contained vertex.  We began by implementing a series of cuts which mitigated the high cosmic background, and identifying events with a $\nu$-induced $\mu$ candidate.  We next identified electromagnetic activity in each plane and used this information to reconstruct 3D shower candidates. Finally, we considered a variety of spatial correlations and shower parameters to create two distinct final samples: a single-shower selected sample, and a two shower selected sample.  With these final selected events, we were able to calculate the uncertainties due to the neutrino flux production, GENIE interaction modeling, and the detector simulation contributions to the final result.  These results are summarized below:\\

\noindent $\sigma\left(\nu_{\mu} + \text{Ar} \rightarrow \mu + 1 \pi^{0} + X\right)_{\geq 2 \text{showers}}=$ (2.56 $\pm$ $0.50_{stat}$ $\pm$ $0.31_{genie}$ $\pm$ $0.37_{flux}$ $\pm$ $0.31_{det}$) $\times$ $10^{-38}$ $\frac{cm^2}{Ar}$ \\

\noindent $\sigma\left(\nu_{\mu} + \text{Ar} \rightarrow \mu + 1 \pi^{0} + X\right)_{\geq 1 \text{shower}}=$ (2.64 $\pm$ $0.33_{stat}$ $\pm$ $0.36_{genie}$ $\pm$ $0.38_{flux}$ $\pm$ $0.35_{det}$) $\times$ $10^{-38}$ $\frac{cm^2}{Ar}$ \\

\noindent This is the first charged current single $\pi^0$ cross section measured on argon.

\begin{figure}[h!]
\centering
\includegraphics[width=1\textwidth]{FinalCrossSectionPlots/Final_statsyst.png}
%\includegraphics[scale=0.35]{XSection_Calc_Section/GenieTruth_stat_sys.png}
\caption{ $\nu_{\mu}+\text{Ar}$ charged current single pion production cross section extracted from GENIE with the MicroBooNE measured cross section using the two- and one-shower paths. The uncertainty bars on the MicroBooNE measurements are inner statistical only and the outer are the quadratic sum of the statistical, flux, detector simulation, and GENIE uncertainties.}
\label{fig:genie_uboone_xsec2}
\end{figure}


\clearpage
\appendix
\clearpage

\section{0 Reconstructed Showers}
\label{sec:AppC}
Here we consider the distributions of events which have 0 showers reconstructed. Signal and background distributions are shown for a variety of kinematic variables in Figures \ref{fig:physics_sel2_0shower_mulen} - \ref{fig:physics_sel2_0shower_z} (uncertainties are purely statistical at this point). 

\begin{figure}[h!]
\centering
\includegraphics[scale=0.3]{Appendix_0Showers/Physics_sel2_0showers_onoffseparate_mult.png}
\hspace{2 mm}
\includegraphics[scale=0.3]{Appendix_0Showers/Physics_sel2_0showers_onoffseparate_mu_len.png}
\caption{ Data to simulation comparison of a) multiplicity and b) $\mu$ length after Selection II filter }
\label{fig:physics_sel2_0shower_mulen}
\end{figure}

\begin{figure}[h!]
\centering
\includegraphics[scale=0.3]{Appendix_0Showers/Physics_sel2_0showers_onoffseparate_mu_angle.png}
\hspace{2 mm}
\includegraphics[scale=0.3]{Appendix_0Showers/Physics_sel2_0showers_onoffseparate_mu_phi.png}
\caption{ Data to simulation comparison of $\mu$ a) $\theta$  and b) $\phi$ after Selection II filter }
\label{fig:physics_sel2_0shower_muphi}
\end{figure}

\begin{figure}[h!]
\centering
\includegraphics[scale=0.3]{Appendix_0Showers/Physics_sel2_0showers_onoffseparate_mu_startx.png}
\includegraphics[scale=0.3]{Appendix_0Showers/Physics_sel2_0showers_onoffseparate_mu_endx.png}
\caption{ Data to simulation comparison of $\mu$ a) start and b) end in x after Selection II filter }
\label{fig:physics_sel2_0shower_x}
\end{figure}

\begin{figure}[h!]
\centering
\includegraphics[scale=0.3]{Appendix_0Showers/Physics_sel2_0showers_onoffseparate_mu_starty.png}
\includegraphics[scale=0.3]{Appendix_0Showers/Physics_sel2_0showers_onoffseparate_mu_endy.png}
\caption{ Data to simulation comparison of $\mu$ a) start and b) end in y after Selection II filter }
\label{fig:physics_sel2_0shower_y}
\end{figure}

\begin{figure}[h!]
\centering
\includegraphics[scale=0.3]{Appendix_0Showers/Physics_sel2_0showers_onoffseparate_mu_startz.png}
\includegraphics[scale=0.3]{Appendix_0Showers/Physics_sel2_0showers_onoffseparate_mu_endz.png}
\caption{ Data to simulation comparison of $\mu$ a) start and b) end in z after Selection II filter }
\label{fig:physics_sel2_0shower_z}
\end{figure}
