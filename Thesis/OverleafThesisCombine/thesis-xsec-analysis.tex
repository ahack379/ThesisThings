\clearpage
\section{Cross Section Analysis}
%\par With samples prepared, we are ready to calculate the cross section of $\nu_\mu$-induced CC 1-$\pi^0$ on argon. 
Before calculating a final cross section, it is important to understand whether any stages of the selection have introduced kinematical bias into the analysis.  

\subsection{Truth and Efficiency Distributions}
This section considers the distributions and selection efficiencies across a variety of kinematic variables.  The events considered in these plots are the same signal CC events with a single neutrino-induced $\pi^0$ originating from a true vertex contained in the FV discussed throughout the note. The efficiency is defined here to be the number of signal at some selection stage divided by the number of signals in the starting sample before any cuts have been applied.  Figures \ref{fig:pi0_effs_0} - \ref{fig:pi0_effs_6} show kinematic distributions and their corresponding efficiencies at each stage of our CC$\pi^0$ selection for both two and one shower paths. Only one legend is shown per line and applies to both surrounding plots.  The flatness within statistical variation of the efficiency distributions described here suggests that no significant kinematically bias exists in the current selection. While this is less important for the absolute cross section measurement calculated here, it will be important going forward with differential and double differential cross section measurements. Only statistical uncertainties are displayed here. 

\begin{figure}[H]
 \begin{subfigure}[t]{0.35\textwidth}
    \includegraphics[scale=0.35]{XSection_Calc_Section/MCVar_pi0_true_nu_e.png}
  \caption{ }
  \end{subfigure} 
  \hspace{15mm}
  \begin{subfigure}[t]{0.35\textwidth}
\includegraphics[scale=0.35]{XSection_Calc_Section/MCVar_eff_pi0_true_nu_e.png}
  \caption{ }
  \end{subfigure} 
\caption{a) $E_\nu$ distribution before and after selection; b) Efficiency as a function of $E_\nu$. }
\label{fig:pi0_effs_0}
\end{figure}
%; note that the efficiency is negligible below 0.275 GeV

\begin{figure}[H]
  \begin{subfigure}[t]{0.35\textwidth}
\includegraphics[scale=0.35]{XSection_Calc_Section/MCVar_pi0_true_pi0_mom.png}
  \caption{ }
  \end{subfigure} 
  \hspace{15mm}
  \begin{subfigure}[t]{0.35\textwidth}
\includegraphics[scale=0.35]{XSection_Calc_Section/MCVar_eff_pi0_true_pi0_mom.png}
  \caption{ }
  \end{subfigure} 
\caption{a) $\pi^0$ momentum distribution across all stages of CC $\pi^0$ selection; b) Efficiency as a function of momentum. }
\label{fig:pi0_effs_1}
\end{figure}

\begin{figure}[H]
  \begin{subfigure}[t]{0.35\textwidth}
\includegraphics[scale=0.35]{XSection_Calc_Section/MCVar_pi0_true_gamma_e_min.png}
  \caption{ }
  \end{subfigure} 
  \hspace{15mm}
  \begin{subfigure}[t]{0.35\textwidth}
\includegraphics[scale=0.35]{XSection_Calc_Section/MCVar_eff_pi0_true_gamma_e_min.png}
  \caption{ }
  \end{subfigure} 
\caption{a) Lower energy shower distribution across all stages of CC-$\pi^0$ selection; b) Efficiency as a function of energy. }
\label{fig:pi0_effs_2}
\end{figure}

\begin{figure}[H]
  \begin{subfigure}[t]{0.35\textwidth}
\includegraphics[scale=0.35]{XSection_Calc_Section/MCVar_pi0_true_gamma_e_max.png}
  \caption{ }
  \end{subfigure} 
  \hspace{15mm}
  \begin{subfigure}[t]{0.35\textwidth}
\includegraphics[scale=0.35]{XSection_Calc_Section/MCVar_eff_pi0_true_gamma_e_max.png}
  \caption{ }
  \end{subfigure} 
\caption{a) Higher energy shower distribution across all stages of CC-$\pi^0$ selection; b) Efficiency as a function of energy. }
\label{fig:pi0_effs_3}
\end{figure}

\begin{figure}[H]
  \begin{subfigure}[t]{0.35\textwidth}
\includegraphics[scale=0.35]{XSection_Calc_Section/MCVar_pi0_true_RL_minE.png}
  \caption{ }
  \end{subfigure} 
  \hspace{15mm}
  \begin{subfigure}[t]{0.35\textwidth}
\includegraphics[scale=0.35]{XSection_Calc_Section/MCVar_eff_pi0_true_RL_minE.png}
  \caption{ }
  \end{subfigure} 
\caption{a) Lower energy shower conversion distances across all stages of CC-$\pi^0$ selection; b) Efficiency as a function of conversion distance. }
\label{fig:pi0_effs_4}
\end{figure}

\begin{figure}[H]
  \begin{subfigure}[t]{0.35\textwidth}
\includegraphics[scale=0.35]{XSection_Calc_Section/MCVar_pi0_true_RL_maxE.png}
  \caption{ }
  \end{subfigure} 
  \hspace{15mm}
  \begin{subfigure}[t]{0.35\textwidth}
\includegraphics[scale=0.35]{XSection_Calc_Section/MCVar_eff_pi0_true_RL_maxE.png}
  \caption{ }
  \end{subfigure} 
\caption{a) Higher energy shower conversion distances across all stages of CC-$\pi^0$ selection; b) Efficiency as a function of energy. }
\label{fig:pi0_effs_5}
\end{figure}


\begin{figure}[H]
  \begin{subfigure}[t]{0.35\textwidth}
\includegraphics[scale=0.35]{XSection_Calc_Section/MCVar_pi0_true_angle.png}
  \caption{ }
  \end{subfigure} 
  \hspace{15mm}
  \begin{subfigure}[t]{0.35\textwidth}
\includegraphics[scale=0.35]{XSection_Calc_Section/MCVar_eff_pi0_true_angle.png}
  \caption{ }
  \end{subfigure} 
\caption{a) $\pi^0$ opening angle distribution across all stages of CC $\pi^0$ selection; b) Efficiency as a function of opening angle. }
\label{fig:pi0_effs_6}
\end{figure}


\clearpage
\subsection{Cross Section}

The first step is to calculate the GENIE flux-averaged, absolute cross section on argon as a baseline of comparison.  The cross section can be calculated according to the following equation:

\begin{equation}
  \sigma = \frac{N_{tagged} - N_{bkgd}}{\epsilon*N_{targ}*\phi}
\end{equation}

\noindent where $N_{tagged}$, $N_{bkgd}$ are the number of tagged and background events respectively, $\epsilon$ is the efficiency, $N_{targ}$ the number of argon nuclei targets and $\phi$ the flux. 
\par The same MicroBooNE neutrino + cosmics simulation used throughout this note is used to perform this initial calculation.  Because MC information is used to calculate a true value here, $\epsilon$ = 1, $N_{bkgd}$=0 and $N_{tagged}$=$N_{signal}$. To calculate the number of argon targets in the FV, the following formula is used

\begin{equation} \label{eq:1}
  N_{targ} = \frac{\rho_{Ar} * V * Avogadro}{m_{mol}} 
\end{equation}
\noindent where $\rho_{Ar}$ is the density of Liquid argon, V is the volume of interest, and $m_{mol}$ is the number of grams per mole of argon.  Using the NIST database at MicroBooNE temperature and pressure for the density of liquid argon, and the FV as the volume of interest, number of targets is calculated to be

\begin{align}
N_{targ} &= \frac{1.3836 [\frac{g}{cm^3}] * 4.246e7 [cm^3] * 6.022e23 [\frac{molec}{mol}]}{39.95 [\frac{g}{mol}]} \\\\
&= 8.855\times10^{29}~\text{molecular targets}.
\end{align}

\par The final step is to calculate the integrated flux.  This is done by integrating over the $\nu_\mu$ flux histogram (Figure \ref{fig:flux}a) provided by the Beam Working Group (internal Ref. \cite{bib:flux}), and normalizing by the 4.232e20 POT in the simulated sample.  Using this POT, the total integrated flux is 3.02e11 $cm^{-2}$ over the full energy range of the flux histograms with $<E>$ = 824 MeV. 
\begin{figure}[H]
\includegraphics[scale=0.36]{XSection_Calc_Section/Misc_numu_MC_flux.png}
\includegraphics[scale=0.36]{XSection_Calc_Section/Misc_numu_flux.png}

\caption{$\nu_\mu$ Flux from Booster Neutrino Beam (BNB) at 470m scaled to a) MicroBooNE simulation of 4.23e20 POT; b) OnBeam 5e19 POT}
\label{fig:flux}
\end{figure}

\noindent Putting it all together, the flux-averaged, absolute, GENIE-extracted cross section for the CC 1$\pi^0$ channel on argon is


\begin{align}
\sigma^{\text{MC}}_{CC\pi^0} &= \frac{7567}{3.02e11 \frac{1}{cm^2} * 8.855e29 Ar } \\\\
&= (2.83 \pm 0.03) *10^{-38} \frac{cm^2}{Ar}
\end{align}


\noindent where the uncertainty presented is purely statistical and dependent only on the number of signal events.  

\par At this point, the cross sections can be extracted from both the two and one shower data samples. The number of targets in this calculation will be the same, but the flux needs to be re-calculated for data POT exposure. The normalized flux is shown in Figure \ref{fig:flux}b and integrated to be 3.51e10 $cm^{-2}$ over the full energy range of the histogram. 

\par The remaining information necessary to calculate the cross section has been discussed earlier in the note. For convenience, this information (along with corresponding Table numbers) is shown in Table \ref{tab:summary_of_xsec_params} for both selections.

\begin{table}[H]
\centering
\captionof{table}{Summary of $N_{OnBeam}$, $N_{OffBeam}$, $N_{MCBkgd}$, and $\epsilon$ pieces that feed into the cross section calculation.  The corresponding Table number that each piece of information can be found in is included in parenthesis next to the entry.  Note that the parameters reported here are not rounded as they are in earlier Tables, in order to give the reader all info needed to reproduce the calculations.  \label{tab:summary_of_xsec_params}}
 \begin{tabular}{|l|l|l|l|l|}
 \hline
 & $N_{OnBeam}$ & $N_{OffBeam}$ & $N_{MCBkgd}$ & $\epsilon~[\%]$ \\ [0.1ex] \hline
2 Shower & 69 $\pm$ 9 (\ref{tab:2shpi0_event_rates}) & 0.00 $\pm$ 2.17 (\ref{tab:2shpi0_event_rates}) & 24.18 $\pm$ 1.68 & 5.6 $\pm$ 0.3 (\ref{tab:pi0_2showers_eventrates})\\ \hline
1 Shower & 257 $\pm$ 16(\ref{tab:pi0_event_rates}) & 15.18 $\pm$ 5.74 (\ref{tab:pi0_event_rates}) & 102.07 $\pm$ 3.44 & 17.0 $\pm$ 0.5 (\ref{tab:pi0_1shower_eventrates}) \\ \hline

\end{tabular}
\end{table}

\par Finally, the cross section is calculated using the two shower results on data:

\begin{align}
\sigma^{\text{Data}}_{CC\pi^0_{2\gamma}} &= \frac{69_{OnBeam} - 0_{OffBeam} - 24.18_{MCBkgd}}{0.056 * 3.51e10 \frac{1}{cm^2} * 8.855e29 Ar} \\\\
&= (2.56 \pm 0.50) *10^{-38} \frac{cm^2}{Ar}
\end{align}

\par The cross section is also calculated using the 1 shower-selection results on data:
\begin{align}
\sigma^{\text{Data}}_{CC\pi^0_{1\gamma}} &= \frac{257_{OnBeam} - 15.18_{OffBeam} - 102.07_{MCBkgd}}{0.17 * 3.51e10 \frac{1}{cm^2} * 8.855e29 Ar} \\\\
&= (2.64 \pm 0.33) *10^{-38} \frac{cm^2}{Ar}
\end{align}

\noindent Both results are within statistical uncertainty of the true calculated cross section above. These results are plotted in Figure \ref{fig:genie_uboone_xsec} on the GENIE extracted cross section plot, along with the flux (arbitrarily normalized).  Only statistical uncertainties are shown for now.

\begin{figure}[h!]
\centering
\includegraphics[width=1\textwidth]{FinalCrossSectionPlots/Final_stat.png}
%\includegraphics[scale=0.35]{XSection_Calc_Section/GenieTruth_stat_sys.png}
\caption{ $\nu_{\mu}+\text{Ar}$ charged current single pion production cross section extracted from GENIE with the MicroBooNE measured cross section using the two and one shower paths shown with statistical uncertainties only. }
\label{fig:genie_uboone_xsec}
\end{figure}

\clearpage
\section{Systematic Uncertainties}
The precision and sensitivity of an experimental measurement depends exactly on how well the contributing models and detector limitations are understood. In the case of MicroBooNE, the nominal cross section uncertainties in the GENIE neutrino generator, modeling of the beam flux, and detector systematics all affect the final measured cross section. The total uncertainty will then ideally be the combination of independent matrices corresponding to each systematic source, as shown below:
% cuts-based variation
\begin{equation}
\label{eq:sys_error}
M^{syst} =  M^{genie} + M^{flux} + M^{detector}
\end{equation}

\noindent In this section, the degree to which each of these sources contributes to the final uncertainty is explored. 
\subsection{Uncertainty Propagation} 
There is a simple prescription to follow when approaching systematic uncertainty evaluation. Contributing parameters and their corresponding degrees of uncertainty must first be identified.  Then, these parameters are varied randomly across many iterations, each time recalculating the cross section.
\par There are two approaches to take in assessing the 1$\sigma$ uncertainties on these variations. If a variation in parameter assignment affects only the rates of event production, a reweighing scheme can be applied to the final distributions rather than re-doing the full simulation for each parameter variation.  This re-weighting strategy can be applied to uncertainty sources such as beam flux and GENIE cross sections.  For example, if the CC-neutrino interaction rate is halved by a parameter adjustment to the underlying neutrino interaction models, we can simply apply a factor of one-half to the final calculations.  On the other hand, if parameter adjustments affect event topology, re-weighting will not be sufficient and a full generation of detector MC must be performed. One example of such a parameter is space charge. Space charge refers to the presence of positively charged ions that are formed when the argon is ionized.  These ions influence the recombination of ionization electrons from new interactions, and can cause distortions in the readout. To handle space charge and other detector effects, samples must be generated for each effect and the impact on our selection studied. A framework exists to do these variations, and studies here are on going. %For now, we limit our systematic evaluations of the flux, GENIE cross sections.

\subsection{GENIE Cross Sections}
The impact of each GENIE parameter variation can be tested by reweighing event distributions using the built-in GENIE event re-weighting framework. Here, each physical parameter P is varied by $\pm$1$\sigma$. The output of this variation is a set of event weights per parameter that represent the output had generation been run with the varied parameter from the beginning. A summary of parameters utilized for this study is summarized in Table \ref{tab:genie_parameters}. MEC parameter variation is currently absent in the re-weighting framework used here, however, this uncertainty will contribute negligibly due to the small portion of MEC events at the final stage of both selection paths.

\begin{table*}
\centering
\captionof{table}{Table of GENIE parameters reproduced for convenience from the GENIE manual \cite{bib:genie} \label{tab:genie_parameters}}
\begin{tabular}{| l | l | l | l |}
\hline
   Param P & Description of P & Value  & $\delta$P/P \\ [0.1ex] \hline
 $M_A^{NCEL}$  & Axial mass for NC elastic & 0.990 GeV & $\pm$25\% \\
 $\eta^{NCEL}$  & Strange axial form factor $\eta$ for NC elastic & 0.120 GeV & $\pm$30\% \\
$M_A^{CCQE}$  & Axial mass for CC quasi-elastic & 0.990 GeV & $\pm$25\% \\
$M_A^{CCRES}$  & Axial mass for CC res $\nu$ production & 1.120 GeV & $\pm$20\% \\
$M_V^{CCRES}$  & Vector mass for CC res $\nu$ production & 0.840 GeV & $\pm$10\% \\
$M_A^{NCRES}$  & Axial mass for NC res $\nu$ production & 1.120 GeV & $\pm$20\% \\
$M_V^{NCRES}$  & Vector mass for NC res $\nu$ production & 0.840 GeV & $\pm$10\% \\
$M_A^{COH}\pi$  & Axial mass for CC,NC coherent $\pi$ production & 1.000 GeV & $\pm$50\% \\
$R_0^{COH}\pi$  & Param. controlling $\pi$ absorp in RS model & 1.000 fm & $\pm$10\% \\ \hline

$A_{HT}^{BY}$  & $A_{HT}$ twist param in scaling variable $\xi_\omega$ & 0.5380 & $\pm$25\% \\
$B_{HT}^{BY}$  & $B_{HT}$ twist param in scaling variable $\xi_\omega$ & 0.305 & $\pm$25\% \\
$C_{V1u}^{BY}$  & $C_{V1u}$ u valence GRV98 PDF correction param & 0.291 & $\pm$30\% \\
$C_{V2u}^{BY}$  & $C_{V2u}$ u valence GRV98 PDF correction param & 0.189 & $\pm$40\% \\ \hline

FormZone  & Hadron formation zone & 0.342 fm & $\pm$50\% \\
BR ($\gamma$)  & Branch ratio for radiative resonance decays & - & $\pm$50\% \\
BR ($\eta$)  & Branch ratio for single-$\eta$ resonance decays & - & $\pm$50\% \\ \hline

$RR_{\nu p}^{CC1\pi}$ & Non-resonance bkg in $\nu$p CC1$\pi$ reactions & - & $\pm$50\% \\ 
$RR_{\nu p}^{CC2\pi}$ & Non-resonance bkg in $\nu$p CC2$\pi$ reactions & - & $\pm$50\% \\ 
$RR_{\nu n}^{CC1\pi}$ & Non-resonance bkg in $\nu$n CC1$\pi$ reactions & - & $\pm$50\% \\ 
$RR_{\nu n}^{CC2\pi}$ & Non-resonance bkg in $\nu$n CC2$\pi$ reactions & - & $\pm$50\% \\ 
$RR_{\nu p}^{NC1\pi}$ & Non-resonance bkg in $\nu$p NC1$\pi$ reactions & - & $\pm$50\% \\ 
$RR_{\nu p}^{NC2\pi}$ & Non-resonance bkg in $\nu$p NC2$\pi$ reactions & - & $\pm$50\% \\ 
$RR_{\nu n}^{NC1\pi}$ & Non-resonance bkg in $\nu$n NC1$\pi$ reactions & - & $\pm$50\% \\ 
$RR_{\nu n}^{NC2\pi}$ & Non-resonance bkg in $\nu$n NC2$\pi$ reactions & - & $\pm$50\% \\  \hline

$x_{abs}^{N}$ & Nucleon absorption probability & - & $\pm$20\% \\ 
$x_{cex}^{N}$ & Nucleon charge exchange probability & - & $\pm$50\% \\ 
$x_{el}^{N}$ & Nucleon elastic reaction probability & - & $\pm$30\% \\ 
$x_{inel}^{N}$ & Nucleon inelastic reaction probability & - & $\pm$40\% \\ 
$x_{mfp}^{N}$ & Nucleon mean free path & - & $\pm$20\% \\
$x_{\pi}^{N}$ & Nucleon $\pi$-production probability & - & $\pm$20\% \\
$x_{abs}^{\pi}$ & $\pi$ absorption probability & - & $\pm$20\% \\
$x_{cex}^{\pi}$ & $\pi$ charge exchange probability & - & $\pm$50\% \\
$x_{el}^{\pi}$ & $\pi$ elastic reaction probability & - & $\pm$10\% \\
$x_{inel}^{\pi}$ & $\pi$ inelastic reaction probability & - & $\pm$40\% \\
$x_{mfp}^{\pi}$ & $\pi$ mean free path & - & $\pm$20\% \\
$x_{\pi}^{\pi}$ & $\pi$ $\pi$-production probability & - & $\pm$20\% \\ \hline
\end{tabular}
\end{table*}

\par Once event re-weighting is done, cross sections are calculated by considering variations to the background B and to the efficiency $\epsilon$, as shown in Equation \ref{eq:genie_xsec_var}. No variation to total tagged events N is considered, as this quantity will be measured directly from data.  To calculate the cross section percentile difference due to a single parameter variation, $\sigma^\pm$ is compared to the nominal $\sigma$ as shown in Equation \ref{eq:genie_xsec_percentdiff}.

\begin{equation} \label{eq:genie_xsec_var}
  \sigma^\pm \propto \frac{N - B^\pm}{\epsilon^\pm} 
\end{equation}

\begin{equation} \label{eq:genie_xsec_percentdiff}
  Percentile\ Difference = \frac{| \sigma - \sigma^\pm |}{\sigma} 
\end{equation}

\par Event re-weighting as described was run over the full 4.232e20 POT MC sample used throughout the note. The results of this re-weighting scheme are shown in Table \ref{tab:genie_results} for the higher variation (i.e., +$\sigma$ or -$\sigma$). In summary, the CC 1-$\pi^0$ analysis chain is most affected by the axial mass for CC resonance production;  this result is supported by Figure \ref{fig:physics_2shower_inttype}, in which resonant processes dominate the neutrino interaction modes of selected events. The final sample is also sensitive to the form zone parameter, nucleon elastic reaction probability, and the nucleon mean free path.   Finally, there is a significant contribution to the total uncertainty from the $\pi$ absorption probability. This is also expected; variation in the probability for $\pi$ absorption in the nucleus is correlated with the resulting number of $\pi^0$'s originating from the nucleus and entering our signal sample.  

\begin{table*}
\centering
\captionof{table}{Percentile Variation from the central value for each parameter variation employed in GENIE. Results for both two and one shower selection chains are shown. \label{tab:genie_results}}
 \begin{tabular}{| l | l | l |}
 \hline
  Parameter P & Perc Var - 2 Shower & Perc Var - 1 Shower  \\ [0.1ex] \hline
$M_A^{NCEL}$ &  0.40\% & 0.35\% \\
$\eta^{NCEL}$  & 0.00\% & 0.01\% \\
$M_A^{CCQE}$  & 1.13\% & 2.66\% \\
$M_V^{CCQE}$  & 0.12\% & 0.16\% \\
$M_A^{CCRES}$  & 5.25\% & 7.00\% \\
$M_V^{CCRES}$  & 3.22\% & 4.19\% \\
$M_A^{NCRES}$  & 1.78\% & 1.24\% \\
$M_V^{NCRES}$  & 0.54\% & 0.31\%\\
$M_A^{COH}\pi$  & 0.00\% & 0.49\% \\
$R_0^{COH}\pi$  & 0.00\% & 0.49\%\\

AGKYpT & 0.00\% & 0.00\% \\
AGKYxF & 0.00\% & 0.00\% \\
DISAth & 0.18\% & 0.14\% \\
DISBth & 0.27\% & 0.21\% \\
DISC$\nu$1u & 0.16\% & 0.09\% \\
DISC$\nu$2u & 0.14\% & 0.08\% \\ \hline

FormZone  & 5.05\% & 4.92\% \\
BR ($\gamma$)  & 0.02\% & 0.27\% \\
BR ($\eta$)  & 3.17\% & 2.24\% \\
BR ($\theta$)  & 2.44\% & 2.00\% \\ \hline

$RR_{\nu p}^{CC1\pi}$ & 0.65\% & 0.30\% \\ 
$RR_{\nu p}^{CC2\pi}$ & 1.00\% & 2.00\% \\
$RR_{\nu n}^{CC1\pi}$ & 2.94\% & 3.24\% \\ 
$RR_{\nu n}^{CC2\pi}$ & 1.06\% & 1.41\% \\ \hline

$x_{abs}^{N}$ & 1.41\% & 1.72\% \\
$x_{cex}^{N}$ & 1.49\% & 1.22\%\\
$x_{el}^{N}$ & 2.52\% & 3.36\% \\
$x_{inel}^{N}$ & 0.79\% & 0.20\% \\
$x_{mfp}^{N}$ & 3.79\% & 3.59\% \\
$x_{\pi}^{N}$ & 0.22\% & 0.46\% \\
$x_{abs}^{\pi}$ & 3.96\% & 3.89\% \\
$x_{cex}^{\pi}$ & 1.49\% & 0.03\% \\
$x_{el}^{\pi}$ & 0.15\% & 0.08\% \\
$x_{inel}^{\pi}$ & 2.57\% & 4.48\% \\
$x_{mfp}^{\pi}$ & 0.40\% & 0.44\% \\
$x_{\pi}^{\pi}$ & 0.21\% & 0.33\% \\
\hline
Total Combined Uncertainty & 12.08\% & 13.80\% \\ \hline
\end{tabular}
\end{table*}


To conclude an assessment of model uncertainties, the uncertainties from each variation are added in quadrature to contribute a 12.08\% uncertainty for the two shower path and a 13.80\% uncertainty for the one shower path.  The higher uncertainty on the one shower sample is due to the lower purity of this selection.

\clearpage
\subsection{Flux Systematics}
When considering uncertainty due to flux, the variations of interest are primarily particle production at the target ($\pi^+$, $\pi^-$, $K^+$, $K^-$, $K^0$) and POT.  The uncertainty due to POT counting (roughly 2\%, measured in the beam hall) is considerably smaller than that due to the flux itself. 
\par To evaluate the flux systematic contribution, variations of a series of parameters are considered.  These parameter variations cover uncertainty on the depth by which the current penetrates into the horn conductor (the ‘skin effect’), the current that the horn is pulsed with, pion and nucleon cross sections (total, inelastic, and quasi-elastic) on aluminum and beryllium and hadron production. In the results report, all non-hadron production uncertainty contributions are combined into one reported value called `FluxUnisim', while the hadronic parameters are kept separate.  Variations to each parameter are determined by spline fits to the HARP cross section data for that parameter used in the flux simulation. 1000 variations of these 6 parameters are performed and 1000 cross sections are calculated per interaction.  This is done using Equation \ref{eq:flux_xsec_var_0} for variation i :

\begin{equation} \label{eq:flux_xsec_var_0}
  \sigma_i \propto \frac{N - B_i}{\epsilon_i \phi_i} 
\end{equation}

\noindent where `N' is the On - OffBeam value from the final stage of each selection path, $B_i$ is the weighted background contribution in the i'th variation, $\epsilon_i$ is the i'th variation efficiency, and $\phi_i$ is the flux renormalization factor for the i'th variation.  The modified fluxes are used to integrate from 0 to 3 GeV and determine the flux through the detector in the i'th variation. Integration across the full energy range increases the overall systematic uncertainty due to the behavior of spline fits to the HARP $\pi^{+}$ production data in regions where there is no HARP data (internal Ref. \cite{bib:flux_uncertainty_tn}). 

Once 1000 cross sections are calculated, 1000 corresponding percentile differences with respect to the nominal are also calculated according to Equation \ref{eq:genie_xsec_percentdiff}. These percentile variations are then used to extract a 1$\sigma$ uncertainty from the nominal value at 0 for each varied flux parameter.  These distributions are shown for the two shower sample in Figures \ref{fig:flux_2shower_unc_plots_0}-\ref{fig:flux_2shower_unc_plots_2} and for the one shower sample in Figures \ref{fig:flux_1shower_unc_plots_0}-\ref{fig:flux_1shower_unc_plots_2}. The uncertainty extracted from each sample is displayed on these plots and additionally summarized in Table \ref{tab:flux_results}.  

\begin{figure}[H]
\centering
\includegraphics[scale=0.35]{XSection_Calc_Section/Flux_perc_var_pi0_FluxUnisim.png}
\includegraphics[scale=0.35]{XSection_Calc_Section/Flux_perc_var_pi0_K0.png}
\caption{ Uncertainty contributions broken down by function for the two shower sample. On the left is the uncertainty contribution from all the non-hadronic processes; on the right are variations due to $K^0$ production. }
\label{fig:flux_2shower_unc_plots_0}
\end{figure}

\begin{figure}[H]
\centering
\includegraphics[scale=0.35]{XSection_Calc_Section/Flux_perc_var_pi0_K+.png}
\includegraphics[scale=0.35]{XSection_Calc_Section/Flux_perc_var_pi0_K-.png}
\caption{ Uncertainty contributions broken down by function for the 2 shower sample. On the left is the uncertainty contribution from variations on $K^+$ production; on the right is the uncertainty due to $K^-$ production. }
\label{fig:flux_2shower_unc_plots_1}
\end{figure}

\begin{figure}[H]
\centering
\includegraphics[scale=0.35]{XSection_Calc_Section/Flux_perc_var_pi0_pi+.png}
\includegraphics[scale=0.35]{XSection_Calc_Section/Flux_perc_var_pi0_pi-.png}
\caption{ Uncertainty contributions broken down by function for the 2 shower sample. On the left is the uncertainty contribution from variations on $\pi^+$ production; on the right is the uncertainty due to $\pi^-$ production. }
\label{fig:flux_2shower_unc_plots_2}
\end{figure}


\begin{figure}[H]
\centering
\includegraphics[scale=0.35]{XSection_Calc_Section/Flux_perc_var_singleshower_FluxUnisim.png}
\includegraphics[scale=0.35]{XSection_Calc_Section/Flux_perc_var_singleshower_K0.png}
\caption{ Uncertainty contributions broken down by function for 1 shower sample. On the left is the uncertainty contribution from all the non-hadronic processes; on the right are variations due to $K^0$ production. }
\label{fig:flux_1shower_unc_plots_0}
\end{figure}

\begin{figure}[H]
\centering
\includegraphics[scale=0.35]{XSection_Calc_Section/Flux_perc_var_singleshower_K+.png}
\includegraphics[scale=0.35]{XSection_Calc_Section/Flux_perc_var_singleshower_K-.png}
\caption{ Uncertainty contributions broken down by function for the 1 shower sample. On the left is the uncertainty contribution from variations on $K^+$ production; on the right is the uncertainty due to $K^-$ production. }
\label{fig:flux_1shower_unc_plots_1}
\end{figure}

\begin{figure}[H]
\centering
\includegraphics[scale=0.35]{XSection_Calc_Section/Flux_perc_var_singleshower_pi+.png}
\includegraphics[scale=0.35]{XSection_Calc_Section/Flux_perc_var_singleshower_pi-.png}
\caption{ Uncertainty contributions broken down by function for the 1 shower sample. On the left is the uncertainty contribution from variations on $\pi^+$ production; on the right is the uncertainty due to $\pi^-$ production. }
\label{fig:flux_1shower_unc_plots_2}
\end{figure}

 \begin{table}[H]
 \centering
 \captionof{table}{ \label{tab:flux_results} Summary of percentile variations to the central value of flux parameters for both 2 and 1 shower paths. }
  \begin{tabular}{| l | l | l |}
  \hline
   Parameter & Perc Var - 2 Shower & Perc Var - 1 Shower  \\ [0.1ex] \hline
 FluxUnisim & 8.3 \% & 9.16 \%  \\
 $K^-$ Production & 0.03\% & 0.34\%\\
 $K^+$ Production &  1.01 \% & 1.07\% \\
 $K^0$ Production & 0.14\% & 0.33\% \\
 $\pi^-$ Production & 0.05\% & 0.22\%\\
 $\pi^+$ Production &  11.75\% & 11.16 \% \\ \hline
 Total Combined Uncertainty & 14.42\% & 14.49\%\\ \hline
\end{tabular}
\end{table}


\clearpage
\subsection{Detector Simulation Systematics}

There are three types of systematics that need to be accounted for when considering the uncertainty on the detector simulation.  There are the uncertainties due to the CORSIKA cosmic models employed in the simulation, the number of targets in the analysis fiducial volume, and finally the overall simulation of various detector effects. 

\subsubsection{MC-based Cosmic Simulation}\label{sec:cosuncert}
 The cosmic ray muon flux has previously been measured to be $141\pm21~\text{Hz~m}^2$ in the MicroBooNE detector hall (internal Ref. \cite{datacosflux}).  This value compares to the CORSIKA-generated flux of $160.9\pm0.3~\text{Hz~m}^2$~\cite{mccosflux} in a comparable location.  Due to the proximity of these values, and the fact that the cosmic showers selected typically originate from cosmic muons (delta-rays, bremsstrahlung, Michel electrons), we know that our simulation of the overall cosmic muon flux is not more than 100\% off.
\par The ``Cosmic + Neutrino'' background makes up 6\% of both the two- and one-shower selected events. To conservatively assess a preliminary uncertainty on the simulation, an overall 100\% uncertainty for both selections is calculated. Given this variation, a resulting systematic on the final cross section of 10\% and 11\% is found for the two and one shower selections respectively. 
 
\subsubsection{Number of Targets}\label{sec:ntarguncert}
The two contributors to potential variation on the calculated number of targets are argon density and fiducial volume. To understand the uncertainty on the density, the temperature and pressure of the detector during the time our data was taken must be understood first. These values are extracted from sensors in the detector and cryogenics system. During the period of data taking for the OnBeam sample in the work, temperature and pressure are measured to be $89.2 \pm 0.3$~K and $1.241 \pm 0.004$~bar respectively. Using these variations, the uncertainty on the liquid argon density is $1.3836^{+0.0019}_{-0.0002}~\frac{\text{g}}{\text{cm}^3}$, a 0.1\% effect. This effect is safely negligible considering the size of the other uncertainties currently taken into account.
\par Second, the effects of space charge on the fiducial volume must be understood. Electric field distortions in the detector (due mainly to passing cosmic ray muons) cause regions of the detector to appear to be within our fiducial volume and vice-versa. To assess an uncertainty on this the TPC volume is divided into $0.0619~\text{cm}^3$ voxels. Simulated space charge displacement maps are then utilized to compute the voxels that moved in and out of the defined fiducial volume. This leads to a change in the fiducial volume of 6.3\%. Treating this as the 1$\sigma$ variation, a fractional change in the cross section measurement of 6.3\% is assigned for both selections. 

\subsubsection{Detector Simulation Unisims}
Detector simulation variations contribute the type of uncertainties that can't easily be modeled by applying a re-weight scheme. To assess these uncertainties, a complementary, full detector simulations per systematic effect must be generated and passed through both selection chains. By using the same set of base events, statistical variation in the sample will be minimized. This is particularly important because the final selection efficiencies are relatively low, and will require a large number of statistics to be able to resolve systematic from statistical effects.  Using these varied simulations, the uncertainties will be assessing by remeasuring the cross section given a central value (CV) sample under the assumption that each unisim represents a 1$\sigma$ shift. Since this is not in principle true, conservative variations in the underlaying parameters will be employed to produce conservative estimates of these uncertainties. This work is on-going, and will conclude in the near future. 

\subsubsection{Total Detector Simulation Uncertainty}
	
The overall detector systematic effect is assessed by taking the quadratic sum of each of the currently estimated effects from Sec.~\ref{sec:cosuncert} and~\ref{sec:ntarguncert}.  This assessment leads to an overall uncertainty on the final cross section of 12\% and 13\% for the two and one shower selections respectively. 

\clearpage 
